\documentclass[12pt]{amsart}
%prepared in AMSLaTeX, under LaTeX2e
\addtolength{\oddsidemargin}{-0.6in} 
\addtolength{\evensidemargin}{-0.6in}
\addtolength{\topmargin}{-.55in}
\addtolength{\textwidth}{1.3in}
\addtolength{\textheight}{.9in}

\renewcommand{\baselinestretch}{1.05}

\usepackage{verbatim} % for "comment" environment

\usepackage{palatino}

\newtheorem*{thm}{Theorem}
\newtheorem*{defn}{Definition}
\newtheorem*{example}{Example}
\newtheorem*{problem}{Problem}
\newtheorem*{remark}{Remark}

\usepackage{fancyvrb,xspace,dsfont}

\usepackage[final]{graphicx}

% macros
\usepackage{amssymb}

\usepackage{hyperref}
\hypersetup{pdfauthor={Ed Bueler},
            pdfcreator={pdflatex},
            colorlinks=true,
            citecolor=blue,
            linkcolor=red,
            urlcolor=blue,
            }

\newcommand{\bn}{\mathbf{n}}
\newcommand{\br}{\mathbf{r}}
\newcommand{\bt}{\mathbf{t}}
\newcommand{\bu}{\mathbf{u}}
\newcommand{\bv}{\mathbf{v}}
\newcommand{\bx}{\mathbf{x}}
\newcommand{\by}{\mathbf{y}}

\newcommand{\cB}{\mathcal{B}}
\newcommand{\cD}{\mathcal{D}}
\newcommand{\cF}{\mathcal{F}}
\newcommand{\cH}{\mathcal{H}}
\newcommand{\cL}{\mathcal{L}}
\newcommand{\cV}{\mathcal{V}}
\newcommand{\cW}{\mathcal{W}}

\newcommand{\CC}{\mathbb{C}}
\newcommand{\NN}{\mathbb{N}}
\newcommand{\RR}{\mathbb{R}}
\newcommand{\ZZ}{\mathbb{Z}}

\renewcommand{\Im}{\mathrm{Im}}
\renewcommand{\Re}{\mathrm{Re}}

\newcommand{\eps}{\epsilon}
\newcommand{\grad}{\nabla}
\newcommand{\lam}{\lambda}
\newcommand{\lap}{\triangle}

\newcommand{\ip}[2]{\ensuremath{\left<#1,#2\right>}}

\newcommand{\image}{\operatorname{im}}
\newcommand{\onull}{\operatorname{null}}
\newcommand{\rank}{\operatorname{rank}}
\newcommand{\range}{\operatorname{range}}
\newcommand{\trace}{\operatorname{tr}}
\newcommand{\Span}{\operatorname{span}}

\newcommand{\prob}[1]{\bigskip\noindent\textbf{#1.}\quad }

\newcommand{\epart}[1]{\medskip\noindent\textbf{(#1)}\quad }
\newcommand{\ppart}[1]{\,\textbf{(#1)}\quad }

\newcommand{\ds}{\displaystyle}

\newcommand{\snew}{s^{\text{new}}}
\newcommand{\rhoi}{\rho_{\text{i}}}



\begin{document}
\scriptsize \hfill \emph{(Bueler) June 2024}
\normalsize\medskip

\Large\centerline{\textbf{Solutions to \underline{Paper} Exercises}}

\normalsize
\medskip
\begin{quote}
\emph{Acronyms: SKE = surface kinematical equation, SIA = shallow ice approximation, CFL = Courant, Friedrichs, Lewy stability criterion.}
\end{quote}

\prob{1}  One way to think about normal vectors is to first construct the general form of a tangent vector.  Ignoring time, in 2D we have a curve $z=s(x)$ and in 3D a surface $z=s(x,y)$.

In the first case, for the tangent line one can move the $x$ value by any number $\Delta x$.  Then by the linearization (differential) of $s$ we have $\Delta s= \frac{\partial s}{\partial x} \Delta x$ as the change in the $z$ direction.  (One could write $\Delta s= \frac{ds}{dx} \Delta x$ in this case.)  Thus any tangent vector is
	$$\bt_s = \left(\Delta x, \frac{\partial s}{\partial x} \Delta x\right)$$
for some value of $\Delta x$.  Now for the given vector $\bn_s = (-\frac{\partial s}{\partial x},1)$ we see that
	$$\bn_s \cdot \bt_s = -\Delta x \frac{\partial s}{\partial x} + \frac{\partial s}{\partial x} \Delta x = 0$$
so $\bn_s$ is orthogonal ($=$perpendicular$=$normal) to any tangent vector.

In 3D the formulas are
\begin{align*}
\bt_s &= \left(\Delta x, \Delta y, \frac{\partial s}{\partial x} \Delta x + \frac{\partial s}{\partial y} \Delta y\right) \\
\bn_s &= \left(-\frac{\partial s}{\partial x},-\frac{\partial s}{\partial y},1\right)
\end{align*}
Note that in $\bt_s$ one can choose any values $\Delta x,\Delta y$; there is a tangent \emph{plane}.  It is easy to check that $\bn_s \cdot \bt_s = 0$.


\prob{2}  For the ice thickness $H=s-b$, the inequality equivalent to $s\ge b$ is $H\ge 0$.

Many geophysical problems involve a layer of viscous fluid or other material.  Sometimes the corresponding models describe (parameterize) the layer geometry using thickness or a surface/bed elevation pair.  Examples of these are ocean models, sea ice models, lahar flows, avalanches, and snow fall.

For subglacial hydrology, some models use water layer thickness, for example thin films and (spatially-averaged) linked-cavities.  Other cases describe geometry in other ways, for instance cross-sectional area $S$ for conduits.  Note ``$S\ge 0$'' is relevant to conduits which freeze shut.

For practical atmosphere models of the Earth, relevant to weather or climate, the ``$H\ge 0$'' constraint is irrelevant because an atmosphere is always present.

For ocean models the thickness goes to zero at the ocean shore, but these models can generally ``get away'' with simple wetting/drying mesh schemes because the shoreline location is not highly dynamic.  Exceptions include estuarine and tsunami run-up models, or the Aral Sea for example; in these models the ``$H\ge 0$'' constraint is an active participant in determining the model domain.

Glacier, subglacial hydrology, and sea ice models all track layer geometry using a thickness (or $s,b$ pair) geometry description, and in cases where the time scales of surface mass processes and flow are comparable.  Thus a margin (terminus, edge, \dots) shape is generated by an ablation process acting simultaneously with a flow (or elastic/plastic displacement, etc.) dynamical process.  The inequality constraint must be actively enforced to determine the modeled shape of the margin.


\prob{3}  The upwind scheme for the SKE is in the slides.  In the case where $a_j=0$ and $w_j=0$, and assuming non-negative horizontal velocity $u_j\ge 0$, the scheme is:
    $$\snew_j = s_j - \Delta t\, u_j \frac{s_j-s_{j-1}}{\Delta x}$$

Collecting terms we get constants in the suggested form:
	$$\snew_j = \underbrace{\left(\frac{u_j \Delta t}{\Delta x}\right)}_{c_{-1}} s_{j-1} + \underbrace{\left(1 - \frac{u_j \Delta t}{\Delta x}\right)}_{c_0} s_j$$
If $u_j\ge 0$ and $|u_j|\Delta t = u_j\Delta t\le \Delta x$ then we see that both constants are nonnegative.  (We are also assuming $\Delta t>0$ and $\Delta x > 0$.)  Furthermore the constants add to one:
	$$c_{-1} + c_0 = \frac{u_j \Delta t}{\Delta x} + 1 - \frac{u_j \Delta t}{\Delta x} = 1.$$
The new surface value is a strict average of the current surface elevation pair $s_{j-1},s_j$.  A similar result holds for rightward flow.  That is, for $u_j < 0$, and assuming CFL $|u_j|\Delta t \le \Delta x$, one gets $\snew_j = c_0 s_j + c_1 s_{j+1}$ as a strict average.

Thus, in this $a=0$ and $w=0$ case the upwind method is stable \emph{if one also uses time steps $\Delta t$ which satisfy CFL}.  Unstable modal growth is then impossible.  Specifically, if the current surface is any wave of any amplitude, the updated surface may still be wavy but its amplitude will be reduced because any bumps are averaged-out.

If $a,w$ are treated as independent of $s$, which is actually quite unrealistic, then one can extend this argument to show that the upwind scheme is still stable in the same basic sense.


\prob{4}  Recall that $H=s-b$.  In what follows we will assume that the SKE holds, that the ice is incompressible, and that the base is non-sliding and non-penetrating.  The result of the derivation will be the mass continuity equation.

One can start with the flux divergence term in the mass continuity equation, then apply the definition of $\bar u$, and then apply the Leibniz rule.\footnote{In this application of the Leibniz rule $t$ is passive.  The action is in the $x,z$ variables.}  Then apply the basal condition on $u$, and use incompressibility to rewrite the $x$ derivative of $u$ as the negative $z$ derivative of $w$:
\begin{align*}
\frac{\partial}{\partial x}(\bar u H) &= \frac{\partial}{\partial x}\left(\int_{b(t,x)}^{s(t,x)} u(t,x,z)\,dz\right) \\
   &= \frac{\partial s}{\partial x}(t,x) u(t,x,s(t,x)) - \frac{\partial b}{\partial x}(x) u(t,x,b(x)) + \int_{b(t,x)}^{s(t,x)} \frac{\partial u}{\partial x}(t,x,z)\,dz \\
   &= \frac{\partial s}{\partial x}(t,x) u(t,x,s(t,x)) - \int_{b(t,x)}^{s(t,x)} \frac{\partial w}{\partial z}(t,x,z)\,dz
\end{align*}
(The $u \partial s/\partial x$ term in the SKE has already appeared.)  Now apply the fundamental theorem of calculus and the non-penetrating condition:
\begin{align*}
\frac{\partial}{\partial x}(\bar u H) &= \frac{\partial s}{\partial x}(t,x) u(t,x,s(t,x)) - w(t,x,s(t,x)) + w(t,x,b(t,x)) \\
  &= \frac{\partial s}{\partial x}(t,x) u(t,x,s(t,x)) - w(t,x,s(t,x))
\end{align*}

Thus we have turned the flux divergence into those dynamic terms which appear in the SKE.  From now on we simplify appearance, not showing the $t,x$ dependence explicitly.  Because $\partial b/\partial t=0$ by assumption, we know $\partial s/\partial t = \partial H/\partial t$.  Also, we can write the SKE as $u \frac{\partial s}{\partial x} - w = -\frac{\partial s}{\partial t} + a$.  Thus
	$$\frac{\partial}{\partial x}(\bar u H) = u \frac{\partial s}{\partial x} - w = -\frac{\partial s}{\partial t} + a = -\frac{\partial H}{\partial t} + a$$
This is the mass continuity equation.


\prob{5}  Starting from the 5 boxed equations on slide 24, the formulas for $w(z)$, $p(z)$, and $\tau_{13}(z)$ on slide 25 are straightforward to get by integration, using the given boundary conditions, under the additional\footnote{I forgot to mention this boundary condition in the slides!  Sorry about that.  The idea is that the atmosphere applies no (significant) shearing to the ice surface.} condition that $\tau_{13}(H)=0$.

Now using the boundary condition $u(0)=u_0$ and the formula for $\tau_{13}(z)$ we have
    $$u(z) = u_0 + 2 A (\rho g \sin\alpha)^3 \int_0^z (H-z')^3\,dz'$$
Use the substitution $\zeta = H-z'$ and get the desired slab-on-slope horizontal velocity formula:
\begin{align*}
u(z) &= u_0 - 2 A (\rho g \sin\alpha)^3 \int_H^{H-z} \zeta^3\,d\zeta \\
     &= u_0 - \frac{2}{4} A (\rho g \sin\alpha)^3  \left((H-z)^4 - H^4\right) \\
     &= u_0 + \frac{1}{2} A (\rho g \sin\alpha)^3  \left(H^4 - (H-z)^4\right)
\end{align*}

The claim is that the SIA uses this slab-on-a-slope formula, but with the local, i.e.~$x$-dependent, surface slope and thickness.  So, again using $\tan\alpha \approx \sin\alpha$, we should regard ``$\sin\alpha$'' as the local slope $\partial s/\partial x$.  On the other hand, we want to consider both positive and negative values for $\sin\alpha$, and we observe that the ice flows downhill whether it is flowing left or flowing right.  So if $m$ is the slope we want the 3rd power to behave properly:
	$$(-m)^3 \qquad \stackrel{\text{replace}}{\to} \qquad -|m|^2 m.$$

Thus we have the non-sliding, isothermal SIA formula
    $$u(t,x,z) = u_0 - \frac{1}{2} A (\rho g)^3 \left|\frac{\partial s}{\partial x}\right|^2 \frac{\partial s}{\partial x} \left((s-b)^4 - (s-z)^4\right).$$
In this expression $s=s(t,x)$, $H=H(t,x)$, and $u_0(t,x)$.  However, the expression only makes sense at a point which is inside the ice:
	$$b(x) < z < s(t,x)$$

Before believing this stuff, please note that one does not get the basal sliding function $u_0(t,x)$ as data!  Rather, one needs to solve stress balances \emph{both} for deformation within the ice (e.g.~$\partial u/\partial z$) \emph{and} for the basal sliding velocity $u_0$.  In this context, there is no way for the SIA simplified model to balance the longitudinal (or ``membrane'') stresses which control the value $u_0$.  For this purpose, 21st-century practitioners recommend that the sliding be determined by solving a non-trivializing stress balance.  That is, one should either use a hybrid balance combining the SIA and the shallow shelf approximation, or a unified balance like the Blatter-Pattyn equations, or even the (full) Stokes model.  In glaciology there is no physically-based model for sliding which is as cheap as the SIA for internal deformation.


\prob{6}  The basic point, when one is thinking about how the non-sliding SIA model converts a given glacier shape into a surface velocity field, is that ice flows downhill in this model.  That is, the SIA model always makes the horizontal component point opposite the surface gradient: $u \sim - \partial s/\partial x$.

The next point is that the magnitude of the horizontal surface velocity $|u|$ scales with the 4th power\footnote{Assuming $n=3$ as usual.} of the ice thickness.  Relatively small increases in thickness correspond to substantial increases in surface speed.

Not so obvious is how the vertical component $w$ works in the non-sliding SIA model.  It acts diffusively.  Specifically, the nontrivial term in the SKE has the following surface value in the planar case:
\begin{align*}
\bu \cdot \bn_s &= \left(u,w\right) \cdot \left(-\frac{\partial s}{\partial x},1\right) \\
    &= \frac{1}{2}  A (\rho g)^3 H^{4} \left|\frac{\partial s}{\partial x}\right|^{4} + \frac{\partial}{\partial x} \left(\frac{2}{5}  A (\rho g)^3 H^{5} \left|\frac{\partial s}{\partial x}\right|^{2} \frac{\partial s}{\partial x}\right)
\end{align*}
The $w$ contribution to the SKE is of the form
	$$w = \frac{\partial}{\partial x} \left(D \frac{\partial s}{\partial x}\right)$$
where the diffusivity $D>0$ has a highly nonlinear power form.  (A 5th power on thickness and a second power on the magnitude of the surface slope.)  This means, roughly speaking, that a smooth maximum of $s$ tends to be decreased ($\partial s/\partial t < 0$) over time.  Correspondingly, a smooth minimum in $s$ will generally be increased.

In summary the non-sliding SIA velocity can be roughly sketched as downhill, but with a vertical component that tends to damp-out surface bumps.

By contrast, it requires more tuition than I have seen, or have experienced, to see at a glance what the velocity from the Stokes model looks like.  At each location $x$ the surface velocity from a non-sliding Stokes model has a highly-nontrivial response to the longitudinal stresses transmitted from neighboring ice.  Nonetheless, this response to longitudinal stress balance is superimposed on a large-scale response to the direction of gravity relative to the general surface slope.  Over several ice thicknesses or more, the Stokes velocity is downhill and damping.\footnote{The SIA model is not lying.  Rather, it only makes sense to use the SIA when you are seeking the response to surface and thickness features over a horizontal range of several ice thickness.}  However, actually running a Stokes model for a given geometry would seem to be most helpful when thinking about the surface dynamical response.  This statement is even more true if there is sliding.

\end{document}
