\documentclass[12pt]{amsart}
%prepared in AMSLaTeX, under LaTeX2e
\addtolength{\oddsidemargin}{-.55in} 
\addtolength{\evensidemargin}{-0.8in}
\addtolength{\topmargin}{-.55in}
\addtolength{\textwidth}{1.3in}
\addtolength{\textheight}{.9in}

\renewcommand{\baselinestretch}{1.05}

\usepackage{verbatim} % for "comment" environment

\usepackage{palatino}

\newtheorem*{thm}{Theorem}
\newtheorem*{defn}{Definition}
\newtheorem*{example}{Example}
\newtheorem*{problem}{Problem}
\newtheorem*{remark}{Remark}

\usepackage{fancyvrb,xspace,dsfont}

\usepackage[final]{graphicx}

% macros
\usepackage{amssymb}

\usepackage{hyperref}
\hypersetup{pdfauthor={Ed Bueler},
            pdfcreator={pdflatex},
            colorlinks=true,
            citecolor=blue,
            linkcolor=red,
            urlcolor=blue,
            }

\newcommand{\bn}{\mathbf{n}}
\newcommand{\br}{\mathbf{r}}
\newcommand{\bt}{\mathbf{t}}
\newcommand{\bv}{\mathbf{v}}
\newcommand{\bx}{\mathbf{x}}
\newcommand{\by}{\mathbf{y}}

\newcommand{\cB}{\mathcal{B}}
\newcommand{\cD}{\mathcal{D}}
\newcommand{\cF}{\mathcal{F}}
\newcommand{\cH}{\mathcal{H}}
\newcommand{\cL}{\mathcal{L}}
\newcommand{\cV}{\mathcal{V}}
\newcommand{\cW}{\mathcal{W}}

\newcommand{\CC}{\mathbb{C}}
\newcommand{\NN}{\mathbb{N}}
\newcommand{\RR}{\mathbb{R}}
\newcommand{\ZZ}{\mathbb{Z}}

\renewcommand{\Im}{\mathrm{Im}}
\renewcommand{\Re}{\mathrm{Re}}

\newcommand{\eps}{\epsilon}
\newcommand{\grad}{\nabla}
\newcommand{\lam}{\lambda}
\newcommand{\lap}{\triangle}

\newcommand{\ip}[2]{\ensuremath{\left<#1,#2\right>}}

\newcommand{\image}{\operatorname{im}}
\newcommand{\onull}{\operatorname{null}}
\newcommand{\rank}{\operatorname{rank}}
\newcommand{\range}{\operatorname{range}}
\newcommand{\trace}{\operatorname{tr}}
\newcommand{\Span}{\operatorname{span}}

\newcommand{\prob}[1]{\bigskip\noindent\textbf{#1.}\quad }

\newcommand{\epart}[1]{\medskip\noindent\textbf{(#1)}\quad }
\newcommand{\ppart}[1]{\,\textbf{(#1)}\quad }

\newcommand{\ds}{\displaystyle}

\newcommand{\snew}{s^{\text{new}}}



\begin{document}
\scriptsize \hfill \emph{(Bueler) June 2024}
\normalsize\medskip

\Large\centerline{\textbf{Solutions to Paper Exercises}}

\normalsize
\medskip
\begin{quote}
\emph{Acronyms: SKE = surface kinematical equation, SIA = shallow ice approximation, CFL = Courant, Friedrichs, Lewy stability criterion.}
\end{quote}

\prob{1}  One way to think about normal vectors is to first construct the general form of a tangent vector.  Ignoring time, in 2D we have a curve $z=s(x)$ and in 3D a surface $z=s(x,y)$.

In the first case, for the tangent line one can move the $x$ value by any number $\Delta x$.  Then by the linearization (differential) of $s$ we have $\Delta s= \frac{\partial s}{\partial x} \Delta x$ as the change in the $z$ direction.  (One could write $\Delta s= \frac{ds}{dx} \Delta x$ in this case.)  Thus any tangent vector is
	$$\bt_s = \left(\Delta x, \frac{\partial s}{\partial x} \Delta x\right)$$
for some value of $\Delta x$.  Now for the given vector $\bn_s = (-\frac{\partial s}{\partial x},1)$ we see that
	$$\bn_s \cdot \bt_s = -\Delta x \frac{\partial s}{\partial x} + \frac{\partial s}{\partial x} \Delta x = 0$$
so $\bn_s$ is orthogonal ($=$perpendicular$=$normal) to any tangent vector.

In 3D the formulas are
\begin{align*}
\bt_s &= \left(\Delta x, \Delta y, \frac{\partial s}{\partial x} \Delta x + \frac{\partial s}{\partial y} \Delta y\right) \\
\bn_s &= \left(-\frac{\partial s}{\partial x},-\frac{\partial s}{\partial y},1\right)
\end{align*}
Note that in $\bt_s$ one can choose any values $\Delta x,\Delta y$; there is a tangent \emph{plane}.  It is easy to check that $\bn_s \cdot \bt_s = 0$.


\prob{2}  For the ice thickness $H=s-b$, the inequality equivalent to $s\ge b$ is $H\ge 0$.

Many geophysical problems involve a layer of viscous fluid or other material.  Sometimes the corresponding models describe (parameterize) the layer geometry using thickness or a surface/bed elevation pair.  Examples of these are ocean models, sea ice models, lahar flows, avalanches, and snow fall.

For subglacial hydrology, some models use water layer thickness, for example thin films and (spatially-averaged) linked-cavities.  Other cases describe geometry in other ways, for instance cross-sectional area $S$ for conduits.  Note ``$S\ge 0$'' is relevant to conduits which freeze shut.

For practical atmosphere models of the Earth, relevant to weather or climate, the ``$H\ge 0$'' constraint is irrelevant because an atmosphere is always present.

For ocean models the thickness goes to zero at the ocean shore, but these models can generally ``get away'' with simple wetting/drying mesh schemes because the shoreline location is not highly dynamic.  Exceptions include estuarine and tsunami run-up models, or the Aral Sea for example; in these models the ``$H\ge 0$'' constraint is an active participant in determining the model domain.

Glacier, subglacial hydrology, and sea ice models all track layer geometry using a thickness (or $s,b$ pair) geometry description, and in cases where the time scales of surface mass processes and flow are comparable.  Thus a margin (terminus, edge, \dots) shape is generated by an ablation process acting simultaneously with a flow (or elastic/plastic displacement, etc.) dynamical process.  The inequality constraint must be actively enforced to determine the modeled shape of the margin.


\prob{3}  The upwind scheme for the SKE is in the slides.  In the case where $a_j=0$ and $w_j=0$, and assuming non-negative horizontal velocity $u_j\ge 0$, the scheme is:
    $$\snew_j = s_j - \Delta t\, u_j \frac{s_j-s_{j-1}}{\Delta x}$$

Collecting terms we get constants in the suggested form:
	$$\snew_j = \underbrace{\left(\frac{u_j \Delta t}{\Delta x}\right)}_{c_{-1}} s_{j-1} + \underbrace{\left(1 - \frac{u_j \Delta t}{\Delta x}\right)}_{c_0} s_j$$
If $u_j\ge 0$ and $|u_j|\Delta t = u_j\Delta t\le \Delta x$ then we see that both constants are nonnegative.  (We are also assuming $\Delta t>0$ and $\Delta x > 0$.)  Furthermore the constants add to one:
	$$c_{-1} + c_0 = \frac{u_j \Delta t}{\Delta x} + 1 - \frac{u_j \Delta t}{\Delta x} = 1.$$
The new surface value is a strict average of the current surface elevation pair $s_{j-1},s_j$.  A similar result holds for rightward flow.  That is, for $u_j < 0$, and assuming CFL $|u_j|\Delta t \le \Delta x$, one gets $\snew_j = c_0 s_j + c_1 s_{j+1}$ as a strict average.

Thus, in this $a=0$ and $w=0$ case the upwind method is stable \emph{if one also uses time steps $\Delta t$ which satisfy CFL}.  Unstable modal growth is then impossible.  Specifically, if the current surface is any wave of any amplitude, the updated surface may still be wavy but its amplitude will be reduced because any bumps are averaged-out.

If $a,w$ are treated as independent of $s$, which is actually quite unrealistic, then one can extend this argument to show that the upwind scheme is still stable in the same basic sense.


\prob{4}  Recall that $H=s-b$.  In what follows we will assume that the SKE holds, that the ice is incompressible, and that the base is non-sliding and non-penetrating.  The result of the derivation will be the mass continuity equation.

One can start with the flux divergence term in the mass continuity equation, then apply the definition of $\bar u$, and then apply the Leibniz rule.\footnote{In this application of the Leibniz rule $t$ is passive.  The action is in the $x,z$ variables.}  Then apply the basal condition on $u$, and use incompressibility to rewrite the $x$ derivative of $u$ as the negative $z$ derivative of $w$:
\begin{align*}
\frac{\partial}{\partial x}(\bar u H) &= \frac{\partial}{\partial x}\left(\int_{b(t,x)}^{s(t,x)} u(t,x,z)\,dz\right) \\
   &= \frac{\partial s}{\partial x}(t,x) u(t,x,s(t,x)) - \frac{\partial b}{\partial x}(x) u(t,x,b(x)) + \int_{b(t,x)}^{s(t,x)} \frac{\partial u}{\partial x}(t,x,z)\,dz \\
   &= \frac{\partial s}{\partial x}(t,x) u(t,x,s(t,x)) - \int_{b(t,x)}^{s(t,x)} \frac{\partial w}{\partial z}(t,x,z)\,dz
\end{align*}
(The $u \partial s/\partial x$ term in the SKE has already appeared.)  Now apply the fundamental theorem of calculus and the non-penetrating condition:
\begin{align*}
\frac{\partial}{\partial x}(\bar u H) &= \frac{\partial s}{\partial x}(t,x) u(t,x,s(t,x)) - w(t,x,s(t,x)) + w(t,x,b(t,x)) \\
  &= \frac{\partial s}{\partial x}(t,x) u(t,x,s(t,x)) - w(t,x,s(t,x))
\end{align*}

Thus we have turned the flux divergence into those dynamic terms which appear in the SKE.  From now on we simplify appearance, not showing the $t,x$ dependence explicitly.  Because $\partial b/\partial t=0$ by assumption, we know $\partial s/\partial t = \partial H/\partial t$.  Also, we can write the SKE as $u \frac{\partial s}{\partial x} - w = -\frac{\partial s}{\partial t} + a$.  Thus
	$$\frac{\partial}{\partial x}(\bar u H) = u \frac{\partial s}{\partial x} - w = -\frac{\partial s}{\partial t} + a = -\frac{\partial H}{\partial t} + a$$
This is the mass continuity equation.

\prob{5}  FIXME Write out the details of the slab-on-a-slope calculation from the slides.  Thereby derive the ($n=3$) velocity formula $u(z) = u_0 + \frac{1}{2} A (\rho g \sin\alpha)^3  \left(H^4 - (H-z)^4\right)$.  Now add in $x$-dependence, to see the velocity formula for the SIA model velocity.

\prob{6}  \emph{(Do with a friend.)}  Sketch a hypothetical planar glacier shape, with smooth surface and bed.  Sketch what you think the non-sliding SIA formulas will generate for the surface values of the horizontal ($u$) and vertical ($w$) velocity components.  How would the non-sliding Stokes model change your pictures?  Repeat with a different hypothetical shape.

%\prob{7}  Modify \href{https://github.com/bueler/mccarthy/blob/master/py/surface1d.py}{\texttt{surface1d.py}} to systematically test its stability.  That is, demonstrate conditional stability of the upwind numerical scheme for the SKE, by varying the grid spacings $\Delta t$ and $\Delta x$, in the given-velocity case.  Confirm that the CFL condition $|u_j| \frac{\Delta t}{\Delta x} \le 1$ predicts which cases destabilize and which remain stable.

%\prob{8}  Modify \href{https://github.com/bueler/mccarthy/blob/master/py/surface1d.py}{\texttt{surface1d.py}} to show that the scheme can generate surfaces which fail $s\ge b$, i.e.~which violate \emph{admissibility}.  (One can increase the ablation rate to cause this.)  Modify the scheme so that it maintains admissibility via \emph{truncation}.  That is, when a preliminary surface elevation ${\tilde s}_j^{\text{new}}$ is calculated, apply $s_j^{\text{new}} = \max\{{\tilde s}_j^{\text{new}},b_j\}$.

%\prob{9}  Construct a case where the horizontal velocity depends on time and space: $u=u(t,x)$.  Modify \href{https://github.com/bueler/mccarthy/blob/master/py/surface1d.py}{\texttt{surface1d.py}} to use this case.  Modify the time-stepping scheme so that the time step is determined via the CFL condition.  That is, at each time step, find the largest horizontal speed and adjust the next $\Delta t$ accordingly.

%\prob{10}  Modify \href{https://github.com/bueler/mccarthy/blob/master/py/surface1d.py}{\texttt{surface1d.py}} so that the data $a,u,w$ can be read from a file.  Check that it runs as before, that is, for values which are the same as in the unmodified code.

%\prob{11}  In a suitable textbook, read about \emph{implicit time-stepping}, and perhaps also about the \emph{method of lines}.  Modify \href{https://github.com/bueler/mccarthy/blob/master/py/surface1d.py}{\texttt{surface1d.py}} to use an implicit scheme.  Do you see benefits?  What replaces truncation---see \textbf{7 c)}---in an implicit scheme?

%\prob{12}  Modify the ice geometry in \href{https://github.com/bueler/mccarthy/blob/master/py/shallowuw.py}{\texttt{shallowuw.py}} to a case where the glacier terminates.  By trying different fractional power shapes for the terminus (margin), show that in some cases the SIA surface velocity goes to zero as one approaches the terminus, and in some cases not.  (\emph{Look up ``Vialov profile'' for a bad case.})  Explain what is going on in terms of the powers which appear in the SIA formulas for surface velocity.

%\prob{13}  Combine \href{https://github.com/bueler/mccarthy/blob/master/py/surface1d.py}{\texttt{surface1d.py}} and \href{https://github.com/bueler/mccarthy/blob/master/py/shallowuw.py}{\texttt{shallowuw.py}} into one code which numerically solves the time-dependent SIA model via explicit time stepping.  That is, alternately compute surface velocity from the current geometry (\href{https://github.com/bueler/mccarthy/blob/master/py/shallowuw.py}{\texttt{shallowuw.py}} does this), and then update the geometry using a time step of the SKE (\href{https://github.com/bueler/mccarthy/blob/master/py/surface1d.py}{\texttt{surface1d.py}} does this).  Now what is the time step condition for stability?  How might one detect instability as the scheme runs?
\end{document}
