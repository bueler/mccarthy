\documentclass[letterpaper,final,12pt,reqno]{amsart}

\usepackage[total={6.1in,9.0in},top=1.1in,left=1.3in]{geometry}

\usepackage{verbatim}
\usepackage{empheq}
\usepackage[dvipsnames]{xcolor}
\usepackage{animate}
\usepackage{graphicx}
\usepackage{fancyvrb}

% hyperref should be the last package we load
\usepackage[pdftex,
colorlinks=true,
plainpages=false, % only if colorlinks=true
linkcolor=blue,   % only if colorlinks=true
citecolor=Red,   % only if colorlinks=true
urlcolor=black     % only if colorlinks=true
]{hyperref}

\pdfinfo{
/Title (Numerical modelling of glaciers, ice sheets, and ice shelves)
/Author (Ed Bueler)
/Subject (numerical modelling of ice sheets)
/Keywords (numerical modelling, numerical analysis, glacier, ice sheet, ice shelf, shallow models)
}

\renewcommand{\baselinestretch}{1.05}

\newcommand{\ddt}[1]{\ensuremath{\frac{\partial #1}{\partial t}}}
\newcommand{\ddx}[1]{\ensuremath{\frac{\partial #1}{\partial x}}}
\newcommand{\ddy}[1]{\ensuremath{\frac{\partial #1}{\partial y}}}
\newcommand{\pp}[2]{\ensuremath{\frac{\partial #1}{\partial #2}}}
\renewcommand{\t}[1]{\texttt{#1}}
\newcommand{\Matlab}{\textsc{Matlab}\xspace}
\newcommand{\bq}{\mathbf{q}}
\newcommand{\bu}{\mathbf{u}}
\newcommand{\bU}{\mathbf{U}}
\newcommand{\eps}{\epsilon}
\newcommand{\grad}{\nabla}
\newcommand{\Div}{\nabla\cdot}
\newcommand{\devstress}{\tau}

\newcommand{\minput}[1]{
\vspace{0.8cm}
\VerbatimInput[frame=single,framesep=3mm,label=\fbox{\normalsize \textsl{\,#1.m\,}},fontfamily=courier,fontsize=\footnotesize]{tmp/#1.slim.m}
\vspace{0.5cm}
}

% usage:  \onefigsize{name}{caption}{width}
\newcommand{\onefigsize}[3]{
\begin{figure}[ht]
\centering
\includegraphics[width=#3,keepaspectratio=true]{#1}
\caption{#2}
\label{fig:#1}
\end{figure}}

% usage:  \onefig{name}{caption}
\newcommand{\onefig}[2]{\onefigsize{#1}{#2}{3.0in}}

% usage:  \twofigsizes{left-name}{right-name}{caption}{left-width}{right-width}
\newcommand{\twofigsizes}[5]{
\begin{figure}[ht]
\centering
\includegraphics[width=#4,keepaspectratio=true]{#1} \quad
\includegraphics[width=#5,keepaspectratio=true]{#2}
\caption{#3}
\label{fig:#1}
\end{figure}}

% usage:  \twofig{left-name}{right-name}{caption}
\newcommand{\twofig}[3]{\twofigsizes{#1}{#2}{#3}{2.5in}{2.5in}}



\begin{document}
\graphicspath{{../figures/}}

\begin{titlepage}

  \begin{center}
  \phantom{foo}
    \vspace{1.0cm}

     {\Large \textsc{Numerical modelling}}
    \vspace{0.7cm}

     {\Large \textsc{of glaciers, ice sheets, and ice shelves}}

    \vspace{1.5cm}

    {\large Ed Bueler}
    \vspace{1cm}

    International Summer School in Glaciology

    McCarthy Alaska

    2010, 2012, 2014, 2016, 2018, 2022, 2024

    \vfill
    
    \includegraphics[width=6.0in]{flowline}
  
    \scriptsize \emph{Illustrates the notation used in these notes.  Figure modified from \cite{SchoofMarine1}.} \normalsize
    
    \vspace{1.5in}
  \end{center}
\end{titlepage}

\clearpage\newpage

\setcounter{page}{2}
%\setcounter{section}{1}
\section{Introduction}  \label{sec:intro}

The greatest importance of mathematical and numerical models to science, and to glaciology in particular, is in \emph{introspection}.  In building and using a model you are asking yourself: When I combine my imperfect and incomplete understanding of glacier processes into this model does it behave as I expect?  A person who does not actually understand your scientific explanation might nod and say ``yes, I understand.''  The dumb computer, however, can only tell you consequences of your model, inadequacies and all.  Numerical models can show how processes interact to give overall behavior, and they can demonstrate flaws in understanding of those processes.

But numerical models must be built with care.  An avoidable bad outcome is to spend time---or worse, reputation---interpreting, explaining, or justifying model behavior that is actually an artifact of buggy computer programming or flawed numerical analysis.

Continuum models, and not computer codes, will be the primary focus of these notes.  (This might be surprising!)  While all codes produce numbers, we want numbers that actually come from the intended continuum model.  Only careful analysis of a numerical implementation can confirm that it matches the model equations.

\subsection*{Scope}  These notes have a limited scope:
  \begin{quote}\emph{shallow approximations of ice flow.}\end{quote}
They adopt a constructive approach:
  \begin{quote}\emph{we provide understandable numerical codes that actually run.}\end{quote}

Our scope includes the shallow ice approximation (SIA) in two horizontal dimensions (2D), the shallow shelf approximation (SSA) in 1D, and the mass continuity and surface kinematical equations.  (These technical terms will be defined in due course.)  We also recall the Stokes model here, and a separate Appendix addresses its numerical solution.

Numerical concepts within our scope include finite difference schemes, solving algebraic systems from stress balances, and the verification of codes using exact solutions.

\subsection*{Notation}  Our notation, which generally follows \cite{GreveBlatter2009}, is covered by Table \ref{tab:notation}.  Cartesian coordinates are $x,y,z$ with $z$ positive-upward, plus time $t$.  These coordinates appear as subscripts to denote partial derivatives: $u_x = \partial u/\partial x$.  Tensor notation uses subscripts from the list $\{1,2,3,i,j\}$.  Thus $\tau_{ij}$, $\tau_{13}$ denote entries of the deviatoric stress tensor.

\begin{table}[ht]
\caption{Notation used in these notes.}
\begin{tabular}{clll}
variable  & description & SI units & value \\
\hline
$A$ & $A=A(T)=$ ice softness in Glen law & $\text{Pa}^{-n}\,\text{s}^{-1}$ \\
$B$ & ice hardness; $B=A^{-1/n}$ & $\text{Pa}\,\text{s}^{1/n}$ \\
$b$ & bedrock elevation & m \\
$c$ & specific heat & J kg$^{-1}$ K$^{-1}$ \\
$\nabla$ & (spatial) gradient & m$^{-1}$ \\
$\nabla\cdot$ & (spatial) divergence & m$^{-1}$ \\
$\mathbf{g}$ & gravity & m s$^{-2}$\phantom{foobar} & 9.81 \\
$H$ & ice thickness & m \\
$h$ & ice surface elevation & m \\
$\kappa$ & temperature conductivity & J s$^{-1}$ m$^{-1}$ K$^{-1}$ \\
$M$ & climatic mass balance & m s$^{-1}$ \\
$n$ & exponent in Glen flow law & & 3 \\
$\nu$ & viscosity & Pa s \\
$p$ & pressure & Pa \\
$\bq$ & map-plane ice flux: $\bq = \int_{b}^{h} \bU\,dx$ & $\text{m}^2\,\text{s}^{-1}$ \\
$\rho$ & density of ice & kg m$^{-3}$ & 910 \\
$\rho_w$ & density of sea water & kg m$^{-3}$ & 1028 \\
$\sigma_{ij}$ & Cauchy stress tensor: $\sigma_{ij} = \tau_{ij} - p\, \delta_{ij}$ & Pa \\
$T$ & temperature & K \\
$\tau_{ij}$ & deviatoric stress tensor & Pa \\
$Du_{ij}$ & strain rate tensor & s$^{-1}$ \\
$\mathbf{U}$ & $=(u,v)$ horizontal ice velocity & m s$^{-1}$ \\
$\mathbf{u}$ & $(u_1,u_2,u_3)=(u,v,w)$ 3D ice velocity & m s$^{-1}$ \\
$x_i$ & $x_1,x_2,x_3=x,y,z$ coordinates & m
\end{tabular}
\label{tab:notation}
\end{table}

\subsection*{Downloading codes}  These notes are based on nineteen Matlab codes, each about one-half page.  All have been tested in Matlab and Octave.  They are distributed by cloning the \texttt{git} repository at
\begin{quote}
\url{https://github.com/bueler/mccarthy}
\end{quote}
\noindent The codes are in the \texttt{mfiles/} subdirectory.  Five of them are printed below with their comments stripped for compactness and clarity.  The electronic versions have generous comments and help files.


\section{Ice flow equations}  \label{sec:continuum}

Our initial goal is an equation for which one might say:
\begin{center}
\emph{by numerically solving this equation one has a usable model for an ice sheet.}
\end{center}
A usable model must be \emph{understood} at least as much as it is \emph{correct} in a physical sense.  Our initial model will not be complete by any modern standard.

First we (briefly!) recall the continuum mechanical equations of ice flow.  Ice in glaciers is a moving fluid so we describe its motion by a velocity field $\mathbf{u}(t,x,y,z)$.  If the ice fluid were faster-moving than in reality, and if it were linearly-viscous like liquid water, then it would be a ``typical'' incompressible fluid and we would use the Navier-Stokes equations as the model:
\begin{align}
\nabla \cdot \mathbf{u} &= 0 &&\text{\emph{incompressibility}} \label{incompressible} \\
\rho \left(\mathbf{u}_t + \mathbf{u}\cdot\nabla \mathbf{u}\right) &= \nabla \cdot (\nu \nabla \mathbf{u}) - \nabla p + \rho \mathbf{g} &&\text{\emph{stress balance}} \label{navierstokes}
\end{align}

Equation \eqref{navierstokes} says ``$ma=F$''; it is Newton's second law.  The term $\mathbf{u}_t + \mathbf{u}\cdot\nabla \mathbf{u}$ is an acceleration while the right-hand side is net force per unit volume, from viscous, pressure, and gravity forces respectively.  If $\sigma$ is the Cauchy stress tensor, the right side can also be written $\nabla \cdot \sigma + \rho \mathbf{g}$.

The space of solutions to \eqref{incompressible} and \eqref{navierstokes} is rich.  It includes turbulent flows, which we will not need here.  Reading a book-length introduction to fluids such as \cite{Acheson} is overkill for glacier modeling, but it is recommended for broader education.

The numerical solution of the Navier-Stokes equations is \emph{computational fluid dynamics} (CFD).  One might ask: is ice flow modelling really a part of CFD?  Does a good general-purpose CFD text like \cite{Wesseling} help a glaciers student?  While glacier flow is a large-scale fluid problem, comparable to atmosphere and ocean circulation, from the point of view of general CFD it is strange.   Topics which might make ocean circulation exciting, for example turbulence, convection, coriolis force, and density stratification, are irelevant to ice flow.  What could be interesting about the flow of slow, thick ice?

\subsection*{The Glen-Stokes equations}  Glacier ice is a slow fluid.  In terms of equation \eqref{navierstokes}, ``slow'' is a reasonably-precise term which means that $\rho \left(\mathbf{u}_t + \mathbf{u}\cdot\nabla \mathbf{u}\right) \approx 0$, in other words that the forces (stresses) of inertia are negligible.  However, ice is also a shear-thinning fluid with a specific kind of nonlinearly-viscous (``non-Newtonian'') behavior in which larger strain rates imply smaller viscosity.  The viscosity $\nu$ in \eqref{navierstokes} is therefore not constant, and one must add an empirically-based flow law.  Combining these ideas gives the standard model for isothermal ice flow:
\begin{align}
\nabla \cdot \mathbf{u} &= 0 &&\text{\emph{incompressibility}} \label{incompressibleagain} \\
- \nabla \cdot \tau_{ij} + \nabla p &= \rho \mathbf{g} &&\text{\emph{stress balance}} \label{forcebalance} \\
Du_{ij} &= A \tau^2 \tau_{ij} &&\text{\emph{$n$=3 Glen flow law}} \label{flowlaw}
\end{align}

In \eqref{flowlaw} the deviatoric stress tensor $\tau_{ij}$ and the strain rate tensor $Du_{ij}$ now appear.  Other lectures cover these ideas, but recall that by definition
    $$Du_{ij} = \frac{1}{2} \left((u_i)_{x_j}+(u_j)_{x_i}\right).$$
Both tensors in \eqref{flowlaw} are symmetric and have trace zero.  Note that $\tau^2 = (1/2) \tau_{ij} \tau_{ij}$, using the summation convention, which defines the \emph{effective stress} $\tau$.

Because the Stokes equations do not contain a time derivative, the combined effect of boundary stresses, gravity $\rho \mathbf{g} = \left<0,0,-\rho g\right>$, and a value for ice softness $A>0$ will determine the velocity and stress fields ($\bu$, $p$, $\tau_{ij}$).  Equations \eqref{incompressibleagain}--\eqref{flowlaw} apply at each instant, even as the glacier geometry and the solution fields change.  Ice flow simulation codes need no memory of prior momentum or velocity.  Said another way, velocity is a ``diagnostic'' output of ice flow models.  Unlike in a Navier-Stokes-based simulation, velocity is not needed for (re)starting a simulation.

Consider now the $x,z$-plane case of equations \eqref{incompressibleagain}--\eqref{flowlaw}.  Such a planar flow has velocity component $v=0$.  All derivatives with respect to $y$ are also zero so the equations are:
\begin{align}
u_x + w_z &= 0 &&\text{\emph{incompressibility}} \label{incompressiblexz} \\
- \tau_{11,x} - \tau_{13,z} + p_x &= 0 &&\text{\emph{stress balance} ($x$)} \label{stokespx} \\
- \tau_{13,x} - \tau_{33,z} + p_z &= - \rho g &&\text{\emph{stress balance} ($z$)} \label{stokespz} \\
u_x &= A \tau^2 \tau_{11} &&\text{\emph{flow law (diagonal)}}  \label{forceflowx} \\
u_z + w _x &= 2 A \tau^2 \tau_{13} &&\text{\emph{flow law (off-diagonal)}} \label{forceflowz}
\end{align}
Note that $\tau_{13}$ is a shear stress while $\tau_{11}$ and $\tau_{33}=-\tau_{11}$ are deviatoric longitudinal stresses.  Also $\tau^2 = \tau_{11}^2+\tau_{13}^2$ in this case.  Equations \eqref{incompressiblexz}--\eqref{forceflowz} form a system of five nonlinear equations in five scalar unknowns ($u,w,p,\tau_{11},\tau_{13}$).

\subsection*{Slab-on-a-slope}  Equations \eqref{incompressiblexz}--\eqref{forceflowz} are complicated enough already!  They should make us pause before jumping into numerical solution methods.  For now we start with a simplified situation, namely a uniform slab of ice, in which we can both solve the Stokes equations exactly and motivate the shallow model in the next subsection.

\onefig{slab}{Rotated axes for a slab-on-a-slope flow calculation.}

We rotate our coordinates only for this example.  The two-dimensional axes ($x$,$z$) shown in Figure \ref{fig:slab} are rotated clockwise at angle $\alpha>0$ so that the gravity vector has components $\mathbf{g} = (g \sin\alpha,- g \cos \alpha)$ in the new axes.  Equations \eqref{stokespx} and \eqref{stokespz} are now
\begin{align}
- \tau_{11,x} - \tau_{13,z} + p_x &= \rho g \sin\alpha, \label{stokespxrot} \\
- \tau_{13,x} + \tau_{11,z} + p_z &= - \rho g \cos\alpha. \label{stokespzrot}
\end{align}

Assume further that there is no variation with $x$. Then the whole set of Stokes equations \eqref{incompressiblexz}, \eqref{forceflowx}--\eqref{stokespzrot} simplifies greatly:
\begin{align}
w_z &= 0 &   \tau_{11} &= 0 \label{stokesslab} \\
\tau_{13,z} &= - \rho g \sin\alpha &   u_z &= 2 A \tau^2 \tau_{13} \notag \\
p_z &= - \rho g \cos\alpha \notag
\end{align}
Each unknown is now a function of $z$ only.  Boundary conditions come from assuming that the ice is not crossing the bed, that the ice surface is moving parallel to the $x$-axis, and that the ice surface is stress free: $w(0)=0$, $p(H)=0$, $\tau_{13}(H)=0$.  Then, by integrating equations \eqref{stokesslab} with respect to $z$, we get $w=0$, $p = (\rho g \cos\alpha) (H-z)$, and $\tau_{13} = (\rho g \sin\alpha) (H-z)$.  Note that $H-z$ is the depth below the ice surface; both pressure and shear stress are proportional to depth.  Integrating the remaining equation in \eqref{stokesslab} yields horizontal velocity:
\begin{align}
u &= u_0 + \frac{1}{2} A (\rho g \sin\alpha)^3  \left(H^4 - (H-z)^4\right)  \label{uslab}
\end{align}
The basal velocity $u_0=u(0)$ remains undetermined for now.

Do we believe formula \eqref{uslab}, which makes a specific prediction about the profile of flow in a glacier?  Figure \ref{fig:slabvel} compares observations of a mountain glacier, showing that our model does a credible job of capturing deformation flow in this case.

\twofigsizes{slabvel}{athabasca-deform}{Left:  Velocity from formula \eqref{uslab}.  Right:  Measured velocity in a glacier (Athabasca Glacier \cite{SavagePaterson}).}{2.0in}{1.8in}

Until section \ref{sec:shelvesandstreams} we will not have a reasonable dynamical model for the sliding velocity $u_0$.  We assume until then that the ice is frozen to the bed and there is no sliding.


\subsection*{Viscosity form of the flow law}  Flow law \eqref{flowlaw} has another form.  Recall $\tau^2 = (1/2) \tau_{ij} \tau_{ij}$; also define $|Du|^2 = (1/2) Du_{ij} Du_{ij}$.  (The scalars $\tau$ and $|Du|$ are tensor \emph{norms}.)  From \eqref{flowlaw} we get $|Du| = A \tau^3$ so $\tau = A^{-1/3} |Du|^{1/3}$.  Thus \eqref{flowlaw} can be rewritten
\begin{equation}
\tau_{ij} = 2 \nu\, Du_{ij}  \qquad \text{\emph{flow law}} \label{viscosityflowlaw}
\end{equation}
where
    $$2\nu = A^{-1/3} |Du|^{-2/3}$$
defines the nonlinear \emph{viscosity} $\nu$.  Often $B = A^{-1/3}$ is called the \emph{ice hardness}.  Equation \eqref{viscosityflowlaw} allows us to eliminate stresses $\tau_{ij}$ from the Stokes equations, resulting in formulas depending only on the strain rates $Du_{ij}$.

\subsection*{The Blatter-Pattyn approximation}  One may approximate the plane-flow Stokes equations \eqref{incompressiblexz}--\eqref{forceflowz} by assuming that horizontal variation in the vertical shear stress is small compared to the other terms.  This removes the single term  $\tau_{13,x}$ from the $z$-component of the stress balance \eqref{stokespz}:
\begin{equation}
p_z = - \tau_{11,z} - \rho g. \label{hydrostaticpz}
\end{equation}
Because $\sigma_{ij} = \tau_{ij} - p \delta_{ij}$, thus $p + \tau_{11} = p - \tau_{33} = - \sigma_{33}$, equation \eqref{hydrostaticpz} says that the vertical normal stress $\sigma_{33}$ is linear in depth.  (A slightly-different model assumes that the pressure $p$ itself is hydrostatic.)  Taking $\sigma_{33}$ to have surface value zero we get
\begin{equation}
p + \tau_{11} = \rho g (h-z). \label{hydrostaticitself}
\end{equation}

Equation \eqref{hydrostaticitself} allows elimination of the pressure $p$ from the model equations.  Furthermore, taking the $x$-derivative of \eqref{hydrostaticitself}, then substituting into \eqref{stokespx} to eliminate $p_x$, and then using the viscosity form \eqref{viscosityflowlaw} leads to this equation \cite{GreveBlatter2009}:
\begin{equation}
\left(4 \nu u_x\right)_x + \left(\nu (u_z+w_x)\right)_z = \rho g h_x \qquad\text{\emph{hydrostatic stress balance}} \label{stresshydrostatic}
\end{equation}
Equations \eqref{incompressiblexz}, \eqref{viscosityflowlaw}, \eqref{stresshydrostatic}, plus appropriate boundary conditions, determine $u$ and $w$ and then all remaining unknowns.  However, although $p$ is gone, equation \eqref{stresshydrostatic} is nontrivially-coupled to incompressibility \eqref{incompressiblexz} because the vertical velocity $w$ appears in both places.

If we additionally drop $w_x$ from equation \eqref{stresshydrostatic} then we get the Blatter-Pattyn model:
\begin{equation}
\left(4 \nu u_x\right)_x + \left(\nu u_z\right)_z = \rho g h_x \qquad\text{\emph{Blatter-Pattyn stress balance}} \label{stressblatter}
\end{equation}
Solving \eqref{stressblatter} as a boundary value problem yields the horizontal velocity $u$.  Then one may recover $w$ from incompressibility \eqref{incompressiblexz}.  Because it is somewhat similar to the well-understood Poisson equation $u_{xx} + u_{zz} = f$, as described in \cite{LeVequeFD,MortonMayers} for example, the process of solving equation \eqref{stressblatter} is more familiar to many professionals than \eqref{stresshydrostatic}.

\subsection*{Plane-flow mass-continuity equation}  The equations so far do not address how a glacier or ice sheet changes shape.  For this we need to add a mass conservation equation.  We derive it here in an informal way, returning to the topic in section \ref{sec:masscont}.

Consider any $x,z$-plane flow with variable thickness and velocity, not just slab-on-a-slope.  Define the vertical average of the horizontal velocity:
	$$\bar U = \frac{1}{H}\int_0^{H} u\,dz.$$
The ice flux $q= \int_0^{H} u\,dz$ is the rate of flow input into the side of the area in Figure \ref{fig:slabmasscontfig}.  Ice can be added by climatic (surface) mass balance $M$ or by a difference in the flux $q=\bar U H$ between the left and right sides.

\onefigsize{slabmasscontfig}{Mass continuity equation \eqref{masscont1D} follows from considering this domain of ice, which has time-dependent area $A$.}{2.5in}

The area $A$ changes according to the sum of all the boundary contributions:
\begin{equation}
\frac{dA}{dt} = \int_{x_1}^{x_2} M(x,t)\,dx + \bar U_1 H_1 - \bar U_2 H_2. \label{masscontintegrated}
\end{equation}
In three-dimensions, \eqref{masscontintegrated} becomes an equation for $dV/dt$, the ice volume rate of change.

If the width $\Delta x=x_2-x_1$ is small then $A\approx \Delta x\, H$.  So we divide by $\Delta x$ and take $\Delta x \to 0$ in \eqref{masscontintegrated} to get
\begin{equation}
H_t = M - \left(\bar U H\right)_x \label{masscont1D}
\end{equation}
This \emph{mass continuity} differential equation describes change in the ice thickness in terms of surface mass balance and ice velocity.  Ice flow simulations can compute the velocity $u$ and then use this equation for the changing glacier geometry.


\section{Shallow ice sheets}   \label{sec:sia}

Ice sheet flow has four outstanding properties as a fluids problem.  Ice sheets are
\renewcommand{\labelenumi}{(\emph{\roman{enumi}})}
\begin{enumerate}
\item slow,
\item shallow,
\item non-Newtonian (shear-thinning), and
\item subject to widespread contact slip (basal sliding).
\end{enumerate}
The first numerical ice flow model in these notes accounts for the first three properties only.  It is the non-sliding, isothermal \emph{shallow ice approximation} (SIA).

Regarding shallowness (\emph{ii}), Figure \ref{fig:green-transect} shows both a no-vertical-exaggeration cross-section of Greenland at $71^\circ$ and the standard vertically-exaggerated version which is more familiar in the glaciological literature.  Without vertical exaggeration we see the ice sheet as a thin blanket, barely three-dimensional.  This illustrates that ice sheets are shallow.  Note, however, that the portion of an ice sheet which you seek to model may not be shallow.

\medskip

\onefigsize{green-transect}{A cross-section of the Greenland ice sheet ($71^\circ$ N), with width about 760 km and maximum thickness 3200 m, is shown by the upper two curves.  The thickened horizontal line (bottom) shows the same cross-section without vertical exaggeration.}{4.0in}

Our earlier slab-on-a-slope example can be converted into a rough derivation of the plane-flow SIA.  We vertically integrate velocity formula \eqref{uslab} in the $u_0=0$ case to get
\begin{equation}
\bar U H = \int_0^H \frac{1}{2} A (\rho g \sin\alpha)^3  \left(H^4 - (H-z)^4\right)\,dz = \frac{2}{5} A (\rho g \sin\alpha)^3 H^5. \label{siaubar}
\end{equation}
Regarding the surface slope, note $\sin \alpha \approx \tan\alpha = - h_x$ when $\alpha$ is a small angle.  Combining these statements with mass continuity \eqref{masscont1D} gives
\begin{equation}
  H_t = M + \left(\frac{2}{5} (\rho g)^3 A H^5 |h_x|^2 h_x\right)_x. \label{sia1D}
\end{equation}

Equation \eqref{sia1D} is the SIA equation for nonsliding plane flow.  Noting $h=H+b$, it models the evolution of an ice sheet's thickness $H$.  The model should, however, be solved subject to the constraint that the thickness is positive ($H\ge 0$) \cite{Bueler2016,JouvetBueler2012}.  For careful SIA derivations, which reduce the Stokes equations under a small depth-to-width ratio assumption, follow the Notes and References at the end.

We now state the complete SIA model in preparation for numerical solutions in section \ref{sec:numericalsia}.  Let $\mathbf{U} = (u,v)$ be the vector horizontal velocity; this is a 2D vector-valued function of 3D coordinates $(x,y,z)$.  The SIA shear stress approximation is $(\tau_{13},\tau_{23}) \approx - \rho g (h-z) \nabla h$, and equation \eqref{flowlaw} gives a formula for shear strain rates
\begin{equation*}
\mathbf{U}_z = 2 A |(\tau_{13},\tau_{23})|^{n-1} (\tau_{13},\tau_{23}) = - 2 A (\rho g)^n (h-z)^n |\nabla h|^{n-1} \nabla h.
\end{equation*}
By integrating vertically we get, in the non-sliding case,
\begin{equation}
\mathbf{U} = - \frac{2 A (\rho g)^n}{n+1} \left[H^{n+1} - (h-z)^{n+1}\right] |\nabla h|^{n-1} \nabla h.  \label{siavelocity}
\end{equation}
Mass continuity in 2D, which generalizes the 1D version \eqref{masscont1D}, says:
\begin{equation}
    H_t = M - \Div\left(\bar{\mathbf{U}} H\right)  \label{masscont}
\end{equation}
Equation \eqref{masscont} may be written $H_t = M - \Div \bq$ in terms of the map-plane flux $\bq = \int_{b}^{h} \mathbf{U}\,dz = \bar{\mathbf{U}}\,H$.  Note $\bar{\mathbf{U}}$ and $\bq$ are functions of 2D coordinates $(x,y)$ only.  Combining equations \eqref{siavelocity} and \eqref{masscont}, we get an equation for the rate of thickness change in terms of mass balance $M$, thickness $H$, and surface gradient $\grad h$:
\begin{equation}
H_t = M + \Div \left(\Gamma H^{n+2} |\grad h|^{n-1} \grad h \right). \label{sia}
\end{equation}
We have defined the positive constant $\Gamma = 2 A (\rho g)^n / (n+2)$ only to improve appearance.

Equation \eqref{sia} is the SIA in 2D.  Recalling our earlier promise, if we can solve \eqref{sia} numerically then we have a usable model for certain ice sheets.  Mary-Anne Mahaffy \cite{Mahaffy} used it in a 1976 numerical model of the Barnes ice cap in Canada, a particularly-simple ice sheet on a rather flat bed.  This was the start of numerical ice sheet modeling.

\subsection*{Analogy with diffusion equations}  Numerical methods for solving \eqref{sia} are modifications of methods for the better-known heat equation, e.g.~ $T_t = D T_{xx}$ for the temperature of a conducting rod, its simplest 1D form.  This model applies when material properties are constant and there are no heat sources.  The positive constant $D$ is the ``diffusivity,'' with SI units which can be read from comparing sides of the equation: $D\sim \text{m}^2 \text{s}^{-1}$.  This equation has a smoothing effect on the solution $T(t,x)$ as it evolves in time, because any local maximum or minimum in the temperature is flattened: $T_{xx}<0$ implies $T_t<0$, so $T$ decreases near a maximum, while $T_{xx}>0$ implies $T_t>0$, so $T$ increases near a minimum.  All extrema of the temperature diffuse away.

The 2D heat equation, which describes the temperature $T(t,x,y)$ of a planar object, is a closer analog of \eqref{sia}.  Recall that Fourier's law for conduction is the formula $\mathbf{Q} = - \kappa \grad T$ for heat flux $\mathbf{Q}$, where $\kappa$ is conductivity.  We will assume, so as to help the analogy to the SIA, that $\kappa(x,y)$ may vary in space, and we suppose the heat source $f(t,x,y)$ is variable.  Adding conservation of internal energy, we find the equation
\begin{equation}
\rho c T_t = f + \Div (\kappa \grad T). \label{heatearly}
\end{equation}
Here $\rho$ is density and $c$ is specific heat capacity.  Assuming $\rho c$ is constant, define the diffusivity $D=\kappa/(\rho c)$.  Defining $F = f/(\rho c)$, a rescaled source term, \eqref{heatearly} becomes
\begin{equation}
T_t = F + \Div (D\, \grad T). \label{heat}
\end{equation}
This 2D heat equation is similar to the above 1D equation $T_t = D T_{xx}$, but now with variable diffusivity and an additional heat source.

Figure \ref{fig:initialheat} shows a solution of \eqref{heat} in which the initial condition is a localized ``hot spot''.  Solutions of the heat equation always involve the spreading, in all directions, of any local concentration of heat; this is diffusion.

\twofigsizes{initialheat}{finalheat}{Left: Initial condition $T(0,x,y)$.   Right: A solution $T(t,x,y)$ of  \eqref{heat}, with $D=1$ and $F=0$, at $t=0.02$.}{2.8in}{2.8in}

The SIA equation \eqref{sia} and the heat equation \eqref{heat} are each diffusive, time-evolving partial differential equations (PDEs).  A side-by-side comparison is appropriate:
\begin{center}
\begin{tabular}{cc}
\vspace{1mm}
SIA:\, $H$ is ice thickness & \phantom{foo bar} heat: $T$ is temperature\phantom{foo bar}  \\
\vspace{1mm}
	$H_t = M + \Div \left({\color{red}\Gamma H^{n+2} |\grad h|^{n-1}}\, \grad h \right)$  &  $T_t = F + \Div (D\, \grad T)$
\end{tabular}
\end{center}
\vspace{1mm}
Notice that the number of derivatives (one time and two space) and the signs of the various terms are the same.  Surface mass balance $M$ is analogous to heat source $F$.  The analogy suggests that we identify the \emph{SIA diffusivity} as:
\begin{equation}
	D = {\color{red}\Gamma H^{n+2} |\grad h|^{n-1}}.  \label{siadiffusivity}
\end{equation}

Diffusion acts quickly where $D$ is large.  SIA diffusivity formula \eqref{siadiffusivity} gives a large value when the ice is thick and steep, because ice flows downhill most-strongly in that case.  Our analogy also explains why the surfaces of ice sheets are smooth, at least once we overlook non-fluid processes like crevassing.

There are, however, conceptual and practical concerns about the \eqref{sia}$\leftrightarrow$\eqref{heat} analogy:
\begin{itemize}
\item The letters ``$H$'' and ``$h$'' generally denote different quantities in \eqref{sia}.  Are we allowed to regard \eqref{sia} as a diffusion anyway, even when the bed is not flat?
\item The diffusivity $D$ in \eqref{siadiffusivity} depends on the solution---both on thickness $H$ and surface slope $|\grad h|$---so equation \eqref{sia} is a nonlinear PDE.  Numerical solutions for nonlinear equations are, at least, more difficult to implement and analyze.
\item The diffusivity $D$ in \eqref{siadiffusivity} goes to zero at margins, where $H\to 0$, and at divides and domes, where $|\grad h|\to 0$.  Thus the solution ($h$ or $H$) is not smooth at these locations, though it is continuous everywhere.  The solution is not smooth even if the bed $b$ and surface mass balance $M$ \emph{are} smooth.  Large numerical errors will arise where the solution is not smooth.
\end{itemize}

Just as important is a physical deficiency of the SIA model, namely
\begin{itemize}
\item Ice flow is not very diffusive when significant longitudinal (membrane) stresses are present, as when ice is floating or sliding, or when the flow is significantly confined by terrain.
\end{itemize}
This concerns a deficiency of the SIA model, not just our analogy to explain it.

All four concerns are areas of active research in mathematical glaciology; see the Notes and References.


\section{Finite difference numerics}  \label{sec:fd}

Now we introduce numerical methods.  The above diffusion analogy suggests that numerical schemes for the heat equation are a good starting point for solving the SIA.  The analogy will lead us to a verified numerical scheme for \eqref{sia}.  We demonstrate only finite difference (FD) schemes, which replace derivatives by difference quotients.

Taylor's theorem says that for a smooth function $f(x)$,
	$$f(x+\Delta) = f(x) + f'(x) \Delta + \frac{1}{2} f''(x) \Delta^2 + \frac{1}{3!} f'''(x) \Delta^3 + \dots$$
One may also replace ``$\Delta$'' by its multiples, for example:
\begin{align*}
f(x+2\Delta) &= f(x) + 2 f'(x) \Delta + 2 f''(x) \Delta^2 + \frac{4}{3} f'''(x) \Delta^3 + \dots \\
f(x-\Delta) &= f(x) - f'(x) \Delta + \frac{1}{2} f''(x) \Delta^2 - \frac{1}{3!} f'''(x) \Delta^3 + \dots
\end{align*}
When constructing FD methods for differential equations the idea is to combine expressions like these to give approximations of the derivatives in the equation.  The resulting algebraic equations, relating unknown gridded solution values via mere arithmetic, will approximate the differential equation.

We must apply Taylor expansions for each variable to approximate the partial derivatives appearing in the heat and SIA equations.  For example, for $u=u(t,x)$ we have
\begin{align*}
u_t(t,x) &= \frac{u(t+\Delta t,x) - u(t,x)}{\Delta t} + O(\Delta t), \\
u_t(t,x) &= \frac{u(t+\Delta t,x) - u(t-\Delta t,x)}{2\Delta t} + O((\Delta t)^2), \\
u_x(t,x) &= \frac{u(t,x+\Delta x) - u(t,x-\Delta x)}{2\Delta x} + O((\Delta x)^2), \\
u_{xx}(t,x) &= \frac{u(t,x+\Delta x) - 2\, u(t,x) + u(t,x-\Delta x)}{\Delta x^2} + O((\Delta x)^2)
\end{align*}
Note that if $\Delta$ is a small number then $O(\Delta^2)$ is smaller than $O(\Delta)$; the first approximation is inferior to the latter three in this sense.

\subsection*{Explicit scheme for the heat equation}  Our simplest scheme approximates the 1D heat equation $T_t=D T_{xx}$ \emph{explicitly}.  It is based on the fact that because $T_t$ and $D T_{xx}$ are equal, these two FD expressions are nearly equal:
\begin{equation}
\frac{T(t+\Delta t,x) - T(t,x)}{\Delta t} \approx D\,\frac{T(t,x+\Delta x) - 2\, T(t,x) + T(t,x-\Delta x)}{\Delta x^2}.  \label{heat1Dapproximated}
\end{equation}
While \eqref{heat1Dapproximated} only approximates the PDE, forcing it to be an equality allows us to determine grid values.  Let $(t_n,x_j)$ denote the points of the time-space grid shown in Figure \ref{fig:timespacegrid}.  It has spacing $\Delta x>0$ and $\Delta t>0$ in the two dimensions.  Denote our approximation of the solution value $T(t_n,x_j)$ by $T_j^n$.  The finite difference scheme is
	$$\frac{T_j^{n+1} - T_j^n}{\Delta t} = D\,\frac{T_{j+1}^n - 2\, T_j^n + T_{j-1}^n}{\Delta x^2}.$$
To get a simplified formula, let $\mu = D \Delta t / (\Delta x)^2$ and solve for $T_j^{n+1}$:
\begin{equation}
  T_j^{n+1} = \mu T_{j+1}^n + (1 - 2 \mu) T_j^n + \mu T_{j-1}^n \label{heat1Dfd}
\end{equation}

\onefigsize{timespacegrid}{A grid for a finite difference solution to the 1D heat equation.}{2.0in}

FD scheme \eqref{heat1Dfd} is \emph{explicit} because it directly (explicitly) computes $T_j^{n+1}$ in terms of values at time $t_n$.  Figure \ref{fig:expstencil} (left) shows the ``stencil'' for scheme \eqref{heat1Dfd}: three values at the current time $t_n$ are combined to update one value at the next time $t_{n+1}$.

\twofigsizes{expstencil}{exp2dstencil}{Left: Space-time stencil for the explicit scheme \eqref{heat1Dfd} for the 1D heat equation.  Right: Spatial-only stencil for scheme \eqref{heat2dexplicit}.}{2.0in}{2.1in}

The expression $T(t_n,x_j)$ is the value of a heat equation solution at a grid point.  This is generally a different number from $T_j^n$, the value actually computed by a scheme like \eqref{heat1Dfd}.  Because the FD expressions become closer to the derivatives they approximate, we intend that the numbers $T(t_n,x_j)$ and $T_j^n$ become closer together as the grid is made finer ($\Delta t \to 0$ and $\Delta x \to 0$).  That is, we intend our FD scheme to \emph{converge} under \emph{grid refinement}.  Proving that such convergence happens is a main topic in numerical textbooks \cite{LeVequeFD,MortonMayers}.  Even if we expect convergence to happen, this needs checking for the actual computer program which is our implementation (\emph{verification}; Section \ref{sec:exactsolutions}).

How accurate is scheme \eqref{heat1Dfd}?  Our construction tells us that the difference between the scheme \eqref{heat1Dfd} and the PDE it solves is $O(\Delta t + (\Delta x)^2)$.  This difference goes to zero as we refine the grid in space and time, a scheme property called \emph{consistency}.  With care about the smoothness of boundary conditions, and using mathematical facts about the heat equation itself, one can also show that the difference between $T_j^n$ and $T(t_n,x_j)$ is $O(\Delta t + (\Delta x)^2)$, the \emph{convergence rate}.  However, to get convergence the scheme must be \emph{stable}, which we address below.  The main theorem for numerical PDE schemes asserts that ``consistency plus stability implies convergence''.  See the Notes and References.

Instead of pursuing theory, in these notes we emphasize something practical, namely verification using exact solutions.  We will find PDE problems for which we already know an exact solution $T(t,x)$, and then compute differences like $|T_j^n - T(t_n,x_j)|$ and confirm that they go to zero under grid refinement.  This demonstrates directly that our actual implementation converges, not just that it should do so in theory.  A verified code can be confidently applied to real situations.

\subsection*{A first implemented scheme}  Our first Matlab implementation is for the 2D heat equation \eqref{heat} with $D$ constant and $F=0$:
\begin{equation}
T_t = D (T_{xx}+T_{yy}).\label{heat2D}
\end{equation}
We solve this equation on the square $-1 < x < 1$, $-1 < y < 1$.  For a simple example problem we set $T=0$ on the boundary of the square and choose a gaussian initial condition: $T(0,x,y) = \exp(-30 (x^2+y^2))$.

Writing $T_{jk}^n \approx T(t_n,x_j,y_k)$, the 2D explicit scheme is
\begin{equation}
	\frac{T_{jk}^{n+1} - T_{jk}^n}{\Delta t} = D\,\left(\frac{T_{j+1,k}^n - 2\, T_{jk}^n + T_{j-1,k}^n}{\Delta x^2} + \frac{T_{j,k+1}^n - 2\, T_{jk}^n + T_{j,k-1}^n}{\Delta y^2}\right). \label{heat2dexplicit}
\end{equation}
The stencil for the right side of \eqref{heat2dexplicit} is in Figure \ref{fig:expstencil}.  Using Matlab colon notation to implement loops over the spatial variables, our scheme becomes \texttt{heat.m} below.  

Consider the example run:
\begin{Verbatim}
>>  heat(1.0,30,30,0.001,20);
\end{Verbatim}
This sets $D=1.0$, uses a $30\times 30$ spatial grid, sets $\Delta t = 0.001$, and takes $N=20$ time steps.  The result was already shown in Figure \ref{fig:initialheat} (right); this is the look of success.

\minput{heat}

However, very similar runs seem to succeed or fail according to some as-yet unclear circumstances.  For example, results from these calls are shown in Figure \ref{fig:stability}:
\begin{Verbatim}
>> heat(1.0,40,40,0.0005,100);    % Figure 8, left
>> heat(1.0,40,40,0.001,50);      % Figure 8, right
\end{Verbatim}
Both runs compute temperature $T$ on the same spatial grid, for the same final time $t_f = N \Delta t = 0.05$, but with different time steps.  Noting the vertical axes, the second run clearly shows ``instability,'' an extreme form of inaccuracy characterized in practice by exponentially-growing solution magnitude.  (In fact the exact solution is bounded.)

\twofig{stability}{instability}{Numerically-computed temperature on $40\times 40$ grids.  The two runs are the same except that the left has $\Delta t=0.0005$ and $D\Delta t/(\Delta x)^2= 0.2$, while the right has $\Delta t=0.001$ and $D\Delta t/(\Delta x)^2= 0.4$.  Compare \eqref{stabcrit}.}

\subsection*{Stability criteria and adaptive time stepping}  To avoid such instabilities we need to understand the scheme better.  While we have not made an implementation error, we must learn how to choose compatible space and time steps.

Recall the 1D explicit scheme in form \eqref{heat1Dfd}: $T_j^{n+1} = \mu T_{j+1}^n + (1 - 2 \mu) T_j^n + \mu T_{j-1}^n$.  The new value $T_j^{n+1}$ is an average of the old values, in the sense that the coefficients sum to one.  Averaging is stable because averaged wiggles are smaller than the wiggles themselves.

However, the scheme is only an average \emph{if} the coefficients, specifically the middle coefficient $1-2\mu$, are positive.  A linear combination with coefficients which add to one is not an average if some coefficients are negative.  For example, one would not accept 15 as an ``average'' of 5 and 7, but: $15 = -4 \times 5 + 5 \times 7$, and $-4+5=1$.

What consequences follow from requiring the middle coefficient in \eqref{heat1Dfd} to be positive?  We find a \emph{stability criterion}, with these equivalent forms:
\begin{equation}
   1 - 2 \mu \ge 0 \quad \iff \quad \frac{D\Delta t}{\Delta x^2} \le \frac{1}{2} \quad \iff \quad \Delta t \le \frac{\Delta x^2}{2 D}.  \label{stabcrit}
\end{equation}
The third form states the criterion as a limitation on the size of $\Delta t$.  It is a \emph{sufficient} stability condition; it guarantees stability though something weaker might do.

In summary, for a given value of $\Delta x$, shortening the time steps $\Delta t$ until \eqref{stabcrit} holds will make FD scheme \eqref{heat1Dfd} for the 1D heat equation into a stable averaging process.  Applying this same idea to the 2D heat equation \eqref{heat2D} leads to the stability condition $1-2\mu^x-2\mu^y \ge 0$, where $\mu^x = D \Delta t / (\Delta x^2)$ and $\mu^y = D \Delta t / (\Delta y^2)$.  In the cases shown in Figure \ref{fig:stability}, wherein $\Delta x=\Delta y$, this condition requires $D \Delta t /(\Delta x^2) \le 0.25$.  This inequality precisely distinguishes between these stable and unstable results, respectively.

Runs of \texttt{heat.m} are unstable if the time step $\Delta t$ is too large relative to the spacing $\Delta x$.  However, the stability criterion \eqref{stabcrit} can be enforced inside the code, and this is called an \emph{adaptive} implementation.  Changing \texttt{heat.m} in this way yields \texttt{heatadapt.m} (not shown).  Furthermore, such adaptive codes can be stable even if the diffusivity $D$ varies in space or time.  On the other hand, if the diffusivity $D$ is large, or if grid spacings $\Delta x$, $\Delta y$ are small, then adaptive explicit implementations must take many short time steps to assure stability.  Adaptive explicit schemes become inefficient on spatially-refined grids.

\subsection*{Implicit schemes}  \emph{Implicitness} is an alternative stability fix to adaptivity.  For example, the finite difference scheme
\begin{equation}
  \frac{T_j^{n+1} - T_j^n}{\Delta t} = D\,\frac{T_{j+1}^{n+1} - 2\, T_j^{n+1} + T_{j-1}^{n+1}}{\Delta x^2} \label{implicit1D}
\end{equation}
is an $O(\Delta t + (\Delta x)^2)$ implicit scheme for the 1D heat equation $T_t = D T_{xx}$.  This implicit \emph{backward Euler} scheme is stable for any positive time step $\Delta t>0$; it is \emph{unconditionally stable} \cite{LeVequeFD,MortonMayers}.  Another well-known implicit scheme for the heat equation is \emph{Crank-Nicolson}, an unconditionally-stable scheme with smaller error $O((\Delta t)^2 +(\Delta x)^2)$.

Implicit schemes are more difficult to implement because the unknown solution at time step $t_{n+1}$ must be treated as a vector, and a (large) linear or nonlinear system of equations must be formed and solved at each time step.  Though the SIA is a highly-nonlinear diffusion equation, implicit schemes are indeed implementable when appropriate solver tools are used \cite{Bueler2016}.  However, for these notes, in the tradeoff between the easy implementability of adaptive explicit schemes and the improved stability of implicit schemes, we choose the former. 

\subsection*{Numerical solution of generalized diffusion equations}  We are trying to numerically model flowing ice, not temperature as seen so far.  In this section, as a step toward an explicit and adaptive SIA code which works on nonflat beds, \texttt{diffstag.m} below is an implemented numerical scheme for a generalized diffusion equation:
\begin{equation}
  T_t = F + \Div \left(D \grad (T + b)\right). \label{gendiffusion}
\end{equation}
Equation \eqref{gendiffusion} has an extra ``tilt'' $b$ inside the gradient.  The source term $F(x,y)$, diffusivity $D(x,y)$, and tilt $b(x,y)$ may all vary in space.

\minput{diffstag}

This function is called for SIA-specific solution schemes, which we build next, using $b(x,y)$ equal to the bedrock elevation.  Its explicit time-stepping method is conditionally-stable, with the same time step restriction as for the constant diffusivity case.  For stability to apply under the same criterion it is important that we evaluate $D(x,y)$ at \emph{staggered} grid points.  That is, where $X=T+b$, we will be using the approximation
\begin{align*}
\Div \left(D \grad X\right) &\approx \frac{D_{j+1/2,k}(X_{j+1,k} - X_{j,k}) - D_{j-1/2,k}(X_{j,k} - X_{j-1,k})}{\Delta x^2} \\
	&\qquad + \frac{D_{j,k+1/2}(X_{j,k+1} - X_{j,k}) - D_{j,k-1/2}(X_{j,k} - X_{j,k-1})}{\Delta y^2}.
\end{align*}
Figure \ref{fig:diffstencil} (left) shows the stencil.  The user supplies the diffusivity $D(x,y)$ on the staggered grid (triangles).  The initial values $T(0,x,y)$, source term $F(x,y)$, and tilt $b(x,y)$ are supplied on the regular grid (diamonds).  When using this code for the flat-bed case of the SIA we will take $b=0$.

\twofigsizes{diffstencil}{mahaffystencil}{Left:  Spatial stencil for staggered-grid (triangles) evaluation of diffusivity $D$ in \eqref{gendiffusion}.  Right: Stencil showing how $D$ is evaluated in the SIA, from surface elevation (diamonds) and thicknesses (squares).}{2.2in}{2.2in}


\section{Numerically solving the SIA} \label{sec:numericalsia}

As already noted, we compute the SIA diffusivity $D$ on a staggered grid.  Various schemes for staggered-grid diffusivity have been proposed, but all involve averaging $H$ and differencing $h$ in a ``balanced'' way onto the staggered grid.  In the code \texttt{siaflat.m} below, which only works for a flat bed and zero surface mass balance, we use the Mahaffy \cite{Mahaffy} method.  The Mahaffy stencil for computing $D$ is shown in Figure \ref{fig:diffstencil} (right).  Note that this SIA solver calls \texttt{diffstag.m} (above).

For non-flat beds and actual climates, namely when the surface mass balance term $M$ can take either sign, \texttt{siaflat.m} does not suffice.  Its deficiences are corrected in an extended code \texttt{siageneral.m} (not shown), which in fact only alters a few lines.

\minput{siaflat}

Addressing ablation ($M<0$) requires a small modification which deserves exposure.  Note that the grounded ice margin occurs in areas of ablation where the SIA thickness solution $H$ goes to zero.  However, the thickness of ice must obviously be greater than or equal to zero everywhere: $H(t,x)\ge 0$.  It follows that geometry-evolving numerical glacier and ice sheet models are solving a \emph{free-boundary problem}.  For an explicit time-stepping scheme it suffices to make the following rule:
\begin{quote}
\emph{If the numerical computation generates an updated thickness which is negative then it is reset to zero:} \, $H_{j,k}^{n+1} \gets \max\{0,H_{j,k}^{n+1}\}$.
\end{quote}
Determining glacier-covered area is a major glaciological purpose for modeling, thus this seemingly-small complication is ubiquitous and unavoidable in explicit time-stepping glacier models if the margin is allowed to move.  Modelers often implement and then overlook the above rule; it violates conservation of mass \cite{Bueler2021conservation}.


\section{Exact solutions and verification} \label{sec:exactsolutions}

Program \texttt{siaflat.m} calls \texttt{diffstag.m}, so we already have a complicated code.  How can we make sure that such an implemented numerical scheme is correct?  Here are three proposed techniques in the modeling community:
\begin{enumerate}
  \item don't make any mistakes, or
  \item compare your numerical results with results from other researchers, and hope the outliers are in error, or
  \item compare your numerical results to an exact solution.   \end{enumerate}
The first two approaches, \emph{infallibility} and \emph{intercomparison} respectively, should be less common than they are.  The last one, preferred by the CFD community generally \cite{Roache,Wesseling}, is \emph{verification}.  It is a simple idea:  A computer code should be tested in cases where we know the right answer.  To do this we must return to the PDE itself.  We must seek-out useful exact solutions.

\subsection*{Exact solutions of heat equation}  Again consider the 1D heat equation $T_t = D T_{xx}$ with constant $D$.  Many exact solutions are known, but we will use the time-dependent \emph{Green's function}, circa 1830, also known as the \emph{heat kernel}.  It starts at time $t=0$ with a delta function $T(0,x)=\delta(x)$ of heat at the origin.  Then it spreads out over time, solving the equation on $-\infty<x<\infty$ and for $t>0$.  We calculate it by a method which generalizes to the SIA.

This Green's function is self-similar over time, in the sense that it changes shape \emph{only} by shrinking the output (vertical) axis and lengthening the input (horizontal) axis (Figure \ref{fig:heatscaling}).  These scalings are related to each other by conservation of energy.  Similarity variables for this solution involve multiplying the input and output of an invariant shape $\phi(s)$ by the same power of $t$:
\begin{equation}
s \stackrel{\text{\emph{input scaling}}}{\phantom{\Big|}=\phantom{\Big|}} t^{-1/2} x, \qquad\qquad T(t,x) \stackrel{\text{\emph{output scaling}}}{\phantom{\Big|}=\phantom{\Big|}} t^{-1/2} \phi(s).  \label{heatscalings}
\end{equation}
One may use $T_t = D T_{xx}$ to show that $\phi(s) = (4 \pi D)^{-1/2}\, e^{-s^2/(4D)}$, and thus
    $$T(t,x) = (4 \pi D t)^{-1/2}\, e^{-x^2/(4Dt)}.$$

To verify a numerical solver for the 1D heat equation using this exact solution we may evaluate this exact solution at $t_0>0$ to provide initial values.  At a later time $t_1>t_0$ the numerical solution should be close to the exact solution $T(t_1,x)$ (see the Exercises).

\onefigsize{heatscaling}{The Green's function of the heat equation has the same shape at each time, but with time-dependent input- and output-scalings.}{2.4in}

\subsection*{Halfar's similarity solution to the SIA}  Now we jump to the year 1981 in which Peter Halfar \cite{Halfar81} published the corresponding similarity solution of the SIA, at least for flat bed and zero surface mass balance.  For the 2D SIA model \eqref{sia} using a Glen exponent $n=3$ this solution \cite{Halfar83} has scalings similar to \eqref{heatscalings} above:
\begin{equation}
s = t^{-1/18} r, \qquad \qquad H(t,r)=t^{-1/9} \phi(s), \label{halfarscalings}
\end{equation}
where $r=(x^2+y^2)^{1/2}$ is the distance from the origin.  These scalings are related to each other by the constancy of total mass/volume, because no mass is gained or lost through the surface (see the Exercises). Scalings \eqref{halfarscalings} imply that, differently from heat, the diffusion of ice slows down dramatically as the shape flattens out.  That is, the powers $t^{-1/9}$ and $t^{-1/18}$ change very slowly for large times $t$.

\onefigsize{siascaling}{Radial sections of Halfar's solution \eqref{halfar} of \eqref{sia}, shown on $H$ (m) versus $r$ (km) axes at $t=1,10,100,1000,10000$ years.}{5.5in}

The Halfar solution formula is remarkably simple given all that it accomplishes:
\begin{equation}
H(t,r) = H_0 \left(\frac{t_0}{t}\right)^{1/9} \left[1 - \left(\left(\frac{t_0}{t}\right)^{1/18} \frac{r}{R_0}\right)^{4/3}\right]^{3/7}. \label{halfar}
\end{equation}
Here the characteristic time $t_0 = (18 \Gamma)^{-1} (7/4)^3 R_0^4 H_0^{-7}$ depends on the radius $R_0$ and the center height $H_0$.  In Figure \ref{fig:siascaling} we see that for times significantly greater than $t_0$ ($t/t_0 \gg 1$) the solution changes very slowly.  For example, the change between years $1$ and $100$ is larger than that between years $1000$ and $10000$.

\subsection*{Using Halfar's solution}  Formula \eqref{halfar} is simple to use for verifying time-dependent SIA models.  The code \texttt{verifysia.m} (not shown) takes as input the number of grid points in each ($x,y$) direction.  It uses the Halfar solution at 200 a as the initial condition, does a numerical run of \texttt{siaflat.m} to a final time 20000 a, and then compares to the Halfar formula at that time.  It computes the thickness \emph{numerical error}, the absolute difference between the numerical and exact thicknesses at the final time:
\small
\begin{verbatim}
>> verifysia(20)
average thickness error  = 22.310
>> verifysia(40)
average thickness error  = 9.459
>> verifysia(80)
average thickness error  = 2.771
>> verifysia(160)
average thickness error  = 1.085
\end{verbatim}
\normalsize

The average numerical thickness error decreases with increasing grid resolution, as expected for a correctly-implemented code.  What is less obvious is that almost any numerical implementation mistake will break such convergence.  That is, the reported errors will not shrink if there is any meaningful bug (see the Exercises).  The fragility of verification is its essential benefit!  One must implement the model correctly, not just generate good-looking results, to get convergence as above.

The Halfar solution has also been used for modelling real ice masses.  Nye and others \cite{NyeIcarus2000} used it to compare the long-time consequences of different flow laws for the south polar cap on Mars.  In particular, they evaluated $\text{CO}_2$ and $\text{H}_2\text{O}$ ice softness parameters by comparing the long-time behavior of the corresponding Halfar solutions to the observed polar cap properties.  Their conclusions:
  \begin{quote}
  \dots none of the three possible [$\text{CO}_2$] flow laws will allow a 3000-m cap, the thickness suggested by stereogrammetry, to survive for $10^7$ years, indicating that the south polar ice cap is probably not composed of pure $\text{CO}_2$ ice [but rather] water ice, \dots
  \end{quote}
This theoretical result has since been confirmed by sampling the surface of Mars.

Are exact solutions like Halfar's always available when needed?  No, of course not.  However, many ice flow models have exact solutions which are relevant to verification; see the Notes and References.  For example, we will use an exact solution for ice shelves in a later section.  An apparent absence of exact solutions to a given model may actually show that not enough thought has gone into the continuum model itself.

\subsection*{A test of robustness}  Verification is an ideal way to test a code, but another worthwhile test is for ``robustness'' in unusual or difficult input cases.  One asks: Does the model break?  Here one may not have a precise expectation of what the code \emph{should} do, but it should not produce obviously-unreasonable outputs.

The robustness test run by \texttt{roughice.m} (not shown) demonstrates that \texttt{siaflat.m} can handle an ice sheet with extraordinarily-large driving stresses.  Recall that the glaciological driving stress $\tau_d = - \rho g H \grad h$ appears in the SIA model as the shear stress $(\tau_{13},\tau_{23})$ at the base of the ice.  The driving stress is largest where the ice is thick and the surface slope $|\nabla h|$ is steep.  In our test we give \texttt{siaflat.m} a randomly-generated initial ice sheet which has huge driving stresses.  It is thick---average of 3000 m---with huge surface slopes because of random surface elevation values.  An initial shape is shown in Figure \ref{fig:roughinitial} (left).  During a run of just 50 model years on a 17 km grid, with time step determined adaptively from \eqref{stabcrit}, the maximum diffusivity $D$ decreases as the surface both lowers and becomes smooth through flow.  The time-step consequently increases, in this case by three orders of magnitude from about 0.0002 years to 0.2 years.  The maximum value of the driving stress decreases from $57$ bar ($= 5.7\times 10^6$ Pa) to $3.6$ bar.  At the end of the run the ice cap has the shape shown in Figure \ref{fig:roughinitial} (right).

\twofigsizes{roughinitial}{roughfinal}{The initial ice sheet at left, with huge driving stresses, evolves to the ice cap at right in only 50 model years.}{2.9in}{2.9in}

The final shape in Figure 12 is close to a Halfar solution, illustrating what he proved \cite{Halfar81,Halfar83}, namely that \emph{all} solutions of the zero-mass-balance SIA on a flat bed asymptotically approach his solution.  The Halfar solution is attracting and thus worth knowing.

\section{Application to the Antarctic ice sheet}   \label{sec:antapp}

Let us apply the SIA model to the Antarctic ice sheet.  To do this we must first modify \texttt{siaflat.m} to allow non-flat bedrock elevation $b(x,y)$ and arbitrary, though here time-independent, climatic mass balance $M(x,y)$.  Also we enforce non-negative thickness at each timestep, and add a minimal model of interaction with the ocean.  Specifically, we calve-off any ice that satisfies the flotation criterion (see Section \ref{sec:shelvesandstreams}).  The result is \texttt{siageneral.m} (not shown), a code only ten lines longer than \texttt{siaflat.m}.

\twofigsizes{antinitial}{antfinal}{Left: Initial surface elevation (m) of Antarctic ice sheet.  Right: Final surface elevation at end of 40 ka model run on 50 km grid.}{2.55in}{3.2in}

We use measured accumulation, bedrock elevation, and surface elevation from the ALBMAPv1 data set \cite{LeBrocqetal2010}.  Melt is not modelled so the climatic mass balance is equal to the precipitation rate.  (This is a reasonable approximation for Antarctica, but not for Greenland.)  These input data are read from a NetCDF file and preprocessed by an additional code \texttt{buildant.m} (not shown).

The code \texttt{ant.m} (not shown) calls \texttt{siageneral.m} to do the simulation in blocks of 500 model years.  The volume is computed at the end of each block.  Figure \ref{fig:antinitial} shows the initial and final surface elevations from a run of 40,000 model years on a $\Delta x = \Delta y = 50$ km grid.  (The runtime on a typical laptop is a few minutes.)  Areas of the actual Antarctic ice sheet with low-slope and fast-flowing ice experience thickening in the model, while near-divide ice in East Antarctica, in particular, thins.

Assuming the present-day Antarctic ice sheet is somewhere near steady state, these thickness differences reflect model inadequacies.  The lack of a sliding mechanism explains the thickening in low-slope areas.  The lack of thermomechanical coupling, or equivalently the constancy of ice softness in the model, and the fact that our isothermal $A$ value is quite ``soft'', explains the thinning near the divide.  We should be modelling floating ice too, but the SIA is completely inappropriate to that purpose; compare Section \ref{sec:shelvesandstreams}.  See the Notes and References for widely-used thermomechanical and membrane-stress modelling techniques which address these inadequacies.

Figure \ref{fig:antvolcompare} compares the ice volume time series for 50 km, 25 km, and 20 km grids.  This result, namely grid dependence of the ice volume, is typical.  One cause is that steep gradients near the ice margin are poorly resolved, but to differing degrees, at such coarse resolutions.  In any case, Figure \ref{fig:antvolcompare} is a warning about the interpretation of model runs.  Even if the data is available only on a fixed grid, the model should be run at different grid spacings to evaluate the dependence of model results on resolution.

\medskip
\onefig{antvolcompare}{Ice volume of the modeled Antarctic ice sheet, in units of $10^6 \, \text{km}^3$, from 50 km (top), 25 km (middle), and 20 km (bottom) grids.}


\section{Mass continuity and kinematical equations}   \label{sec:masscont}

In the SIA the stress balance becomes formula \eqref{siavelocity} for the velocity.  Combined with the mass continuity equation \eqref{masscont}, this gives model \eqref{sia} for evolution of the ice sheet thickness.  Equation \eqref{sia} therefore combines two concepts which we now want to think about separately, and in greater generality.

The most basic shallow assumption made by most ice flow models is that the surface and base of the ice are differentiable functions $z=h(t,x,y)$ and $z=b(t,x,y)$.  That is, though the Stokes theory allows the fluid boundary to be any closed surface in three-dimensional space, ice sheet and glacier models take a map-plane perspective and have a well-defined ice thickness: $H=h-b$.  For example, surface overhang is rarely allowed in models.

\emph{Kinematical} equations apply on the upper and lower surfaces of the ice sheet.  Let $a$ be the climatic mass balance function ($a>0$ is accumulation) and $s$ be the basal melt rate function ($s>0$ is basal melting).  We need these fields in thickness-per-time units; convert from mass flux units by assuming an ice density.  The net map-plane mass balance $M=a-s$, which appears in the mass continuity equation \eqref{masscont}, is the difference of these surface fluxes.  The \emph{(upper) surface kinematical equation} is
\begin{equation}
h_t = a - \mathbf{U}\big|_h \cdot \grad h + w\big|_h,  \label{surfkine}
\end{equation}
and the \emph{base kinematical equation} is
\begin{equation}
b_t = s - \mathbf{U}\big|_b \cdot \grad b + w\big|_b.  \label{basekine}
\end{equation}
In these equations $\mathbf{U}$ is the horizontal ice velocity and $w$ the vertical ice velocity.  Equations \eqref{surfkine} and \eqref{basekine} describe the movement of the ice surfaces caused by the velocity of the ice and the mass balance functions.

An important mathematical fact follows from the assumption of well-defined surface elevations and incompressibility.  Namely, that any \emph{pair} of these three equations implies the third:
  \begin{itemize}
  \item the surface kinematical equation \eqref{surfkine},
  \item the base kinematical equation \eqref{basekine}, and
  \item the map-plane mass continuity equation \eqref{masscont}.
  \end{itemize}
One proves the any-two-implies-the-other implications by using equation \eqref{incompressible} and the Leibniz rule for differentiating integrals (see the Exercises).  Thus the surface kinematical and mass continuity equations are closely-related.

\subsection*{Prognostic models}  Consider an explicit, geometry-evolving, isothermal ice sheet model, called a \emph{prognostic} model in glaciology.  Each time step follows this recipe:
  \begin{itemize}
  \item numerically solve an incompressible stress balance, yielding velocity $\mathbf{u}=(u,v,w)$,
    \begin{itemize}
    \item[$\circ$] the ice geometry and the stress boundary conditions determine $\mathbf{u}$
    \item[$\circ$] if the stress balance yields $\mathbf{U}=(u,v)$ then $w$ comes from equation \eqref{incompressible}
    \end{itemize}
  \item decide on a time step $\Delta t$ for \eqref{masscont} based on velocities and/or diffusivities,
  \item from the horizontal velocity $\mathbf{U}=(u,v)$, compute the flux $\bq = \int_b^h \bU dz = \bar{\bU} H$
  \item update mass balance $M=a-s$ and do a time-step of \eqref{masscont} to get $H_t$
  \item update the upper surface elevation and thickness, and repeat.
  \end{itemize}
The last step uses the rules $H \mapsto H + H_t \Delta t$ and $h \mapsto h + H_t \Delta t$, but it also enforces $H\ge 0$ and $h\ge b$ as mentioned above.  Most ice sheet models use the mass continuity equation \eqref{masscont} to describe changes in ice sheet geometry, but \eqref{surfkine} may be used instead.

The above ``standard explicit ice sheet model'' has many variations.  Some glaciological questions are answered just by solving the stress balance for the velocity, so the geometry does not evolve.  Sometimes the goal is the steady-state configuration of the glacier, both geometry and velocity, in which case the constraint $H\ge 0$ should be regarded as a part of the continuum formulation and the solver \cite{Bueler2016,JouvetBueler2012}.

Other processes are often simulated at each time step, such as the conservation of energy within the ice, or subglacial and supraglacial processes.  Understanding the diverse time scales associated to these processes is always an important step in designing a model which extends our ``introductory'' isothermal SIA model.

It may seem that when using SIA equation \eqref{sia} one bypasses the computation of the ice velocity.  That is because the mass continuity equation can be written as a diffusion, with $\bq=-D\nabla h$ for the flux.  Fast flow in ice sheets is associated with sliding and floating ice, however, and for these flows the ice geometry evolution is not a diffusion.  A complete ice flow model will be neither diffusive nor wave-like (hyperbolic), but the general formula $\bq = \bar{\bU} H$ will apply everywhere.  The ice velocity is always related in some way to the local surface gradient, but usually one must solve membrane-stress-balancing equations.  Solving the stress balance for the velocity field is both a generally-nontrivial and obligatory step in a numerical ice flow model.  The next section explores one important case.


\section{Shelves and streams}  \label{sec:shelvesandstreams}

As its name suggests, the \emph{shallow shelf approximation} (SSA) stress balance applies to floating \emph{ice shelves}.  It also applies reasonably well to grounded \emph{ice streams} which have not-too-steep bed topography and low basal resistance (Figure \ref{fig:siple}).  But what is, and is not, an ice stream?  Ice streams slide at $50$ to $1000 \,\text{m}\,\text{a}^{-1}$, they have a concentration of vertical shear in a thin layer near base, and typically they flow into ice shelves.  Pressurized liquid water at their beds plays a critical role enabling their fast flow.  However, there are other fast-flowing grounded parts of ice sheets,  generally called \emph{outlet glaciers}.  They can have even faster surface speed (up to $10 \,\text{km}\,\text{a}^{-1}$), but an unknown fraction of this speed is from sliding at the base.  In an outlet glacier there is substantial vertical shear within the ice column, sometimes caused by a thick layer of soft temperate ice above the bed.  Outlet glaciers are typically controlled by fjord-like, large-slope bedrock topography.  Few simplifying assumptions are appropriate for outlet glaciers, and the SSA is often not a sufficient model.

\twofigsizes{siple}{streamisbrae}{Left:  The SSA model applies to these Siple Coast ice streams \cite{Joughinetal2002}; color is radar-derived surface speed.  Right: Cross sections, with no vertical exaggeration, of (\textbf{a}) Jakobshavns Isbrae outlet glacier, Greenland and (\textbf{b}) Whillans Ice Stream, Siple Coast, Antarctica \cite{TrufferEchelmeyer}.}{2.8in}{2.9in}

\subsection*{The shallow shelf approximation (SSA)}  We state the SSA stress balance equation only in the plane flow, isothermal case:
\begin{equation}
  \left(2 B H |u_x|^{1/n - 1} u_x\right)_x - C|u|^{m-1}u = \rho g H h_x \label{ssaearly}
\end{equation}
The term in parentheses is the vertically-integrated longitudinal stress, also called the \emph{membrane stress} in two horizontal dimensions.  The second term $\tau_b = - C|u|^{m-1}u$ is the basal resistance, which is zero ($C=0$) in an ice shelf.  The term on the right is the negative of the driving stress ($\tau_d = - \rho g H h_x$).  Thus the longitudinal strain rates are determined by a balance involving the integrated ice hardness (coefficient $BH$), the slipperyness of the bed (coefficient $C$ and the power $m$) and the geometry of the ice (thickness $H$ and surface slope $h_x$).

In \eqref{ssaearly} the velocity $u$ is independent of $z$.  We assume that the ice hardness $B=A^{-1/n}$ is also independent of depth.  (Non-isothermal models compute the vertical average of the temperature-dependent hardness.)  The coefficient $\bar \nu = B |u_x|^{1/n-1}$ is the \emph{effective viscosity}, which depends on $u$; \eqref{ssaearly} can be written
\begin{equation}
  \left(2 \,\bar \nu\, H u_x\right)_x - C |u|^{m-1} u = \rho g H h_x.  \label{ssa}
\end{equation}

The inequality
\begin{equation}
\rho H < - \rho_w b     \label{flotation}
\end{equation}
is called the \emph{flotation criterion}.  In our simple mechanical model, a \emph{grounding line} occurs where $\rho H = - \rho_w b$.  For grounded ice ($\rho H \ge - \rho_w b$) the driving stress uses $h = H+b$.  For floating ice one uses $h = (1-\rho/\rho_w) H$ in the driving stress so that equation \eqref{ssa} simplifies to
\begin{equation}
   \left(2 \,\bar\nu\, H u_x\right)_x = \rho g (1-\rho/\rho_w) H H_x. \label{ssafloat}
\end{equation}

\subsection*{Steady ice shelf exact solution}  For a steady 1D ice shelf, in which $H_t=0$, van der Veen \cite{vanderVeen83} built an exact solution by noting that both sides in equation \eqref{ssafloat} are derivatives, thus the equation is integrable.  In fact, because the mass continuity equation \eqref{masscont} also reduces to $M=(uH)_x$ in the steady case, velocity and thickness simultaneously solving \eqref{masscont} and \eqref{ssafloat} can be found when $M$ is constant ($M=M_0$).  This exact solution is parameterized by $M_0$ and the ice thickness $H_g$ and velocity $u_g$ at the grounding line (i.e.~the inflow).

Supposing $H_g=500$ m, $u_g = 50 \,\text{m}\,\text{a}^{-1}$, and $M_0=30 \,\text{cm}\,\text{a}^{-1}$ we get the exact results shown in Figure \ref{fig:steadyshelfprofile}.  We will use this exact solution to verify a numerical SSA code.  Because driving stresses are highest near the grounding line, the highest longitudinal strain rates and thinning rates also occur there.

\twofig{steadyshelfprofile}{steadyshelfvelocity}{The upper and lower surface elevation (m; left) and velocity (m/a; right) of the exact ice shelf solution; $x=0$ is the grounding line.}

\subsection*{Numerical solution of the SSA}  Suppose the ice thickness is a known function $H(x)$.  To find the velocity we must solve the nonlinear PDE \eqref{ssa} or \eqref{ssafloat} for $u(x)$.  When we do this numerically an iteration is needed because of the nonlinearity with respect to $u$.  The simplest iterative approach is to use an initial guess at the velocity, then compute an effective viscosity, then get a new velocity solution from a linear PDE problem, and repeat until the change is as small as desired.  We will demonstrate such a \emph{fixed point} or \emph{Picard iteration}.  \emph{Newton iteration}, which would converge faster, is more complicated to implement and may require a better initial guess to converge at all.

Denote the previous velocity iterate as $u^{(k-1)}$ and the current iterate as $u^{(k)}$.  Compute $\bar \nu^{(k-1)} = B |u^{(k-1)}_x|^{1/n-1}$ and define $W^{(k-1)} = 2 \bar \nu^{(k-1)} H$.  Solving the following linear elliptic PDE for the unknown $u^{(k)}$ is a Picard iteration for \eqref{ssa}:
\begin{equation}
   \left(W^{(k-1)} u^{(k)}_x\right)_x - C |u^{(k-1)}|^{m-1} u^{(k)} = \rho g H h_x. \label{picardssa}
\end{equation}
If the difference between $u^{(k-1)}$ and $u^{(k)}$ were zero then we would have a solution of \eqref{ssa}.  In practice we stop the iteration when the difference is smaller than some tolerance.

Equation \eqref{picardssa} is a linear boundary value problem, written abstractly as
\begin{equation}
  \left(W(x)\, u_x\right)_x - \alpha(x)\, u = \beta(x)  \label{innerlinear}
\end{equation}
where the functions $W(x)$, $\alpha(x)$, $\beta(x)$ are known.  Equation \eqref{innerlinear} applies for $x$ in an interval $[x_g,x_c]$ where $x_g,x_c$ are the locations of the grounding line and calving front, respectively.  At $x_g$ we have $u(x_g)=u_g$.  In the ice shelf case we also have the calving front condition
\begin{equation}
  2 B H |u_x|^{1/n - 1} u_x = \frac{1}{2}\rho (1-\rho/\rho_w) g H^2  \label{calvingstress}
\end{equation}
at $x=x_c$; see Notes and References.  Notice that \eqref{calvingstress} can be solved for $u_x(x_c)=\gamma$ in terms of the thickness $H(x_c)$ at the calving front.

Where do we get an initial guess $u^{(0)}$?  Generally this may require effort, but the choice is straightforward in the 1D case.  For grounded ice, we may assume ice is held by basal resistance only: $u^{(0)}(x) = \left(-C^{-1} \rho g H h_x\right)^{1/m}$.  For floating ice, an initial velocity iterate comes from assuming a uniform strain rate provided by the boundary conditions: $u^{(0)}(x) = u_g + \gamma (x-x_g)$.

Suppose equation \eqref{innerlinear} applies on $[x_g,x_c]=[0,L]$ and choose a grid with equal spacing $\Delta x$.  For $j=1,2,\dots,J+1$ we let $x_j=(j-1)\Delta x$ so that $x_1 = 0$ and $x_{J+1} = L$ are the boundary points.  The coefficient $W(x)$ is needed on a staggered grid, for stability and accuracy reasons similar to those for the SIA diffusivity: $W_{j+1/2}$ at $x_{j+1/2} = x_j + \Delta x/2$.  Our finite difference approximation of \eqref{innerlinear} is, therefore,
\begin{equation}
  \frac{W_{j+1/2} (u_{j+1} - u_j) - W_{j-1/2} (u_{j} - u_{j-1})}{\Delta x^2} - \alpha_j u_j = \beta_j  \label{discreteinnerlinear}
\end{equation}

For the left end boundary condition we have $u_1 = u_g$ given, which is easy to include in the linear system (below).  For the right end boundary condition we have $u_x(L)=\gamma$, which requires more thought.  First introduce a notional point $x_{J+2}$, then require $(u_{J+2} - u_J)/(2 \Delta x) = \gamma$, which is a centered approximation to ``$u_x(x_c)=\gamma$,'' and then, using equation \eqref{discreteinnerlinear} in the $j=J+1$ case, eliminate the $u_{J+2}$ variable.  This procedure generates the last equation in our linear system.

Thus each iteration solves a linear system of $J+1$ equations of the form $A \mathbf{u} = \mathbf{b}$ or
\begin{equation}
\begin{bmatrix}
1 &  &  &  &  \\
W_{3/2} & a_{22} & W_{5/2} &  &  \\
 & W_{5/2} & a_{33} &  &  \\
 &  & \ddots & \ddots &  \\
 &  & W_{J-1/2} & a_{JJ} & W_{J+1/2} \\
 &  &  & a_{J+1,J} & a_{J+1,J+1} \\
\end{bmatrix}\,
\begin{bmatrix}
u_1 \\ u_2 \\ u_3 \\ \vdots \\ u_J \\ u_{J+1}
\end{bmatrix}
=
\begin{bmatrix}
u_g \\ \beta_2 \Delta x^2 \\ \beta_3 \Delta x^2 \\ \vdots \\ \beta_J \Delta x^2 \\ b_{J+1}
\end{bmatrix}  \label{discretematrixform}
\end{equation}
The diagonal entries $a_{jj}$ are
  $$a_{22} = -(W_{3/2}+W_{5/2}+\alpha_2 \Delta x^2), \quad \dots, \quad a_{JJ} = -(W_{J-1/2}+W_{J+1/2}+\alpha_J \Delta x^2).$$
There are special cases for the coefficients in the last equation:
  $$a_{J+1,J} = 2 W_{J+1/2}, \quad a_{J+1,J+1} = -(2 W_{J+1/2}+\alpha_{J+1}\Delta x^2).$$
For the right side of the last equation, $b_{J+1} = -2 \gamma \Delta x W_{J+3/2} + \beta_{J+1} \Delta x^2$.

System \eqref{discretematrixform} is a \emph{tridiagonal} linear system.  It is fully-appropriate to hand this linear system to Matlab's linear solver, the ``backslash'' operator ($\mathbf{u} = A\, \backslash\, \mathbf{b}$), especially at this initial implementation stage. Matlab will identify it as a tridiagonal system and solve it efficiently.  (These notes will not worry further about solving linear systems arising from discretization, but see \cite{Bueler2021}.)  Collecting these ideas yields the following function for the abstracted problem \eqref{innerlinear}:

\minput{flowline}

By \emph{manufacturing} exact solutions to \eqref{innerlinear}---see Notes and References---we can test this first piece before proceeding to solve the actual nonlinear SSA problem.   Results from \texttt{testflowline.m} (not shown) show that our implemented numerical scheme converges at the correct rate $O(\Delta x^2)$.

\subsection*{The case of an ice shelf}  The code \texttt{ssaflowline.m} below numerically computes the velocity for an ice shelf.  It calls \texttt{ssainit.m} (not shown) to get an initial iterate $u^{(0)}(x)$, and then \texttt{flowline.m} at each iteration, thereby implementing Picard iteration \eqref{picardssa} to solve \eqref{ssafloat}.  The thickness $H(x)$ is assumed given, so we are not yet addressing the full, \emph{coupled} ice shelf problem in which one simultaneously solves the mass continuity \eqref{masscont1D} and stress balance \eqref{ssafloat} equations for $H$ and $u$.

\minput{ssaflowline}

Does \texttt{ssaflowline.m} work correctly?  The exact thickness and velocity solution shown in Figure \ref{fig:steadyshelfprofile}, as evaluated by \texttt{exactshelf.m} (not shown), allows us to evaluate the numerical error.  The convergence result in Figure \ref{fig:shelfconv} is from codes \texttt{testshelf.m} and \texttt{shelfconv.m} (not shown), which call \texttt{ssaflowline.m}.  Even on a coarse $4$ km grid the maximum velocity error is less than 1 m/a, while the maximum velocity itself is $\sim 300$ m/a.  In particular, at screen resolution our numerical velocity solutions are the same as shown in Figure \ref{fig:steadyshelfprofile}.

\onefig{shelfconv}{The numerical SSA velocity solution from \texttt{ssaflowline.m} converges to the exact solution, at nearly the optimal rate $O(\Delta x^2)$, as the grid is refined from spacing $\Delta x=8000$ m to $\Delta x=100$ m.}

\subsection*{Realistic ice shelf modelling}  Real ice shelves have two horizontal variables, are confined in bays, and experience \emph{confining stresses} or side drag, which is to say membrane stresses in 2D.  Their velocities vary spatially and temporally along their grounding lines, which are now complicated curves.  Furthermore real ice shelves have nontrivial mass and energy processes on their surfaces, including high basal melt near grounding lines, marine ice basal freeze-on, and fracturing which nears full thickness at the calving front.  So the modeling of real ice shelves is complicated.

Nonetheless \emph{diagnostic} (fixed geometry) ice shelf modelling in two horizontal variables, where the velocity is unknown but the thickness is known and fixed, is quite successful using only the isothermal SSA model.  For example, Figure \ref{fig:rossquiver} shows a Parallel Ice Sheet Model (PISM) \cite{BBssasliding,Winkelmannetal2011} result for the Ross ice shelf, compared to observed velocities.  There is only one tuned parameter, the constant value of the ice hardness $B$.  Note that observed velocities for grounded ice were applied as boundary conditions.  Many current ice shelf models yield comparable match \cite{MacAyealetal}.

\twofigsizes{rossquiver}{rossscatter}{Left: Observed (white) and PISM-modeled (black) ice velocities are nearly the same across the whole Ross ice shelf.  Right: A scatter plot shows the same points by observed-vs-modeled ice speed.}{2.85in}{3.0in}


\section{Summary} \label{sec:summary}

These brief notes give an incomplete view of numerical ice flow models for glaciers and ice sheets.  As a meager conclusion here are four high-level views of such models:
\begin{itemize}
\item The mass continuity equation, a conservation equation in the map-plane, describes ice thickness changes.  In most ice sheet models it is the component which describes how the geometry evolves.  The numerical approach to this equation depends on the form of the stress balance which supplies the ice velocity, but, since it is a conservation equation, it is common to evaluate the ice flux on the staggered grid (finite difference thinking) or on cell boundaries (finite volume thinking).
\item The mass continuity equation has diffusive character (SIA-like) at larger spatial scales and in areas with significant basal resistance, but it is \emph{not} very diffusive for membrane-stress dominated areas.  Ice streams and shelves are often modeled by the membrane-stress-resolving SSA.
\item The SIA stress balance is exceptional because it contains no horizontal derivatives of the velocities.  It follows that in the SIA the velocity can be found simply by vertical integration of geometric quantities (e.g.~the driving stress).  No other stress balance is like this.
\item Membrane-stress-resolving balance equations, such as the SSA, Blatter-Pattyn, hydrostatic, and Stokes models, all determine horizontal velocity from solving spatial, elliptic-like PDEs using the ice geometry and stress boundary conditions.  Because of the Glen law these PDEs are nonlinear, so iteration is necessary.  At each iteration a sparse matrix ``inner'' problem is typically solved using a linear-algebra solver package, e.g.~a high-performance library like PETSc \cite{Bueler2021}.
\end{itemize}

We have illustrated some best practices for numerical modelling:
\begin{itemize}
\item Return often to the continuum model.
\item Modularize your codes.
\item Test your modules:  Does it show convergence to exact solutions?  Is it robust?
\end{itemize}


\section{Notes} \label{sec:nr}

Continuum modelling of ice flows is covered in textbooks \cite{CuffeyPaterson,GreveBlatter2009,vanderVeen}, and in the excellent review \cite{SchoofHewitt2013}, among other sources.

The SIA model, which was originally derived by several authors \cite{FowlerLarson1978,Hutter,MorlandJohnson}, follows from scaling the Stokes equations using the aspect ratio $\eps = [H]/[L]$, where $[H]$ is a typical thickness of an ice sheet and $[L]$ is a typical horizontal dimension.  After scaling one drops small terms in the $\eps\to 0$ limit \cite{Fowler,Hutter}, a ``small-parameter argument''.  In Fowler's form of the scaling argument \cite{Fowler} there are no $O(\eps)$ terms in the scaled equations so one only drops $O(\eps^2)$ terms.  The SIA is re-formulated in \cite{JouvetBueler2012} as a well-posed free boundary problem, by adding the constraint that the thickness is positive, thereby providing the correct boundary condition at grounded margins; a similar approach is in \cite{Bueler2016}.  The Mahaffy \cite{Mahaffy} scheme for diffusivity used here is not the only one \cite{HindmarshPayne}.  The Mahaffy scheme is the basis for an unconditionally-stable implicit scheme \cite{Bueler2016}.

The SSA model \cite{WeisGreveHutter} was first applied to ice shelves in \cite{Morland} and ice streams in \cite{MacAyeal}.  In deriving the SSA, the aspect ratio $\eps$ is one of two small parameters; the other describes the magnitude of surface undulations \cite{SchoofStream,SchoofHindmarsh}.  A well-posed model for the emergence of ice streams though till failure, using only the SSA, is in \cite{SchoofStream}.  Little theory is known about the well-posedness of the SSA+(mass continuity) coupled equations.

A major modelling issue omitted from these notes is thermomechanical coupling \cite{Blatteretal2010}.  Temperature is important because the ice softness varies by three orders of magnitude in the temperature range relevant to ice sheet modelling, and because temperature gives ice sheets a long memory of past climate.  While the geothermal flux is a significant input in slow-flowing parts of ice sheets, the dissipation of gravitational potential energy is often the dominant heat source, especially as it melts basal ice and facilitates sliding.  For example, each year the ice in the Jakobshavn drainage basin in Greenland dissipates enough gravitational potential energy to fully melt more than $1\,\text{km}^3$ of ice \cite{AschwandenBuelerKhroulevBlatter}.  In practice this is reflected as high basal water production.  Furthermore it heats large volumes of ice up to the pressure-melting temperature, thereby generating thick layers of temperate ice in outlet glaciers.  (Beautiful observation evidence appears in \cite{Luethietal2009}.)  These physical effects motivate modellers to solve the conservation of energy equation simultaneously with the mass conservation (mass continuity) and momentum conservation (stress balance) equations.  The conservation of energy equation is strongly advection-dominated for ice flows.  While some models only use temperature as the state variable \cite{BBL}, perhaps suitable for their coldest parts, ice sheets are generally polythermal \cite{FowlerLarson1978,Greve}.  Enthalpy methods track the energy content of polythermal ice sheets and glaciers \cite{AschwandenBuelerKhroulevBlatter,GreveBlatter2016}, though one may also solve a separate water-content equation in temperate ice \cite{Greve}.

Pressurized basal water is usually required for ice to slide at significant speed.  To model the production of such water one must at least compute the basal melt rate via the energy conservation equation \cite{BBssasliding,BuelervanPelt2015,Clarke05,Raymondenergy,Tulaczyketal2000b}.  Beyond this statement there is little consensus on what is essential, or even constructive, in terms of process sub-models to improve the modeling of ice sheet sliding.  Belief in one's preferred cartoon of subglacial morphology will lead to continuum equations for subglacial hydrology, and then numerical experiments will proceed.

One of the most significant issues in modelling ice sheets using shallow models is to describe the ``switch'', in space and time, between shear-dominated and membrane-stress-dominated flow.  It is not a good idea to abruptly switch from the SIA model to the SSA model at the edge of an ice stream, though this was attempted in early models \cite{HulbeMacAyeal,Ritzetal2001}.  Instead \emph{hybrid} schemes exist which solve the SIA and SSA everywhere in the ice sheet \cite{BBssasliding,PollardDeConto,Winkelmannetal2011} and then combine the stresses or velocities according to different schemes.  Coherent vertically-integrated models have also arisen \cite{BrinkerhoffJohnson2013,Goldberg2011}.

Solving the Stokes stress balance normally accomodates the incompressibility constraint by using pressure as an unknown \cite{Bueler2021,JouvetRappaz2011,Lengetal2012,ISMIPHOM}.  (Separate Appendix A shows a 2D example.)  Numerical approximations of this stress balance are indefinite, thus harder to solve.  In plane flow one may address the incompressibility constraint by using stream functions \cite{BaliseRaymond1985}.

``Higher-order'' three-dimensional approximations of the Stokes stress balance, such as the Blatter-Pattyn model \eqref{stressblatter} \cite{Blatter,Pattyn03,SchoofHindmarsh}, are also shallow approximations in fact.  Even Stokes ice sheet models typically apply the most-basic shallow assumption of well-defined thickness (see main text).  Higher-order models add an assumption such as hydrostatic normal stress \cite{GreveBlatter2009}.  Computational limitations generally restrict either the spatial extent, the spatial resolution, or the run duration of these models relative to SIA-SSA hybrids, but careful numerical analysis may generate more computationally-efficient solutions \cite{Brown2013}.  Vertically-integrated and hybrid models will always allow finer map-plane resolution and longer time scales than higher-order and Stokes models, simply because 2D stress balance equations are easier to solve.  On the other hand, questions remain about what are the most important deficiencies, relative to the Stokes model, when using either higher-order \cite{ISMIPHOM} or hybrid models.

The finite difference material in these notes should probably be read with a reference like \cite{LeVequeFD} or \cite{MortonMayers} in hand.  The ``main theorem for numerical PDE schemes'' mentioned in the text is the Lax equivalence theorem.  Alternative numerical discretization techniques include the finite element \cite{Braess}, finite volume \cite{LeVeque}, and spectral \cite{Trefethen} methods.  Newton iteration for the nonlinear discrete equations is superior to Picard iteration used here, in terms of rapid convergence once iterates are near the solution, but care is needed to globalize convergence \cite{Kelley}.  High-performance Newton solvers for nonlinear PDEs are demonstrated in \cite{Bueler2021}.

Which are the best numerical models for moving grounding lines?  Even when the minimal SSA stress balance is used, this is still something of an open question \cite{Feldmannetal2014,Goldbergetal2009,MISMIP3d2013,MISMIP2012,SchoofMarine1}.  The physics requires at least that the quantities $H$ and $u$ are continuous there.  Multiple stress balance regimes exist in the vicinity of the grounding line \cite{SchoofMarine2}.

Where to find exact solutions for ice flow models?  Textbooks by van der Veen \cite{vanderVeen} and Greve and Blatter \cite{GreveBlatter2009} have a few.  Halfar's similarity solution to the SIA \cite{Halfar81,Halfar83} has been generalized to non-zero mass balance \cite{BLKCB}.  There are flow-line \cite{Bodvardsson,vanderVeen83} and cross-flow \cite{SchoofStream} solutions to the SSA model, and one can even construct an exact, steady marine ice sheet in the flow-line case \cite{Bueler2014exactmarine}.  For the Stokes equations themselves there are plane flow solutions for constant viscosity \cite{BaliseRaymond1985}.

More generally, for numerical verification purposes one can \emph{manufacture} exact solutions by starting with a specified solution and then deriving a source term so that the specified function is actually a solution \cite{Bueler2021,Roache}.  The literature contains such manufactured solutions to the thermomechanically-coupled SIA \cite{BBL}, plane flow Blatter-Pattyn model \cite{GlowinskiRappaz}, and Glen-law Stokes equations \cite{JouvetRappaz2011,Lengetal2012,SargentFastook2010}.

\footnotesize

\bigskip
\bigskip
%\bibliographystyle{siam}
\documentclass[letterpaper,final,12pt,reqno]{amsart}

\usepackage[total={6.3in,9.2in},top=1.1in,left=1.1in]{geometry}

\usepackage{verbatim}
\usepackage{empheq}
\usepackage[dvipsnames]{xcolor}
\usepackage{animate}
\usepackage{graphicx}
\usepackage{fancyvrb}

%\usepackage{palatino}

% hyperref should be the last package we load
\usepackage[pdftex,
colorlinks=true,
plainpages=false, % only if colorlinks=true
linkcolor=blue,   % only if colorlinks=true
citecolor=Red,   % only if colorlinks=true
urlcolor=ForestGreen     % only if colorlinks=true
]{hyperref}

\pdfinfo{
/Title (Numerical modelling of ice sheets, streams, and shelves)
/Author (Ed Bueler)
/Subject (numerical modelling of ice sheets)
/Keywords (numerical modelling, numerical analysis, glacier, ice sheet, ice shelf, shallow models)
}

\renewcommand{\baselinestretch}{1.05}

\newcommand{\ddt}[1]{\ensuremath{\frac{\partial #1}{\partial t}}}
\newcommand{\ddx}[1]{\ensuremath{\frac{\partial #1}{\partial x}}}
\newcommand{\ddy}[1]{\ensuremath{\frac{\partial #1}{\partial y}}}
\newcommand{\pp}[2]{\ensuremath{\frac{\partial #1}{\partial #2}}}
\renewcommand{\t}[1]{\texttt{#1}}
\newcommand{\Matlab}{\textsc{Matlab}\xspace}
\newcommand{\bq}{\mathbf{q}}
\newcommand{\bu}{\mathbf{u}}
\newcommand{\bU}{\mathbf{U}}
\newcommand{\eps}{\epsilon}
\newcommand{\grad}{\nabla}
\newcommand{\Div}{\nabla\cdot}
\newcommand{\devstress}{\tau}

\newcommand{\minput}[1]{
\vspace{0.8cm}
\VerbatimInput[frame=single,framesep=3mm,label=\fbox{\normalsize \textsl{\,#1.m\,}},fontfamily=courier,fontsize=\footnotesize]{tmp/#1.slim.m}
\vspace{0.5cm}
}

% usage:  \onefigsize{name}{caption}{width}
\newcommand{\onefigsize}[3]{
\begin{figure}[ht]
\centering
\includegraphics[width=#3,keepaspectratio=true]{#1}
\caption{#2}
\label{fig:#1}
\end{figure}}

% usage:  \onefig{name}{caption}
\newcommand{\onefig}[2]{\onefigsize{#1}{#2}{3.0in}}

% usage:  \twofigsizes{left-name}{right-name}{caption}{left-width}{right-width}
\newcommand{\twofigsizes}[5]{
\begin{figure}[ht]
\centering
\includegraphics[width=#4,keepaspectratio=true]{#1} \quad
\includegraphics[width=#5,keepaspectratio=true]{#2}
\caption{#3}
\label{fig:#1}
\end{figure}}

% usage:  \twofig{left-name}{right-name}{caption}
\newcommand{\twofig}[3]{\twofigsizes{#1}{#2}{#3}{2.5in}{2.5in}}



\begin{document}
\graphicspath{{../photos/}{../pdffigs/}}

\begin{titlepage}

  \begin{center}
  \phantom{foo}
    \vspace{1.0cm}

     {\Large \textsc{Numerical modelling}}
    \vspace{0.7cm}

     {\Large \textsc{of ice sheets, streams, and shelves}}

    \vspace{1.5cm}

    {\large Ed Bueler}
    \vspace{1cm}

    Summer School in Glaciology, McCarthy Alaska, June 2016 

    \vfill
    
    \includegraphics[width=6.0in]{flowline}
  
    \scriptsize \emph{Illustrates the notation used in these notes.  Figure modified from \cite{SchoofMarine1}.} \normalsize
    
    \vspace{1.5in}
  \end{center}
\end{titlepage}

\clearpage\newpage

%\setcounter{section}{1}
\section{Introduction}

The most common use of numerical models in glaciology may be to help you ask: When I put together my incomplete understanding of glacier processes into a mathematical model, does the combination behave as I expect?  Numerical models can at least demonstrate flaws in our understanding of glacier processes, and they can show us how these processes combine to give overall behavior, but they should be built with care.  The worst outcome is to spend time (and perhaps reputation) explaining, through physical argumentation and perhaps using observational evidence, numerical model behavior that is an artifact of poor computer programming or numerical analysis.

So the reader of these notes may be surprised that a continuum model, and not a computer code, seems to be our focus much of the time.  While all codes produce some numbers, we want numbers that actually come from our continuum model.  We will therefore analyse numerical implementations to see if they match the continuum model and its solutions.

These notes have a limited scope:
  \begin{quote}\emph{shallow approximations of ice flow.}\end{quote}
They adopt a constructive approach; we provide:
  \begin{quote}\emph{example numerical codes that actually work.}\end{quote}
Within our scope are the shallow ice approximation (SIA) in two horizontal dimensions (2D), the shallow shelf approximation (SSA) in 1D, and the mass continuity and surface kinematical equations.  We recall the Stokes model, but we do not address its numerical solution.  Our numerical concepts include finite difference schemes, solving algebraic systems from stress balances, and the verification of codes using exact solutions.

Our notation, which generally follows \cite{GreveBlatter2009}, is common in the glaciological literature, but see Table \ref{tab:notation}.  Cartesian coordinates $x,y,z$ have $z$ perpendicular to the geoid and positive-upward.  If these coordinates or ``$t$'' appear as subscripts then they denote partial derivatives: $u_x = \partial u/\partial x$.  Tensor notation uses subscripts from the list $\{1,2,3,i,j\}$.  For example, ``$\tau_{ij}$'' or ``$\tau_{13}$'' denote entries of the deviatoric stress tensor.

These notes are based on eighteen Matlab codes, each about one-half page.  All have been tested in Matlab and Octave.  They are distributed by cloning the repository
\begin{quote}
\url{https://github.com/bueler/mccarthy}
\end{quote}
\noindent and looking in the \texttt{mfiles/} subdirectory.  Though only five of the codes are printed here, with their comments stripped for compactness and clarity, the electronic versions have generous comments and help files.

\begin{table}[ht]
\caption{Notation used in these notes, with values for some constants.}
\begin{tabular}{clll}
variable  & description & SI units & value \\
\hline
$A$ & $A=A(T)=$ ice softness in the Glen flow law & $\text{Pa}^{-n}\,\text{s}^{-1}$ \\
$B$ & ice hardness; $B=A^{-1/n}$ & $\text{Pa}\,\text{s}^{1/n}$ \\
$b$ & bedrock elevation & m \\
$c$ & specific heat in general & J kg$^{-1}$ K$^{-1}$ \\
$\nabla$ & (spatial) gradient & m$^{-1}$ \\
$\nabla\cdot$ & (spatial) divergence & m$^{-1}$ \\
$\mathbf{g}$ & gravity & m s$^{-2}$\phantom{foobar} & 9.81 \\
$H$ & ice thickness & m \\
$h$ & ice surface elevation & m \\
$\kappa$ & conductivity in general & J s$^{-1}$ m$^{-1}$ K$^{-1}$ \\
$M$ & climatic mass balance & m s$^{-1}$ \\
$n$ & exponent in Glen flow law & & 3 \\
$\nu$ & viscosity & Pa s \\
$p$ & pressure & Pa \\
$\bq$ & map-plane ice flux: $\bq = \int_{b}^{h} \bU\,dx = \bar{\bU} H$ & $\text{m}^2\,\text{s}^{-1}$ \\
$\rho$ & (1) density in general & kg m$^{-3}$ & \\
  & (2) density of ice & kg m$^{-3}$ & 910 \\
$\rho_w$ & density of sea water & kg m$^{-3}$ & 1028 \\
$T$ & temperature & K \\
$\tau$ & magnitude of $\tau_{ij}$: $2 \tau^2 = \sum_{ij} \tau_{ij}^2$ & Pa \\
$\tau_{ij}$ & deviatoric stress tensor & Pa \\
$Du_{ij}$ & strain rate tensor & s$^{-1}$ \\
$\mathbf{U}$ & $=(u,v)$ horizontal ice velocity & m s$^{-1}$ \\
$\mathbf{u}$ & $=(u,v,w)$ 3D ice velocity & m s$^{-1}$ \\
\end{tabular}
\label{tab:notation}
\end{table}


\section{Ice flow equations}

My first goal in these notes is to get to an equation for which I can say:
\begin{center}
\emph{by numerically solving this equation, we have a usable model for an ice sheet.}
\end{center}
\noindent A ``usable'' model tends to be \emph{understood} as much as it is \emph{correct}.  Also, this first model will not be complete by any modern standard.  To get to my goal I first (briefly!) recall the continuum mechanical equations of ice flow.  

Ice in glaciers is a moving fluid so we describe its motion by a velocity field $\mathbf{u}(t,x,y,z)$.  If the ice fluid were faster-moving than it actually is, and if it were linearly-viscous like liquid water, then it would be a ``typical'' incompressible fluid.  We would use the Navier-Stokes equations as the model:
\begin{align}
\nabla \cdot \mathbf{u} &= 0 &&\text{\emph{incompressibility}} \label{incompressible} \\
\rho \left(\mathbf{u}_t + \mathbf{u}\cdot\nabla \mathbf{u}\right) &= -\nabla p + \nabla \cdot (\nu \nabla \mathbf{u}) + \rho \mathbf{g} &&\text{\emph{stress balance}} \label{navierstokes}
\end{align}
In equation \eqref{navierstokes} the term $\mathbf{u}_t + \mathbf{u}\cdot\nabla \mathbf{u}$ is an acceleration.  The right-hand side of \eqref{navierstokes} is the total stress, and so equation \eqref{navierstokes} says ``$ma=F$'', i.e.~it is Newton's second law.  Much time has been spent to get partial understanding of the rich solutions of these Navier-Stokes equations; a book-length introduction like \cite{Acheson} is recommended.  The numerical solution of these equations is \emph{computational fluid dynamics} (CFD).

But, is numerical ice flow modelling a part of CFD?  Does a well-written general-purpose CFD text like \cite{Wesseling} help the glaciers student?  Ice sheet flow is a large-scale fluid problem like atmosphere and ocean circulation in climate systems, but it is an odd one.  Consider some topics which might make ocean circulation modelling exciting, for example:
  \begin{center} turbulence \qquad convection \qquad  coriolis force  \qquad density stratification
  \end{center}
None of these topics are relevant to ice flow.  What could be interesting about the flow of slow and old, and surely boring, ice?

First observe that ice is indeed a slow fluid.  In terms of equation \eqref{navierstokes}, ``slow'' means $\rho \left(\mathbf{u}_t + \mathbf{u}\cdot\nabla \mathbf{u}\right) \approx 0$, which says that the forces (stresses) of inertia are negligible.  However, ice is also a shear-thinning fluid with a specific kind of nonlinearly-viscous (``non-Newtonian'') behavior in which larger strain rates imply smaller viscosity.  The viscosity $\nu$ in \eqref{navierstokes} is therefore not constant, and so we separately state an empirically-based flow law below.

\subsection*{Stokes equations}  So now the standard model for isothermal flow is this set of Stokes equations:
\begin{align}
\nabla \cdot \mathbf{u} &= 0 &&\text{\emph{incompressibility}} \label{incompressibleagain} \\
0 &= - \nabla p + \nabla \cdot \tau_{ij} + \rho \mathbf{g} &&\text{\emph{stress balance}} \label{forcebalance} \\
Du_{ij} &= A \tau^2 \tau_{ij} &&\text{\emph{$n$=3 Glen flow law}} \label{flowlaw}
\end{align}
In the flow law \eqref{flowlaw}, the deviatoric stress tensor $\tau_{ij}$ and the strain rate tensor $Du_{ij}$ appear; previous lectures cover these.  Here we merely note that: $Du_{ij} = (1/2)((u_i)_{x_j}+(u_j)_{x_i})$ if we index coordinates by $x_1,x_2,x_3=x,y,z$, each tensor in \eqref{flowlaw} is symmetric and has trace zero, and $\tau^2 = (1/2) \tau_{ij} \tau_{ij}$ defines ``$\tau^2$'' in \eqref{flowlaw} if we use the summation convention.

The Stokes equations do not contain a time derivative.  Thus boundary stresses, the force of gravity $\rho \mathbf{g}$, and ice softness $A$ together determine the velocity and stress fields (i.e.~$\bu$, $p$, $\tau_{ij}$) instantaneously.  Thus ice flow simulation codes have no memory of prior momentum or velocity.  Said another way, velocity is a ``diagnostic'' output of ice flow codes, because it is not needed for (re)starting a simulation.

\subsection*{Plane-flow Stokes equations}  Consider now the $x,z$-plane case of equations \eqref{incompressibleagain}, \eqref{forcebalance}, and \eqref{flowlaw}.  ``Plane-flow'' means that velocity component $v$ is zero and that all derivatives with respect to $y$ are zero:
\begin{align}
u_x + w_z &= 0 &&\text{\emph{incompressibility}} \label{incompressiblexz} \\
p_x &= \tau_{11,x} + \tau_{13,z} &&\text{\emph{stress balance} ($x$)} \label{stokespx} \\
p_z &= \tau_{13,x} - \tau_{11,z} - \rho g &&\text{\emph{stress balance} ($z$)} \label{stokespz} \\
u_x &= A \tau^2 \tau_{11} &&\text{\emph{flow law} (diagonal)}  \label{forceflowx} \\
u_z + w _x &= 2 A \tau^2 \tau_{13} &&\text{\emph{flow law} (off-diagonal)} \label{forceflowz}
\end{align}
Note that $\tau_{13}$ is a shear stress while $\tau_{11}$ and $\tau_{33}=-\tau_{11}$ are deviatoric longitudinal stresses.  Also $\tau^2 = \tau_{11}^2+\tau_{13}^2$ in this case.  Equations \eqref{incompressiblexz}--\eqref{forceflowz} form a system of five nonlinear equations in five scalar unknowns ($u,w,p,\tau_{11},\tau_{13}$).

\subsection*{Slab-on-a-slope}  Equations \eqref{incompressiblexz}--\eqref{forceflowz} are complicated enough to make us pause before jumping in to numerical solution methods, but  we can handle a simplified situation first.  A uniform slab of ice, or a ``slab-on-a-slope'', is both a case in which we actually solve the Stokes equations, and a motivation for the shallow model in the next subsection.

\onefig{slab}{Rotated axes for a slab-on-a-slope flow calculation.}

We rotate our coordinates only for this example and not elsewhere in these notes.  The two-dimensional axes ($x$,$z$) shown in Figure \ref{fig:slab} are rotated downward (clockwise) at angle $\alpha>0$ so that the gravity vector has components $\mathbf{g} = (g \sin\alpha,- g \cos \alpha)$.  Equations \eqref{stokespx} and \eqref{stokespz} in these rotated coordinates are
\begin{align}
p_x &= \tau_{11,x} + \tau_{13,z} + \rho g \sin\alpha, \label{stokespxrot} \\
p_z &= \tau_{13,x} - \tau_{11,z} - \rho g \cos\alpha. \label{stokespzrot}
\end{align}
Assuming also that there is no variation with $x$, the whole set of Stokes equations \eqref{incompressiblexz}, \eqref{forceflowx}, \eqref{forceflowz}, \eqref{stokespxrot}, \eqref{stokespzrot} simplifies greatly:
\begin{align}
w_z &= 0 &   0 &= \tau_{11} \label{stokesslab} \\
\tau_{13,z} &= - \rho g \sin\alpha &   u_z &= 2 A \tau^2 \tau_{13} \notag \\
p_z &= -\tau_{11,z} = - \rho g \cos\alpha \notag
\end{align}
We apply boundary conditions for these functions of $z$: $w(0)=0$, $p(H)=0$, $u(0)=u_0$.  The basal velocity $u_0$ will remain undetermined for now.

By integrating equations \eqref{stokesslab} vertically and using $\tau_{11}=0$, we get $w=0$, $p = \rho g \cos\alpha (H-z)$, and $\tau_{13} = \rho g \sin\alpha (H-z)$.  Note that $H-z$ is the depth below the ice surface, so both the pressure $p$ and shear stress $\tau_{13}$ are proportional to depth.  Because $u_z = 2 A \tau^2 \tau_{13}$, by integrating vertically again we find the horizontal velocity:
\begin{align}
u &= u_0 + \frac{1}{2} A (\rho g \sin\alpha)^3  \left(H^4 - (H-z)^4\right)  \label{uslab}
\end{align}

Do we believe formula \eqref{uslab}?  Figure \ref{fig:slabvel} compares to observations of a mountain glacier.  This comparison shows we have at least done a credible job of capturing deformation flow velocity, though we do not yet have a model for the sliding velocity $u_0$ (i.e.~basal motion).  

\twofigsizes{slabvel}{athabasca_deform}{Left:  Velocity from slab-on-a-slope formula \eqref{uslab}.  Right:  Inclinometry-measured velocity in a glacier (Athabasca Glacier \cite{SavagePaterson}).}{2.0in}{1.8in}

\subsection*{Plane-flow mass-continuity equation}  Observe that the equations so far do not address the change in shape of the glacier or ice sheet.  For this we need another equation, the \emph{mass continuity equation}.  First, define the vertical average of velocity:
	$$\bar U = \frac{1}{H}\int_0^{H} u\,dz.$$
The flux $q=\bar U\, H$ is the rate of flow input into the side of the area in Figure \ref{fig:slabmasscontfig}.

\onefigsize{slabmasscontfig}{Mass continuity equation \eqref{masscont1D} follows from considering the changing area $A$ of ice in a planar flow.  Ice can be added by surface mass balance $M$ or a difference of flux $q=\bar u H$ into the left and right sides.}{2.5in}

The ice area $A$ in Figure \ref{fig:slabmasscontfig} changes by adding all the boundary contributions,
\begin{equation}
\frac{dA}{dt} = \int_{x_1}^{x_2} M(x)\,dx + \bar U_1 H_1 - \bar U_2 H_2: \label{masscontintegrated}
\end{equation}
Here $M(x)$ is the climatic mass balance at the ice surface.  (In three-dimensions, equation \eqref{masscontintegrated} would be an equation for $dV/dt$, the rate of change of ice volume.)

If the width $\Delta x=x_2-x_1$ is small then $A\approx \Delta x\, H$.  So we divide by $\Delta x$ and take $\Delta x \to 0$ in \eqref{masscontintegrated} and get
\begin{equation}
H_t = M - \left(\bar U H\right)_x \label{masscont1D}
\end{equation}
This mass continuity equation describes change in the ice thickness in terms of surface mass balance and ice velocity.  It is a major ``use'' of the velocity in ice flow simulations.

\subsection*{Viscosity form of the flow law}  The flow law \eqref{flowlaw} has another form which we will use next, and later in describing ice shelf and stream flow.  Recall $\tau^2 = (1/2) \tau_{ij} \tau_{ij}$, which uses the summation convention.  Also define $|Du|^2 = (1/2) Du_{ij} Du_{ij}$.  The scalars $\tau$ and $|Du|$ are ``norms'' (also ``second invariants'') of the tensors $\tau_{ij}$ and $Du_{ij}$, respectively.  By taking these norms of both sides of \eqref{flowlaw} we get $|Du| = A \tau^3$.  But then $\tau = A^{-1/3} |Du|^{1/3}$, so \eqref{flowlaw} can be rewritten
\begin{equation}
\tau_{ij} = 2 \nu\, Du_{ij}  \qquad \text{\emph{flow law (viscosity form)}} \label{viscosityflowlaw}
\end{equation}
where $\nu = (1/2) A^{-1/3} |Du|^{-2/3}$ is the nonlinear viscosity.  Often $B = A^{-1/3}$ is called the ice ``hardness''.  The derivation of \eqref{viscosityflowlaw} is worth knowing in detail; see the Exercises.

Form \eqref{viscosityflowlaw} of the flow law allows us to eliminate stresses $\tau_{ij}$ from the Stokes equations by replacing them with formulas depending on derivatives of the velocity, that is, on the strain rates only.  The next two approximate models also use this idea.

\subsection*{The hydrostatic and Blatter-Pattyn approximations}  We return now briefly to plane-flow Stokes equations \eqref{incompressiblexz}--\eqref{forceflowz}, and reconsider how to simplify them.  One simplification step, present in all shallow models, is the ``hydrostatic'' approximation.  It drops the single term  $\tau_{13,x}$ from the $z$-component of the stress balance \eqref{stokespz}.  That is, it assumes that horizontal variation in the vertical shear stress is small compared to the other terms:
\begin{equation}
p_z = - \tau_{11,z} - \rho g. \label{hydrostaticpz}
\end{equation}
Because the (Cauchy) stress tensor $\sigma_{ij}$ is related to the deviatoric stress tensor by $\sigma_{ij} = \tau_{ij} - p \delta_{ij}$, and thus $p + \tau_{11} = p - \tau_{33} = - \sigma_{33}$, equation \eqref{hydrostaticpz} says that the vertical normal stress $\sigma_{33}$ is linear in depth.  Taking it to have surface value zero we get
\begin{equation}
p + \tau_{11} = \rho g (h-z). \label{hydrostaticitself}
\end{equation}

Equation \eqref{hydrostaticitself} is the major hydrostatic statement, and it allows one to eliminate $p$ from the model equations.  Furthermore, taking the $x$-derivative of \eqref{hydrostaticitself} and substituting into \eqref{stokespx}, then using the viscosity form \eqref{viscosityflowlaw}, leads to this equation:
\begin{equation}
\left(4 \nu u_x\right)_x + \left(\nu (u_z+w_x)\right)_z = \rho g h_x \qquad\text{\emph{hydrostatic stress balance}} \label{stresshydrostatic}
\end{equation}
The hydrostatic stress balance equation \eqref{stresshydrostatic} is nontrivially-coupled to incompressibility \eqref{incompressiblexz} because derivatives of the vertical velocity $w$ appear in both equations, though $p$ is gone.  Nonetheless coupled equations \eqref{incompressiblexz} and \eqref{stresshydrostatic}, along with the formula $\nu = (1/2) A^{-1/3} |Du|^{-2/3}$ and appropriate boundary conditions, determine $u$ and $w$.

If we drop $w_x$ from equation \eqref{stresshydrostatic} then we get the Blatter-Pattyn model
\begin{equation}
\left(4 \nu u_x\right)_x + \left(\nu u_z\right)_z = \rho g h_x \qquad\text{\emph{Blatter-Pattyn stress balance}} \label{stressblatter}
\end{equation}
Using this equation one can solve first for the horizontal velocity $u$ and then afterward recover $w$ from \eqref{incompressiblexz}; stress balance and incompressibility are decoupled.

\section{Shallow ice sheets}

Ice sheets have four outstanding properties as fluids.  They are (\emph{i}) slow, (\emph{ii}) shallow,  (\emph{iii}) non-Newtonian, and (\emph{iv}) they experience some contact slip (basal sliding).  The first ice flow model we consider, the non-sliding, isothermal \emph{shallow ice approximation} (SIA), accounts for (\emph{i})--(\emph{iii}).

Regarding the property of shallowness, Figure \ref{fig:green_transect} shows both a no-vertical-exaggeration cross-section of Greenland at $71^\circ$, as well as the standard vertically-exaggerated version which is more familiar in the glaciological literature.  Ice sheets are shallow, though of course the portion of an ice sheet which you want to model may not be.

\onefig{green_transect}{A vertically-exaggerated cross-section of the Greenland ice sheet ($71^\circ$ N) is shown by the green and blue curves.  Without exaggeration it appears as nearly a horizontal line (red).}

Our slab-on-a-slope example gives us a rough explanation of the SIA.  To show the SIA in its plane-flow form, we vertically integrate velocity formula \eqref{uslab} in the $u_0=0$ (non-sliding) case to get
\begin{equation}
\bar u H = \int_0^H \frac{1}{2} A (\rho g \sin\alpha)^3  \left(H^4 - (H-z)^4\right)\,dz = \frac{2}{5} A (\rho g \sin\alpha)^3 H^5. \label{siaubar}
\end{equation}
Note $\sin \alpha \approx \tan\alpha = - h_x$.  Combining these statements with mass continuity \eqref{masscont1D} gives
\begin{equation}
  H_t = M + \left(\frac{2}{5} (\rho g)^3 A H^5 |h_x|^2 h_x\right)_x. \label{sia1D}
\end{equation}
Equation \eqref{sia1D} is the SIA equation for nonsliding plane flow.  It is a model for the evolution of an ice sheet's thickness $H$ in terms of surface mass balance $M$, ice softness $A$, and bed elevation $b$ (because $h=H+b$).

Additional arguments are needed to demonstrate that the SIA is more general-purpose than the special case of a simple slab; see Notes and References.  Such arguments reduce the Stokes equations under the assumption that the surface and bed slopes, and the depth-to-width ratio, are small.

We will numerically solve the SIA in section \ref{sec:numericalsia}, but first we state it in two horizontal dimensions.  Let $\mathbf{U} = (u,v)$ be the vector horizontal velocity.  The shear stress approximation is $(\tau_{13},\tau_{23}) \approx - \rho g (h-z) \nabla h$, which appeared as ``$\tau_{13}= \rho g \sin \alpha (h-z)$ and $\sin \alpha \approx -h_x$'' in the previous section, becomes an equality in the SIA.  Equation \eqref{flowlaw} then gives the SIA formula for shear strain rates
\begin{equation*}
\mathbf{U}_z = 2 A |(\tau_{13},\tau_{23})|^{n-1} (\tau_{13},\tau_{23}) = - 2 A (\rho g)^n (h-z)^n |\nabla h|^{n-1} \nabla h.
\end{equation*}
By integrating vertically we get, in the non-sliding case,
\begin{equation}
\mathbf{U} = - \frac{2 A (\rho g)^n}{n+1} \left[H^{n+1} - (h-z)^{n+1}\right] |\nabla h|^{n-1} \nabla h.  \label{siavelocity}
\end{equation}

Mass continuity in two horizontal dimensions, which generalizes the 1D version \eqref{masscont1D}, also applies:
\begin{equation}
    H_t = M - \Div\left(\bar{\mathbf{U}} H\right)  \label{masscont}
\end{equation}
Equation \eqref{masscont} may be written $H_t = M - \Div \bq$ in terms of the map-plane flux $\bq = \int_{b}^{h} \mathbf{U}\,dz = \bar{\mathbf{U}}\,H$.

Combining Equations \eqref{siavelocity} and \eqref{masscont}, we get an equation for the rate of thickness change in terms of mass balance $M$, thickness, and surface slope $\grad h$:
\begin{equation}
H_t = M + \Div \left(\Gamma H^{n+2} |\grad h|^{n-1} \grad h \right), \label{sia}
\end{equation}
where we have defined the positive constant $\Gamma = 2 A (\rho g)^n / (n+2)$.  Equation \eqref{sia} is the SIA in two dimensions.  Recalling our earlier promise, if we can solve \eqref{sia} numerically then we have, following Mahaffy \cite{Mahaffy}, a usable model for the Barnes ice cap in Canada, a particularly simple ice sheet on a rather flat bed.

\subsection*{Analogy with the heat equation}  The SIA model is easy to compare with the better-known heat equation.  All numerical methods for solving \eqref{sia} can be understood as modifications of well-known heat equation methods.

In the simplest one-dimensional (1D) case, the heat equation for the temperature $T(t,x)$ of a conducting rod is
\begin{equation}
  T_t = D T_{xx}. \label{heat1D}
\end{equation}
This form applies when material properties are constant and there are no heat sources.  The positive constant $D$ is the ``diffusivity,'' with units which can be read from comparing sides of the equation: $D\sim \text{m}^2 \text{s}^{-1}$.  Observe that equation \eqref{heat1D} has a smoothing effect on the solution $T$ as it evolves in time, because any local maximum in the temperature is flattened (i.e.~$T_{xx}<0$ implies $T_t<0$ so $T$ decreases), while any local minimum is also flattened (i.e.~$T_{xx}>0$ implies $T_t>0$ so $T$ increases).

The 2D heat equation, analogous to equation \eqref{sia}, describes the temperature $T(t,x,y)$ at position $x,y$ and time $t$.  Recall that Fourier's law for conduction is the formula $\mathbf{Q} = - \kappa \grad T$ for heat flux $\mathbf{Q}$, where $\kappa$ is conductivity.  We will assume, for the purposes of our analogy, that $\kappa(x,y)$ may vary in space.  Also suppose a variable heat source $f(t,x,y)$, with units of Watts per cubic meter.  Then conservation of internal energy says
\begin{equation}
\rho c T_t = f + \Div (\kappa \grad T). \label{heatearly}
\end{equation}
Here $\rho$ is density and $c$ is specific heat capacity.  Assuming $\rho c$ is constant, define the ``diffusivity'' $D=\kappa/(\rho c)$ and the rescaled source term $F = f/(\rho c)$.  The revised 2D heat equation is
\begin{equation}
T_t = F + \Div (D\, \grad T). \label{heat}
\end{equation}
Figure \ref{fig:initialheat} shows a solution of this heat equation, where the initial condition is a localized ``hot spot''.  Solutions of equation \eqref{heat} always involve the spreading, in all directions, of any local heat maxima or minima, that is, diffusion.

\twofigsizes{initialheat}{finalheat}{A solution of heat equation \eqref{heat} with $D=1$ and $F=0$.  Left: Initial condition $T(0,x,y)$.   Right: Solution $T(t,x,y)$ at $t=0.02$.}{2.8in}{2.8in}

The SIA equation \eqref{sia} and the heat equation \eqref{heat} are each diffusive, time-evolving partial differential equations (PDEs).  A side-by-side comparison is illuminating:
\begin{center}
\begin{tabular}{cc}
\vspace{1mm}
SIA:\, $H$ is ice thickness & \phantom{foo bar} heat: $T$ is temperature\phantom{foo bar}  \\
\vspace{1mm}
	$H_t = M + \Div \left({\color{red}\Gamma H^{n+2} |\grad h|^{n-1}}\, \grad h \right)$  &  $T_t = F + \Div (D\, \grad T)$
\end{tabular}
\end{center}
\vspace{1mm}
Notice that the number of derivatives (one time and two space derivatives) and the signs are the same.  Surface mass balance $M$ is analogous to heat source $F$.  

The analogy suggests that we identify the \emph{diffusivity in the SIA} as:
	$$D = {\color{red}\Gamma H^{n+2} |\grad h|^{n-1}}.$$
A non-sliding SIA flow diffuses the thickness of the ice sheet.  When this $D$, a product of $\Gamma$ and the powers of $H$ and $|\grad h|$, comes out large then the diffusion acts most quickly.

This diffusion equation analogy explains generally why the surfaces of ice sheets are smooth, at least if we overlook non-flow processes like crevassing and wind-driven (snow) dunes.  There are, however, some issues with the analogy:
\begin{itemize}
\item The diffusivity $D$ depends on the solution, both the thickness $H$ and surface elevation $h$.
\item The diffusivity $D$ goes to zero at margins, where $H\to 0$, and at divides and domes, where $|\grad h|\to 0$.
\end{itemize}
More important is a deficiency of the SIA model and not \emph{per se} the analogy, namely
\begin{itemize}
\item Ice flow is much less diffusive when longitudinal (membrane) stresses are important, as when ice is floating or sliding or when the flow is confined by terrain.
\end{itemize}
But we will continue with the SIA, working toward a verified numerical scheme for \eqref{sia} in Section \ref{sec:numericalsia}.

\section{Finite difference numerics} 

Numerical schemes for the heat equation are a good starting place for solving the SIA equation \eqref{sia}.  Here we demonstrate only finite difference (FD) schemes.  These schemes replace derivatives in a differential equation by mere arithmetic.

The basic fact on which FD schemes are based is \emph{Taylor's theorem}, which says that for a smooth function $f(x)$,
	$$f(x+\Delta) = f(x) + f'(x) \Delta + \frac{1}{2} f''(x) \Delta^2 + \frac{1}{3!} f'''(x) \Delta^3 + \dots$$
You can replace ``$\Delta$'' by its multiples, for example:
\begin{align*}
f(x+2\Delta) &= f(x) + 2 f'(x) \Delta + 2 f''(x) \Delta^2 + \frac{4}{3} f'''(x) \Delta^3 + \dots \\
f(x-\Delta) &= f(x) - f'(x) \Delta + \frac{1}{2} f''(x) \Delta^2 - \frac{1}{3!} f'''(x) \Delta^3 + \dots
\end{align*}
The idea for constructing FD schemes is to combine expressions like these to give approximations of derivatives.  Thereby function values on a grid combine to approximate the differential equation.

Here we want partial derivative approximations, so we apply the Taylor's expansions one variable at a time.  For example, with a general function $u=u(t,x)$,
\begin{align*}
u_t(t,x) &= \frac{u(t+\Delta t,x) - u(t,x)}{\Delta t} + O(\Delta t), \\
u_t(t,x) &= \frac{u(t+\Delta t,x) - u(t-\Delta t,x)}{2\Delta t} + O((\Delta t)^2), \\
u_x(t,x) &= \frac{u(t,x+\Delta x) - u(t,x-\Delta x)}{2\Delta x} + O((\Delta x)^2), \\
u_{xx}(t,x) &= \frac{u(t,x+\Delta x) - 2\, u(t,x) + u(t,x-\Delta x)}{\Delta x^2} + O((\Delta x)^2)
\end{align*}
Note that if $\Delta$ is a small number then ``$+O(\Delta^2)$'' is smaller than ``$+O(\Delta)$'', so the approximation is closer when you drop it.

\subsection*{Explicit scheme for the heat equation}  We can build the simplest ``explicit'' scheme which approximates the 1D heat equation \eqref{heat1D} by observing that these two FD expressions are nearly equal:
\begin{equation}
\frac{T(t+\Delta t,x) - T(t,x)}{\Delta t} \approx D\,\frac{T(t,x+\Delta x) - 2\, T(t,x) + T(t,x-\Delta x)}{\Delta x^2}.  \label{heat1Dapproximated}
\end{equation}
The FD scheme itself is not just an approximation of a PDE, like \eqref{heat1Dapproximated}, but an actual method for computing numbers on a grid.  Let $(t_n,x_j)$ denote the time-space grid points.  Denote our approximation of the solution value $T(t_n,x_j)$ by $T_j^n$.  Then the finite difference scheme is
	$$\frac{T_j^{n+1} - T_j^n}{\Delta t} = D\,\frac{T_{j+1}^n - 2\, T_j^n + T_{j-1}^n}{\Delta x^2}.$$
To get a computable formula, let $\mu = D \Delta t / (\Delta x)^2$ and solve for $T_j^{n+1}$:
\begin{equation}
  T_j^{n+1} = \mu T_{j+1}^n + (1 - 2 \mu) T_j^n + \mu T_{j-1}^n \label{heat1Dfd}
\end{equation}

FD scheme \eqref{heat1Dfd} is \emph{explicit} because it directly computes $T_j^{n+1}$ in terms of values at time $t_n$.  Figure \ref{fig:expstencil} shows the ``stencil'' for scheme \eqref{heat1Dfd}: three values at the current time $t_n$ are combined to update the one value at the next time $t_{n+1}$.

Before moving on, notice that evaluating a heat equation solution at a grid point (i.e.~the expression ``$T(t_n,x_j)$'') generally gives a different value from the value $T_j^n$ computed by a scheme like \eqref{heat1Dfd}.  Of course we plan that these numbers will be close, but that needs checking (``verification'') or an \emph{a priori} proof.  Specifically we intend that the numbers $T(t_n,x_j)$ and $T_j^n$ become close to each other when the grid is made finer (i.e.~$\Delta t \to 0$ and $\Delta x \to 0$), as the FD expressions become closer to the derivatives they approximate.  That is, we intend our FD scheme to \emph{converge} under \emph{grid refinement}.

How accurate is scheme \eqref{heat1Dfd}?  Its construction tells us that the difference between the scheme \eqref{heat1Dfd} and the PDE \eqref{heat1D} is $O(\Delta t + (\Delta x)^2)$, so this difference goes to zero as we refine the grid in space and time, a property called \emph{consistency}.  With care about the smoothness of boundary conditions, and using mathematical facts about the heat equation itself, one can show that the difference between $T_j^n$ and $T(t_n,x_j)$ is also $O(\Delta t + (\Delta x)^2)$, which is thus the \emph{convergence rate}; see Notes and References.

To get convergence the PDE problem must generate adequately smooth solutions, and also scheme \eqref{heat1Dfd} must be \emph{stable}, which we address below.  (The main theorem for numerical PDE schemes is ``consistency plus stability implies convergence''; see Notes and References.)  In these notes we do something rather practical, namely verification.  We find problems for which we already know an exact solution $T(t,x)$, and then we compute the differences $|T_j^n - T(t_n,x_j)|$.  This determines directly whether the \emph{implementation} (i.e.~computer code form) of our FD scheme actually converges, not just in theory.

\subsection*{A first implemented scheme}  Our first Matlab implementation we consider the two spatial dimension Equation \eqref{heat} with $D$ constant and $F=0$:
\begin{equation}
T_t = D (T_{xx}+T_{yy}).\label{heat2D}
\end{equation}
Writing $T_{jk}^n \approx T(t_n,x_j,y_k)$, the 2D explicit scheme is
\begin{equation}
	\frac{T_{jk}^{n+1} - T_{jk}^n}{\Delta t} = D\,\left(\frac{T_{j+1,k}^n - 2\, T_{jk}^n + T_{j-1,k}^n}{\Delta x^2} + \frac{T_{j,k+1}^n - 2\, T_{jk}^n + T_{j,k-1}^n}{\Delta y^2}\right). \label{heat2dexplicit}
\end{equation}
The stencil for the right-hand side of \eqref{heat2dexplicit} is in Figure \ref{fig:expstencil}.

\twofigsizes{expstencil}{exp2dstencil}{Left: Space-time stencil for the explicit scheme \eqref{heat1Dfd} for the 1D heat equation.  Right: Spatial-only stencil for scheme \eqref{heat2dexplicit}.}{2.0in}{2.1in}

Scheme \eqref{heat2dexplicit} has implementation \texttt{heat.m} below.  For simplicity we set $T=0$ on the boundary of the square $-1 < x < 1$, $-1 < y < 1$.  The initial condition is gaussian: $T(0,x,y) = \exp(-30 (x^2+y^2))$.  The code uses Matlab ``colon'' notation to remove loops over spatial variables.  Here is an example run:
\begin{Verbatim}
>>  heat(1.0,30,30,0.001,20);
\end{Verbatim}
This sets $D=1.0$ and uses a $30\times 30$ spatial grid.  We take $N=20$ time steps of $\Delta t = 0.001$.  The result is shown in Figure \ref{fig:initialheat}, right.  This is the look of success.

\minput{heat}

However, very similar runs seem to succeed or fail according to some yet unclear circumstance.  For example, results from these calls are shown in Figure \ref{fig:stability}:
\begin{Verbatim}
>> heat(1.0,40,40,0.0005,100);    % Figure 7, left
>> heat(1.0,40,40,0.001,50);      % Figure 7, right
\end{Verbatim}
Both runs compute temperature $T$ on the same spatial grid, for the same final time $t_f = N \Delta t = 0.05$, but with different time steps.  The second run clearly shows instability.

\twofig{stability}{instability}{Numerically-computed temperature on $40\times 40$ grids.  The two runs are the same except that the left has $\Delta t=0.0005$ so $D\Delta t/(\Delta x)^2= 0.2$, while the right has $\Delta t=0.001$ so $D\Delta t/(\Delta x)^2= 0.4$.  Compare \eqref{stabcrit}.}

%>> heat(1.0,40,40,0.001,50);
%  doing N = 50 steps of dt = 0.00100 for 0.0 < t < 0.050
%  nu = 1 * dt / dx^2 = 0.40000
%>> print -dpdf instability.pdf
%>> heat(1.0,40,40,0.0005,100);
%  doing N = 100 steps of dt = 0.00050 for 0.0 < t < 0.050
%  nu = 1 * dt / dx^2 = 0.20000
%>> print -dpdf stability.pdf

\subsection*{Stability criteria and adaptive time stepping}  To avoid the instability shown at right in Figure \ref{fig:stability}, we need to understand the scheme better.  It turns out we have not made an implementation error, but more care is required when choosing space and time steps.

Recall the 1D explicit scheme in form \eqref{heat1Dfd}: $T_j^{n+1} = \mu T_{j+1}^n + (1 - 2 \mu) T_j^n + \mu T_{j-1}^n$.  The new value $T_j^{n+1}$ is an average of the old values, in the sense that the coefficients add to one.  Averaging is stable because averaged wiggles are smaller than the wiggles themselves.  Actually, however, the scheme is only an average \emph{if} the middle coefficient is positive, as a linear combination with coefficients which add to one is not an average if any coefficients are negative.  (For example, we would not accept 15 as an ``average'' of 5 and 7, but we can write $15 = -4 \times 5 + 5 \times 7$, and $-4+5=1$.)

So, what follows from requiring the middle coefficient in \eqref{heat1Dfd} to be positive so it computes such an average?  A \emph{stability criterion} follows, with these equivalent forms:
\begin{equation}
   1 - 2 \mu \ge 0 \quad \iff \quad \frac{D\Delta t}{\Delta x^2} \le \frac{1}{2} \quad \iff \quad \Delta t \le \frac{\Delta x^2}{2 D}.  \label{stabcrit}
\end{equation}
This condition on the size of $\Delta t$ is a \emph{sufficient} stability criterion; it is enough to guarantee stability, though something weaker might do.  In summary, for given $\Delta x$, shortening the time steps $\Delta t$ so that \eqref{stabcrit} holds will make FD scheme \eqref{heat1Dfd} into an averaging process.

Applying this same idea to the 2D heat equation \eqref{heat2D} leads to the stability condition that $1-2\mu^x-2\mu^y \ge 0$ where $\mu^x = D \Delta t / (\Delta x^2)$ and $\mu^y = D \Delta t / (\Delta y^2)$.  In the cases like those shown in Figure \ref{fig:stability} with $\Delta x=\Delta y$, this condition requires $D \Delta t /(\Delta x^2) \le 0.25$, which precisely distinguishes between the two parts of the Figure.  Runs of \texttt{heat.m} are unstable if the time step $\Delta t$ is too big relative to the spacing $\Delta x$.

The stability criterion above is easily satisfied by making each time step shorter.  Doing so at each time step makes an \emph{adaptive} implementation which can be stable even if the diffusivity $D$ is changing in time.  To show how easy this is to implement, \texttt{heatadapt.m} (not shown) is the same as \texttt{heat.m} except that the time step comes from the stability criterion, so it cannot generate the instability seen in Figure \ref{fig:stability}.  However, if the diffusivity $D$ is large or the grid spacings $\Delta x$, $\Delta y$ are small, then adaptive explicit implementations must take many short time steps to assure stability.


\subsection*{Implicit schemes}  There is an alternative stability fix instead of adaptivity, namely ``implicitness.''  For example, the finite difference scheme
\begin{equation}
  \frac{T_j^{n+1} - T_j^n}{\Delta t} = D\,\frac{T_{j+1}^{n+1} - 2\, T_j^{n+1} + T_{j-1}^{n+1}}{\Delta x^2} \label{implicit1D}
\end{equation}
is an $O(\Delta t + (\Delta x)^2)$ implicit scheme for Equation \eqref{heat1D}.  Such implicit schemes for the heat equation are stable for \emph{any} positive time step $\Delta t>0$ (``unconditionally stable''); see Notes and References.  Another well-known implicit scheme is \emph{Crank-Nicolson}, which is unconditionally stable for the heat equation, but with smaller error $O((\Delta t)^2 +(\Delta x)^2)$.

Implicit schemes are harder to implement because the unknown solution values at time step $t_{n+1}$ are treated as a vector in a large system of equations which must be formed and solved at each time step.  If the PDE is nonlinear then the system of equations may be hard to solve.  Of course, the SIA is a highly nonlinear diffusion equation.

Generally, there is a tradeoff between the easy implementability of adaptive explicit schemes and the better stability of implicit schemes.  For these notes we stay with adaptive explicit FD schemes.

\subsection*{Numerical solution of diffusion equations}  We are trying to numerically model ice flows, not heat conduction.  We have an analogy, however, which says that the SIA is diffusive like the heat equation.  In this section, because we wish to solve the SIA on real bedrock, we construct a numerical scheme for a more general diffusion equation which has an extra ``shift'' inside the gradient, namely
\begin{equation}
  T_t = F + \Div \left(D \grad (T + b)\right). \label{gendiffusion}
\end{equation}
In equation \eqref{gendiffusion}, the source term $F(x,y)$, the diffusivity $D(x,y)$, and the ``shift'' $b(x,y)$ may all vary in space.

The following code solves \eqref{gendiffusion}.  It is called by the SIA-specific schemes we build next.

\minput{diffusion}

This adaptive explicit method for the diffusion equation is conditionally stable, with the same essential time step restriction as for the constant diffusivity case, as long as we evaluate $D(x,y)$ at \emph{staggered} grid points.  That is, we use this expression for the second derivative:
\begin{align*}
\Div \left(D \grad X\right) &\approx \frac{D_{j+1/2,k}(X_{j+1,k} - X_{j,k}) - D_{j-1/2,k}(X_{j,k} - X_{j-1,k})}{\Delta x^2} \\
	&\qquad + \frac{D_{j,k+1/2}(X_{j,k+1} - X_{j,k}) - D_{j,k-1/2}(X_{j,k} - X_{j,k-1})}{\Delta y^2},
\end{align*}
where $X=T+b$.  The left part of Figure \ref{fig:diffstencil} shows the stencil.

\twofigsizes{diffstencil}{mahaffystencil}{Left:  Spatial stencil for staggered grid evaluation of diffusivity (at triangles) in the diffusion equation \eqref{gendiffusion}.  Right: Stencil showing how the staggered-grid diffusivity (triangle) can be evaluated in the SIA, from surface elevation (diamonds) and thicknesses (squares).}{2.2in}{2.2in}

The user supplies the diffusivity $D(x,y)$ to \texttt{diffusion.m} on the staggered grid.  The initial temperature $T(0,x,y)$, source term $F(x,y)$, and ``shift'' $b(x,y)$ are also supplied on the regular grid.  When using this code for standard diffusions, or for the flat-bed case of the SIA, we would take $b=0$.


\section{Numerically solving the SIA} \label{sec:numericalsia}

In SIA equation \eqref{sia} we have diffusivity $D = \Gamma H^{n+2} |\grad h|^{n-1}$.  There are two interesting aspects of this formula.  First, as already noted, $D$ goes to zero, i.e.~it ``degenerates,'' when either $H\to 0$ or $\grad h \to 0$.  Degenerate diffusion equations are automatically free boundary problems, but this aspect of the thickness evolution problem is no surprise to a glaciologist.  Determining the location of the margin is an obvious part of modelling a glacier or ice sheet.  To address this free boundary issue in our explicit time-stepping code it suffices to numerically compute new thicknesses and then set them to zero if they come out negative.

Second, for numerical stability and mass conservation we should compute $D$ on a ``staggered'' grid.  Various finite difference schemes for computing it have been proposed.  All of these schemes involve averaging $H$ and differencing $h$ in a ``balanced'' way onto the staggered grid.  In the code \texttt{siaflat.m} below we use the Mahaffy \cite{Mahaffy} method, with the stencil for computing $D$ shown in Figure \ref{fig:diffstencil}.  This code only works for the flat bed, zero surface mass balance case, but we will correct these deficiencies later.

\minput{siaflat}


\section{Exact solutions and verification}

In \texttt{siaflat.m}, which calls \texttt{diffusion.m}, we already have a fairly complicated code.  How do we make sure that such an implemented numerical scheme is correct?  Here are three proposed techniques:
\begin{enumerate}
  \item don't make any mistakes, or
  \item compare your numerical results with results from other researchers, and hope that the outliers are in error, or
  \item compare your numerical results to an exact solution.   \end{enumerate}
The last one of these, which we prefer to the first two when possible, is called ``verification.''  That is, when we build a new computer code we should test it in cases where we know the right answer.  To do so we need to return to the PDE itself, to get useful exact solutions.

\subsection*{Exact solution of heat equation}  First we consider the simpler case of the 1D heat equation with constant $D$, namely $T_t = D T_{xx}$.  Many exact solutions $T(t,x)$ to this heat equation are known, but let's consider the time-dependent ``Green's function,'' also known as the ``heat kernel''.  It starts at time $t=0$ with a delta function $T(0,x)=\delta(x)$ of heat at the origin $x=0$.  Then it spreads out over time.  It is a solution of the heat equation on the whole line $-\infty<x<\infty$ and for all $t>0$.

We will calculate this exact solution by a method which generalizes to the SIA.  The Green's function of the heat equation is ``self-similar'' over time, in the sense that it changes shape \emph{only} by shrinking the output (vertical) axis and lengthening the input (horizontal) axis, as shown in Figure \ref{fig:heatscaling}.  These scalings are related to each other by conservation of energy, which says that the total heat energy is independent of time.

\onefigsize{heatscaling}{The heat equation Green's function in 1D has the same shape at each time, but with time-dependent input- and output-scalings.}{2.4in}

In particular, the Green's function of the 1D heat equation is
  $$T(t,x) = (4 \pi D t)^{-1/2}\, e^{-x^2/(4Dt)}.$$
``Similarity'' variables for this solution, the above-mentioned scalings, involving multiplying the input and output of an invariant shape function $\phi(s) = (4 \pi D)^{-1/2}\, e^{-s^2/(4D)}$ by the same power of $t$:
\begin{equation}
s \stackrel{\text{\emph{input scaling}}}{\phantom{\Big|}=\phantom{\Big|}} t^{-1/2} x, \qquad\qquad T(t,x) \stackrel{\text{\emph{output scaling}}}{\phantom{\Big|}=\phantom{\Big|}} t^{-1/2} \phi(s).  \label{heatscalings}
\end{equation}
Note that all time-dependence is from the input and output scalings.

A numerical solver for the 1D heat equation which starts with initial values $T(t_0,x)$ taken from this exact solution should, at a later time $t$, produce numbers which are close to the exact solution $T(t,x)$; see the Exercises.

\subsection*{Halfar's similarity solution to the SIA}  Now we jump from Green's idea in about 1830 to the year 1981.  That is when P.~Halfar published the similarity solution of the SIA in the case of flat bed and zero surface mass balance.  Halfar's solution to the 2D SIA model \eqref{sia}, using a Glen exponent $n=3$, has scalings which are powers of $t$ like \eqref{heatscalings} above:
\begin{equation}
s = t^{-1/18} r, \qquad \qquad H(t,r)=t^{-1/9} \phi(s). \label{halfarscalings}
\end{equation}
Here $r=(x^2+y^2)^{1/2}$ is the distance from the origin.  These scalings are related to each other by conservation of mass, because no mass is gained or lost through the surface. Scalings \eqref{halfarscalings} imply that, quite differently from heat, the diffusion of ice slows down severely as the shape flattens out.  The powers $t^{-1/9}$ and $t^{-1/18}$ change very slowly for large times $t$.

\onefigsize{siascaling}{A case of Halfar's solution \eqref{halfar} of the SIA equation \eqref{sia} on a flat bed with zero mass balance.  The solution is shown on $H$ (m) versus $r$ (km) axes for times $t=1,10,100,1000,10000$ years.}{5.5in}

The formula for the Halfar solution to the SIA is remarkably simple given all that it accomplishes:
\begin{equation}
H(t,r) = H_0 \left(\frac{t_0}{t}\right)^{1/9} \left[1 - \left(\left(\frac{t_0}{t}\right)^{1/18} \frac{r}{R_0}\right)^{4/3}\right]^{3/7}. \label{halfar}
\end{equation}
Here the ``characteristic time'' $t_0 = (18 \Gamma)^{-1} (7/4)^3 R_0^4 H_0^{-7}$ is a parameter which can be determined by choosing center height $H_0$ and radius $R_0$.

Formula \eqref{halfar} is plotted in Figure \ref{fig:siascaling}.  We see that for times significantly greater than $t_0$ (i.e.~$t/t_0 \gg 1$) the solution changes very slowly.  For example, the change between years $1$ and $100$ is larger than that between years $1000$ and $10000$.  The \emph{volume} of ice in this Halfar ice cap is, however, constant with $t$; see the Exercises.

\subsection*{Using Halfar's solution}  Formula \eqref{halfar} is simple enough to use for verifying time-dependent SIA models.  The code \texttt{verifysia.m} (not shown) takes as input the number of grid points in each ($x,y$) direction.  It uses the Halfar solution at 200 a as the initial condition, does a numerical run of \texttt{siaflat.m} above to a final time 20000 a, and then compares to the Halfar formula for that time.  By ``compares'' we mean it computes the thickness error, the absolute values of the differences between the numerical and exact thickness solutions at the final time:
\small
\begin{verbatim}
>> verifysia(20)
average thickness error     = 22.310
>> verifysia(40)
average thickness error     = 9.490
>> verifysia(80)
average thickness error     = 2.800
>> verifysia(160)
average thickness error     = 1.059
\end{verbatim}
\normalsize
We see that the average thickness error decreases with increasing grid resolution.  This is as expected for a correctly-implemented code.  What is less obvious, perhaps, is that almost any numerical implementation mistake---almost any bug---will break this property, and these errors will not shrink.

You might ask, is the Halfar solution ever useful for modelling real ice masses?  The answer is yes.  In fact, J.~Nye and others (2000; \cite{NyeIcarus2000}) compared the long-time consequences of different flow laws for the south polar cap on Mars.  In particular, they evaluated $\text{CO}_2$ ice and $\text{H}_2\text{O}$ ice softness parameters by comparing the long-time behavior of the corresponding Halfar solutions to the observed polar cap properties.  Their conclusions:
  \begin{quote}
  \dots none of the three possible [$\text{CO}_2$] flow laws will allow a 3000-m cap, the thickness suggested by stereogrammetry, to survive for $10^7$ years, indicating that the south polar ice cap is probably not composed of pure $\text{CO}_2$ ice [but rather] water ice, with an unknown admixture of dust.
  \end{quote}
This theoretical result has since been confirmed by the observation and sampling of the polar geology of Mars.

Are exact solutions like Halfar's always available when needed?  The answer is ``no'', of course, though many ice flow models do have exact solutions which are relevant to verification; see the Notes and References.  For example, we will use van der Veen's solution for ice shelves in a later section.  On the other hand, the absence of exact solutions may show that not enough thought has gone into the continuum model itself.

\subsection*{A test of robustness}  Verification is an ideal way to start testing a code.  Another kind of test is for ``robustness''.  One asks: Does the model break when you ask it to do hard things?  Unlike for verification, we might not have precise knowledge of what it should do, but a well-implemented model should act in a ``reasonable'' way.

The robustness test in the program \texttt{roughice.m} (not shown) demonstrates that \texttt{siaflat.m} can handle an ice sheet with extraordinarily large ``driving stresses.''  Recall that glaciological driving stress is $\tau_d = - \rho g H \grad h$.  This quantity appears in the slab-on-a-slope example, and thus in the SIA model, as the value of the shear stress $(\tau_{13},\tau_{23})$ at the base of the ice.  The driving stress is, obviously, large when the ice is both thick and has steep surface slope $|\nabla h|$.

In \texttt{roughice.m} we give \texttt{siaflat.m} a randomly-generated initial ice sheet which is of the worst possible sort because it is both thick (average of 3000 m) and it has large surface slopes.  The initial shape is shown in the left side of Figure \ref{fig:roughinitial}.  During the run of 50 model years, the time step is determined adaptively from \eqref{stabcrit}, increasing from 0.0002 years to about 0.2 years as the maximum diffusivity $D$ decreases correspondingly.  The maximum value of the driving stress decreases from $57$ bar ($= 5.7\times 10^6$ Pa) to $3.6$ bar.  At the end the ice cap has the shape shown at right in Figure \ref{fig:roughinitial}.

\twofig{roughinitial}{roughfinal}{The SIA model evolves the huge-driving-stress initial ice sheet at left to the ice cap at right in only 50 model years.}

The shape at right in Figure \ref{fig:roughinitial} is rather close to a Halfar solution.  Indeed Halfar proved that all solutions of the zero-mass-balance SIA on a flat bed asymptotically approach the Halfar solution.


\section{Applying our numerical ice sheet model}

Finally we apply the model to the Antarctic ice sheet.  To do this we must first modify \texttt{siaflat.m} to allow non-flat bedrock elevation $b(x,y)$ and arbitrary surface mass balance $M(x,y)$.  Also we calve floating ice, and we enforce non-negative thickness at each timestep.  The result is \texttt{siageneral.m} (not shown), a code only ten lines longer than \texttt{siaflat.m}.

\twofigsizes{antinitial}{antfinal}{Left: Initial surface elevation (m) of Antarctic ice sheet.  Right: Final surface elevation at end of 40 ka model run on 50 km grid.}{2.55in}{3.2in}

We use measured accumulation, bedrock elevation, and surface elevation from ALBMAPv1 data \cite{LeBrocqetal2010}.  Melt is not modelled so the surface mass balance is the accumulation rate.  These input data are read from a NetCDF file and preprocessed by an additional code \texttt{buildant.m} (not shown).

\onefig{antvolcompare}{Ice volume of the modeled Antarctic ice sheet, in units of $10^6 \, \text{km}^3$, from runs on 50 km (red), 25 km (green), and 20 km (blue) grids.}

The code \texttt{ant.m} (not shown) calls \texttt{siageneral.m} to do the simulation in blocks of 500 model years.  The volume is computed at the end of each block.  Figure \ref{fig:antinitial} shows the initial and final surface elevations from a run of 40,000 model years on a $\Delta x = \Delta y = 50$ km grid.  The runtime on a typical laptop is a few minutes.

Areas of the Antarctic ice sheet with low-slope and (actual) fast-flowing ice experience thickening in the model, while near-divide ice in East Antarctica, in particular, thins.  Assuming the present-day Antarctic ice sheet is near steady state, these most-obvious thickness differences reflect model inadequacies.  The lack of a sliding mechanism explains the thickening in low-slope areas.  The lack of thermomechanical coupling, or equivalently the constancy of ice softness, explains the thinning near the divide.  And of course we should be modeling floating ice too, but the SIA is completely inappropriate to that purpose.  See section \ref{sec:shelvesandstreams} and Notes and References on modelling techniques which address these inadequacies.

Figure \ref{fig:antvolcompare} compares the ice volume time series for 50 km, 25 km, and 20 km grids.  This result, namely grid dependence of the ice volume, is typical.  One cause is that most steep gradients near the ice margin are poorly resolved, and this is true to differing degrees at these coarse resolutions.  Mainly Figure \ref{fig:antvolcompare} is a warning about the interpretation of model runs:  Even if the data is available only on a fixed grid, the model should be run at different resolutions to evaluate the robustness of the model results.


\section{Interlude: Mass continuity and kinematical equations}

Recall that in the SIA the ``stress balance'' is essentially formula \eqref{siavelocity} for the velocity.  It combines with the mass continuity equation \eqref{masscont} to give model \eqref{sia} for the ice sheet thickness.  The major SIA equation \eqref{sia} thus combines two concepts which we will now think about separately, and in greater generality, in the remainder of these notes.

The basic shallow assumption made by most ice flow theories\footnote{There are several inequivalent shallow theories: SIA, SSA, hybrids, Blatter-Pattyn, \dots} is that the surface and base of the ice are differentiable functions $z=h(t,x,y)$ and $z=b(t,x,y)$.  Thus surface overhang is not allowed, though, by contrast, the Stokes theory of slow viscous fluids only needs a closed surface in three-dimensional space as a boundary surface for the fluid.  Most ice sheet and glacier models take a map-plane perspective, however, and they have a well-defined ice thickness: $H=h-b$.

To pursue such ideas a bit further, let us state the ``kinematical equations'' which apply at upper and lower surfaces of the ice sheet.  Let $a$ be the upper surface (climatic) mass balance function ($a>0$ is accumulation) and $s$ be the basal melt rate function ($s>0$ is basal melting).  In the equations which follow these are measured in thickness-per-time units, but they could be in mass-per-area-per-time units also.  The net map-plane mass balance $M=a-s$, which already appears in the mass continuity equation \eqref{masscont}, is the difference of these surface fluxes.

The \emph{(upper) surface kinematical equation} is 
\begin{equation}
h_t = a - \mathbf{U}\big|_h \cdot \grad h + w\big|_h,  \label{surfkine}
\end{equation}
and the \emph{base kinematical equation} is
\begin{equation}
b_t = s - \mathbf{U}\big|_b \cdot \grad b + w\big|_b.  \label{basekine}
\end{equation}
(Recall $\mathbf{U}$ is the horizontal ice velocity and $w$ the vertical ice velocity.)  Equations \eqref{surfkine} and \eqref{basekine} describe the movement of the ice's upper surface and lower surfaces, respectively, from the velocity of the ice and the mass balance functions at the respective surfaces.

We can now state an important mathematical fact which follows merely from the assumption of well-defined upper and basal surface elevations.  Namely, that the surface kinematical and mass continuity equations are closely-related.  More precisely, any pair of these equations implies the third:
  \begin{itemize}
  \item the surface kinematical equation \eqref{surfkine},
  \item the base kinematical equation \eqref{basekine}, and
  \item the map-plane mass continuity equation \eqref{masscont}.
  \end{itemize}
One proves these facts by using the incompressibility of ice \eqref{incompressible} and the Leibniz rule for differentiating integrals.  The details are left for exercises.

The bedrock is often regarded as fixed (i.e.~$b_t=0$), and in fact the basal kinematical equation is often not explicitly mentioned.  Instead one gets a simplified view.  In the case of non-deformable bedrock and no sliding, for example, the basal value of the vertical velocity equals the basal melt rate.  This simplification corresponds to $b_t=0$ and $\mathbf{U}\big|_b=0$ so that $w\big|_b=-s$ from \eqref{basekine}.

\subsection*{Prognostic models}  We can now sketch the structure of a general ``prognostic,'' i.e.~ice geometry evolving, isothermal ice sheet model.  Each time step follows this recipe:
  \begin{itemize}
  \item numerically solve a stress balance, which gives velocity $\mathbf{u}=(u,v,w)$,
    \begin{itemize}
    \item[$\circ$] if the stress balance only gives $\mathbf{U}=(u,v)$, get $w$ from incompressibility \eqref{incompressible},
    \end{itemize}
  \item decide on a time step $\Delta t$ for \eqref{masscont} based on velocities and/or diffusivities,
  \item from the horizontal velocity $\mathbf{U}=(u,v)$, compute the flux $\bq = \bar{\bU} H$,
  \item update mass balance $M=a-s$ and do a time-step of \eqref{masscont} to get $H_t$,
  \item update the upper surface elevation and thickness (e.g.~$h \mapsto h + H_t \Delta t$), and repeat.
  \end{itemize}
Like most ice sheet models, we use the mass continuity equation \eqref{masscont} to describe changes in ice sheet geometry, but we could use the surface kinematical equation \eqref{surfkine} instead.

The above ``standard'' ice sheet model has many variations.  Some glaciological questions are answered just by solving the stress balance for the velocity.  Sometimes the goal is the steady state configuration of the glacier, which might be computed more quickly by iteratively solving steady state equations than by time-stepping physical evolution equations to steady state.  Other processes are usually simulated at each time step, such as the conservation of energy within the ice, or subglacial and supraglacial processes.  Understanding the diverse time scales associated to these processes is usually an important step in designing the coupled model.

When using the SIA equation \eqref{sia}, one can seemingly bypass the computation of the velocity.  That is because we could write the mass continuity equation as a diffusion, with $\bq=-D\nabla h$ for the flux instead of the more general $\bq = \bar{\bU} H$.  Fast flow in ice sheets is associated with sliding and floating ice, however, and for these flows the ice geometry evolution is not a diffusion, and so only ``$\bq = \bar{\bU} H$'' applies.  Solving the stress balance for the velocity field is then an obligatory, and usually nontrivial, step.  We consider such a stress balance next.


\section{Shelves and streams} \label{sec:shelvesandstreams}

The shallow shelf approximation (SSA) stress balance applies to ice shelves as its name suggests.  The SSA also applies reasonably well to ice streams, like those in Figure \ref{fig:siple} which have not-too-steep bed topography and low basal resistance.

\twofigsizes{siple}{streamisbrae}{Left:  The SSA model applies to ice streams like these on the Siple Coast in Antarctica.  Color shows radar-derived surface speed.  Right: Cross sections, \emph{without} vertical exaggeration, of the Jakobshavns Isbrae outlet glacier in Greenland (\textbf{a}) and the Whillans Ice Stream on the Siple Coast (\textbf{b}); this is Figure 1 in \cite{TrufferEchelmeyer}.}{2.8in}{2.9in}

But what is, and is not, an ice stream?  Ice streams slide at $50$ to $1000 \,\text{m}\,\text{a}^{-1}$, they have a concentration of vertical shear in a thin layer near base, and typically they flow into ice shelves.  Pressurized liquid water at their beds plays a critical role enabling their fast flow.  There are other fast-flowing grounded parts of ice sheets, however, called ``outlet glaciers''.  They can have even faster surface speed (up to $10 \,\text{km}\,\text{a}^{-1}$), but it is typically uncertain how much of this speed is from sliding at the base.  In an outlet glacier there is substantial vertical shear ``up'' in the ice column, sometimes caused by soft temperate ice in a significant fraction of the thickness.  Furthermore, outlet glaciers are strongly controlled by fjord-like, high slope bedrock topography.  Figure \ref{fig:siple} (right) compares the shallowness and bedrock topography of an outlet glacier and an ice stream.  Thus, few simplifying assumptions are appropriate for outlet glaciers, and the SSA may not be a sufficient model.

\subsection*{The shallow shelf approximation (SSA)}  We state this stress balance equation only in the plane flow (``flow-line'') case:
\begin{equation}
  \left(2 B H |u_x|^{1/n - 1} u_x\right)_x - C|u|^{m-1}u = \rho g H h_x \label{ssaearly}
\end{equation}
The term in parentheses is the vertically-integrated longitudinal stress, also called the ``membrane'' stress when there are two horizontal variables.  The second term $\tau_b = - C|u|^{m-1}u$ is the basal resistance, which is zero (i.e.~$C=0$) in an ice shelf.  The term on the right is the driving stress ($\tau_d = - \rho g H h_x$).  Thus the SSA equation is a balance wherein longitudinal strain rates are determined by the integrated ice hardness (i.e.~the coefficient $BH$), the slipperyness of the bed (i.e.~by the coefficient $C$ and the power $m$) and the geometry of the ice sheet (i.e.~the thickness $H$ and the surface slope $h_x$).

In \eqref{ssaearly} the velocity $u$ is independent of the vertical coordinate $z$.  We assume that the ice hardness $B=A^{-1/n}$ is also independent of depth.  Models which are not isothermal compute the vertical average of the temperature-dependent hardness.  The formula for the basal resistance $\tau_b$ is often called a ``sliding law'' in power law form.

The coefficient $\bar \nu = B |u_x|^{1/n-1}$ in \eqref{ssaearly} is called the ``effective viscosity'', so that \eqref{ssaearly} can be written
\begin{equation}
  \left(2 \,\bar \nu\, H u_x\right)_x - C |u|^{m-1} u = \rho g H h_x.  \label{ssa}
\end{equation}
In form \eqref{ssa} it is understood that the viscosity $\bar\nu$ depends on the velocity solution $u$.

The inequality ``$\,\rho H < - \rho_w b\,$'' is sometimes called the \emph{flotation criterion}.  For grounded ice we know $\rho H > - \rho_w b$ and the driving stress $\tau_d = - \rho g H h_x$ uses $h = H+b$.  On the floating side we know $\rho H < - \rho_w b$ and, by Archimedes principle, we use $h = (1-\rho/\rho_w) H$ in the driving stress.

Equation \eqref{ssa} simplifies if the ice is floating.  The ice surface elevation is proportional to the thickness if the ice is floating.  Also we assume zero resistance ($C=0$) is applied by the ocean.  Thus the SSA becomes
\begin{equation}
   \left(2 \,\bar\nu\, H u_x\right)_x = \rho g (1-\rho/\rho_w) H H_x \label{ssafloat}
\end{equation}
for floating ice.  A useful observation about flow line equation \eqref{ssafloat} is that both left- and right-hand expressions are derivatives; this can be used to build a 1D exact solution.

\subsection*{Steady ice shelf exact solution}  For a steady 1D ice shelf, in which $H_t=0$, the mass continuity equation \eqref{masscont} reduces to $M=(uH)_x$.  Because of the relative simplicity of the SSA equation \eqref{ssafloat} and the steady mass continuity equation for 1D floating ice, the exact velocity and thickness for a steady ice shelf can be computed \cite{vanderVeen83}.  This exact solution depends on the ice thickness $H_g$ and velocity $u_g$ at the grounding line.  For the surface mass balance $M$ we choose a positive constant $M_0$.  These choices determine a unique solution, the derivation of which is left to the exercises.

Supposing $H_g=500$ m, $u_g = 50 \,\text{m}\,\text{a}^{-1}$, and $M_0=30 \,\text{cm}\,\text{a}^{-1}$ we get the results in Figure \ref{fig:steadyshelfprofile}, which are from code \texttt{exactshelf.m} (not shown).  We will use this exact solution to verify a numerical SSA code.  Note that driving stresses are much higher near the grounding line than away from it, and thus the highest longitudinal stresses, strain rates, and thinning rates occur near the grounding line.

\twofig{steadyshelfprofile}{steadyshelfvelocity}{The upper and lower surface elevation (m; left) of the exact ice shelf solution and its velocity (m/a; right); $x=0$ is the grounding line.}

\subsection*{Numerical solution of the SSA}  Suppose the ice thickness is a fixed function $H(x)$.  To find the velocity we must solve the nonlinear PDE \eqref{ssa} or \eqref{ssafloat} for the unknown $u(x)$.  When we do this numerically an iteration is needed because of the nonlinearity.  The simplest iteration idea is to use an initial guess at the velocity, which allows us to compute an effective viscosity and then get a new velocity solution from a linear PDE problem.  Then we recompute the effective viscosity, solve for a new velocity, and repeat until things stop changing.  This is often called a ``Picard'' iteration, in contrast to a ``Newton'' iteration which should converge faster.

Denote the previous velocity iterate as $u^{(k-1)}$ and the current iterate as $u^{(k)}$.  Compute $\bar \nu^{(k-1)} = B |u^{(k-1)}_x|^{1/n-1}$ and define $W^{(k-1)} = 2 \bar \nu^{(k-1)} H$.  Solving this linear elliptic PDE for the unknown $u^{(k)}$ is a Picard iteration for \eqref{ssa}:
\begin{equation}
   \left(W^{(k-1)} u^{(k)}_x\right)_x - C |u^{(k-1)}|^{m-1} u^{(k)} = \rho g H h_x. \label{picardssa}
\end{equation}
If the difference between $u^{(k-1)}$ and $u^{(k)}$ were zero then we would have a solution of \eqref{ssa}, while in practice we stop the iteration when the difference is smaller than some tolerance.

Equation \eqref{picardssa} is a linear boundary value problem.  We can write it abstractly
\begin{equation}
  \left(W(x)\, u_x\right)_x - \alpha(x)\, u = \beta(x)  \label{innerlinear}
\end{equation}
where the functions $W(x)$, $\alpha(x)$, $\beta(x)$ are known.  Equation \eqref{innerlinear} applies on an interval of the $x$-axis.  For one boundary condition we will suppose that $x=x_g$ is a location where the velocity is known, $u(x_g)=u_g$, as in Figure \ref{fig:steadyshelfprofile}.  In the ice shelf case we also have the calving front condition (see Notes and References)
\begin{equation}
  2 B H |u_x|^{1/n - 1} u_x = \frac{1}{2}\rho (1-\rho/\rho_w) g H^2  \label{calvingstress}
\end{equation}
at the end of the ice shelf $x=x_c$.  Boundary condition \eqref{calvingstress} can be solved for $u_x(x_c)=\gamma$ in terms of known quantities including the thickness at the calving front.

Where to get an initial guess $u^{(0)}$?  Generally this may require effort, but we will use these choices for our 1D case.  For floating ice, an initial velocity comes from assuming a uniform strain rate provided by the calving front condition: $u^{(0)}(x) = \gamma (x-x_g) + u_g$.  For grounded ice, we may assume ice is held by basal resistance only: $u^{(0)}(x) = \left(-C^{-1} \rho g H h_x\right)^{1/m}$.

\subsection*{Numerics of the linear boundary value problem}  Suppose equation \eqref{innerlinear} applies on $[x_g,x_c]=[0,L]$.  We choose a grid with equal spacing $\Delta x$ and index $j=1,2,\dots,J+1$, so that $x_1 = 0$ and $x_{J+1} = L$ are endpoints.  The coefficient $W(x)$ is needed on a staggered grid, for stability and accuracy reasons similar to those for the SIA diffusivity.  Our finite difference approximation of \eqref{innerlinear} is, therefore,
\begin{equation}
  \frac{W_{j+1/2} (u_{j+1} - u_j) - W_{j-1/2} (u_{j} - u_{j-1})}{\Delta x^2} - \alpha_j u_j = \beta_j  \label{discreteinnerlinear}
\end{equation}

For the left end boundary condition we have $u_1 = u_g$ given, which is easy to include in the linear system (below).  For the right end boundary condition we have $u_x(L)=\gamma$, which requires more thought.  First introduce a notional point $x_{J+2}$.  Now require $(u_{J+2} - u_J)/(2 \Delta x) = \gamma$, which is a centered approximation to ``$u_x(x_c)=\gamma$.''  Using equation \eqref{discreteinnerlinear} in $j=J+1$ case, eliminate the $u_{J+2}$ variable ``by-hand''.  This determines the form of the last equation in our linear system.

Now observe that each iteration to solve the SSA stress balance has the form
\begin{equation}
   A \mathbf{v} = \mathbf{b}. \label{Aveqb}
\end{equation}
Indeed, at each location $x_1,\dots,x_{J+1}$ we can write an equation, including a row of the matrix $A$ in \eqref{Aveqb}, involving the unknown velocities.  It is this linear system of $J+1$ equations:
\begin{equation}
\begin{bmatrix}
1 &  &  &  &  \\
W_{3/2} & A_{22} & W_{5/2} &  &  \\
 & W_{5/2} & A_{33} &  &  \\
 &  & \ddots & \ddots &  \\
 &  & W_{J-1/2} & A_{JJ} & W_{J+1/2} \\
 &  &  & A_{J+1,J} & A_{J+1,J+1} \\
\end{bmatrix}\,
\begin{bmatrix}
u_1 \\ u_2 \\ u_3 \\ \vdots \\ u_J \\ u_{J+1}
\end{bmatrix}
=
\begin{bmatrix}
u_g \\ \beta_2 \Delta x^2 \\ \beta_3 \Delta x^2 \\ \vdots \\ \beta_J \Delta x^2 \\ b_{J+1}
\end{bmatrix}  \label{discretematrixform}
\end{equation}
The diagonal entries ``$A_{ij}$'' are
  $$A_{22} = -(W_{3/2}+W_{5/2}+\alpha_2 \Delta x^2), \quad \dots, \quad A_{JJ} = -(W_{J-1/2}+W_{J+1/2}+\alpha_J \Delta x^2),$$
except for special cases for the coefficients in the last equation,
  $$A_{J+1,J} = 2 W_{J+1/2}, \quad A_{J+1,J+1} = -(2 W_{J+1/2}+\alpha_{J+1}\Delta x^2).$$
For the right side of the last equation, $b_{J+1} = -2 \gamma \Delta x W_{J+3/2} + \beta_{J+1} \Delta x^2$.

System \eqref{discretematrixform} is a tridiagonal linear system.  But don't bother looking up how to solve such a linear system unless you really need to!  It is fully appropriate to give system \eqref{discretematrixform} to Matlab's linear solver, the ``backslash'' operator $\mathbf{v} = A\, \backslash\, \mathbf{b}$, especially at this initial implementation stage.  In these notes we will not worry further about solving finite linear systems.  We now have a code to solve \eqref{innerlinear} by finite differences and linear algebra, namely \texttt{flowline.m} below.

\minput{flowline}

By ``manufacturing'' exact solutions to \eqref{innerlinear}---see Notes and References---we can test this first piece of our SSA-solving codes before proceeding to solve the actual nonlinear SSA problem.   In fact, results from \texttt{testflowline.m} (not shown) demonstrate that our implemented numerical scheme converges at the optimal rate $O(\Delta x^2)$.

\subsection*{Solving the stress balance for an ice shelf}  The code \texttt{ssaflowline.m} (below) numerically computes the velocity for an ice shelf.  The thickness is assumed to be given, so we are not yet addressing the full, ``coupled'' ice shelf problem, simultaneously solving the applicable mass continuity \eqref{masscont1D} and stress balance \eqref{ssafloat} equations.  We are only solving the latter.

This code implements Picard iteration \eqref{picardssa}, in the floating case, to solve the nonlinear equation \eqref{ssafloat}.  It calls \texttt{ssainit.m} (not shown) to get the initial iterate $u^{(0)}(x)$, as already described, and it calls \texttt{flowline.m} at each iteration.  It also calls small helper functions \texttt{stagav(),regslope(),stagslope()}, at the end of the code, to computed certain gridded values.

\minput{ssaflowline}

Now we can ask precisely: Does \texttt{ssaflowline.m} work correctly?  The exact velocity solution shown in Figure \ref{fig:steadyshelfprofile}, computed by \texttt{exactshelf.m}, allows us to compare the numerical to the exact velocities by finding the maximum difference between them.  For this to work we take the exact thickness shown in Figure \ref{fig:steadyshelfprofile}, also from \texttt{exactshelf.m}.  A convergence comparison, shown in Figure \ref{fig:shelfconv}, is done by codes \texttt{testshelf.m} and \texttt{shelfconv.m} (not shown).  Each circle in the Figure gives the maximum velocity error on a given grid.

\onefig{shelfconv}{The numerical SSA velocity solution from \texttt{ssaflowline.m} converges to the exact solution, at nearly the optimal rate $O(\Delta x^2)$, as the grid is refined from spacing $\Delta x=4$ km to $\Delta x=62$ m.}

Even on the coarsest $\Delta x = 4$ km grid we see in Figure \ref{fig:shelfconv} that the maximum velocity error (i.e.~difference) is less than 1 m/a, while the maximum velocity itself is $\sim 300$ m/a.  We can conclude from this comparison that, at screen resolution, our numerical velocity solutions are essentially identical to that shown in the right part of Figure \ref{fig:steadyshelfprofile}.  There is not even a reason to show the numerical solutions!

\subsection*{Realistic ice shelf modelling}  Real ice shelves have two horizontal variables.  They are frequently confined in bays, and thus they experience ``side drag''.  Their velocities vary spatially and temporally along their grounding lines, which are the curves where the flotation criterion is an equality.  Furthermore real ice shelves have interesting boundary processes, including high basal melt near grounding lines, marine ice basal freeze-on, and fracturing which nears full thickness at the calving front.  It is a bit complicated.

Nonetheless ``diagnostic'' (i.e.~fixed geometry) ice shelf modelling in two horizontal variables, done like the above example where the velocity is unknown but the thickness is known and fixed, is quite successful using only the isothermal SSA model.  For example, Figure \ref{fig:rossquiver} shows a Parallel Ice Sheet Model (PISM) result for the Ross ice shelf, compared to observed velocities.  There is only one tuned parameter, the constant value of the ice hardness $B$.  In this run, observed velocities for grounded ice were applied as boundary conditions.  Many current ice shelf models yield comparable match \cite{MacAyealetal}.

\twofigsizes{rossquiver}{rossscatter}{Results from PISM.  Left: Observed (white) and modeled (black) ice velocities are nearly coincident across the whole Ross ice shelf.  The grounding line is the thin black curve.  Right: In this scatter plot there is one point for each arrow at left.}{3.0in}{3.0in}


\section{A summary of numerical ice sheet modelling}

These notes are brief, and so they give a very incomplete view of numerical models for glaciers and ice sheets.  They do, however, illustrate some general principles about numerical modelling.  One should:
\begin{itemize}
\item Return often to the continuum model.
\item Modularize codes.
\item Test the parts: Is the component robust? Does it show convergence?
\end{itemize}

Regarding the specific ice flow models covered in these notes, here are three high-level points, as a meager conclusion:
\begin{itemize}
\item The mass continuity equation is the part of an ice sheet model which describes how the ice geometry evolves.  It is a kind of transport equation in the map-plane, but with diffusive character at larger spatial scales.  The numerical approach to this equation depends on which is the stress balance which supplies the ice velocity or ice flux.  Mass continuity is a diffusion for frozen bed, large scale flows, and in that case the SIA is a good choice.  Mass continuity is \emph{not} very diffusive for membrane stresses (e.g.~SSA), especially with no basal resistance as in ice shelves.  It has some diffusiveness for ice streams, though how much is hard to quantify.
\item The SIA stress balance is exceptional because it is not horizontally-distributed.  In the SIA, velocity follows immediately by vertical integration of the driving stress.
\item Membrane stress balance equations like the SSA (and the Blatter-Pattyn, hydrostatic, and Stokes models also) determine horizontal velocity from geometry and boundary conditions.  Because of the Glen law these equations are nonlinear, so iteration is necessary.  At each iteration a sparse matrix ``inner'' problem is solved; non-experts should give this matrix problem to a solver package.
\end{itemize}



%\small
\section{Notes} \label{sec:nr}

Recent and recommended books and reviews which extend the continuum modeling content of these notes include \cite{CuffeyPaterson,GreveBlatter2009,SchoofHewitt2013,vanderVeen}.

The SIA model, which was derived by several authors \cite{FowlerLarson1978,Hutter,MorlandJohnson}, follows by scaling the Stokes equations using the aspect ratio $\eps = [H]/[L]$, where $[H]$ is a typical thickness of an ice sheet and $[L]$ is a typical horizontal dimension.  After scaling one drops the terms that are small if $\eps$ is small \cite{Fowler,Hutter}; this is a ``small-parameter argument''.  In one scaling there are no $O(\eps)$ terms in the scaled equations so one only drops $O(\eps^2)$ terms \cite{Fowler}.  The SIA is re-formulated as a well-posed free boundary problem in \cite{JouvetBueler2012}, which provides the correct boundary condition at grounded margins.  The Mahaffy \cite{Mahaffy} scheme for diffusivity used here is not the only one \cite{HindmarshPayne}.

The SSA model \cite{WeisGreveHutter} was derived in \cite{Morland} for ice shelves and in \cite{MacAyeal} for ice streams.  In deriving the SSA, the aspect ratio $\eps$ above is one small parameter but additionally a second parameter describing the magnitude of surface undulations must be assumed to be small  \cite{SchoofStream,SchoofHindmarsh}.  A well-posed model for the emergence of ice streams though till failure, using only the SSA, is in \cite{SchoofStream}.

A key modelling issue omitted in these notes is thermomechanical coupling.  Temperature is important because the ice softness varies by three orders of magnitude in the temperature range relevant to ice sheet modelling.  Ice temperature therefore gives ice sheet dynamics a long memory of past climate, and because the geothermal flux is a significant input in slow-flowing parts of ice sheets.  Equally important, dissipation of gravitational potential energy is a major part of the energy balance, and basal melt in particular.  For example, each year the ice in the Jakobshavn drainage basin in Greenland dissipates enough gravitational potential energy to fully melt more than $1\,\text{km}^3$ of ice \cite{AschwandenBuelerKhroulevBlatter}.  Beautiful evidence that, as a result, outlet glaciers have thick temperate ice is in \cite{Luethietal2009}.  These physical effects motivate modelers to solve the conservation of energy equation simultaneously with the mass conservation (continuity) and momentum conservation (stress balance) equations.  Traditionally the conservation of energy equation uses only temperature as the state variable \cite{BBL}, and this may be suitable for cold ice sheets, but ice sheets are generically polythermal.  Enthalpy methods are a good way to track the energy content of polythermal ice sheets and glaciers \cite{AschwandenBuelerKhroulevBlatter}, though one can also have a separate water-content equation for temperate ice \cite{Greve}.  In any case, the conservation of energy equation is strongly advection-dominated in general \cite{BBL}.

Pressurized basal water is required for most ice sliding.  To model the production of such water in ice sheets one must at least compute the ice base temperature and the basal melt rate through the energy conservation equation \cite{BBssasliding,Clarke05,Raymondenergy,Tulaczyketal2000b}.

One of the most significant issues in modelling ice sheets using shallow models is to describe the ``switch'', in space and time, between shear-dominated and membrane-stress-dominated flow.  It is not a good idea to abruptly switch from the SIA model to the SSA model at the edge of an ice stream, by whatever criterion that switch might be applied, though this has been attempted \cite{HulbeMacAyeal,Ritzetal2001}.  However, ``hybrid'' schemes exist which solve the SIA and SSA everywhere in the ice sheet \cite{BBssasliding,Winkelmannetal2011}, or solve a related vertically-integrated model \cite{Goldberg2011,PollardDeConto}, then combining the stresses or velocities according to different schemes.

``Higher-order'' three-dimensional approximations of the Stokes stress balance, such as the Blatter-Pattyn model \cite{Blatter,Pattyn03}, also use shallow approximations, at minimum including both the most-basic shallow assumption of well-defined thickness (see main text) \emph{and} an assumption of hydrostatic normal stress \cite{GreveBlatter2009}.  Computational limitations generally restrict either the spatial extent, the spatial resolution, or the run duration of these more complete models, primarily because 3D stress balances involve more memory.  Vertically-integrated hybrids can generally be used at higher spatial resolution and longer time scales than higher-order models because the 2D stress balance equations are easier to solve.

As both the SIA and the SSA are derived by small-parameter arguments from the Stokes equations, one might ask whether there is a common shallow antecedent model of both SIA and SSA?  Schoof and Hindmarsh \cite{SchoofHindmarsh} answer that Blatter-Pattyn is one.

Solving the Stokes stress balance itself \cite{JouvetRappaz2011,Lengetal2012,ISMIPHOM} requires explicit accounting for incompressibility through a pressure variable.  Numerical approximations of this stress balance are indefinite, thus harder to solve, essentially because incompressibility is an equality constraint.  In plane flow one can address the incompressibility constraint by using stream functions \cite{BaliseRaymond1985}.  Questions remain about what are the most important deficiencies, relative to the Stokes model, when using either higher-order \cite{ISMIPHOM} or hybrid models.

The finite difference material in these notes should probably be read with reference \cite{MortonMayers} or similar in hand.  The ``main theorem for numerical PDE schemes'' mentioned in the text is the Lax equivalence theorem \cite{MortonMayers}.  Alternative numerical discretization techniques include the finite element \cite{Braess}, finite volume \cite{LeVeque}, and spectral \cite{Trefethen} methods.  Newton iteration for the nonlinear discrete equations is superior to Picard iteration used here, in terms of rapid convergence once iterates are near the solution, but implementation care is needed \cite{Kelley}.

Which are the best numerical models for moving grounding lines?  Even when the minimal SSA stress balance is used, this is still something of an open question \cite{Goldbergetal2009,MISMIP3d2013,MISMIP2012,SchoofMarine1}.  The physics requires at least that the quantities $H$ and $u$ are continuous there, but several stress balance regimes exist near the grounding line, with increasing complexity as one focusses-in on the line \cite{SchoofMarine2}.

Where to find exact solutions for ice flow models?  The textbook Greve and Blatter \cite{GreveBlatter2009} has a few.  Halfar's similarity solution to the SIA \cite{Halfar81,Halfar83} has been generalized to non-zero mass balance \cite{BLKCB}.  There are flow-line \cite{Bodvardsson,vanderVeen83} and cross-flow \cite{SchoofStream} solutions to the SSA model, and one can even construct an exact, steady marine ice sheet in the flow-line case \cite{Bueler2014exactmarine}.  For the Stokes equations themselves there are plane flow solutions for constant viscosity \cite{BaliseRaymond1985}.

As a last resort for numerical verification, one can ``manufacture'' exact solutions by starting with a specified solution and then deriving a source term so that the specified function is actually a solution \cite{Roache}.  There are such manufactured solutions to the thermomechanically-coupled SIA \cite{BBL}, plane flow Blatter-Pattyn model \cite{GlowinskiRappaz}, and Glen-law Stokes equations \cite{JouvetRappaz2011,Lengetal2012,SargentFastook2010}.

\clearpage\newpage
\footnotesize

%\bibliography{ice-bib}
%\bibliographystyle{siam}
\documentclass[letterpaper,final,12pt,reqno]{amsart}

\usepackage[total={6.3in,9.2in},top=1.1in,left=1.1in]{geometry}

\usepackage{verbatim}
\usepackage{empheq}
\usepackage[dvipsnames]{xcolor}
\usepackage{animate}
\usepackage{graphicx}
\usepackage{fancyvrb}

%\usepackage{palatino}

% hyperref should be the last package we load
\usepackage[pdftex,
colorlinks=true,
plainpages=false, % only if colorlinks=true
linkcolor=blue,   % only if colorlinks=true
citecolor=Red,   % only if colorlinks=true
urlcolor=ForestGreen     % only if colorlinks=true
]{hyperref}

\pdfinfo{
/Title (Numerical modelling of ice sheets, streams, and shelves)
/Author (Ed Bueler)
/Subject (numerical modelling of ice sheets)
/Keywords (numerical modelling, numerical analysis, glacier, ice sheet, ice shelf, shallow models)
}

\renewcommand{\baselinestretch}{1.05}

\newcommand{\ddt}[1]{\ensuremath{\frac{\partial #1}{\partial t}}}
\newcommand{\ddx}[1]{\ensuremath{\frac{\partial #1}{\partial x}}}
\newcommand{\ddy}[1]{\ensuremath{\frac{\partial #1}{\partial y}}}
\newcommand{\pp}[2]{\ensuremath{\frac{\partial #1}{\partial #2}}}
\renewcommand{\t}[1]{\texttt{#1}}
\newcommand{\Matlab}{\textsc{Matlab}\xspace}
\newcommand{\bq}{\mathbf{q}}
\newcommand{\bu}{\mathbf{u}}
\newcommand{\bU}{\mathbf{U}}
\newcommand{\eps}{\epsilon}
\newcommand{\grad}{\nabla}
\newcommand{\Div}{\nabla\cdot}
\newcommand{\devstress}{\tau}

\newcommand{\minput}[1]{
\vspace{0.8cm}
\VerbatimInput[frame=single,framesep=3mm,label=\fbox{\normalsize \textsl{\,#1.m\,}},fontfamily=courier,fontsize=\footnotesize]{tmp/#1.slim.m}
\vspace{0.5cm}
}

% usage:  \onefigsize{name}{caption}{width}
\newcommand{\onefigsize}[3]{
\begin{figure}[ht]
\centering
\includegraphics[width=#3,keepaspectratio=true]{#1}
\caption{#2}
\label{fig:#1}
\end{figure}}

% usage:  \onefig{name}{caption}
\newcommand{\onefig}[2]{\onefigsize{#1}{#2}{3.0in}}

% usage:  \twofigsizes{left-name}{right-name}{caption}{left-width}{right-width}
\newcommand{\twofigsizes}[5]{
\begin{figure}[ht]
\centering
\includegraphics[width=#4,keepaspectratio=true]{#1} \quad
\includegraphics[width=#5,keepaspectratio=true]{#2}
\caption{#3}
\label{fig:#1}
\end{figure}}

% usage:  \twofig{left-name}{right-name}{caption}
\newcommand{\twofig}[3]{\twofigsizes{#1}{#2}{#3}{2.5in}{2.5in}}



\begin{document}
\graphicspath{{../photos/}{../pdffigs/}}

\begin{titlepage}

  \begin{center}
  \phantom{foo}
    \vspace{1.0cm}

     {\Large \textsc{Numerical modelling}}
    \vspace{0.7cm}

     {\Large \textsc{of ice sheets, streams, and shelves}}

    \vspace{1.5cm}

    {\large Ed Bueler}
    \vspace{1cm}

    Summer School in Glaciology, McCarthy Alaska, June 2016 

    \vfill
    
    \includegraphics[width=6.0in]{flowline}
  
    \scriptsize \emph{Illustrates the notation used in these notes.  Figure modified from \cite{SchoofMarine1}.} \normalsize
    
    \vspace{1.5in}
  \end{center}
\end{titlepage}

\clearpage\newpage

%\setcounter{section}{1}
\section{Introduction}

The most common use of numerical models in glaciology may be to help you ask: When I put together my incomplete understanding of glacier processes into a mathematical model, does the combination behave as I expect?  Numerical models can at least demonstrate flaws in our understanding of glacier processes, and they can show us how these processes combine to give overall behavior, but they should be built with care.  The worst outcome is to spend time (and perhaps reputation) explaining, through physical argumentation and perhaps using observational evidence, numerical model behavior that is an artifact of poor computer programming or numerical analysis.

So the reader of these notes may be surprised that a continuum model, and not a computer code, seems to be our focus much of the time.  While all codes produce some numbers, we want numbers that actually come from our continuum model.  We will therefore analyse numerical implementations to see if they match the continuum model and its solutions.

These notes have a limited scope:
  \begin{quote}\emph{shallow approximations of ice flow.}\end{quote}
They adopt a constructive approach; we provide:
  \begin{quote}\emph{example numerical codes that actually work.}\end{quote}
Within our scope are the shallow ice approximation (SIA) in two horizontal dimensions (2D), the shallow shelf approximation (SSA) in 1D, and the mass continuity and surface kinematical equations.  We recall the Stokes model, but we do not address its numerical solution.  Our numerical concepts include finite difference schemes, solving algebraic systems from stress balances, and the verification of codes using exact solutions.

Our notation, which generally follows \cite{GreveBlatter2009}, is common in the glaciological literature, but see Table \ref{tab:notation}.  Cartesian coordinates $x,y,z$ have $z$ perpendicular to the geoid and positive-upward.  If these coordinates or ``$t$'' appear as subscripts then they denote partial derivatives: $u_x = \partial u/\partial x$.  Tensor notation uses subscripts from the list $\{1,2,3,i,j\}$.  For example, ``$\tau_{ij}$'' or ``$\tau_{13}$'' denote entries of the deviatoric stress tensor.

These notes are based on eighteen Matlab codes, each about one-half page.  All have been tested in Matlab and Octave.  They are distributed by cloning the repository
\begin{quote}
\url{https://github.com/bueler/mccarthy}
\end{quote}
\noindent and looking in the \texttt{mfiles/} subdirectory.  Though only five of the codes are printed here, with their comments stripped for compactness and clarity, the electronic versions have generous comments and help files.

\begin{table}[ht]
\caption{Notation used in these notes, with values for some constants.}
\begin{tabular}{clll}
variable  & description & SI units & value \\
\hline
$A$ & $A=A(T)=$ ice softness in the Glen flow law & $\text{Pa}^{-n}\,\text{s}^{-1}$ \\
$B$ & ice hardness; $B=A^{-1/n}$ & $\text{Pa}\,\text{s}^{1/n}$ \\
$b$ & bedrock elevation & m \\
$c$ & specific heat in general & J kg$^{-1}$ K$^{-1}$ \\
$\nabla$ & (spatial) gradient & m$^{-1}$ \\
$\nabla\cdot$ & (spatial) divergence & m$^{-1}$ \\
$\mathbf{g}$ & gravity & m s$^{-2}$\phantom{foobar} & 9.81 \\
$H$ & ice thickness & m \\
$h$ & ice surface elevation & m \\
$\kappa$ & conductivity in general & J s$^{-1}$ m$^{-1}$ K$^{-1}$ \\
$M$ & climatic mass balance & m s$^{-1}$ \\
$n$ & exponent in Glen flow law & & 3 \\
$\nu$ & viscosity & Pa s \\
$p$ & pressure & Pa \\
$\bq$ & map-plane ice flux: $\bq = \int_{b}^{h} \bU\,dx = \bar{\bU} H$ & $\text{m}^2\,\text{s}^{-1}$ \\
$\rho$ & (1) density in general & kg m$^{-3}$ & \\
  & (2) density of ice & kg m$^{-3}$ & 910 \\
$\rho_w$ & density of sea water & kg m$^{-3}$ & 1028 \\
$T$ & temperature & K \\
$\tau$ & magnitude of $\tau_{ij}$: $2 \tau^2 = \sum_{ij} \tau_{ij}^2$ & Pa \\
$\tau_{ij}$ & deviatoric stress tensor & Pa \\
$Du_{ij}$ & strain rate tensor & s$^{-1}$ \\
$\mathbf{U}$ & $=(u,v)$ horizontal ice velocity & m s$^{-1}$ \\
$\mathbf{u}$ & $=(u,v,w)$ 3D ice velocity & m s$^{-1}$ \\
\end{tabular}
\label{tab:notation}
\end{table}


\section{Ice flow equations}

My first goal in these notes is to get to an equation for which I can say:
\begin{center}
\emph{by numerically solving this equation, we have a usable model for an ice sheet.}
\end{center}
\noindent A ``usable'' model tends to be \emph{understood} as much as it is \emph{correct}.  Also, this first model will not be complete by any modern standard.  To get to my goal I first (briefly!) recall the continuum mechanical equations of ice flow.  

Ice in glaciers is a moving fluid so we describe its motion by a velocity field $\mathbf{u}(t,x,y,z)$.  If the ice fluid were faster-moving than it actually is, and if it were linearly-viscous like liquid water, then it would be a ``typical'' incompressible fluid.  We would use the Navier-Stokes equations as the model:
\begin{align}
\nabla \cdot \mathbf{u} &= 0 &&\text{\emph{incompressibility}} \label{incompressible} \\
\rho \left(\mathbf{u}_t + \mathbf{u}\cdot\nabla \mathbf{u}\right) &= -\nabla p + \nabla \cdot (\nu \nabla \mathbf{u}) + \rho \mathbf{g} &&\text{\emph{stress balance}} \label{navierstokes}
\end{align}
In equation \eqref{navierstokes} the term $\mathbf{u}_t + \mathbf{u}\cdot\nabla \mathbf{u}$ is an acceleration.  The right-hand side of \eqref{navierstokes} is the total stress, and so equation \eqref{navierstokes} says ``$ma=F$'', i.e.~it is Newton's second law.  Much time has been spent to get partial understanding of the rich solutions of these Navier-Stokes equations; a book-length introduction like \cite{Acheson} is recommended.  The numerical solution of these equations is \emph{computational fluid dynamics} (CFD).

But, is numerical ice flow modelling a part of CFD?  Does a well-written general-purpose CFD text like \cite{Wesseling} help the glaciers student?  Ice sheet flow is a large-scale fluid problem like atmosphere and ocean circulation in climate systems, but it is an odd one.  Consider some topics which might make ocean circulation modelling exciting, for example:
  \begin{center} turbulence \qquad convection \qquad  coriolis force  \qquad density stratification
  \end{center}
None of these topics are relevant to ice flow.  What could be interesting about the flow of slow and old, and surely boring, ice?

First observe that ice is indeed a slow fluid.  In terms of equation \eqref{navierstokes}, ``slow'' means $\rho \left(\mathbf{u}_t + \mathbf{u}\cdot\nabla \mathbf{u}\right) \approx 0$, which says that the forces (stresses) of inertia are negligible.  However, ice is also a shear-thinning fluid with a specific kind of nonlinearly-viscous (``non-Newtonian'') behavior in which larger strain rates imply smaller viscosity.  The viscosity $\nu$ in \eqref{navierstokes} is therefore not constant, and so we separately state an empirically-based flow law below.

\subsection*{Stokes equations}  So now the standard model for isothermal flow is this set of Stokes equations:
\begin{align}
\nabla \cdot \mathbf{u} &= 0 &&\text{\emph{incompressibility}} \label{incompressibleagain} \\
0 &= - \nabla p + \nabla \cdot \tau_{ij} + \rho \mathbf{g} &&\text{\emph{stress balance}} \label{forcebalance} \\
Du_{ij} &= A \tau^2 \tau_{ij} &&\text{\emph{$n$=3 Glen flow law}} \label{flowlaw}
\end{align}
In the flow law \eqref{flowlaw}, the deviatoric stress tensor $\tau_{ij}$ and the strain rate tensor $Du_{ij}$ appear; previous lectures cover these.  Here we merely note that: $Du_{ij} = (1/2)((u_i)_{x_j}+(u_j)_{x_i})$ if we index coordinates by $x_1,x_2,x_3=x,y,z$, each tensor in \eqref{flowlaw} is symmetric and has trace zero, and $\tau^2 = (1/2) \tau_{ij} \tau_{ij}$ defines ``$\tau^2$'' in \eqref{flowlaw} if we use the summation convention.

The Stokes equations do not contain a time derivative.  Thus boundary stresses, the force of gravity $\rho \mathbf{g}$, and ice softness $A$ together determine the velocity and stress fields (i.e.~$\bu$, $p$, $\tau_{ij}$) instantaneously.  Thus ice flow simulation codes have no memory of prior momentum or velocity.  Said another way, velocity is a ``diagnostic'' output of ice flow codes, because it is not needed for (re)starting a simulation.

\subsection*{Plane-flow Stokes equations}  Consider now the $x,z$-plane case of equations \eqref{incompressibleagain}, \eqref{forcebalance}, and \eqref{flowlaw}.  ``Plane-flow'' means that velocity component $v$ is zero and that all derivatives with respect to $y$ are zero:
\begin{align}
u_x + w_z &= 0 &&\text{\emph{incompressibility}} \label{incompressiblexz} \\
p_x &= \tau_{11,x} + \tau_{13,z} &&\text{\emph{stress balance} ($x$)} \label{stokespx} \\
p_z &= \tau_{13,x} - \tau_{11,z} - \rho g &&\text{\emph{stress balance} ($z$)} \label{stokespz} \\
u_x &= A \tau^2 \tau_{11} &&\text{\emph{flow law} (diagonal)}  \label{forceflowx} \\
u_z + w _x &= 2 A \tau^2 \tau_{13} &&\text{\emph{flow law} (off-diagonal)} \label{forceflowz}
\end{align}
Note that $\tau_{13}$ is a shear stress while $\tau_{11}$ and $\tau_{33}=-\tau_{11}$ are deviatoric longitudinal stresses.  Also $\tau^2 = \tau_{11}^2+\tau_{13}^2$ in this case.  Equations \eqref{incompressiblexz}--\eqref{forceflowz} form a system of five nonlinear equations in five scalar unknowns ($u,w,p,\tau_{11},\tau_{13}$).

\subsection*{Slab-on-a-slope}  Equations \eqref{incompressiblexz}--\eqref{forceflowz} are complicated enough to make us pause before jumping in to numerical solution methods, but  we can handle a simplified situation first.  A uniform slab of ice, or a ``slab-on-a-slope'', is both a case in which we actually solve the Stokes equations, and a motivation for the shallow model in the next subsection.

\onefig{slab}{Rotated axes for a slab-on-a-slope flow calculation.}

We rotate our coordinates only for this example and not elsewhere in these notes.  The two-dimensional axes ($x$,$z$) shown in Figure \ref{fig:slab} are rotated downward (clockwise) at angle $\alpha>0$ so that the gravity vector has components $\mathbf{g} = (g \sin\alpha,- g \cos \alpha)$.  Equations \eqref{stokespx} and \eqref{stokespz} in these rotated coordinates are
\begin{align}
p_x &= \tau_{11,x} + \tau_{13,z} + \rho g \sin\alpha, \label{stokespxrot} \\
p_z &= \tau_{13,x} - \tau_{11,z} - \rho g \cos\alpha. \label{stokespzrot}
\end{align}
Assuming also that there is no variation with $x$, the whole set of Stokes equations \eqref{incompressiblexz}, \eqref{forceflowx}, \eqref{forceflowz}, \eqref{stokespxrot}, \eqref{stokespzrot} simplifies greatly:
\begin{align}
w_z &= 0 &   0 &= \tau_{11} \label{stokesslab} \\
\tau_{13,z} &= - \rho g \sin\alpha &   u_z &= 2 A \tau^2 \tau_{13} \notag \\
p_z &= -\tau_{11,z} = - \rho g \cos\alpha \notag
\end{align}
We apply boundary conditions for these functions of $z$: $w(0)=0$, $p(H)=0$, $u(0)=u_0$.  The basal velocity $u_0$ will remain undetermined for now.

By integrating equations \eqref{stokesslab} vertically and using $\tau_{11}=0$, we get $w=0$, $p = \rho g \cos\alpha (H-z)$, and $\tau_{13} = \rho g \sin\alpha (H-z)$.  Note that $H-z$ is the depth below the ice surface, so both the pressure $p$ and shear stress $\tau_{13}$ are proportional to depth.  Because $u_z = 2 A \tau^2 \tau_{13}$, by integrating vertically again we find the horizontal velocity:
\begin{align}
u &= u_0 + \frac{1}{2} A (\rho g \sin\alpha)^3  \left(H^4 - (H-z)^4\right)  \label{uslab}
\end{align}

Do we believe formula \eqref{uslab}?  Figure \ref{fig:slabvel} compares to observations of a mountain glacier.  This comparison shows we have at least done a credible job of capturing deformation flow velocity, though we do not yet have a model for the sliding velocity $u_0$ (i.e.~basal motion).  

\twofigsizes{slabvel}{athabasca_deform}{Left:  Velocity from slab-on-a-slope formula \eqref{uslab}.  Right:  Inclinometry-measured velocity in a glacier (Athabasca Glacier \cite{SavagePaterson}).}{2.0in}{1.8in}

\subsection*{Plane-flow mass-continuity equation}  Observe that the equations so far do not address the change in shape of the glacier or ice sheet.  For this we need another equation, the \emph{mass continuity equation}.  First, define the vertical average of velocity:
	$$\bar U = \frac{1}{H}\int_0^{H} u\,dz.$$
The flux $q=\bar U\, H$ is the rate of flow input into the side of the area in Figure \ref{fig:slabmasscontfig}.

\onefigsize{slabmasscontfig}{Mass continuity equation \eqref{masscont1D} follows from considering the changing area $A$ of ice in a planar flow.  Ice can be added by surface mass balance $M$ or a difference of flux $q=\bar u H$ into the left and right sides.}{2.5in}

The ice area $A$ in Figure \ref{fig:slabmasscontfig} changes by adding all the boundary contributions,
\begin{equation}
\frac{dA}{dt} = \int_{x_1}^{x_2} M(x)\,dx + \bar U_1 H_1 - \bar U_2 H_2: \label{masscontintegrated}
\end{equation}
Here $M(x)$ is the climatic mass balance at the ice surface.  (In three-dimensions, equation \eqref{masscontintegrated} would be an equation for $dV/dt$, the rate of change of ice volume.)

If the width $\Delta x=x_2-x_1$ is small then $A\approx \Delta x\, H$.  So we divide by $\Delta x$ and take $\Delta x \to 0$ in \eqref{masscontintegrated} and get
\begin{equation}
H_t = M - \left(\bar U H\right)_x \label{masscont1D}
\end{equation}
This mass continuity equation describes change in the ice thickness in terms of surface mass balance and ice velocity.  It is a major ``use'' of the velocity in ice flow simulations.

\subsection*{Viscosity form of the flow law}  The flow law \eqref{flowlaw} has another form which we will use next, and later in describing ice shelf and stream flow.  Recall $\tau^2 = (1/2) \tau_{ij} \tau_{ij}$, which uses the summation convention.  Also define $|Du|^2 = (1/2) Du_{ij} Du_{ij}$.  The scalars $\tau$ and $|Du|$ are ``norms'' (also ``second invariants'') of the tensors $\tau_{ij}$ and $Du_{ij}$, respectively.  By taking these norms of both sides of \eqref{flowlaw} we get $|Du| = A \tau^3$.  But then $\tau = A^{-1/3} |Du|^{1/3}$, so \eqref{flowlaw} can be rewritten
\begin{equation}
\tau_{ij} = 2 \nu\, Du_{ij}  \qquad \text{\emph{flow law (viscosity form)}} \label{viscosityflowlaw}
\end{equation}
where $\nu = (1/2) A^{-1/3} |Du|^{-2/3}$ is the nonlinear viscosity.  Often $B = A^{-1/3}$ is called the ice ``hardness''.  The derivation of \eqref{viscosityflowlaw} is worth knowing in detail; see the Exercises.

Form \eqref{viscosityflowlaw} of the flow law allows us to eliminate stresses $\tau_{ij}$ from the Stokes equations by replacing them with formulas depending on derivatives of the velocity, that is, on the strain rates only.  The next two approximate models also use this idea.

\subsection*{The hydrostatic and Blatter-Pattyn approximations}  We return now briefly to plane-flow Stokes equations \eqref{incompressiblexz}--\eqref{forceflowz}, and reconsider how to simplify them.  One simplification step, present in all shallow models, is the ``hydrostatic'' approximation.  It drops the single term  $\tau_{13,x}$ from the $z$-component of the stress balance \eqref{stokespz}.  That is, it assumes that horizontal variation in the vertical shear stress is small compared to the other terms:
\begin{equation}
p_z = - \tau_{11,z} - \rho g. \label{hydrostaticpz}
\end{equation}
Because the (Cauchy) stress tensor $\sigma_{ij}$ is related to the deviatoric stress tensor by $\sigma_{ij} = \tau_{ij} - p \delta_{ij}$, and thus $p + \tau_{11} = p - \tau_{33} = - \sigma_{33}$, equation \eqref{hydrostaticpz} says that the vertical normal stress $\sigma_{33}$ is linear in depth.  Taking it to have surface value zero we get
\begin{equation}
p + \tau_{11} = \rho g (h-z). \label{hydrostaticitself}
\end{equation}

Equation \eqref{hydrostaticitself} is the major hydrostatic statement, and it allows one to eliminate $p$ from the model equations.  Furthermore, taking the $x$-derivative of \eqref{hydrostaticitself} and substituting into \eqref{stokespx}, then using the viscosity form \eqref{viscosityflowlaw}, leads to this equation:
\begin{equation}
\left(4 \nu u_x\right)_x + \left(\nu (u_z+w_x)\right)_z = \rho g h_x \qquad\text{\emph{hydrostatic stress balance}} \label{stresshydrostatic}
\end{equation}
The hydrostatic stress balance equation \eqref{stresshydrostatic} is nontrivially-coupled to incompressibility \eqref{incompressiblexz} because derivatives of the vertical velocity $w$ appear in both equations, though $p$ is gone.  Nonetheless coupled equations \eqref{incompressiblexz} and \eqref{stresshydrostatic}, along with the formula $\nu = (1/2) A^{-1/3} |Du|^{-2/3}$ and appropriate boundary conditions, determine $u$ and $w$.

If we drop $w_x$ from equation \eqref{stresshydrostatic} then we get the Blatter-Pattyn model
\begin{equation}
\left(4 \nu u_x\right)_x + \left(\nu u_z\right)_z = \rho g h_x \qquad\text{\emph{Blatter-Pattyn stress balance}} \label{stressblatter}
\end{equation}
Using this equation one can solve first for the horizontal velocity $u$ and then afterward recover $w$ from \eqref{incompressiblexz}; stress balance and incompressibility are decoupled.

\section{Shallow ice sheets}

Ice sheets have four outstanding properties as fluids.  They are (\emph{i}) slow, (\emph{ii}) shallow,  (\emph{iii}) non-Newtonian, and (\emph{iv}) they experience some contact slip (basal sliding).  The first ice flow model we consider, the non-sliding, isothermal \emph{shallow ice approximation} (SIA), accounts for (\emph{i})--(\emph{iii}).

Regarding the property of shallowness, Figure \ref{fig:green_transect} shows both a no-vertical-exaggeration cross-section of Greenland at $71^\circ$, as well as the standard vertically-exaggerated version which is more familiar in the glaciological literature.  Ice sheets are shallow, though of course the portion of an ice sheet which you want to model may not be.

\onefig{green_transect}{A vertically-exaggerated cross-section of the Greenland ice sheet ($71^\circ$ N) is shown by the green and blue curves.  Without exaggeration it appears as nearly a horizontal line (red).}

Our slab-on-a-slope example gives us a rough explanation of the SIA.  To show the SIA in its plane-flow form, we vertically integrate velocity formula \eqref{uslab} in the $u_0=0$ (non-sliding) case to get
\begin{equation}
\bar u H = \int_0^H \frac{1}{2} A (\rho g \sin\alpha)^3  \left(H^4 - (H-z)^4\right)\,dz = \frac{2}{5} A (\rho g \sin\alpha)^3 H^5. \label{siaubar}
\end{equation}
Note $\sin \alpha \approx \tan\alpha = - h_x$.  Combining these statements with mass continuity \eqref{masscont1D} gives
\begin{equation}
  H_t = M + \left(\frac{2}{5} (\rho g)^3 A H^5 |h_x|^2 h_x\right)_x. \label{sia1D}
\end{equation}
Equation \eqref{sia1D} is the SIA equation for nonsliding plane flow.  It is a model for the evolution of an ice sheet's thickness $H$ in terms of surface mass balance $M$, ice softness $A$, and bed elevation $b$ (because $h=H+b$).

Additional arguments are needed to demonstrate that the SIA is more general-purpose than the special case of a simple slab; see Notes and References.  Such arguments reduce the Stokes equations under the assumption that the surface and bed slopes, and the depth-to-width ratio, are small.

We will numerically solve the SIA in section \ref{sec:numericalsia}, but first we state it in two horizontal dimensions.  Let $\mathbf{U} = (u,v)$ be the vector horizontal velocity.  The shear stress approximation is $(\tau_{13},\tau_{23}) \approx - \rho g (h-z) \nabla h$, which appeared as ``$\tau_{13}= \rho g \sin \alpha (h-z)$ and $\sin \alpha \approx -h_x$'' in the previous section, becomes an equality in the SIA.  Equation \eqref{flowlaw} then gives the SIA formula for shear strain rates
\begin{equation*}
\mathbf{U}_z = 2 A |(\tau_{13},\tau_{23})|^{n-1} (\tau_{13},\tau_{23}) = - 2 A (\rho g)^n (h-z)^n |\nabla h|^{n-1} \nabla h.
\end{equation*}
By integrating vertically we get, in the non-sliding case,
\begin{equation}
\mathbf{U} = - \frac{2 A (\rho g)^n}{n+1} \left[H^{n+1} - (h-z)^{n+1}\right] |\nabla h|^{n-1} \nabla h.  \label{siavelocity}
\end{equation}

Mass continuity in two horizontal dimensions, which generalizes the 1D version \eqref{masscont1D}, also applies:
\begin{equation}
    H_t = M - \Div\left(\bar{\mathbf{U}} H\right)  \label{masscont}
\end{equation}
Equation \eqref{masscont} may be written $H_t = M - \Div \bq$ in terms of the map-plane flux $\bq = \int_{b}^{h} \mathbf{U}\,dz = \bar{\mathbf{U}}\,H$.

Combining Equations \eqref{siavelocity} and \eqref{masscont}, we get an equation for the rate of thickness change in terms of mass balance $M$, thickness, and surface slope $\grad h$:
\begin{equation}
H_t = M + \Div \left(\Gamma H^{n+2} |\grad h|^{n-1} \grad h \right), \label{sia}
\end{equation}
where we have defined the positive constant $\Gamma = 2 A (\rho g)^n / (n+2)$.  Equation \eqref{sia} is the SIA in two dimensions.  Recalling our earlier promise, if we can solve \eqref{sia} numerically then we have, following Mahaffy \cite{Mahaffy}, a usable model for the Barnes ice cap in Canada, a particularly simple ice sheet on a rather flat bed.

\subsection*{Analogy with the heat equation}  The SIA model is easy to compare with the better-known heat equation.  All numerical methods for solving \eqref{sia} can be understood as modifications of well-known heat equation methods.

In the simplest one-dimensional (1D) case, the heat equation for the temperature $T(t,x)$ of a conducting rod is
\begin{equation}
  T_t = D T_{xx}. \label{heat1D}
\end{equation}
This form applies when material properties are constant and there are no heat sources.  The positive constant $D$ is the ``diffusivity,'' with units which can be read from comparing sides of the equation: $D\sim \text{m}^2 \text{s}^{-1}$.  Observe that equation \eqref{heat1D} has a smoothing effect on the solution $T$ as it evolves in time, because any local maximum in the temperature is flattened (i.e.~$T_{xx}<0$ implies $T_t<0$ so $T$ decreases), while any local minimum is also flattened (i.e.~$T_{xx}>0$ implies $T_t>0$ so $T$ increases).

The 2D heat equation, analogous to equation \eqref{sia}, describes the temperature $T(t,x,y)$ at position $x,y$ and time $t$.  Recall that Fourier's law for conduction is the formula $\mathbf{Q} = - \kappa \grad T$ for heat flux $\mathbf{Q}$, where $\kappa$ is conductivity.  We will assume, for the purposes of our analogy, that $\kappa(x,y)$ may vary in space.  Also suppose a variable heat source $f(t,x,y)$, with units of Watts per cubic meter.  Then conservation of internal energy says
\begin{equation}
\rho c T_t = f + \Div (\kappa \grad T). \label{heatearly}
\end{equation}
Here $\rho$ is density and $c$ is specific heat capacity.  Assuming $\rho c$ is constant, define the ``diffusivity'' $D=\kappa/(\rho c)$ and the rescaled source term $F = f/(\rho c)$.  The revised 2D heat equation is
\begin{equation}
T_t = F + \Div (D\, \grad T). \label{heat}
\end{equation}
Figure \ref{fig:initialheat} shows a solution of this heat equation, where the initial condition is a localized ``hot spot''.  Solutions of equation \eqref{heat} always involve the spreading, in all directions, of any local heat maxima or minima, that is, diffusion.

\twofigsizes{initialheat}{finalheat}{A solution of heat equation \eqref{heat} with $D=1$ and $F=0$.  Left: Initial condition $T(0,x,y)$.   Right: Solution $T(t,x,y)$ at $t=0.02$.}{2.8in}{2.8in}

The SIA equation \eqref{sia} and the heat equation \eqref{heat} are each diffusive, time-evolving partial differential equations (PDEs).  A side-by-side comparison is illuminating:
\begin{center}
\begin{tabular}{cc}
\vspace{1mm}
SIA:\, $H$ is ice thickness & \phantom{foo bar} heat: $T$ is temperature\phantom{foo bar}  \\
\vspace{1mm}
	$H_t = M + \Div \left({\color{red}\Gamma H^{n+2} |\grad h|^{n-1}}\, \grad h \right)$  &  $T_t = F + \Div (D\, \grad T)$
\end{tabular}
\end{center}
\vspace{1mm}
Notice that the number of derivatives (one time and two space derivatives) and the signs are the same.  Surface mass balance $M$ is analogous to heat source $F$.  

The analogy suggests that we identify the \emph{diffusivity in the SIA} as:
	$$D = {\color{red}\Gamma H^{n+2} |\grad h|^{n-1}}.$$
A non-sliding SIA flow diffuses the thickness of the ice sheet.  When this $D$, a product of $\Gamma$ and the powers of $H$ and $|\grad h|$, comes out large then the diffusion acts most quickly.

This diffusion equation analogy explains generally why the surfaces of ice sheets are smooth, at least if we overlook non-flow processes like crevassing and wind-driven (snow) dunes.  There are, however, some issues with the analogy:
\begin{itemize}
\item The diffusivity $D$ depends on the solution, both the thickness $H$ and surface elevation $h$.
\item The diffusivity $D$ goes to zero at margins, where $H\to 0$, and at divides and domes, where $|\grad h|\to 0$.
\end{itemize}
More important is a deficiency of the SIA model and not \emph{per se} the analogy, namely
\begin{itemize}
\item Ice flow is much less diffusive when longitudinal (membrane) stresses are important, as when ice is floating or sliding or when the flow is confined by terrain.
\end{itemize}
But we will continue with the SIA, working toward a verified numerical scheme for \eqref{sia} in Section \ref{sec:numericalsia}.

\section{Finite difference numerics} 

Numerical schemes for the heat equation are a good starting place for solving the SIA equation \eqref{sia}.  Here we demonstrate only finite difference (FD) schemes.  These schemes replace derivatives in a differential equation by mere arithmetic.

The basic fact on which FD schemes are based is \emph{Taylor's theorem}, which says that for a smooth function $f(x)$,
	$$f(x+\Delta) = f(x) + f'(x) \Delta + \frac{1}{2} f''(x) \Delta^2 + \frac{1}{3!} f'''(x) \Delta^3 + \dots$$
You can replace ``$\Delta$'' by its multiples, for example:
\begin{align*}
f(x+2\Delta) &= f(x) + 2 f'(x) \Delta + 2 f''(x) \Delta^2 + \frac{4}{3} f'''(x) \Delta^3 + \dots \\
f(x-\Delta) &= f(x) - f'(x) \Delta + \frac{1}{2} f''(x) \Delta^2 - \frac{1}{3!} f'''(x) \Delta^3 + \dots
\end{align*}
The idea for constructing FD schemes is to combine expressions like these to give approximations of derivatives.  Thereby function values on a grid combine to approximate the differential equation.

Here we want partial derivative approximations, so we apply the Taylor's expansions one variable at a time.  For example, with a general function $u=u(t,x)$,
\begin{align*}
u_t(t,x) &= \frac{u(t+\Delta t,x) - u(t,x)}{\Delta t} + O(\Delta t), \\
u_t(t,x) &= \frac{u(t+\Delta t,x) - u(t-\Delta t,x)}{2\Delta t} + O((\Delta t)^2), \\
u_x(t,x) &= \frac{u(t,x+\Delta x) - u(t,x-\Delta x)}{2\Delta x} + O((\Delta x)^2), \\
u_{xx}(t,x) &= \frac{u(t,x+\Delta x) - 2\, u(t,x) + u(t,x-\Delta x)}{\Delta x^2} + O((\Delta x)^2)
\end{align*}
Note that if $\Delta$ is a small number then ``$+O(\Delta^2)$'' is smaller than ``$+O(\Delta)$'', so the approximation is closer when you drop it.

\subsection*{Explicit scheme for the heat equation}  We can build the simplest ``explicit'' scheme which approximates the 1D heat equation \eqref{heat1D} by observing that these two FD expressions are nearly equal:
\begin{equation}
\frac{T(t+\Delta t,x) - T(t,x)}{\Delta t} \approx D\,\frac{T(t,x+\Delta x) - 2\, T(t,x) + T(t,x-\Delta x)}{\Delta x^2}.  \label{heat1Dapproximated}
\end{equation}
The FD scheme itself is not just an approximation of a PDE, like \eqref{heat1Dapproximated}, but an actual method for computing numbers on a grid.  Let $(t_n,x_j)$ denote the time-space grid points.  Denote our approximation of the solution value $T(t_n,x_j)$ by $T_j^n$.  Then the finite difference scheme is
	$$\frac{T_j^{n+1} - T_j^n}{\Delta t} = D\,\frac{T_{j+1}^n - 2\, T_j^n + T_{j-1}^n}{\Delta x^2}.$$
To get a computable formula, let $\mu = D \Delta t / (\Delta x)^2$ and solve for $T_j^{n+1}$:
\begin{equation}
  T_j^{n+1} = \mu T_{j+1}^n + (1 - 2 \mu) T_j^n + \mu T_{j-1}^n \label{heat1Dfd}
\end{equation}

FD scheme \eqref{heat1Dfd} is \emph{explicit} because it directly computes $T_j^{n+1}$ in terms of values at time $t_n$.  Figure \ref{fig:expstencil} shows the ``stencil'' for scheme \eqref{heat1Dfd}: three values at the current time $t_n$ are combined to update the one value at the next time $t_{n+1}$.

Before moving on, notice that evaluating a heat equation solution at a grid point (i.e.~the expression ``$T(t_n,x_j)$'') generally gives a different value from the value $T_j^n$ computed by a scheme like \eqref{heat1Dfd}.  Of course we plan that these numbers will be close, but that needs checking (``verification'') or an \emph{a priori} proof.  Specifically we intend that the numbers $T(t_n,x_j)$ and $T_j^n$ become close to each other when the grid is made finer (i.e.~$\Delta t \to 0$ and $\Delta x \to 0$), as the FD expressions become closer to the derivatives they approximate.  That is, we intend our FD scheme to \emph{converge} under \emph{grid refinement}.

How accurate is scheme \eqref{heat1Dfd}?  Its construction tells us that the difference between the scheme \eqref{heat1Dfd} and the PDE \eqref{heat1D} is $O(\Delta t + (\Delta x)^2)$, so this difference goes to zero as we refine the grid in space and time, a property called \emph{consistency}.  With care about the smoothness of boundary conditions, and using mathematical facts about the heat equation itself, one can show that the difference between $T_j^n$ and $T(t_n,x_j)$ is also $O(\Delta t + (\Delta x)^2)$, which is thus the \emph{convergence rate}; see Notes and References.

To get convergence the PDE problem must generate adequately smooth solutions, and also scheme \eqref{heat1Dfd} must be \emph{stable}, which we address below.  (The main theorem for numerical PDE schemes is ``consistency plus stability implies convergence''; see Notes and References.)  In these notes we do something rather practical, namely verification.  We find problems for which we already know an exact solution $T(t,x)$, and then we compute the differences $|T_j^n - T(t_n,x_j)|$.  This determines directly whether the \emph{implementation} (i.e.~computer code form) of our FD scheme actually converges, not just in theory.

\subsection*{A first implemented scheme}  Our first Matlab implementation we consider the two spatial dimension Equation \eqref{heat} with $D$ constant and $F=0$:
\begin{equation}
T_t = D (T_{xx}+T_{yy}).\label{heat2D}
\end{equation}
Writing $T_{jk}^n \approx T(t_n,x_j,y_k)$, the 2D explicit scheme is
\begin{equation}
	\frac{T_{jk}^{n+1} - T_{jk}^n}{\Delta t} = D\,\left(\frac{T_{j+1,k}^n - 2\, T_{jk}^n + T_{j-1,k}^n}{\Delta x^2} + \frac{T_{j,k+1}^n - 2\, T_{jk}^n + T_{j,k-1}^n}{\Delta y^2}\right). \label{heat2dexplicit}
\end{equation}
The stencil for the right-hand side of \eqref{heat2dexplicit} is in Figure \ref{fig:expstencil}.

\twofigsizes{expstencil}{exp2dstencil}{Left: Space-time stencil for the explicit scheme \eqref{heat1Dfd} for the 1D heat equation.  Right: Spatial-only stencil for scheme \eqref{heat2dexplicit}.}{2.0in}{2.1in}

Scheme \eqref{heat2dexplicit} has implementation \texttt{heat.m} below.  For simplicity we set $T=0$ on the boundary of the square $-1 < x < 1$, $-1 < y < 1$.  The initial condition is gaussian: $T(0,x,y) = \exp(-30 (x^2+y^2))$.  The code uses Matlab ``colon'' notation to remove loops over spatial variables.  Here is an example run:
\begin{Verbatim}
>>  heat(1.0,30,30,0.001,20);
\end{Verbatim}
This sets $D=1.0$ and uses a $30\times 30$ spatial grid.  We take $N=20$ time steps of $\Delta t = 0.001$.  The result is shown in Figure \ref{fig:initialheat}, right.  This is the look of success.

\minput{heat}

However, very similar runs seem to succeed or fail according to some yet unclear circumstance.  For example, results from these calls are shown in Figure \ref{fig:stability}:
\begin{Verbatim}
>> heat(1.0,40,40,0.0005,100);    % Figure 7, left
>> heat(1.0,40,40,0.001,50);      % Figure 7, right
\end{Verbatim}
Both runs compute temperature $T$ on the same spatial grid, for the same final time $t_f = N \Delta t = 0.05$, but with different time steps.  The second run clearly shows instability.

\twofig{stability}{instability}{Numerically-computed temperature on $40\times 40$ grids.  The two runs are the same except that the left has $\Delta t=0.0005$ so $D\Delta t/(\Delta x)^2= 0.2$, while the right has $\Delta t=0.001$ so $D\Delta t/(\Delta x)^2= 0.4$.  Compare \eqref{stabcrit}.}

%>> heat(1.0,40,40,0.001,50);
%  doing N = 50 steps of dt = 0.00100 for 0.0 < t < 0.050
%  nu = 1 * dt / dx^2 = 0.40000
%>> print -dpdf instability.pdf
%>> heat(1.0,40,40,0.0005,100);
%  doing N = 100 steps of dt = 0.00050 for 0.0 < t < 0.050
%  nu = 1 * dt / dx^2 = 0.20000
%>> print -dpdf stability.pdf

\subsection*{Stability criteria and adaptive time stepping}  To avoid the instability shown at right in Figure \ref{fig:stability}, we need to understand the scheme better.  It turns out we have not made an implementation error, but more care is required when choosing space and time steps.

Recall the 1D explicit scheme in form \eqref{heat1Dfd}: $T_j^{n+1} = \mu T_{j+1}^n + (1 - 2 \mu) T_j^n + \mu T_{j-1}^n$.  The new value $T_j^{n+1}$ is an average of the old values, in the sense that the coefficients add to one.  Averaging is stable because averaged wiggles are smaller than the wiggles themselves.  Actually, however, the scheme is only an average \emph{if} the middle coefficient is positive, as a linear combination with coefficients which add to one is not an average if any coefficients are negative.  (For example, we would not accept 15 as an ``average'' of 5 and 7, but we can write $15 = -4 \times 5 + 5 \times 7$, and $-4+5=1$.)

So, what follows from requiring the middle coefficient in \eqref{heat1Dfd} to be positive so it computes such an average?  A \emph{stability criterion} follows, with these equivalent forms:
\begin{equation}
   1 - 2 \mu \ge 0 \quad \iff \quad \frac{D\Delta t}{\Delta x^2} \le \frac{1}{2} \quad \iff \quad \Delta t \le \frac{\Delta x^2}{2 D}.  \label{stabcrit}
\end{equation}
This condition on the size of $\Delta t$ is a \emph{sufficient} stability criterion; it is enough to guarantee stability, though something weaker might do.  In summary, for given $\Delta x$, shortening the time steps $\Delta t$ so that \eqref{stabcrit} holds will make FD scheme \eqref{heat1Dfd} into an averaging process.

Applying this same idea to the 2D heat equation \eqref{heat2D} leads to the stability condition that $1-2\mu^x-2\mu^y \ge 0$ where $\mu^x = D \Delta t / (\Delta x^2)$ and $\mu^y = D \Delta t / (\Delta y^2)$.  In the cases like those shown in Figure \ref{fig:stability} with $\Delta x=\Delta y$, this condition requires $D \Delta t /(\Delta x^2) \le 0.25$, which precisely distinguishes between the two parts of the Figure.  Runs of \texttt{heat.m} are unstable if the time step $\Delta t$ is too big relative to the spacing $\Delta x$.

The stability criterion above is easily satisfied by making each time step shorter.  Doing so at each time step makes an \emph{adaptive} implementation which can be stable even if the diffusivity $D$ is changing in time.  To show how easy this is to implement, \texttt{heatadapt.m} (not shown) is the same as \texttt{heat.m} except that the time step comes from the stability criterion, so it cannot generate the instability seen in Figure \ref{fig:stability}.  However, if the diffusivity $D$ is large or the grid spacings $\Delta x$, $\Delta y$ are small, then adaptive explicit implementations must take many short time steps to assure stability.


\subsection*{Implicit schemes}  There is an alternative stability fix instead of adaptivity, namely ``implicitness.''  For example, the finite difference scheme
\begin{equation}
  \frac{T_j^{n+1} - T_j^n}{\Delta t} = D\,\frac{T_{j+1}^{n+1} - 2\, T_j^{n+1} + T_{j-1}^{n+1}}{\Delta x^2} \label{implicit1D}
\end{equation}
is an $O(\Delta t + (\Delta x)^2)$ implicit scheme for Equation \eqref{heat1D}.  Such implicit schemes for the heat equation are stable for \emph{any} positive time step $\Delta t>0$ (``unconditionally stable''); see Notes and References.  Another well-known implicit scheme is \emph{Crank-Nicolson}, which is unconditionally stable for the heat equation, but with smaller error $O((\Delta t)^2 +(\Delta x)^2)$.

Implicit schemes are harder to implement because the unknown solution values at time step $t_{n+1}$ are treated as a vector in a large system of equations which must be formed and solved at each time step.  If the PDE is nonlinear then the system of equations may be hard to solve.  Of course, the SIA is a highly nonlinear diffusion equation.

Generally, there is a tradeoff between the easy implementability of adaptive explicit schemes and the better stability of implicit schemes.  For these notes we stay with adaptive explicit FD schemes.

\subsection*{Numerical solution of diffusion equations}  We are trying to numerically model ice flows, not heat conduction.  We have an analogy, however, which says that the SIA is diffusive like the heat equation.  In this section, because we wish to solve the SIA on real bedrock, we construct a numerical scheme for a more general diffusion equation which has an extra ``shift'' inside the gradient, namely
\begin{equation}
  T_t = F + \Div \left(D \grad (T + b)\right). \label{gendiffusion}
\end{equation}
In equation \eqref{gendiffusion}, the source term $F(x,y)$, the diffusivity $D(x,y)$, and the ``shift'' $b(x,y)$ may all vary in space.

The following code solves \eqref{gendiffusion}.  It is called by the SIA-specific schemes we build next.

\minput{diffusion}

This adaptive explicit method for the diffusion equation is conditionally stable, with the same essential time step restriction as for the constant diffusivity case, as long as we evaluate $D(x,y)$ at \emph{staggered} grid points.  That is, we use this expression for the second derivative:
\begin{align*}
\Div \left(D \grad X\right) &\approx \frac{D_{j+1/2,k}(X_{j+1,k} - X_{j,k}) - D_{j-1/2,k}(X_{j,k} - X_{j-1,k})}{\Delta x^2} \\
	&\qquad + \frac{D_{j,k+1/2}(X_{j,k+1} - X_{j,k}) - D_{j,k-1/2}(X_{j,k} - X_{j,k-1})}{\Delta y^2},
\end{align*}
where $X=T+b$.  The left part of Figure \ref{fig:diffstencil} shows the stencil.

\twofigsizes{diffstencil}{mahaffystencil}{Left:  Spatial stencil for staggered grid evaluation of diffusivity (at triangles) in the diffusion equation \eqref{gendiffusion}.  Right: Stencil showing how the staggered-grid diffusivity (triangle) can be evaluated in the SIA, from surface elevation (diamonds) and thicknesses (squares).}{2.2in}{2.2in}

The user supplies the diffusivity $D(x,y)$ to \texttt{diffusion.m} on the staggered grid.  The initial temperature $T(0,x,y)$, source term $F(x,y)$, and ``shift'' $b(x,y)$ are also supplied on the regular grid.  When using this code for standard diffusions, or for the flat-bed case of the SIA, we would take $b=0$.


\section{Numerically solving the SIA} \label{sec:numericalsia}

In SIA equation \eqref{sia} we have diffusivity $D = \Gamma H^{n+2} |\grad h|^{n-1}$.  There are two interesting aspects of this formula.  First, as already noted, $D$ goes to zero, i.e.~it ``degenerates,'' when either $H\to 0$ or $\grad h \to 0$.  Degenerate diffusion equations are automatically free boundary problems, but this aspect of the thickness evolution problem is no surprise to a glaciologist.  Determining the location of the margin is an obvious part of modelling a glacier or ice sheet.  To address this free boundary issue in our explicit time-stepping code it suffices to numerically compute new thicknesses and then set them to zero if they come out negative.

Second, for numerical stability and mass conservation we should compute $D$ on a ``staggered'' grid.  Various finite difference schemes for computing it have been proposed.  All of these schemes involve averaging $H$ and differencing $h$ in a ``balanced'' way onto the staggered grid.  In the code \texttt{siaflat.m} below we use the Mahaffy \cite{Mahaffy} method, with the stencil for computing $D$ shown in Figure \ref{fig:diffstencil}.  This code only works for the flat bed, zero surface mass balance case, but we will correct these deficiencies later.

\minput{siaflat}


\section{Exact solutions and verification}

In \texttt{siaflat.m}, which calls \texttt{diffusion.m}, we already have a fairly complicated code.  How do we make sure that such an implemented numerical scheme is correct?  Here are three proposed techniques:
\begin{enumerate}
  \item don't make any mistakes, or
  \item compare your numerical results with results from other researchers, and hope that the outliers are in error, or
  \item compare your numerical results to an exact solution.   \end{enumerate}
The last one of these, which we prefer to the first two when possible, is called ``verification.''  That is, when we build a new computer code we should test it in cases where we know the right answer.  To do so we need to return to the PDE itself, to get useful exact solutions.

\subsection*{Exact solution of heat equation}  First we consider the simpler case of the 1D heat equation with constant $D$, namely $T_t = D T_{xx}$.  Many exact solutions $T(t,x)$ to this heat equation are known, but let's consider the time-dependent ``Green's function,'' also known as the ``heat kernel''.  It starts at time $t=0$ with a delta function $T(0,x)=\delta(x)$ of heat at the origin $x=0$.  Then it spreads out over time.  It is a solution of the heat equation on the whole line $-\infty<x<\infty$ and for all $t>0$.

We will calculate this exact solution by a method which generalizes to the SIA.  The Green's function of the heat equation is ``self-similar'' over time, in the sense that it changes shape \emph{only} by shrinking the output (vertical) axis and lengthening the input (horizontal) axis, as shown in Figure \ref{fig:heatscaling}.  These scalings are related to each other by conservation of energy, which says that the total heat energy is independent of time.

\onefigsize{heatscaling}{The heat equation Green's function in 1D has the same shape at each time, but with time-dependent input- and output-scalings.}{2.4in}

In particular, the Green's function of the 1D heat equation is
  $$T(t,x) = (4 \pi D t)^{-1/2}\, e^{-x^2/(4Dt)}.$$
``Similarity'' variables for this solution, the above-mentioned scalings, involving multiplying the input and output of an invariant shape function $\phi(s) = (4 \pi D)^{-1/2}\, e^{-s^2/(4D)}$ by the same power of $t$:
\begin{equation}
s \stackrel{\text{\emph{input scaling}}}{\phantom{\Big|}=\phantom{\Big|}} t^{-1/2} x, \qquad\qquad T(t,x) \stackrel{\text{\emph{output scaling}}}{\phantom{\Big|}=\phantom{\Big|}} t^{-1/2} \phi(s).  \label{heatscalings}
\end{equation}
Note that all time-dependence is from the input and output scalings.

A numerical solver for the 1D heat equation which starts with initial values $T(t_0,x)$ taken from this exact solution should, at a later time $t$, produce numbers which are close to the exact solution $T(t,x)$; see the Exercises.

\subsection*{Halfar's similarity solution to the SIA}  Now we jump from Green's idea in about 1830 to the year 1981.  That is when P.~Halfar published the similarity solution of the SIA in the case of flat bed and zero surface mass balance.  Halfar's solution to the 2D SIA model \eqref{sia}, using a Glen exponent $n=3$, has scalings which are powers of $t$ like \eqref{heatscalings} above:
\begin{equation}
s = t^{-1/18} r, \qquad \qquad H(t,r)=t^{-1/9} \phi(s). \label{halfarscalings}
\end{equation}
Here $r=(x^2+y^2)^{1/2}$ is the distance from the origin.  These scalings are related to each other by conservation of mass, because no mass is gained or lost through the surface. Scalings \eqref{halfarscalings} imply that, quite differently from heat, the diffusion of ice slows down severely as the shape flattens out.  The powers $t^{-1/9}$ and $t^{-1/18}$ change very slowly for large times $t$.

\onefigsize{siascaling}{A case of Halfar's solution \eqref{halfar} of the SIA equation \eqref{sia} on a flat bed with zero mass balance.  The solution is shown on $H$ (m) versus $r$ (km) axes for times $t=1,10,100,1000,10000$ years.}{5.5in}

The formula for the Halfar solution to the SIA is remarkably simple given all that it accomplishes:
\begin{equation}
H(t,r) = H_0 \left(\frac{t_0}{t}\right)^{1/9} \left[1 - \left(\left(\frac{t_0}{t}\right)^{1/18} \frac{r}{R_0}\right)^{4/3}\right]^{3/7}. \label{halfar}
\end{equation}
Here the ``characteristic time'' $t_0 = (18 \Gamma)^{-1} (7/4)^3 R_0^4 H_0^{-7}$ is a parameter which can be determined by choosing center height $H_0$ and radius $R_0$.

Formula \eqref{halfar} is plotted in Figure \ref{fig:siascaling}.  We see that for times significantly greater than $t_0$ (i.e.~$t/t_0 \gg 1$) the solution changes very slowly.  For example, the change between years $1$ and $100$ is larger than that between years $1000$ and $10000$.  The \emph{volume} of ice in this Halfar ice cap is, however, constant with $t$; see the Exercises.

\subsection*{Using Halfar's solution}  Formula \eqref{halfar} is simple enough to use for verifying time-dependent SIA models.  The code \texttt{verifysia.m} (not shown) takes as input the number of grid points in each ($x,y$) direction.  It uses the Halfar solution at 200 a as the initial condition, does a numerical run of \texttt{siaflat.m} above to a final time 20000 a, and then compares to the Halfar formula for that time.  By ``compares'' we mean it computes the thickness error, the absolute values of the differences between the numerical and exact thickness solutions at the final time:
\small
\begin{verbatim}
>> verifysia(20)
average thickness error     = 22.310
>> verifysia(40)
average thickness error     = 9.490
>> verifysia(80)
average thickness error     = 2.800
>> verifysia(160)
average thickness error     = 1.059
\end{verbatim}
\normalsize
We see that the average thickness error decreases with increasing grid resolution.  This is as expected for a correctly-implemented code.  What is less obvious, perhaps, is that almost any numerical implementation mistake---almost any bug---will break this property, and these errors will not shrink.

You might ask, is the Halfar solution ever useful for modelling real ice masses?  The answer is yes.  In fact, J.~Nye and others (2000; \cite{NyeIcarus2000}) compared the long-time consequences of different flow laws for the south polar cap on Mars.  In particular, they evaluated $\text{CO}_2$ ice and $\text{H}_2\text{O}$ ice softness parameters by comparing the long-time behavior of the corresponding Halfar solutions to the observed polar cap properties.  Their conclusions:
  \begin{quote}
  \dots none of the three possible [$\text{CO}_2$] flow laws will allow a 3000-m cap, the thickness suggested by stereogrammetry, to survive for $10^7$ years, indicating that the south polar ice cap is probably not composed of pure $\text{CO}_2$ ice [but rather] water ice, with an unknown admixture of dust.
  \end{quote}
This theoretical result has since been confirmed by the observation and sampling of the polar geology of Mars.

Are exact solutions like Halfar's always available when needed?  The answer is ``no'', of course, though many ice flow models do have exact solutions which are relevant to verification; see the Notes and References.  For example, we will use van der Veen's solution for ice shelves in a later section.  On the other hand, the absence of exact solutions may show that not enough thought has gone into the continuum model itself.

\subsection*{A test of robustness}  Verification is an ideal way to start testing a code.  Another kind of test is for ``robustness''.  One asks: Does the model break when you ask it to do hard things?  Unlike for verification, we might not have precise knowledge of what it should do, but a well-implemented model should act in a ``reasonable'' way.

The robustness test in the program \texttt{roughice.m} (not shown) demonstrates that \texttt{siaflat.m} can handle an ice sheet with extraordinarily large ``driving stresses.''  Recall that glaciological driving stress is $\tau_d = - \rho g H \grad h$.  This quantity appears in the slab-on-a-slope example, and thus in the SIA model, as the value of the shear stress $(\tau_{13},\tau_{23})$ at the base of the ice.  The driving stress is, obviously, large when the ice is both thick and has steep surface slope $|\nabla h|$.

In \texttt{roughice.m} we give \texttt{siaflat.m} a randomly-generated initial ice sheet which is of the worst possible sort because it is both thick (average of 3000 m) and it has large surface slopes.  The initial shape is shown in the left side of Figure \ref{fig:roughinitial}.  During the run of 50 model years, the time step is determined adaptively from \eqref{stabcrit}, increasing from 0.0002 years to about 0.2 years as the maximum diffusivity $D$ decreases correspondingly.  The maximum value of the driving stress decreases from $57$ bar ($= 5.7\times 10^6$ Pa) to $3.6$ bar.  At the end the ice cap has the shape shown at right in Figure \ref{fig:roughinitial}.

\twofig{roughinitial}{roughfinal}{The SIA model evolves the huge-driving-stress initial ice sheet at left to the ice cap at right in only 50 model years.}

The shape at right in Figure \ref{fig:roughinitial} is rather close to a Halfar solution.  Indeed Halfar proved that all solutions of the zero-mass-balance SIA on a flat bed asymptotically approach the Halfar solution.


\section{Applying our numerical ice sheet model}

Finally we apply the model to the Antarctic ice sheet.  To do this we must first modify \texttt{siaflat.m} to allow non-flat bedrock elevation $b(x,y)$ and arbitrary surface mass balance $M(x,y)$.  Also we calve floating ice, and we enforce non-negative thickness at each timestep.  The result is \texttt{siageneral.m} (not shown), a code only ten lines longer than \texttt{siaflat.m}.

\twofigsizes{antinitial}{antfinal}{Left: Initial surface elevation (m) of Antarctic ice sheet.  Right: Final surface elevation at end of 40 ka model run on 50 km grid.}{2.55in}{3.2in}

We use measured accumulation, bedrock elevation, and surface elevation from ALBMAPv1 data \cite{LeBrocqetal2010}.  Melt is not modelled so the surface mass balance is the accumulation rate.  These input data are read from a NetCDF file and preprocessed by an additional code \texttt{buildant.m} (not shown).

\onefig{antvolcompare}{Ice volume of the modeled Antarctic ice sheet, in units of $10^6 \, \text{km}^3$, from runs on 50 km (red), 25 km (green), and 20 km (blue) grids.}

The code \texttt{ant.m} (not shown) calls \texttt{siageneral.m} to do the simulation in blocks of 500 model years.  The volume is computed at the end of each block.  Figure \ref{fig:antinitial} shows the initial and final surface elevations from a run of 40,000 model years on a $\Delta x = \Delta y = 50$ km grid.  The runtime on a typical laptop is a few minutes.

Areas of the Antarctic ice sheet with low-slope and (actual) fast-flowing ice experience thickening in the model, while near-divide ice in East Antarctica, in particular, thins.  Assuming the present-day Antarctic ice sheet is near steady state, these most-obvious thickness differences reflect model inadequacies.  The lack of a sliding mechanism explains the thickening in low-slope areas.  The lack of thermomechanical coupling, or equivalently the constancy of ice softness, explains the thinning near the divide.  And of course we should be modeling floating ice too, but the SIA is completely inappropriate to that purpose.  See section \ref{sec:shelvesandstreams} and Notes and References on modelling techniques which address these inadequacies.

Figure \ref{fig:antvolcompare} compares the ice volume time series for 50 km, 25 km, and 20 km grids.  This result, namely grid dependence of the ice volume, is typical.  One cause is that most steep gradients near the ice margin are poorly resolved, and this is true to differing degrees at these coarse resolutions.  Mainly Figure \ref{fig:antvolcompare} is a warning about the interpretation of model runs:  Even if the data is available only on a fixed grid, the model should be run at different resolutions to evaluate the robustness of the model results.


\section{Interlude: Mass continuity and kinematical equations}

Recall that in the SIA the ``stress balance'' is essentially formula \eqref{siavelocity} for the velocity.  It combines with the mass continuity equation \eqref{masscont} to give model \eqref{sia} for the ice sheet thickness.  The major SIA equation \eqref{sia} thus combines two concepts which we will now think about separately, and in greater generality, in the remainder of these notes.

The basic shallow assumption made by most ice flow theories\footnote{There are several inequivalent shallow theories: SIA, SSA, hybrids, Blatter-Pattyn, \dots} is that the surface and base of the ice are differentiable functions $z=h(t,x,y)$ and $z=b(t,x,y)$.  Thus surface overhang is not allowed, though, by contrast, the Stokes theory of slow viscous fluids only needs a closed surface in three-dimensional space as a boundary surface for the fluid.  Most ice sheet and glacier models take a map-plane perspective, however, and they have a well-defined ice thickness: $H=h-b$.

To pursue such ideas a bit further, let us state the ``kinematical equations'' which apply at upper and lower surfaces of the ice sheet.  Let $a$ be the upper surface (climatic) mass balance function ($a>0$ is accumulation) and $s$ be the basal melt rate function ($s>0$ is basal melting).  In the equations which follow these are measured in thickness-per-time units, but they could be in mass-per-area-per-time units also.  The net map-plane mass balance $M=a-s$, which already appears in the mass continuity equation \eqref{masscont}, is the difference of these surface fluxes.

The \emph{(upper) surface kinematical equation} is 
\begin{equation}
h_t = a - \mathbf{U}\big|_h \cdot \grad h + w\big|_h,  \label{surfkine}
\end{equation}
and the \emph{base kinematical equation} is
\begin{equation}
b_t = s - \mathbf{U}\big|_b \cdot \grad b + w\big|_b.  \label{basekine}
\end{equation}
(Recall $\mathbf{U}$ is the horizontal ice velocity and $w$ the vertical ice velocity.)  Equations \eqref{surfkine} and \eqref{basekine} describe the movement of the ice's upper surface and lower surfaces, respectively, from the velocity of the ice and the mass balance functions at the respective surfaces.

We can now state an important mathematical fact which follows merely from the assumption of well-defined upper and basal surface elevations.  Namely, that the surface kinematical and mass continuity equations are closely-related.  More precisely, any pair of these equations implies the third:
  \begin{itemize}
  \item the surface kinematical equation \eqref{surfkine},
  \item the base kinematical equation \eqref{basekine}, and
  \item the map-plane mass continuity equation \eqref{masscont}.
  \end{itemize}
One proves these facts by using the incompressibility of ice \eqref{incompressible} and the Leibniz rule for differentiating integrals.  The details are left for exercises.

The bedrock is often regarded as fixed (i.e.~$b_t=0$), and in fact the basal kinematical equation is often not explicitly mentioned.  Instead one gets a simplified view.  In the case of non-deformable bedrock and no sliding, for example, the basal value of the vertical velocity equals the basal melt rate.  This simplification corresponds to $b_t=0$ and $\mathbf{U}\big|_b=0$ so that $w\big|_b=-s$ from \eqref{basekine}.

\subsection*{Prognostic models}  We can now sketch the structure of a general ``prognostic,'' i.e.~ice geometry evolving, isothermal ice sheet model.  Each time step follows this recipe:
  \begin{itemize}
  \item numerically solve a stress balance, which gives velocity $\mathbf{u}=(u,v,w)$,
    \begin{itemize}
    \item[$\circ$] if the stress balance only gives $\mathbf{U}=(u,v)$, get $w$ from incompressibility \eqref{incompressible},
    \end{itemize}
  \item decide on a time step $\Delta t$ for \eqref{masscont} based on velocities and/or diffusivities,
  \item from the horizontal velocity $\mathbf{U}=(u,v)$, compute the flux $\bq = \bar{\bU} H$,
  \item update mass balance $M=a-s$ and do a time-step of \eqref{masscont} to get $H_t$,
  \item update the upper surface elevation and thickness (e.g.~$h \mapsto h + H_t \Delta t$), and repeat.
  \end{itemize}
Like most ice sheet models, we use the mass continuity equation \eqref{masscont} to describe changes in ice sheet geometry, but we could use the surface kinematical equation \eqref{surfkine} instead.

The above ``standard'' ice sheet model has many variations.  Some glaciological questions are answered just by solving the stress balance for the velocity.  Sometimes the goal is the steady state configuration of the glacier, which might be computed more quickly by iteratively solving steady state equations than by time-stepping physical evolution equations to steady state.  Other processes are usually simulated at each time step, such as the conservation of energy within the ice, or subglacial and supraglacial processes.  Understanding the diverse time scales associated to these processes is usually an important step in designing the coupled model.

When using the SIA equation \eqref{sia}, one can seemingly bypass the computation of the velocity.  That is because we could write the mass continuity equation as a diffusion, with $\bq=-D\nabla h$ for the flux instead of the more general $\bq = \bar{\bU} H$.  Fast flow in ice sheets is associated with sliding and floating ice, however, and for these flows the ice geometry evolution is not a diffusion, and so only ``$\bq = \bar{\bU} H$'' applies.  Solving the stress balance for the velocity field is then an obligatory, and usually nontrivial, step.  We consider such a stress balance next.


\section{Shelves and streams} \label{sec:shelvesandstreams}

The shallow shelf approximation (SSA) stress balance applies to ice shelves as its name suggests.  The SSA also applies reasonably well to ice streams, like those in Figure \ref{fig:siple} which have not-too-steep bed topography and low basal resistance.

\twofigsizes{siple}{streamisbrae}{Left:  The SSA model applies to ice streams like these on the Siple Coast in Antarctica.  Color shows radar-derived surface speed.  Right: Cross sections, \emph{without} vertical exaggeration, of the Jakobshavns Isbrae outlet glacier in Greenland (\textbf{a}) and the Whillans Ice Stream on the Siple Coast (\textbf{b}); this is Figure 1 in \cite{TrufferEchelmeyer}.}{2.8in}{2.9in}

But what is, and is not, an ice stream?  Ice streams slide at $50$ to $1000 \,\text{m}\,\text{a}^{-1}$, they have a concentration of vertical shear in a thin layer near base, and typically they flow into ice shelves.  Pressurized liquid water at their beds plays a critical role enabling their fast flow.  There are other fast-flowing grounded parts of ice sheets, however, called ``outlet glaciers''.  They can have even faster surface speed (up to $10 \,\text{km}\,\text{a}^{-1}$), but it is typically uncertain how much of this speed is from sliding at the base.  In an outlet glacier there is substantial vertical shear ``up'' in the ice column, sometimes caused by soft temperate ice in a significant fraction of the thickness.  Furthermore, outlet glaciers are strongly controlled by fjord-like, high slope bedrock topography.  Figure \ref{fig:siple} (right) compares the shallowness and bedrock topography of an outlet glacier and an ice stream.  Thus, few simplifying assumptions are appropriate for outlet glaciers, and the SSA may not be a sufficient model.

\subsection*{The shallow shelf approximation (SSA)}  We state this stress balance equation only in the plane flow (``flow-line'') case:
\begin{equation}
  \left(2 B H |u_x|^{1/n - 1} u_x\right)_x - C|u|^{m-1}u = \rho g H h_x \label{ssaearly}
\end{equation}
The term in parentheses is the vertically-integrated longitudinal stress, also called the ``membrane'' stress when there are two horizontal variables.  The second term $\tau_b = - C|u|^{m-1}u$ is the basal resistance, which is zero (i.e.~$C=0$) in an ice shelf.  The term on the right is the driving stress ($\tau_d = - \rho g H h_x$).  Thus the SSA equation is a balance wherein longitudinal strain rates are determined by the integrated ice hardness (i.e.~the coefficient $BH$), the slipperyness of the bed (i.e.~by the coefficient $C$ and the power $m$) and the geometry of the ice sheet (i.e.~the thickness $H$ and the surface slope $h_x$).

In \eqref{ssaearly} the velocity $u$ is independent of the vertical coordinate $z$.  We assume that the ice hardness $B=A^{-1/n}$ is also independent of depth.  Models which are not isothermal compute the vertical average of the temperature-dependent hardness.  The formula for the basal resistance $\tau_b$ is often called a ``sliding law'' in power law form.

The coefficient $\bar \nu = B |u_x|^{1/n-1}$ in \eqref{ssaearly} is called the ``effective viscosity'', so that \eqref{ssaearly} can be written
\begin{equation}
  \left(2 \,\bar \nu\, H u_x\right)_x - C |u|^{m-1} u = \rho g H h_x.  \label{ssa}
\end{equation}
In form \eqref{ssa} it is understood that the viscosity $\bar\nu$ depends on the velocity solution $u$.

The inequality ``$\,\rho H < - \rho_w b\,$'' is sometimes called the \emph{flotation criterion}.  For grounded ice we know $\rho H > - \rho_w b$ and the driving stress $\tau_d = - \rho g H h_x$ uses $h = H+b$.  On the floating side we know $\rho H < - \rho_w b$ and, by Archimedes principle, we use $h = (1-\rho/\rho_w) H$ in the driving stress.

Equation \eqref{ssa} simplifies if the ice is floating.  The ice surface elevation is proportional to the thickness if the ice is floating.  Also we assume zero resistance ($C=0$) is applied by the ocean.  Thus the SSA becomes
\begin{equation}
   \left(2 \,\bar\nu\, H u_x\right)_x = \rho g (1-\rho/\rho_w) H H_x \label{ssafloat}
\end{equation}
for floating ice.  A useful observation about flow line equation \eqref{ssafloat} is that both left- and right-hand expressions are derivatives; this can be used to build a 1D exact solution.

\subsection*{Steady ice shelf exact solution}  For a steady 1D ice shelf, in which $H_t=0$, the mass continuity equation \eqref{masscont} reduces to $M=(uH)_x$.  Because of the relative simplicity of the SSA equation \eqref{ssafloat} and the steady mass continuity equation for 1D floating ice, the exact velocity and thickness for a steady ice shelf can be computed \cite{vanderVeen83}.  This exact solution depends on the ice thickness $H_g$ and velocity $u_g$ at the grounding line.  For the surface mass balance $M$ we choose a positive constant $M_0$.  These choices determine a unique solution, the derivation of which is left to the exercises.

Supposing $H_g=500$ m, $u_g = 50 \,\text{m}\,\text{a}^{-1}$, and $M_0=30 \,\text{cm}\,\text{a}^{-1}$ we get the results in Figure \ref{fig:steadyshelfprofile}, which are from code \texttt{exactshelf.m} (not shown).  We will use this exact solution to verify a numerical SSA code.  Note that driving stresses are much higher near the grounding line than away from it, and thus the highest longitudinal stresses, strain rates, and thinning rates occur near the grounding line.

\twofig{steadyshelfprofile}{steadyshelfvelocity}{The upper and lower surface elevation (m; left) of the exact ice shelf solution and its velocity (m/a; right); $x=0$ is the grounding line.}

\subsection*{Numerical solution of the SSA}  Suppose the ice thickness is a fixed function $H(x)$.  To find the velocity we must solve the nonlinear PDE \eqref{ssa} or \eqref{ssafloat} for the unknown $u(x)$.  When we do this numerically an iteration is needed because of the nonlinearity.  The simplest iteration idea is to use an initial guess at the velocity, which allows us to compute an effective viscosity and then get a new velocity solution from a linear PDE problem.  Then we recompute the effective viscosity, solve for a new velocity, and repeat until things stop changing.  This is often called a ``Picard'' iteration, in contrast to a ``Newton'' iteration which should converge faster.

Denote the previous velocity iterate as $u^{(k-1)}$ and the current iterate as $u^{(k)}$.  Compute $\bar \nu^{(k-1)} = B |u^{(k-1)}_x|^{1/n-1}$ and define $W^{(k-1)} = 2 \bar \nu^{(k-1)} H$.  Solving this linear elliptic PDE for the unknown $u^{(k)}$ is a Picard iteration for \eqref{ssa}:
\begin{equation}
   \left(W^{(k-1)} u^{(k)}_x\right)_x - C |u^{(k-1)}|^{m-1} u^{(k)} = \rho g H h_x. \label{picardssa}
\end{equation}
If the difference between $u^{(k-1)}$ and $u^{(k)}$ were zero then we would have a solution of \eqref{ssa}, while in practice we stop the iteration when the difference is smaller than some tolerance.

Equation \eqref{picardssa} is a linear boundary value problem.  We can write it abstractly
\begin{equation}
  \left(W(x)\, u_x\right)_x - \alpha(x)\, u = \beta(x)  \label{innerlinear}
\end{equation}
where the functions $W(x)$, $\alpha(x)$, $\beta(x)$ are known.  Equation \eqref{innerlinear} applies on an interval of the $x$-axis.  For one boundary condition we will suppose that $x=x_g$ is a location where the velocity is known, $u(x_g)=u_g$, as in Figure \ref{fig:steadyshelfprofile}.  In the ice shelf case we also have the calving front condition (see Notes and References)
\begin{equation}
  2 B H |u_x|^{1/n - 1} u_x = \frac{1}{2}\rho (1-\rho/\rho_w) g H^2  \label{calvingstress}
\end{equation}
at the end of the ice shelf $x=x_c$.  Boundary condition \eqref{calvingstress} can be solved for $u_x(x_c)=\gamma$ in terms of known quantities including the thickness at the calving front.

Where to get an initial guess $u^{(0)}$?  Generally this may require effort, but we will use these choices for our 1D case.  For floating ice, an initial velocity comes from assuming a uniform strain rate provided by the calving front condition: $u^{(0)}(x) = \gamma (x-x_g) + u_g$.  For grounded ice, we may assume ice is held by basal resistance only: $u^{(0)}(x) = \left(-C^{-1} \rho g H h_x\right)^{1/m}$.

\subsection*{Numerics of the linear boundary value problem}  Suppose equation \eqref{innerlinear} applies on $[x_g,x_c]=[0,L]$.  We choose a grid with equal spacing $\Delta x$ and index $j=1,2,\dots,J+1$, so that $x_1 = 0$ and $x_{J+1} = L$ are endpoints.  The coefficient $W(x)$ is needed on a staggered grid, for stability and accuracy reasons similar to those for the SIA diffusivity.  Our finite difference approximation of \eqref{innerlinear} is, therefore,
\begin{equation}
  \frac{W_{j+1/2} (u_{j+1} - u_j) - W_{j-1/2} (u_{j} - u_{j-1})}{\Delta x^2} - \alpha_j u_j = \beta_j  \label{discreteinnerlinear}
\end{equation}

For the left end boundary condition we have $u_1 = u_g$ given, which is easy to include in the linear system (below).  For the right end boundary condition we have $u_x(L)=\gamma$, which requires more thought.  First introduce a notional point $x_{J+2}$.  Now require $(u_{J+2} - u_J)/(2 \Delta x) = \gamma$, which is a centered approximation to ``$u_x(x_c)=\gamma$.''  Using equation \eqref{discreteinnerlinear} in $j=J+1$ case, eliminate the $u_{J+2}$ variable ``by-hand''.  This determines the form of the last equation in our linear system.

Now observe that each iteration to solve the SSA stress balance has the form
\begin{equation}
   A \mathbf{v} = \mathbf{b}. \label{Aveqb}
\end{equation}
Indeed, at each location $x_1,\dots,x_{J+1}$ we can write an equation, including a row of the matrix $A$ in \eqref{Aveqb}, involving the unknown velocities.  It is this linear system of $J+1$ equations:
\begin{equation}
\begin{bmatrix}
1 &  &  &  &  \\
W_{3/2} & A_{22} & W_{5/2} &  &  \\
 & W_{5/2} & A_{33} &  &  \\
 &  & \ddots & \ddots &  \\
 &  & W_{J-1/2} & A_{JJ} & W_{J+1/2} \\
 &  &  & A_{J+1,J} & A_{J+1,J+1} \\
\end{bmatrix}\,
\begin{bmatrix}
u_1 \\ u_2 \\ u_3 \\ \vdots \\ u_J \\ u_{J+1}
\end{bmatrix}
=
\begin{bmatrix}
u_g \\ \beta_2 \Delta x^2 \\ \beta_3 \Delta x^2 \\ \vdots \\ \beta_J \Delta x^2 \\ b_{J+1}
\end{bmatrix}  \label{discretematrixform}
\end{equation}
The diagonal entries ``$A_{ij}$'' are
  $$A_{22} = -(W_{3/2}+W_{5/2}+\alpha_2 \Delta x^2), \quad \dots, \quad A_{JJ} = -(W_{J-1/2}+W_{J+1/2}+\alpha_J \Delta x^2),$$
except for special cases for the coefficients in the last equation,
  $$A_{J+1,J} = 2 W_{J+1/2}, \quad A_{J+1,J+1} = -(2 W_{J+1/2}+\alpha_{J+1}\Delta x^2).$$
For the right side of the last equation, $b_{J+1} = -2 \gamma \Delta x W_{J+3/2} + \beta_{J+1} \Delta x^2$.

System \eqref{discretematrixform} is a tridiagonal linear system.  But don't bother looking up how to solve such a linear system unless you really need to!  It is fully appropriate to give system \eqref{discretematrixform} to Matlab's linear solver, the ``backslash'' operator $\mathbf{v} = A\, \backslash\, \mathbf{b}$, especially at this initial implementation stage.  In these notes we will not worry further about solving finite linear systems.  We now have a code to solve \eqref{innerlinear} by finite differences and linear algebra, namely \texttt{flowline.m} below.

\minput{flowline}

By ``manufacturing'' exact solutions to \eqref{innerlinear}---see Notes and References---we can test this first piece of our SSA-solving codes before proceeding to solve the actual nonlinear SSA problem.   In fact, results from \texttt{testflowline.m} (not shown) demonstrate that our implemented numerical scheme converges at the optimal rate $O(\Delta x^2)$.

\subsection*{Solving the stress balance for an ice shelf}  The code \texttt{ssaflowline.m} (below) numerically computes the velocity for an ice shelf.  The thickness is assumed to be given, so we are not yet addressing the full, ``coupled'' ice shelf problem, simultaneously solving the applicable mass continuity \eqref{masscont1D} and stress balance \eqref{ssafloat} equations.  We are only solving the latter.

This code implements Picard iteration \eqref{picardssa}, in the floating case, to solve the nonlinear equation \eqref{ssafloat}.  It calls \texttt{ssainit.m} (not shown) to get the initial iterate $u^{(0)}(x)$, as already described, and it calls \texttt{flowline.m} at each iteration.  It also calls small helper functions \texttt{stagav(),regslope(),stagslope()}, at the end of the code, to computed certain gridded values.

\minput{ssaflowline}

Now we can ask precisely: Does \texttt{ssaflowline.m} work correctly?  The exact velocity solution shown in Figure \ref{fig:steadyshelfprofile}, computed by \texttt{exactshelf.m}, allows us to compare the numerical to the exact velocities by finding the maximum difference between them.  For this to work we take the exact thickness shown in Figure \ref{fig:steadyshelfprofile}, also from \texttt{exactshelf.m}.  A convergence comparison, shown in Figure \ref{fig:shelfconv}, is done by codes \texttt{testshelf.m} and \texttt{shelfconv.m} (not shown).  Each circle in the Figure gives the maximum velocity error on a given grid.

\onefig{shelfconv}{The numerical SSA velocity solution from \texttt{ssaflowline.m} converges to the exact solution, at nearly the optimal rate $O(\Delta x^2)$, as the grid is refined from spacing $\Delta x=4$ km to $\Delta x=62$ m.}

Even on the coarsest $\Delta x = 4$ km grid we see in Figure \ref{fig:shelfconv} that the maximum velocity error (i.e.~difference) is less than 1 m/a, while the maximum velocity itself is $\sim 300$ m/a.  We can conclude from this comparison that, at screen resolution, our numerical velocity solutions are essentially identical to that shown in the right part of Figure \ref{fig:steadyshelfprofile}.  There is not even a reason to show the numerical solutions!

\subsection*{Realistic ice shelf modelling}  Real ice shelves have two horizontal variables.  They are frequently confined in bays, and thus they experience ``side drag''.  Their velocities vary spatially and temporally along their grounding lines, which are the curves where the flotation criterion is an equality.  Furthermore real ice shelves have interesting boundary processes, including high basal melt near grounding lines, marine ice basal freeze-on, and fracturing which nears full thickness at the calving front.  It is a bit complicated.

Nonetheless ``diagnostic'' (i.e.~fixed geometry) ice shelf modelling in two horizontal variables, done like the above example where the velocity is unknown but the thickness is known and fixed, is quite successful using only the isothermal SSA model.  For example, Figure \ref{fig:rossquiver} shows a Parallel Ice Sheet Model (PISM) result for the Ross ice shelf, compared to observed velocities.  There is only one tuned parameter, the constant value of the ice hardness $B$.  In this run, observed velocities for grounded ice were applied as boundary conditions.  Many current ice shelf models yield comparable match \cite{MacAyealetal}.

\twofigsizes{rossquiver}{rossscatter}{Results from PISM.  Left: Observed (white) and modeled (black) ice velocities are nearly coincident across the whole Ross ice shelf.  The grounding line is the thin black curve.  Right: In this scatter plot there is one point for each arrow at left.}{3.0in}{3.0in}


\section{A summary of numerical ice sheet modelling}

These notes are brief, and so they give a very incomplete view of numerical models for glaciers and ice sheets.  They do, however, illustrate some general principles about numerical modelling.  One should:
\begin{itemize}
\item Return often to the continuum model.
\item Modularize codes.
\item Test the parts: Is the component robust? Does it show convergence?
\end{itemize}

Regarding the specific ice flow models covered in these notes, here are three high-level points, as a meager conclusion:
\begin{itemize}
\item The mass continuity equation is the part of an ice sheet model which describes how the ice geometry evolves.  It is a kind of transport equation in the map-plane, but with diffusive character at larger spatial scales.  The numerical approach to this equation depends on which is the stress balance which supplies the ice velocity or ice flux.  Mass continuity is a diffusion for frozen bed, large scale flows, and in that case the SIA is a good choice.  Mass continuity is \emph{not} very diffusive for membrane stresses (e.g.~SSA), especially with no basal resistance as in ice shelves.  It has some diffusiveness for ice streams, though how much is hard to quantify.
\item The SIA stress balance is exceptional because it is not horizontally-distributed.  In the SIA, velocity follows immediately by vertical integration of the driving stress.
\item Membrane stress balance equations like the SSA (and the Blatter-Pattyn, hydrostatic, and Stokes models also) determine horizontal velocity from geometry and boundary conditions.  Because of the Glen law these equations are nonlinear, so iteration is necessary.  At each iteration a sparse matrix ``inner'' problem is solved; non-experts should give this matrix problem to a solver package.
\end{itemize}



%\small
\section{Notes} \label{sec:nr}

Recent and recommended books and reviews which extend the continuum modeling content of these notes include \cite{CuffeyPaterson,GreveBlatter2009,SchoofHewitt2013,vanderVeen}.

The SIA model, which was derived by several authors \cite{FowlerLarson1978,Hutter,MorlandJohnson}, follows by scaling the Stokes equations using the aspect ratio $\eps = [H]/[L]$, where $[H]$ is a typical thickness of an ice sheet and $[L]$ is a typical horizontal dimension.  After scaling one drops the terms that are small if $\eps$ is small \cite{Fowler,Hutter}; this is a ``small-parameter argument''.  In one scaling there are no $O(\eps)$ terms in the scaled equations so one only drops $O(\eps^2)$ terms \cite{Fowler}.  The SIA is re-formulated as a well-posed free boundary problem in \cite{JouvetBueler2012}, which provides the correct boundary condition at grounded margins.  The Mahaffy \cite{Mahaffy} scheme for diffusivity used here is not the only one \cite{HindmarshPayne}.

The SSA model \cite{WeisGreveHutter} was derived in \cite{Morland} for ice shelves and in \cite{MacAyeal} for ice streams.  In deriving the SSA, the aspect ratio $\eps$ above is one small parameter but additionally a second parameter describing the magnitude of surface undulations must be assumed to be small  \cite{SchoofStream,SchoofHindmarsh}.  A well-posed model for the emergence of ice streams though till failure, using only the SSA, is in \cite{SchoofStream}.

A key modelling issue omitted in these notes is thermomechanical coupling.  Temperature is important because the ice softness varies by three orders of magnitude in the temperature range relevant to ice sheet modelling.  Ice temperature therefore gives ice sheet dynamics a long memory of past climate, and because the geothermal flux is a significant input in slow-flowing parts of ice sheets.  Equally important, dissipation of gravitational potential energy is a major part of the energy balance, and basal melt in particular.  For example, each year the ice in the Jakobshavn drainage basin in Greenland dissipates enough gravitational potential energy to fully melt more than $1\,\text{km}^3$ of ice \cite{AschwandenBuelerKhroulevBlatter}.  Beautiful evidence that, as a result, outlet glaciers have thick temperate ice is in \cite{Luethietal2009}.  These physical effects motivate modelers to solve the conservation of energy equation simultaneously with the mass conservation (continuity) and momentum conservation (stress balance) equations.  Traditionally the conservation of energy equation uses only temperature as the state variable \cite{BBL}, and this may be suitable for cold ice sheets, but ice sheets are generically polythermal.  Enthalpy methods are a good way to track the energy content of polythermal ice sheets and glaciers \cite{AschwandenBuelerKhroulevBlatter}, though one can also have a separate water-content equation for temperate ice \cite{Greve}.  In any case, the conservation of energy equation is strongly advection-dominated in general \cite{BBL}.

Pressurized basal water is required for most ice sliding.  To model the production of such water in ice sheets one must at least compute the ice base temperature and the basal melt rate through the energy conservation equation \cite{BBssasliding,Clarke05,Raymondenergy,Tulaczyketal2000b}.

One of the most significant issues in modelling ice sheets using shallow models is to describe the ``switch'', in space and time, between shear-dominated and membrane-stress-dominated flow.  It is not a good idea to abruptly switch from the SIA model to the SSA model at the edge of an ice stream, by whatever criterion that switch might be applied, though this has been attempted \cite{HulbeMacAyeal,Ritzetal2001}.  However, ``hybrid'' schemes exist which solve the SIA and SSA everywhere in the ice sheet \cite{BBssasliding,Winkelmannetal2011}, or solve a related vertically-integrated model \cite{Goldberg2011,PollardDeConto}, then combining the stresses or velocities according to different schemes.

``Higher-order'' three-dimensional approximations of the Stokes stress balance, such as the Blatter-Pattyn model \cite{Blatter,Pattyn03}, also use shallow approximations, at minimum including both the most-basic shallow assumption of well-defined thickness (see main text) \emph{and} an assumption of hydrostatic normal stress \cite{GreveBlatter2009}.  Computational limitations generally restrict either the spatial extent, the spatial resolution, or the run duration of these more complete models, primarily because 3D stress balances involve more memory.  Vertically-integrated hybrids can generally be used at higher spatial resolution and longer time scales than higher-order models because the 2D stress balance equations are easier to solve.

As both the SIA and the SSA are derived by small-parameter arguments from the Stokes equations, one might ask whether there is a common shallow antecedent model of both SIA and SSA?  Schoof and Hindmarsh \cite{SchoofHindmarsh} answer that Blatter-Pattyn is one.

Solving the Stokes stress balance itself \cite{JouvetRappaz2011,Lengetal2012,ISMIPHOM} requires explicit accounting for incompressibility through a pressure variable.  Numerical approximations of this stress balance are indefinite, thus harder to solve, essentially because incompressibility is an equality constraint.  In plane flow one can address the incompressibility constraint by using stream functions \cite{BaliseRaymond1985}.  Questions remain about what are the most important deficiencies, relative to the Stokes model, when using either higher-order \cite{ISMIPHOM} or hybrid models.

The finite difference material in these notes should probably be read with reference \cite{MortonMayers} or similar in hand.  The ``main theorem for numerical PDE schemes'' mentioned in the text is the Lax equivalence theorem \cite{MortonMayers}.  Alternative numerical discretization techniques include the finite element \cite{Braess}, finite volume \cite{LeVeque}, and spectral \cite{Trefethen} methods.  Newton iteration for the nonlinear discrete equations is superior to Picard iteration used here, in terms of rapid convergence once iterates are near the solution, but implementation care is needed \cite{Kelley}.

Which are the best numerical models for moving grounding lines?  Even when the minimal SSA stress balance is used, this is still something of an open question \cite{Goldbergetal2009,MISMIP3d2013,MISMIP2012,SchoofMarine1}.  The physics requires at least that the quantities $H$ and $u$ are continuous there, but several stress balance regimes exist near the grounding line, with increasing complexity as one focusses-in on the line \cite{SchoofMarine2}.

Where to find exact solutions for ice flow models?  The textbook Greve and Blatter \cite{GreveBlatter2009} has a few.  Halfar's similarity solution to the SIA \cite{Halfar81,Halfar83} has been generalized to non-zero mass balance \cite{BLKCB}.  There are flow-line \cite{Bodvardsson,vanderVeen83} and cross-flow \cite{SchoofStream} solutions to the SSA model, and one can even construct an exact, steady marine ice sheet in the flow-line case \cite{Bueler2014exactmarine}.  For the Stokes equations themselves there are plane flow solutions for constant viscosity \cite{BaliseRaymond1985}.

As a last resort for numerical verification, one can ``manufacture'' exact solutions by starting with a specified solution and then deriving a source term so that the specified function is actually a solution \cite{Roache}.  There are such manufactured solutions to the thermomechanically-coupled SIA \cite{BBL}, plane flow Blatter-Pattyn model \cite{GlowinskiRappaz}, and Glen-law Stokes equations \cite{JouvetRappaz2011,Lengetal2012,SargentFastook2010}.

\clearpage\newpage
\footnotesize

%\bibliography{ice-bib}
%\bibliographystyle{siam}
\documentclass[letterpaper,final,12pt,reqno]{amsart}

\usepackage[total={6.3in,9.2in},top=1.1in,left=1.1in]{geometry}

\usepackage{verbatim}
\usepackage{empheq}
\usepackage[dvipsnames]{xcolor}
\usepackage{animate}
\usepackage{graphicx}
\usepackage{fancyvrb}

%\usepackage{palatino}

% hyperref should be the last package we load
\usepackage[pdftex,
colorlinks=true,
plainpages=false, % only if colorlinks=true
linkcolor=blue,   % only if colorlinks=true
citecolor=Red,   % only if colorlinks=true
urlcolor=ForestGreen     % only if colorlinks=true
]{hyperref}

\pdfinfo{
/Title (Numerical modelling of ice sheets, streams, and shelves)
/Author (Ed Bueler)
/Subject (numerical modelling of ice sheets)
/Keywords (numerical modelling, numerical analysis, glacier, ice sheet, ice shelf, shallow models)
}

\renewcommand{\baselinestretch}{1.05}

\newcommand{\ddt}[1]{\ensuremath{\frac{\partial #1}{\partial t}}}
\newcommand{\ddx}[1]{\ensuremath{\frac{\partial #1}{\partial x}}}
\newcommand{\ddy}[1]{\ensuremath{\frac{\partial #1}{\partial y}}}
\newcommand{\pp}[2]{\ensuremath{\frac{\partial #1}{\partial #2}}}
\renewcommand{\t}[1]{\texttt{#1}}
\newcommand{\Matlab}{\textsc{Matlab}\xspace}
\newcommand{\bq}{\mathbf{q}}
\newcommand{\bu}{\mathbf{u}}
\newcommand{\bU}{\mathbf{U}}
\newcommand{\eps}{\epsilon}
\newcommand{\grad}{\nabla}
\newcommand{\Div}{\nabla\cdot}
\newcommand{\devstress}{\tau}

\newcommand{\minput}[1]{
\vspace{0.8cm}
\VerbatimInput[frame=single,framesep=3mm,label=\fbox{\normalsize \textsl{\,#1.m\,}},fontfamily=courier,fontsize=\footnotesize]{tmp/#1.slim.m}
\vspace{0.5cm}
}

% usage:  \onefigsize{name}{caption}{width}
\newcommand{\onefigsize}[3]{
\begin{figure}[ht]
\centering
\includegraphics[width=#3,keepaspectratio=true]{#1}
\caption{#2}
\label{fig:#1}
\end{figure}}

% usage:  \onefig{name}{caption}
\newcommand{\onefig}[2]{\onefigsize{#1}{#2}{3.0in}}

% usage:  \twofigsizes{left-name}{right-name}{caption}{left-width}{right-width}
\newcommand{\twofigsizes}[5]{
\begin{figure}[ht]
\centering
\includegraphics[width=#4,keepaspectratio=true]{#1} \quad
\includegraphics[width=#5,keepaspectratio=true]{#2}
\caption{#3}
\label{fig:#1}
\end{figure}}

% usage:  \twofig{left-name}{right-name}{caption}
\newcommand{\twofig}[3]{\twofigsizes{#1}{#2}{#3}{2.5in}{2.5in}}



\begin{document}
\graphicspath{{../photos/}{../pdffigs/}}

\begin{titlepage}

  \begin{center}
  \phantom{foo}
    \vspace{1.0cm}

     {\Large \textsc{Numerical modelling}}
    \vspace{0.7cm}

     {\Large \textsc{of ice sheets, streams, and shelves}}

    \vspace{1.5cm}

    {\large Ed Bueler}
    \vspace{1cm}

    Summer School in Glaciology, McCarthy Alaska, June 2016 

    \vfill
    
    \includegraphics[width=6.0in]{flowline}
  
    \scriptsize \emph{Illustrates the notation used in these notes.  Figure modified from \cite{SchoofMarine1}.} \normalsize
    
    \vspace{1.5in}
  \end{center}
\end{titlepage}

\clearpage\newpage

%\setcounter{section}{1}
\section{Introduction}

The most common use of numerical models in glaciology may be to help you ask: When I put together my incomplete understanding of glacier processes into a mathematical model, does the combination behave as I expect?  Numerical models can at least demonstrate flaws in our understanding of glacier processes, and they can show us how these processes combine to give overall behavior, but they should be built with care.  The worst outcome is to spend time (and perhaps reputation) explaining, through physical argumentation and perhaps using observational evidence, numerical model behavior that is an artifact of poor computer programming or numerical analysis.

So the reader of these notes may be surprised that a continuum model, and not a computer code, seems to be our focus much of the time.  While all codes produce some numbers, we want numbers that actually come from our continuum model.  We will therefore analyse numerical implementations to see if they match the continuum model and its solutions.

These notes have a limited scope:
  \begin{quote}\emph{shallow approximations of ice flow.}\end{quote}
They adopt a constructive approach; we provide:
  \begin{quote}\emph{example numerical codes that actually work.}\end{quote}
Within our scope are the shallow ice approximation (SIA) in two horizontal dimensions (2D), the shallow shelf approximation (SSA) in 1D, and the mass continuity and surface kinematical equations.  We recall the Stokes model, but we do not address its numerical solution.  Our numerical concepts include finite difference schemes, solving algebraic systems from stress balances, and the verification of codes using exact solutions.

Our notation, which generally follows \cite{GreveBlatter2009}, is common in the glaciological literature, but see Table \ref{tab:notation}.  Cartesian coordinates $x,y,z$ have $z$ perpendicular to the geoid and positive-upward.  If these coordinates or ``$t$'' appear as subscripts then they denote partial derivatives: $u_x = \partial u/\partial x$.  Tensor notation uses subscripts from the list $\{1,2,3,i,j\}$.  For example, ``$\tau_{ij}$'' or ``$\tau_{13}$'' denote entries of the deviatoric stress tensor.

These notes are based on eighteen Matlab codes, each about one-half page.  All have been tested in Matlab and Octave.  They are distributed by cloning the repository
\begin{quote}
\url{https://github.com/bueler/mccarthy}
\end{quote}
\noindent and looking in the \texttt{mfiles/} subdirectory.  Though only five of the codes are printed here, with their comments stripped for compactness and clarity, the electronic versions have generous comments and help files.

\begin{table}[ht]
\caption{Notation used in these notes, with values for some constants.}
\begin{tabular}{clll}
variable  & description & SI units & value \\
\hline
$A$ & $A=A(T)=$ ice softness in the Glen flow law & $\text{Pa}^{-n}\,\text{s}^{-1}$ \\
$B$ & ice hardness; $B=A^{-1/n}$ & $\text{Pa}\,\text{s}^{1/n}$ \\
$b$ & bedrock elevation & m \\
$c$ & specific heat in general & J kg$^{-1}$ K$^{-1}$ \\
$\nabla$ & (spatial) gradient & m$^{-1}$ \\
$\nabla\cdot$ & (spatial) divergence & m$^{-1}$ \\
$\mathbf{g}$ & gravity & m s$^{-2}$\phantom{foobar} & 9.81 \\
$H$ & ice thickness & m \\
$h$ & ice surface elevation & m \\
$\kappa$ & conductivity in general & J s$^{-1}$ m$^{-1}$ K$^{-1}$ \\
$M$ & climatic mass balance & m s$^{-1}$ \\
$n$ & exponent in Glen flow law & & 3 \\
$\nu$ & viscosity & Pa s \\
$p$ & pressure & Pa \\
$\bq$ & map-plane ice flux: $\bq = \int_{b}^{h} \bU\,dx = \bar{\bU} H$ & $\text{m}^2\,\text{s}^{-1}$ \\
$\rho$ & (1) density in general & kg m$^{-3}$ & \\
  & (2) density of ice & kg m$^{-3}$ & 910 \\
$\rho_w$ & density of sea water & kg m$^{-3}$ & 1028 \\
$T$ & temperature & K \\
$\tau$ & magnitude of $\tau_{ij}$: $2 \tau^2 = \sum_{ij} \tau_{ij}^2$ & Pa \\
$\tau_{ij}$ & deviatoric stress tensor & Pa \\
$Du_{ij}$ & strain rate tensor & s$^{-1}$ \\
$\mathbf{U}$ & $=(u,v)$ horizontal ice velocity & m s$^{-1}$ \\
$\mathbf{u}$ & $=(u,v,w)$ 3D ice velocity & m s$^{-1}$ \\
\end{tabular}
\label{tab:notation}
\end{table}


\section{Ice flow equations}

My first goal in these notes is to get to an equation for which I can say:
\begin{center}
\emph{by numerically solving this equation, we have a usable model for an ice sheet.}
\end{center}
\noindent A ``usable'' model tends to be \emph{understood} as much as it is \emph{correct}.  Also, this first model will not be complete by any modern standard.  To get to my goal I first (briefly!) recall the continuum mechanical equations of ice flow.  

Ice in glaciers is a moving fluid so we describe its motion by a velocity field $\mathbf{u}(t,x,y,z)$.  If the ice fluid were faster-moving than it actually is, and if it were linearly-viscous like liquid water, then it would be a ``typical'' incompressible fluid.  We would use the Navier-Stokes equations as the model:
\begin{align}
\nabla \cdot \mathbf{u} &= 0 &&\text{\emph{incompressibility}} \label{incompressible} \\
\rho \left(\mathbf{u}_t + \mathbf{u}\cdot\nabla \mathbf{u}\right) &= -\nabla p + \nabla \cdot (\nu \nabla \mathbf{u}) + \rho \mathbf{g} &&\text{\emph{stress balance}} \label{navierstokes}
\end{align}
In equation \eqref{navierstokes} the term $\mathbf{u}_t + \mathbf{u}\cdot\nabla \mathbf{u}$ is an acceleration.  The right-hand side of \eqref{navierstokes} is the total stress, and so equation \eqref{navierstokes} says ``$ma=F$'', i.e.~it is Newton's second law.  Much time has been spent to get partial understanding of the rich solutions of these Navier-Stokes equations; a book-length introduction like \cite{Acheson} is recommended.  The numerical solution of these equations is \emph{computational fluid dynamics} (CFD).

But, is numerical ice flow modelling a part of CFD?  Does a well-written general-purpose CFD text like \cite{Wesseling} help the glaciers student?  Ice sheet flow is a large-scale fluid problem like atmosphere and ocean circulation in climate systems, but it is an odd one.  Consider some topics which might make ocean circulation modelling exciting, for example:
  \begin{center} turbulence \qquad convection \qquad  coriolis force  \qquad density stratification
  \end{center}
None of these topics are relevant to ice flow.  What could be interesting about the flow of slow and old, and surely boring, ice?

First observe that ice is indeed a slow fluid.  In terms of equation \eqref{navierstokes}, ``slow'' means $\rho \left(\mathbf{u}_t + \mathbf{u}\cdot\nabla \mathbf{u}\right) \approx 0$, which says that the forces (stresses) of inertia are negligible.  However, ice is also a shear-thinning fluid with a specific kind of nonlinearly-viscous (``non-Newtonian'') behavior in which larger strain rates imply smaller viscosity.  The viscosity $\nu$ in \eqref{navierstokes} is therefore not constant, and so we separately state an empirically-based flow law below.

\subsection*{Stokes equations}  So now the standard model for isothermal flow is this set of Stokes equations:
\begin{align}
\nabla \cdot \mathbf{u} &= 0 &&\text{\emph{incompressibility}} \label{incompressibleagain} \\
0 &= - \nabla p + \nabla \cdot \tau_{ij} + \rho \mathbf{g} &&\text{\emph{stress balance}} \label{forcebalance} \\
Du_{ij} &= A \tau^2 \tau_{ij} &&\text{\emph{$n$=3 Glen flow law}} \label{flowlaw}
\end{align}
In the flow law \eqref{flowlaw}, the deviatoric stress tensor $\tau_{ij}$ and the strain rate tensor $Du_{ij}$ appear; previous lectures cover these.  Here we merely note that: $Du_{ij} = (1/2)((u_i)_{x_j}+(u_j)_{x_i})$ if we index coordinates by $x_1,x_2,x_3=x,y,z$, each tensor in \eqref{flowlaw} is symmetric and has trace zero, and $\tau^2 = (1/2) \tau_{ij} \tau_{ij}$ defines ``$\tau^2$'' in \eqref{flowlaw} if we use the summation convention.

The Stokes equations do not contain a time derivative.  Thus boundary stresses, the force of gravity $\rho \mathbf{g}$, and ice softness $A$ together determine the velocity and stress fields (i.e.~$\bu$, $p$, $\tau_{ij}$) instantaneously.  Thus ice flow simulation codes have no memory of prior momentum or velocity.  Said another way, velocity is a ``diagnostic'' output of ice flow codes, because it is not needed for (re)starting a simulation.

\subsection*{Plane-flow Stokes equations}  Consider now the $x,z$-plane case of equations \eqref{incompressibleagain}, \eqref{forcebalance}, and \eqref{flowlaw}.  ``Plane-flow'' means that velocity component $v$ is zero and that all derivatives with respect to $y$ are zero:
\begin{align}
u_x + w_z &= 0 &&\text{\emph{incompressibility}} \label{incompressiblexz} \\
p_x &= \tau_{11,x} + \tau_{13,z} &&\text{\emph{stress balance} ($x$)} \label{stokespx} \\
p_z &= \tau_{13,x} - \tau_{11,z} - \rho g &&\text{\emph{stress balance} ($z$)} \label{stokespz} \\
u_x &= A \tau^2 \tau_{11} &&\text{\emph{flow law} (diagonal)}  \label{forceflowx} \\
u_z + w _x &= 2 A \tau^2 \tau_{13} &&\text{\emph{flow law} (off-diagonal)} \label{forceflowz}
\end{align}
Note that $\tau_{13}$ is a shear stress while $\tau_{11}$ and $\tau_{33}=-\tau_{11}$ are deviatoric longitudinal stresses.  Also $\tau^2 = \tau_{11}^2+\tau_{13}^2$ in this case.  Equations \eqref{incompressiblexz}--\eqref{forceflowz} form a system of five nonlinear equations in five scalar unknowns ($u,w,p,\tau_{11},\tau_{13}$).

\subsection*{Slab-on-a-slope}  Equations \eqref{incompressiblexz}--\eqref{forceflowz} are complicated enough to make us pause before jumping in to numerical solution methods, but  we can handle a simplified situation first.  A uniform slab of ice, or a ``slab-on-a-slope'', is both a case in which we actually solve the Stokes equations, and a motivation for the shallow model in the next subsection.

\onefig{slab}{Rotated axes for a slab-on-a-slope flow calculation.}

We rotate our coordinates only for this example and not elsewhere in these notes.  The two-dimensional axes ($x$,$z$) shown in Figure \ref{fig:slab} are rotated downward (clockwise) at angle $\alpha>0$ so that the gravity vector has components $\mathbf{g} = (g \sin\alpha,- g \cos \alpha)$.  Equations \eqref{stokespx} and \eqref{stokespz} in these rotated coordinates are
\begin{align}
p_x &= \tau_{11,x} + \tau_{13,z} + \rho g \sin\alpha, \label{stokespxrot} \\
p_z &= \tau_{13,x} - \tau_{11,z} - \rho g \cos\alpha. \label{stokespzrot}
\end{align}
Assuming also that there is no variation with $x$, the whole set of Stokes equations \eqref{incompressiblexz}, \eqref{forceflowx}, \eqref{forceflowz}, \eqref{stokespxrot}, \eqref{stokespzrot} simplifies greatly:
\begin{align}
w_z &= 0 &   0 &= \tau_{11} \label{stokesslab} \\
\tau_{13,z} &= - \rho g \sin\alpha &   u_z &= 2 A \tau^2 \tau_{13} \notag \\
p_z &= -\tau_{11,z} = - \rho g \cos\alpha \notag
\end{align}
We apply boundary conditions for these functions of $z$: $w(0)=0$, $p(H)=0$, $u(0)=u_0$.  The basal velocity $u_0$ will remain undetermined for now.

By integrating equations \eqref{stokesslab} vertically and using $\tau_{11}=0$, we get $w=0$, $p = \rho g \cos\alpha (H-z)$, and $\tau_{13} = \rho g \sin\alpha (H-z)$.  Note that $H-z$ is the depth below the ice surface, so both the pressure $p$ and shear stress $\tau_{13}$ are proportional to depth.  Because $u_z = 2 A \tau^2 \tau_{13}$, by integrating vertically again we find the horizontal velocity:
\begin{align}
u &= u_0 + \frac{1}{2} A (\rho g \sin\alpha)^3  \left(H^4 - (H-z)^4\right)  \label{uslab}
\end{align}

Do we believe formula \eqref{uslab}?  Figure \ref{fig:slabvel} compares to observations of a mountain glacier.  This comparison shows we have at least done a credible job of capturing deformation flow velocity, though we do not yet have a model for the sliding velocity $u_0$ (i.e.~basal motion).  

\twofigsizes{slabvel}{athabasca_deform}{Left:  Velocity from slab-on-a-slope formula \eqref{uslab}.  Right:  Inclinometry-measured velocity in a glacier (Athabasca Glacier \cite{SavagePaterson}).}{2.0in}{1.8in}

\subsection*{Plane-flow mass-continuity equation}  Observe that the equations so far do not address the change in shape of the glacier or ice sheet.  For this we need another equation, the \emph{mass continuity equation}.  First, define the vertical average of velocity:
	$$\bar U = \frac{1}{H}\int_0^{H} u\,dz.$$
The flux $q=\bar U\, H$ is the rate of flow input into the side of the area in Figure \ref{fig:slabmasscontfig}.

\onefigsize{slabmasscontfig}{Mass continuity equation \eqref{masscont1D} follows from considering the changing area $A$ of ice in a planar flow.  Ice can be added by surface mass balance $M$ or a difference of flux $q=\bar u H$ into the left and right sides.}{2.5in}

The ice area $A$ in Figure \ref{fig:slabmasscontfig} changes by adding all the boundary contributions,
\begin{equation}
\frac{dA}{dt} = \int_{x_1}^{x_2} M(x)\,dx + \bar U_1 H_1 - \bar U_2 H_2: \label{masscontintegrated}
\end{equation}
Here $M(x)$ is the climatic mass balance at the ice surface.  (In three-dimensions, equation \eqref{masscontintegrated} would be an equation for $dV/dt$, the rate of change of ice volume.)

If the width $\Delta x=x_2-x_1$ is small then $A\approx \Delta x\, H$.  So we divide by $\Delta x$ and take $\Delta x \to 0$ in \eqref{masscontintegrated} and get
\begin{equation}
H_t = M - \left(\bar U H\right)_x \label{masscont1D}
\end{equation}
This mass continuity equation describes change in the ice thickness in terms of surface mass balance and ice velocity.  It is a major ``use'' of the velocity in ice flow simulations.

\subsection*{Viscosity form of the flow law}  The flow law \eqref{flowlaw} has another form which we will use next, and later in describing ice shelf and stream flow.  Recall $\tau^2 = (1/2) \tau_{ij} \tau_{ij}$, which uses the summation convention.  Also define $|Du|^2 = (1/2) Du_{ij} Du_{ij}$.  The scalars $\tau$ and $|Du|$ are ``norms'' (also ``second invariants'') of the tensors $\tau_{ij}$ and $Du_{ij}$, respectively.  By taking these norms of both sides of \eqref{flowlaw} we get $|Du| = A \tau^3$.  But then $\tau = A^{-1/3} |Du|^{1/3}$, so \eqref{flowlaw} can be rewritten
\begin{equation}
\tau_{ij} = 2 \nu\, Du_{ij}  \qquad \text{\emph{flow law (viscosity form)}} \label{viscosityflowlaw}
\end{equation}
where $\nu = (1/2) A^{-1/3} |Du|^{-2/3}$ is the nonlinear viscosity.  Often $B = A^{-1/3}$ is called the ice ``hardness''.  The derivation of \eqref{viscosityflowlaw} is worth knowing in detail; see the Exercises.

Form \eqref{viscosityflowlaw} of the flow law allows us to eliminate stresses $\tau_{ij}$ from the Stokes equations by replacing them with formulas depending on derivatives of the velocity, that is, on the strain rates only.  The next two approximate models also use this idea.

\subsection*{The hydrostatic and Blatter-Pattyn approximations}  We return now briefly to plane-flow Stokes equations \eqref{incompressiblexz}--\eqref{forceflowz}, and reconsider how to simplify them.  One simplification step, present in all shallow models, is the ``hydrostatic'' approximation.  It drops the single term  $\tau_{13,x}$ from the $z$-component of the stress balance \eqref{stokespz}.  That is, it assumes that horizontal variation in the vertical shear stress is small compared to the other terms:
\begin{equation}
p_z = - \tau_{11,z} - \rho g. \label{hydrostaticpz}
\end{equation}
Because the (Cauchy) stress tensor $\sigma_{ij}$ is related to the deviatoric stress tensor by $\sigma_{ij} = \tau_{ij} - p \delta_{ij}$, and thus $p + \tau_{11} = p - \tau_{33} = - \sigma_{33}$, equation \eqref{hydrostaticpz} says that the vertical normal stress $\sigma_{33}$ is linear in depth.  Taking it to have surface value zero we get
\begin{equation}
p + \tau_{11} = \rho g (h-z). \label{hydrostaticitself}
\end{equation}

Equation \eqref{hydrostaticitself} is the major hydrostatic statement, and it allows one to eliminate $p$ from the model equations.  Furthermore, taking the $x$-derivative of \eqref{hydrostaticitself} and substituting into \eqref{stokespx}, then using the viscosity form \eqref{viscosityflowlaw}, leads to this equation:
\begin{equation}
\left(4 \nu u_x\right)_x + \left(\nu (u_z+w_x)\right)_z = \rho g h_x \qquad\text{\emph{hydrostatic stress balance}} \label{stresshydrostatic}
\end{equation}
The hydrostatic stress balance equation \eqref{stresshydrostatic} is nontrivially-coupled to incompressibility \eqref{incompressiblexz} because derivatives of the vertical velocity $w$ appear in both equations, though $p$ is gone.  Nonetheless coupled equations \eqref{incompressiblexz} and \eqref{stresshydrostatic}, along with the formula $\nu = (1/2) A^{-1/3} |Du|^{-2/3}$ and appropriate boundary conditions, determine $u$ and $w$.

If we drop $w_x$ from equation \eqref{stresshydrostatic} then we get the Blatter-Pattyn model
\begin{equation}
\left(4 \nu u_x\right)_x + \left(\nu u_z\right)_z = \rho g h_x \qquad\text{\emph{Blatter-Pattyn stress balance}} \label{stressblatter}
\end{equation}
Using this equation one can solve first for the horizontal velocity $u$ and then afterward recover $w$ from \eqref{incompressiblexz}; stress balance and incompressibility are decoupled.

\section{Shallow ice sheets}

Ice sheets have four outstanding properties as fluids.  They are (\emph{i}) slow, (\emph{ii}) shallow,  (\emph{iii}) non-Newtonian, and (\emph{iv}) they experience some contact slip (basal sliding).  The first ice flow model we consider, the non-sliding, isothermal \emph{shallow ice approximation} (SIA), accounts for (\emph{i})--(\emph{iii}).

Regarding the property of shallowness, Figure \ref{fig:green_transect} shows both a no-vertical-exaggeration cross-section of Greenland at $71^\circ$, as well as the standard vertically-exaggerated version which is more familiar in the glaciological literature.  Ice sheets are shallow, though of course the portion of an ice sheet which you want to model may not be.

\onefig{green_transect}{A vertically-exaggerated cross-section of the Greenland ice sheet ($71^\circ$ N) is shown by the green and blue curves.  Without exaggeration it appears as nearly a horizontal line (red).}

Our slab-on-a-slope example gives us a rough explanation of the SIA.  To show the SIA in its plane-flow form, we vertically integrate velocity formula \eqref{uslab} in the $u_0=0$ (non-sliding) case to get
\begin{equation}
\bar u H = \int_0^H \frac{1}{2} A (\rho g \sin\alpha)^3  \left(H^4 - (H-z)^4\right)\,dz = \frac{2}{5} A (\rho g \sin\alpha)^3 H^5. \label{siaubar}
\end{equation}
Note $\sin \alpha \approx \tan\alpha = - h_x$.  Combining these statements with mass continuity \eqref{masscont1D} gives
\begin{equation}
  H_t = M + \left(\frac{2}{5} (\rho g)^3 A H^5 |h_x|^2 h_x\right)_x. \label{sia1D}
\end{equation}
Equation \eqref{sia1D} is the SIA equation for nonsliding plane flow.  It is a model for the evolution of an ice sheet's thickness $H$ in terms of surface mass balance $M$, ice softness $A$, and bed elevation $b$ (because $h=H+b$).

Additional arguments are needed to demonstrate that the SIA is more general-purpose than the special case of a simple slab; see Notes and References.  Such arguments reduce the Stokes equations under the assumption that the surface and bed slopes, and the depth-to-width ratio, are small.

We will numerically solve the SIA in section \ref{sec:numericalsia}, but first we state it in two horizontal dimensions.  Let $\mathbf{U} = (u,v)$ be the vector horizontal velocity.  The shear stress approximation is $(\tau_{13},\tau_{23}) \approx - \rho g (h-z) \nabla h$, which appeared as ``$\tau_{13}= \rho g \sin \alpha (h-z)$ and $\sin \alpha \approx -h_x$'' in the previous section, becomes an equality in the SIA.  Equation \eqref{flowlaw} then gives the SIA formula for shear strain rates
\begin{equation*}
\mathbf{U}_z = 2 A |(\tau_{13},\tau_{23})|^{n-1} (\tau_{13},\tau_{23}) = - 2 A (\rho g)^n (h-z)^n |\nabla h|^{n-1} \nabla h.
\end{equation*}
By integrating vertically we get, in the non-sliding case,
\begin{equation}
\mathbf{U} = - \frac{2 A (\rho g)^n}{n+1} \left[H^{n+1} - (h-z)^{n+1}\right] |\nabla h|^{n-1} \nabla h.  \label{siavelocity}
\end{equation}

Mass continuity in two horizontal dimensions, which generalizes the 1D version \eqref{masscont1D}, also applies:
\begin{equation}
    H_t = M - \Div\left(\bar{\mathbf{U}} H\right)  \label{masscont}
\end{equation}
Equation \eqref{masscont} may be written $H_t = M - \Div \bq$ in terms of the map-plane flux $\bq = \int_{b}^{h} \mathbf{U}\,dz = \bar{\mathbf{U}}\,H$.

Combining Equations \eqref{siavelocity} and \eqref{masscont}, we get an equation for the rate of thickness change in terms of mass balance $M$, thickness, and surface slope $\grad h$:
\begin{equation}
H_t = M + \Div \left(\Gamma H^{n+2} |\grad h|^{n-1} \grad h \right), \label{sia}
\end{equation}
where we have defined the positive constant $\Gamma = 2 A (\rho g)^n / (n+2)$.  Equation \eqref{sia} is the SIA in two dimensions.  Recalling our earlier promise, if we can solve \eqref{sia} numerically then we have, following Mahaffy \cite{Mahaffy}, a usable model for the Barnes ice cap in Canada, a particularly simple ice sheet on a rather flat bed.

\subsection*{Analogy with the heat equation}  The SIA model is easy to compare with the better-known heat equation.  All numerical methods for solving \eqref{sia} can be understood as modifications of well-known heat equation methods.

In the simplest one-dimensional (1D) case, the heat equation for the temperature $T(t,x)$ of a conducting rod is
\begin{equation}
  T_t = D T_{xx}. \label{heat1D}
\end{equation}
This form applies when material properties are constant and there are no heat sources.  The positive constant $D$ is the ``diffusivity,'' with units which can be read from comparing sides of the equation: $D\sim \text{m}^2 \text{s}^{-1}$.  Observe that equation \eqref{heat1D} has a smoothing effect on the solution $T$ as it evolves in time, because any local maximum in the temperature is flattened (i.e.~$T_{xx}<0$ implies $T_t<0$ so $T$ decreases), while any local minimum is also flattened (i.e.~$T_{xx}>0$ implies $T_t>0$ so $T$ increases).

The 2D heat equation, analogous to equation \eqref{sia}, describes the temperature $T(t,x,y)$ at position $x,y$ and time $t$.  Recall that Fourier's law for conduction is the formula $\mathbf{Q} = - \kappa \grad T$ for heat flux $\mathbf{Q}$, where $\kappa$ is conductivity.  We will assume, for the purposes of our analogy, that $\kappa(x,y)$ may vary in space.  Also suppose a variable heat source $f(t,x,y)$, with units of Watts per cubic meter.  Then conservation of internal energy says
\begin{equation}
\rho c T_t = f + \Div (\kappa \grad T). \label{heatearly}
\end{equation}
Here $\rho$ is density and $c$ is specific heat capacity.  Assuming $\rho c$ is constant, define the ``diffusivity'' $D=\kappa/(\rho c)$ and the rescaled source term $F = f/(\rho c)$.  The revised 2D heat equation is
\begin{equation}
T_t = F + \Div (D\, \grad T). \label{heat}
\end{equation}
Figure \ref{fig:initialheat} shows a solution of this heat equation, where the initial condition is a localized ``hot spot''.  Solutions of equation \eqref{heat} always involve the spreading, in all directions, of any local heat maxima or minima, that is, diffusion.

\twofigsizes{initialheat}{finalheat}{A solution of heat equation \eqref{heat} with $D=1$ and $F=0$.  Left: Initial condition $T(0,x,y)$.   Right: Solution $T(t,x,y)$ at $t=0.02$.}{2.8in}{2.8in}

The SIA equation \eqref{sia} and the heat equation \eqref{heat} are each diffusive, time-evolving partial differential equations (PDEs).  A side-by-side comparison is illuminating:
\begin{center}
\begin{tabular}{cc}
\vspace{1mm}
SIA:\, $H$ is ice thickness & \phantom{foo bar} heat: $T$ is temperature\phantom{foo bar}  \\
\vspace{1mm}
	$H_t = M + \Div \left({\color{red}\Gamma H^{n+2} |\grad h|^{n-1}}\, \grad h \right)$  &  $T_t = F + \Div (D\, \grad T)$
\end{tabular}
\end{center}
\vspace{1mm}
Notice that the number of derivatives (one time and two space derivatives) and the signs are the same.  Surface mass balance $M$ is analogous to heat source $F$.  

The analogy suggests that we identify the \emph{diffusivity in the SIA} as:
	$$D = {\color{red}\Gamma H^{n+2} |\grad h|^{n-1}}.$$
A non-sliding SIA flow diffuses the thickness of the ice sheet.  When this $D$, a product of $\Gamma$ and the powers of $H$ and $|\grad h|$, comes out large then the diffusion acts most quickly.

This diffusion equation analogy explains generally why the surfaces of ice sheets are smooth, at least if we overlook non-flow processes like crevassing and wind-driven (snow) dunes.  There are, however, some issues with the analogy:
\begin{itemize}
\item The diffusivity $D$ depends on the solution, both the thickness $H$ and surface elevation $h$.
\item The diffusivity $D$ goes to zero at margins, where $H\to 0$, and at divides and domes, where $|\grad h|\to 0$.
\end{itemize}
More important is a deficiency of the SIA model and not \emph{per se} the analogy, namely
\begin{itemize}
\item Ice flow is much less diffusive when longitudinal (membrane) stresses are important, as when ice is floating or sliding or when the flow is confined by terrain.
\end{itemize}
But we will continue with the SIA, working toward a verified numerical scheme for \eqref{sia} in Section \ref{sec:numericalsia}.

\section{Finite difference numerics} 

Numerical schemes for the heat equation are a good starting place for solving the SIA equation \eqref{sia}.  Here we demonstrate only finite difference (FD) schemes.  These schemes replace derivatives in a differential equation by mere arithmetic.

The basic fact on which FD schemes are based is \emph{Taylor's theorem}, which says that for a smooth function $f(x)$,
	$$f(x+\Delta) = f(x) + f'(x) \Delta + \frac{1}{2} f''(x) \Delta^2 + \frac{1}{3!} f'''(x) \Delta^3 + \dots$$
You can replace ``$\Delta$'' by its multiples, for example:
\begin{align*}
f(x+2\Delta) &= f(x) + 2 f'(x) \Delta + 2 f''(x) \Delta^2 + \frac{4}{3} f'''(x) \Delta^3 + \dots \\
f(x-\Delta) &= f(x) - f'(x) \Delta + \frac{1}{2} f''(x) \Delta^2 - \frac{1}{3!} f'''(x) \Delta^3 + \dots
\end{align*}
The idea for constructing FD schemes is to combine expressions like these to give approximations of derivatives.  Thereby function values on a grid combine to approximate the differential equation.

Here we want partial derivative approximations, so we apply the Taylor's expansions one variable at a time.  For example, with a general function $u=u(t,x)$,
\begin{align*}
u_t(t,x) &= \frac{u(t+\Delta t,x) - u(t,x)}{\Delta t} + O(\Delta t), \\
u_t(t,x) &= \frac{u(t+\Delta t,x) - u(t-\Delta t,x)}{2\Delta t} + O((\Delta t)^2), \\
u_x(t,x) &= \frac{u(t,x+\Delta x) - u(t,x-\Delta x)}{2\Delta x} + O((\Delta x)^2), \\
u_{xx}(t,x) &= \frac{u(t,x+\Delta x) - 2\, u(t,x) + u(t,x-\Delta x)}{\Delta x^2} + O((\Delta x)^2)
\end{align*}
Note that if $\Delta$ is a small number then ``$+O(\Delta^2)$'' is smaller than ``$+O(\Delta)$'', so the approximation is closer when you drop it.

\subsection*{Explicit scheme for the heat equation}  We can build the simplest ``explicit'' scheme which approximates the 1D heat equation \eqref{heat1D} by observing that these two FD expressions are nearly equal:
\begin{equation}
\frac{T(t+\Delta t,x) - T(t,x)}{\Delta t} \approx D\,\frac{T(t,x+\Delta x) - 2\, T(t,x) + T(t,x-\Delta x)}{\Delta x^2}.  \label{heat1Dapproximated}
\end{equation}
The FD scheme itself is not just an approximation of a PDE, like \eqref{heat1Dapproximated}, but an actual method for computing numbers on a grid.  Let $(t_n,x_j)$ denote the time-space grid points.  Denote our approximation of the solution value $T(t_n,x_j)$ by $T_j^n$.  Then the finite difference scheme is
	$$\frac{T_j^{n+1} - T_j^n}{\Delta t} = D\,\frac{T_{j+1}^n - 2\, T_j^n + T_{j-1}^n}{\Delta x^2}.$$
To get a computable formula, let $\mu = D \Delta t / (\Delta x)^2$ and solve for $T_j^{n+1}$:
\begin{equation}
  T_j^{n+1} = \mu T_{j+1}^n + (1 - 2 \mu) T_j^n + \mu T_{j-1}^n \label{heat1Dfd}
\end{equation}

FD scheme \eqref{heat1Dfd} is \emph{explicit} because it directly computes $T_j^{n+1}$ in terms of values at time $t_n$.  Figure \ref{fig:expstencil} shows the ``stencil'' for scheme \eqref{heat1Dfd}: three values at the current time $t_n$ are combined to update the one value at the next time $t_{n+1}$.

Before moving on, notice that evaluating a heat equation solution at a grid point (i.e.~the expression ``$T(t_n,x_j)$'') generally gives a different value from the value $T_j^n$ computed by a scheme like \eqref{heat1Dfd}.  Of course we plan that these numbers will be close, but that needs checking (``verification'') or an \emph{a priori} proof.  Specifically we intend that the numbers $T(t_n,x_j)$ and $T_j^n$ become close to each other when the grid is made finer (i.e.~$\Delta t \to 0$ and $\Delta x \to 0$), as the FD expressions become closer to the derivatives they approximate.  That is, we intend our FD scheme to \emph{converge} under \emph{grid refinement}.

How accurate is scheme \eqref{heat1Dfd}?  Its construction tells us that the difference between the scheme \eqref{heat1Dfd} and the PDE \eqref{heat1D} is $O(\Delta t + (\Delta x)^2)$, so this difference goes to zero as we refine the grid in space and time, a property called \emph{consistency}.  With care about the smoothness of boundary conditions, and using mathematical facts about the heat equation itself, one can show that the difference between $T_j^n$ and $T(t_n,x_j)$ is also $O(\Delta t + (\Delta x)^2)$, which is thus the \emph{convergence rate}; see Notes and References.

To get convergence the PDE problem must generate adequately smooth solutions, and also scheme \eqref{heat1Dfd} must be \emph{stable}, which we address below.  (The main theorem for numerical PDE schemes is ``consistency plus stability implies convergence''; see Notes and References.)  In these notes we do something rather practical, namely verification.  We find problems for which we already know an exact solution $T(t,x)$, and then we compute the differences $|T_j^n - T(t_n,x_j)|$.  This determines directly whether the \emph{implementation} (i.e.~computer code form) of our FD scheme actually converges, not just in theory.

\subsection*{A first implemented scheme}  Our first Matlab implementation we consider the two spatial dimension Equation \eqref{heat} with $D$ constant and $F=0$:
\begin{equation}
T_t = D (T_{xx}+T_{yy}).\label{heat2D}
\end{equation}
Writing $T_{jk}^n \approx T(t_n,x_j,y_k)$, the 2D explicit scheme is
\begin{equation}
	\frac{T_{jk}^{n+1} - T_{jk}^n}{\Delta t} = D\,\left(\frac{T_{j+1,k}^n - 2\, T_{jk}^n + T_{j-1,k}^n}{\Delta x^2} + \frac{T_{j,k+1}^n - 2\, T_{jk}^n + T_{j,k-1}^n}{\Delta y^2}\right). \label{heat2dexplicit}
\end{equation}
The stencil for the right-hand side of \eqref{heat2dexplicit} is in Figure \ref{fig:expstencil}.

\twofigsizes{expstencil}{exp2dstencil}{Left: Space-time stencil for the explicit scheme \eqref{heat1Dfd} for the 1D heat equation.  Right: Spatial-only stencil for scheme \eqref{heat2dexplicit}.}{2.0in}{2.1in}

Scheme \eqref{heat2dexplicit} has implementation \texttt{heat.m} below.  For simplicity we set $T=0$ on the boundary of the square $-1 < x < 1$, $-1 < y < 1$.  The initial condition is gaussian: $T(0,x,y) = \exp(-30 (x^2+y^2))$.  The code uses Matlab ``colon'' notation to remove loops over spatial variables.  Here is an example run:
\begin{Verbatim}
>>  heat(1.0,30,30,0.001,20);
\end{Verbatim}
This sets $D=1.0$ and uses a $30\times 30$ spatial grid.  We take $N=20$ time steps of $\Delta t = 0.001$.  The result is shown in Figure \ref{fig:initialheat}, right.  This is the look of success.

\minput{heat}

However, very similar runs seem to succeed or fail according to some yet unclear circumstance.  For example, results from these calls are shown in Figure \ref{fig:stability}:
\begin{Verbatim}
>> heat(1.0,40,40,0.0005,100);    % Figure 7, left
>> heat(1.0,40,40,0.001,50);      % Figure 7, right
\end{Verbatim}
Both runs compute temperature $T$ on the same spatial grid, for the same final time $t_f = N \Delta t = 0.05$, but with different time steps.  The second run clearly shows instability.

\twofig{stability}{instability}{Numerically-computed temperature on $40\times 40$ grids.  The two runs are the same except that the left has $\Delta t=0.0005$ so $D\Delta t/(\Delta x)^2= 0.2$, while the right has $\Delta t=0.001$ so $D\Delta t/(\Delta x)^2= 0.4$.  Compare \eqref{stabcrit}.}

%>> heat(1.0,40,40,0.001,50);
%  doing N = 50 steps of dt = 0.00100 for 0.0 < t < 0.050
%  nu = 1 * dt / dx^2 = 0.40000
%>> print -dpdf instability.pdf
%>> heat(1.0,40,40,0.0005,100);
%  doing N = 100 steps of dt = 0.00050 for 0.0 < t < 0.050
%  nu = 1 * dt / dx^2 = 0.20000
%>> print -dpdf stability.pdf

\subsection*{Stability criteria and adaptive time stepping}  To avoid the instability shown at right in Figure \ref{fig:stability}, we need to understand the scheme better.  It turns out we have not made an implementation error, but more care is required when choosing space and time steps.

Recall the 1D explicit scheme in form \eqref{heat1Dfd}: $T_j^{n+1} = \mu T_{j+1}^n + (1 - 2 \mu) T_j^n + \mu T_{j-1}^n$.  The new value $T_j^{n+1}$ is an average of the old values, in the sense that the coefficients add to one.  Averaging is stable because averaged wiggles are smaller than the wiggles themselves.  Actually, however, the scheme is only an average \emph{if} the middle coefficient is positive, as a linear combination with coefficients which add to one is not an average if any coefficients are negative.  (For example, we would not accept 15 as an ``average'' of 5 and 7, but we can write $15 = -4 \times 5 + 5 \times 7$, and $-4+5=1$.)

So, what follows from requiring the middle coefficient in \eqref{heat1Dfd} to be positive so it computes such an average?  A \emph{stability criterion} follows, with these equivalent forms:
\begin{equation}
   1 - 2 \mu \ge 0 \quad \iff \quad \frac{D\Delta t}{\Delta x^2} \le \frac{1}{2} \quad \iff \quad \Delta t \le \frac{\Delta x^2}{2 D}.  \label{stabcrit}
\end{equation}
This condition on the size of $\Delta t$ is a \emph{sufficient} stability criterion; it is enough to guarantee stability, though something weaker might do.  In summary, for given $\Delta x$, shortening the time steps $\Delta t$ so that \eqref{stabcrit} holds will make FD scheme \eqref{heat1Dfd} into an averaging process.

Applying this same idea to the 2D heat equation \eqref{heat2D} leads to the stability condition that $1-2\mu^x-2\mu^y \ge 0$ where $\mu^x = D \Delta t / (\Delta x^2)$ and $\mu^y = D \Delta t / (\Delta y^2)$.  In the cases like those shown in Figure \ref{fig:stability} with $\Delta x=\Delta y$, this condition requires $D \Delta t /(\Delta x^2) \le 0.25$, which precisely distinguishes between the two parts of the Figure.  Runs of \texttt{heat.m} are unstable if the time step $\Delta t$ is too big relative to the spacing $\Delta x$.

The stability criterion above is easily satisfied by making each time step shorter.  Doing so at each time step makes an \emph{adaptive} implementation which can be stable even if the diffusivity $D$ is changing in time.  To show how easy this is to implement, \texttt{heatadapt.m} (not shown) is the same as \texttt{heat.m} except that the time step comes from the stability criterion, so it cannot generate the instability seen in Figure \ref{fig:stability}.  However, if the diffusivity $D$ is large or the grid spacings $\Delta x$, $\Delta y$ are small, then adaptive explicit implementations must take many short time steps to assure stability.


\subsection*{Implicit schemes}  There is an alternative stability fix instead of adaptivity, namely ``implicitness.''  For example, the finite difference scheme
\begin{equation}
  \frac{T_j^{n+1} - T_j^n}{\Delta t} = D\,\frac{T_{j+1}^{n+1} - 2\, T_j^{n+1} + T_{j-1}^{n+1}}{\Delta x^2} \label{implicit1D}
\end{equation}
is an $O(\Delta t + (\Delta x)^2)$ implicit scheme for Equation \eqref{heat1D}.  Such implicit schemes for the heat equation are stable for \emph{any} positive time step $\Delta t>0$ (``unconditionally stable''); see Notes and References.  Another well-known implicit scheme is \emph{Crank-Nicolson}, which is unconditionally stable for the heat equation, but with smaller error $O((\Delta t)^2 +(\Delta x)^2)$.

Implicit schemes are harder to implement because the unknown solution values at time step $t_{n+1}$ are treated as a vector in a large system of equations which must be formed and solved at each time step.  If the PDE is nonlinear then the system of equations may be hard to solve.  Of course, the SIA is a highly nonlinear diffusion equation.

Generally, there is a tradeoff between the easy implementability of adaptive explicit schemes and the better stability of implicit schemes.  For these notes we stay with adaptive explicit FD schemes.

\subsection*{Numerical solution of diffusion equations}  We are trying to numerically model ice flows, not heat conduction.  We have an analogy, however, which says that the SIA is diffusive like the heat equation.  In this section, because we wish to solve the SIA on real bedrock, we construct a numerical scheme for a more general diffusion equation which has an extra ``shift'' inside the gradient, namely
\begin{equation}
  T_t = F + \Div \left(D \grad (T + b)\right). \label{gendiffusion}
\end{equation}
In equation \eqref{gendiffusion}, the source term $F(x,y)$, the diffusivity $D(x,y)$, and the ``shift'' $b(x,y)$ may all vary in space.

The following code solves \eqref{gendiffusion}.  It is called by the SIA-specific schemes we build next.

\minput{diffusion}

This adaptive explicit method for the diffusion equation is conditionally stable, with the same essential time step restriction as for the constant diffusivity case, as long as we evaluate $D(x,y)$ at \emph{staggered} grid points.  That is, we use this expression for the second derivative:
\begin{align*}
\Div \left(D \grad X\right) &\approx \frac{D_{j+1/2,k}(X_{j+1,k} - X_{j,k}) - D_{j-1/2,k}(X_{j,k} - X_{j-1,k})}{\Delta x^2} \\
	&\qquad + \frac{D_{j,k+1/2}(X_{j,k+1} - X_{j,k}) - D_{j,k-1/2}(X_{j,k} - X_{j,k-1})}{\Delta y^2},
\end{align*}
where $X=T+b$.  The left part of Figure \ref{fig:diffstencil} shows the stencil.

\twofigsizes{diffstencil}{mahaffystencil}{Left:  Spatial stencil for staggered grid evaluation of diffusivity (at triangles) in the diffusion equation \eqref{gendiffusion}.  Right: Stencil showing how the staggered-grid diffusivity (triangle) can be evaluated in the SIA, from surface elevation (diamonds) and thicknesses (squares).}{2.2in}{2.2in}

The user supplies the diffusivity $D(x,y)$ to \texttt{diffusion.m} on the staggered grid.  The initial temperature $T(0,x,y)$, source term $F(x,y)$, and ``shift'' $b(x,y)$ are also supplied on the regular grid.  When using this code for standard diffusions, or for the flat-bed case of the SIA, we would take $b=0$.


\section{Numerically solving the SIA} \label{sec:numericalsia}

In SIA equation \eqref{sia} we have diffusivity $D = \Gamma H^{n+2} |\grad h|^{n-1}$.  There are two interesting aspects of this formula.  First, as already noted, $D$ goes to zero, i.e.~it ``degenerates,'' when either $H\to 0$ or $\grad h \to 0$.  Degenerate diffusion equations are automatically free boundary problems, but this aspect of the thickness evolution problem is no surprise to a glaciologist.  Determining the location of the margin is an obvious part of modelling a glacier or ice sheet.  To address this free boundary issue in our explicit time-stepping code it suffices to numerically compute new thicknesses and then set them to zero if they come out negative.

Second, for numerical stability and mass conservation we should compute $D$ on a ``staggered'' grid.  Various finite difference schemes for computing it have been proposed.  All of these schemes involve averaging $H$ and differencing $h$ in a ``balanced'' way onto the staggered grid.  In the code \texttt{siaflat.m} below we use the Mahaffy \cite{Mahaffy} method, with the stencil for computing $D$ shown in Figure \ref{fig:diffstencil}.  This code only works for the flat bed, zero surface mass balance case, but we will correct these deficiencies later.

\minput{siaflat}


\section{Exact solutions and verification}

In \texttt{siaflat.m}, which calls \texttt{diffusion.m}, we already have a fairly complicated code.  How do we make sure that such an implemented numerical scheme is correct?  Here are three proposed techniques:
\begin{enumerate}
  \item don't make any mistakes, or
  \item compare your numerical results with results from other researchers, and hope that the outliers are in error, or
  \item compare your numerical results to an exact solution.   \end{enumerate}
The last one of these, which we prefer to the first two when possible, is called ``verification.''  That is, when we build a new computer code we should test it in cases where we know the right answer.  To do so we need to return to the PDE itself, to get useful exact solutions.

\subsection*{Exact solution of heat equation}  First we consider the simpler case of the 1D heat equation with constant $D$, namely $T_t = D T_{xx}$.  Many exact solutions $T(t,x)$ to this heat equation are known, but let's consider the time-dependent ``Green's function,'' also known as the ``heat kernel''.  It starts at time $t=0$ with a delta function $T(0,x)=\delta(x)$ of heat at the origin $x=0$.  Then it spreads out over time.  It is a solution of the heat equation on the whole line $-\infty<x<\infty$ and for all $t>0$.

We will calculate this exact solution by a method which generalizes to the SIA.  The Green's function of the heat equation is ``self-similar'' over time, in the sense that it changes shape \emph{only} by shrinking the output (vertical) axis and lengthening the input (horizontal) axis, as shown in Figure \ref{fig:heatscaling}.  These scalings are related to each other by conservation of energy, which says that the total heat energy is independent of time.

\onefigsize{heatscaling}{The heat equation Green's function in 1D has the same shape at each time, but with time-dependent input- and output-scalings.}{2.4in}

In particular, the Green's function of the 1D heat equation is
  $$T(t,x) = (4 \pi D t)^{-1/2}\, e^{-x^2/(4Dt)}.$$
``Similarity'' variables for this solution, the above-mentioned scalings, involving multiplying the input and output of an invariant shape function $\phi(s) = (4 \pi D)^{-1/2}\, e^{-s^2/(4D)}$ by the same power of $t$:
\begin{equation}
s \stackrel{\text{\emph{input scaling}}}{\phantom{\Big|}=\phantom{\Big|}} t^{-1/2} x, \qquad\qquad T(t,x) \stackrel{\text{\emph{output scaling}}}{\phantom{\Big|}=\phantom{\Big|}} t^{-1/2} \phi(s).  \label{heatscalings}
\end{equation}
Note that all time-dependence is from the input and output scalings.

A numerical solver for the 1D heat equation which starts with initial values $T(t_0,x)$ taken from this exact solution should, at a later time $t$, produce numbers which are close to the exact solution $T(t,x)$; see the Exercises.

\subsection*{Halfar's similarity solution to the SIA}  Now we jump from Green's idea in about 1830 to the year 1981.  That is when P.~Halfar published the similarity solution of the SIA in the case of flat bed and zero surface mass balance.  Halfar's solution to the 2D SIA model \eqref{sia}, using a Glen exponent $n=3$, has scalings which are powers of $t$ like \eqref{heatscalings} above:
\begin{equation}
s = t^{-1/18} r, \qquad \qquad H(t,r)=t^{-1/9} \phi(s). \label{halfarscalings}
\end{equation}
Here $r=(x^2+y^2)^{1/2}$ is the distance from the origin.  These scalings are related to each other by conservation of mass, because no mass is gained or lost through the surface. Scalings \eqref{halfarscalings} imply that, quite differently from heat, the diffusion of ice slows down severely as the shape flattens out.  The powers $t^{-1/9}$ and $t^{-1/18}$ change very slowly for large times $t$.

\onefigsize{siascaling}{A case of Halfar's solution \eqref{halfar} of the SIA equation \eqref{sia} on a flat bed with zero mass balance.  The solution is shown on $H$ (m) versus $r$ (km) axes for times $t=1,10,100,1000,10000$ years.}{5.5in}

The formula for the Halfar solution to the SIA is remarkably simple given all that it accomplishes:
\begin{equation}
H(t,r) = H_0 \left(\frac{t_0}{t}\right)^{1/9} \left[1 - \left(\left(\frac{t_0}{t}\right)^{1/18} \frac{r}{R_0}\right)^{4/3}\right]^{3/7}. \label{halfar}
\end{equation}
Here the ``characteristic time'' $t_0 = (18 \Gamma)^{-1} (7/4)^3 R_0^4 H_0^{-7}$ is a parameter which can be determined by choosing center height $H_0$ and radius $R_0$.

Formula \eqref{halfar} is plotted in Figure \ref{fig:siascaling}.  We see that for times significantly greater than $t_0$ (i.e.~$t/t_0 \gg 1$) the solution changes very slowly.  For example, the change between years $1$ and $100$ is larger than that between years $1000$ and $10000$.  The \emph{volume} of ice in this Halfar ice cap is, however, constant with $t$; see the Exercises.

\subsection*{Using Halfar's solution}  Formula \eqref{halfar} is simple enough to use for verifying time-dependent SIA models.  The code \texttt{verifysia.m} (not shown) takes as input the number of grid points in each ($x,y$) direction.  It uses the Halfar solution at 200 a as the initial condition, does a numerical run of \texttt{siaflat.m} above to a final time 20000 a, and then compares to the Halfar formula for that time.  By ``compares'' we mean it computes the thickness error, the absolute values of the differences between the numerical and exact thickness solutions at the final time:
\small
\begin{verbatim}
>> verifysia(20)
average thickness error     = 22.310
>> verifysia(40)
average thickness error     = 9.490
>> verifysia(80)
average thickness error     = 2.800
>> verifysia(160)
average thickness error     = 1.059
\end{verbatim}
\normalsize
We see that the average thickness error decreases with increasing grid resolution.  This is as expected for a correctly-implemented code.  What is less obvious, perhaps, is that almost any numerical implementation mistake---almost any bug---will break this property, and these errors will not shrink.

You might ask, is the Halfar solution ever useful for modelling real ice masses?  The answer is yes.  In fact, J.~Nye and others (2000; \cite{NyeIcarus2000}) compared the long-time consequences of different flow laws for the south polar cap on Mars.  In particular, they evaluated $\text{CO}_2$ ice and $\text{H}_2\text{O}$ ice softness parameters by comparing the long-time behavior of the corresponding Halfar solutions to the observed polar cap properties.  Their conclusions:
  \begin{quote}
  \dots none of the three possible [$\text{CO}_2$] flow laws will allow a 3000-m cap, the thickness suggested by stereogrammetry, to survive for $10^7$ years, indicating that the south polar ice cap is probably not composed of pure $\text{CO}_2$ ice [but rather] water ice, with an unknown admixture of dust.
  \end{quote}
This theoretical result has since been confirmed by the observation and sampling of the polar geology of Mars.

Are exact solutions like Halfar's always available when needed?  The answer is ``no'', of course, though many ice flow models do have exact solutions which are relevant to verification; see the Notes and References.  For example, we will use van der Veen's solution for ice shelves in a later section.  On the other hand, the absence of exact solutions may show that not enough thought has gone into the continuum model itself.

\subsection*{A test of robustness}  Verification is an ideal way to start testing a code.  Another kind of test is for ``robustness''.  One asks: Does the model break when you ask it to do hard things?  Unlike for verification, we might not have precise knowledge of what it should do, but a well-implemented model should act in a ``reasonable'' way.

The robustness test in the program \texttt{roughice.m} (not shown) demonstrates that \texttt{siaflat.m} can handle an ice sheet with extraordinarily large ``driving stresses.''  Recall that glaciological driving stress is $\tau_d = - \rho g H \grad h$.  This quantity appears in the slab-on-a-slope example, and thus in the SIA model, as the value of the shear stress $(\tau_{13},\tau_{23})$ at the base of the ice.  The driving stress is, obviously, large when the ice is both thick and has steep surface slope $|\nabla h|$.

In \texttt{roughice.m} we give \texttt{siaflat.m} a randomly-generated initial ice sheet which is of the worst possible sort because it is both thick (average of 3000 m) and it has large surface slopes.  The initial shape is shown in the left side of Figure \ref{fig:roughinitial}.  During the run of 50 model years, the time step is determined adaptively from \eqref{stabcrit}, increasing from 0.0002 years to about 0.2 years as the maximum diffusivity $D$ decreases correspondingly.  The maximum value of the driving stress decreases from $57$ bar ($= 5.7\times 10^6$ Pa) to $3.6$ bar.  At the end the ice cap has the shape shown at right in Figure \ref{fig:roughinitial}.

\twofig{roughinitial}{roughfinal}{The SIA model evolves the huge-driving-stress initial ice sheet at left to the ice cap at right in only 50 model years.}

The shape at right in Figure \ref{fig:roughinitial} is rather close to a Halfar solution.  Indeed Halfar proved that all solutions of the zero-mass-balance SIA on a flat bed asymptotically approach the Halfar solution.


\section{Applying our numerical ice sheet model}

Finally we apply the model to the Antarctic ice sheet.  To do this we must first modify \texttt{siaflat.m} to allow non-flat bedrock elevation $b(x,y)$ and arbitrary surface mass balance $M(x,y)$.  Also we calve floating ice, and we enforce non-negative thickness at each timestep.  The result is \texttt{siageneral.m} (not shown), a code only ten lines longer than \texttt{siaflat.m}.

\twofigsizes{antinitial}{antfinal}{Left: Initial surface elevation (m) of Antarctic ice sheet.  Right: Final surface elevation at end of 40 ka model run on 50 km grid.}{2.55in}{3.2in}

We use measured accumulation, bedrock elevation, and surface elevation from ALBMAPv1 data \cite{LeBrocqetal2010}.  Melt is not modelled so the surface mass balance is the accumulation rate.  These input data are read from a NetCDF file and preprocessed by an additional code \texttt{buildant.m} (not shown).

\onefig{antvolcompare}{Ice volume of the modeled Antarctic ice sheet, in units of $10^6 \, \text{km}^3$, from runs on 50 km (red), 25 km (green), and 20 km (blue) grids.}

The code \texttt{ant.m} (not shown) calls \texttt{siageneral.m} to do the simulation in blocks of 500 model years.  The volume is computed at the end of each block.  Figure \ref{fig:antinitial} shows the initial and final surface elevations from a run of 40,000 model years on a $\Delta x = \Delta y = 50$ km grid.  The runtime on a typical laptop is a few minutes.

Areas of the Antarctic ice sheet with low-slope and (actual) fast-flowing ice experience thickening in the model, while near-divide ice in East Antarctica, in particular, thins.  Assuming the present-day Antarctic ice sheet is near steady state, these most-obvious thickness differences reflect model inadequacies.  The lack of a sliding mechanism explains the thickening in low-slope areas.  The lack of thermomechanical coupling, or equivalently the constancy of ice softness, explains the thinning near the divide.  And of course we should be modeling floating ice too, but the SIA is completely inappropriate to that purpose.  See section \ref{sec:shelvesandstreams} and Notes and References on modelling techniques which address these inadequacies.

Figure \ref{fig:antvolcompare} compares the ice volume time series for 50 km, 25 km, and 20 km grids.  This result, namely grid dependence of the ice volume, is typical.  One cause is that most steep gradients near the ice margin are poorly resolved, and this is true to differing degrees at these coarse resolutions.  Mainly Figure \ref{fig:antvolcompare} is a warning about the interpretation of model runs:  Even if the data is available only on a fixed grid, the model should be run at different resolutions to evaluate the robustness of the model results.


\section{Interlude: Mass continuity and kinematical equations}

Recall that in the SIA the ``stress balance'' is essentially formula \eqref{siavelocity} for the velocity.  It combines with the mass continuity equation \eqref{masscont} to give model \eqref{sia} for the ice sheet thickness.  The major SIA equation \eqref{sia} thus combines two concepts which we will now think about separately, and in greater generality, in the remainder of these notes.

The basic shallow assumption made by most ice flow theories\footnote{There are several inequivalent shallow theories: SIA, SSA, hybrids, Blatter-Pattyn, \dots} is that the surface and base of the ice are differentiable functions $z=h(t,x,y)$ and $z=b(t,x,y)$.  Thus surface overhang is not allowed, though, by contrast, the Stokes theory of slow viscous fluids only needs a closed surface in three-dimensional space as a boundary surface for the fluid.  Most ice sheet and glacier models take a map-plane perspective, however, and they have a well-defined ice thickness: $H=h-b$.

To pursue such ideas a bit further, let us state the ``kinematical equations'' which apply at upper and lower surfaces of the ice sheet.  Let $a$ be the upper surface (climatic) mass balance function ($a>0$ is accumulation) and $s$ be the basal melt rate function ($s>0$ is basal melting).  In the equations which follow these are measured in thickness-per-time units, but they could be in mass-per-area-per-time units also.  The net map-plane mass balance $M=a-s$, which already appears in the mass continuity equation \eqref{masscont}, is the difference of these surface fluxes.

The \emph{(upper) surface kinematical equation} is 
\begin{equation}
h_t = a - \mathbf{U}\big|_h \cdot \grad h + w\big|_h,  \label{surfkine}
\end{equation}
and the \emph{base kinematical equation} is
\begin{equation}
b_t = s - \mathbf{U}\big|_b \cdot \grad b + w\big|_b.  \label{basekine}
\end{equation}
(Recall $\mathbf{U}$ is the horizontal ice velocity and $w$ the vertical ice velocity.)  Equations \eqref{surfkine} and \eqref{basekine} describe the movement of the ice's upper surface and lower surfaces, respectively, from the velocity of the ice and the mass balance functions at the respective surfaces.

We can now state an important mathematical fact which follows merely from the assumption of well-defined upper and basal surface elevations.  Namely, that the surface kinematical and mass continuity equations are closely-related.  More precisely, any pair of these equations implies the third:
  \begin{itemize}
  \item the surface kinematical equation \eqref{surfkine},
  \item the base kinematical equation \eqref{basekine}, and
  \item the map-plane mass continuity equation \eqref{masscont}.
  \end{itemize}
One proves these facts by using the incompressibility of ice \eqref{incompressible} and the Leibniz rule for differentiating integrals.  The details are left for exercises.

The bedrock is often regarded as fixed (i.e.~$b_t=0$), and in fact the basal kinematical equation is often not explicitly mentioned.  Instead one gets a simplified view.  In the case of non-deformable bedrock and no sliding, for example, the basal value of the vertical velocity equals the basal melt rate.  This simplification corresponds to $b_t=0$ and $\mathbf{U}\big|_b=0$ so that $w\big|_b=-s$ from \eqref{basekine}.

\subsection*{Prognostic models}  We can now sketch the structure of a general ``prognostic,'' i.e.~ice geometry evolving, isothermal ice sheet model.  Each time step follows this recipe:
  \begin{itemize}
  \item numerically solve a stress balance, which gives velocity $\mathbf{u}=(u,v,w)$,
    \begin{itemize}
    \item[$\circ$] if the stress balance only gives $\mathbf{U}=(u,v)$, get $w$ from incompressibility \eqref{incompressible},
    \end{itemize}
  \item decide on a time step $\Delta t$ for \eqref{masscont} based on velocities and/or diffusivities,
  \item from the horizontal velocity $\mathbf{U}=(u,v)$, compute the flux $\bq = \bar{\bU} H$,
  \item update mass balance $M=a-s$ and do a time-step of \eqref{masscont} to get $H_t$,
  \item update the upper surface elevation and thickness (e.g.~$h \mapsto h + H_t \Delta t$), and repeat.
  \end{itemize}
Like most ice sheet models, we use the mass continuity equation \eqref{masscont} to describe changes in ice sheet geometry, but we could use the surface kinematical equation \eqref{surfkine} instead.

The above ``standard'' ice sheet model has many variations.  Some glaciological questions are answered just by solving the stress balance for the velocity.  Sometimes the goal is the steady state configuration of the glacier, which might be computed more quickly by iteratively solving steady state equations than by time-stepping physical evolution equations to steady state.  Other processes are usually simulated at each time step, such as the conservation of energy within the ice, or subglacial and supraglacial processes.  Understanding the diverse time scales associated to these processes is usually an important step in designing the coupled model.

When using the SIA equation \eqref{sia}, one can seemingly bypass the computation of the velocity.  That is because we could write the mass continuity equation as a diffusion, with $\bq=-D\nabla h$ for the flux instead of the more general $\bq = \bar{\bU} H$.  Fast flow in ice sheets is associated with sliding and floating ice, however, and for these flows the ice geometry evolution is not a diffusion, and so only ``$\bq = \bar{\bU} H$'' applies.  Solving the stress balance for the velocity field is then an obligatory, and usually nontrivial, step.  We consider such a stress balance next.


\section{Shelves and streams} \label{sec:shelvesandstreams}

The shallow shelf approximation (SSA) stress balance applies to ice shelves as its name suggests.  The SSA also applies reasonably well to ice streams, like those in Figure \ref{fig:siple} which have not-too-steep bed topography and low basal resistance.

\twofigsizes{siple}{streamisbrae}{Left:  The SSA model applies to ice streams like these on the Siple Coast in Antarctica.  Color shows radar-derived surface speed.  Right: Cross sections, \emph{without} vertical exaggeration, of the Jakobshavns Isbrae outlet glacier in Greenland (\textbf{a}) and the Whillans Ice Stream on the Siple Coast (\textbf{b}); this is Figure 1 in \cite{TrufferEchelmeyer}.}{2.8in}{2.9in}

But what is, and is not, an ice stream?  Ice streams slide at $50$ to $1000 \,\text{m}\,\text{a}^{-1}$, they have a concentration of vertical shear in a thin layer near base, and typically they flow into ice shelves.  Pressurized liquid water at their beds plays a critical role enabling their fast flow.  There are other fast-flowing grounded parts of ice sheets, however, called ``outlet glaciers''.  They can have even faster surface speed (up to $10 \,\text{km}\,\text{a}^{-1}$), but it is typically uncertain how much of this speed is from sliding at the base.  In an outlet glacier there is substantial vertical shear ``up'' in the ice column, sometimes caused by soft temperate ice in a significant fraction of the thickness.  Furthermore, outlet glaciers are strongly controlled by fjord-like, high slope bedrock topography.  Figure \ref{fig:siple} (right) compares the shallowness and bedrock topography of an outlet glacier and an ice stream.  Thus, few simplifying assumptions are appropriate for outlet glaciers, and the SSA may not be a sufficient model.

\subsection*{The shallow shelf approximation (SSA)}  We state this stress balance equation only in the plane flow (``flow-line'') case:
\begin{equation}
  \left(2 B H |u_x|^{1/n - 1} u_x\right)_x - C|u|^{m-1}u = \rho g H h_x \label{ssaearly}
\end{equation}
The term in parentheses is the vertically-integrated longitudinal stress, also called the ``membrane'' stress when there are two horizontal variables.  The second term $\tau_b = - C|u|^{m-1}u$ is the basal resistance, which is zero (i.e.~$C=0$) in an ice shelf.  The term on the right is the driving stress ($\tau_d = - \rho g H h_x$).  Thus the SSA equation is a balance wherein longitudinal strain rates are determined by the integrated ice hardness (i.e.~the coefficient $BH$), the slipperyness of the bed (i.e.~by the coefficient $C$ and the power $m$) and the geometry of the ice sheet (i.e.~the thickness $H$ and the surface slope $h_x$).

In \eqref{ssaearly} the velocity $u$ is independent of the vertical coordinate $z$.  We assume that the ice hardness $B=A^{-1/n}$ is also independent of depth.  Models which are not isothermal compute the vertical average of the temperature-dependent hardness.  The formula for the basal resistance $\tau_b$ is often called a ``sliding law'' in power law form.

The coefficient $\bar \nu = B |u_x|^{1/n-1}$ in \eqref{ssaearly} is called the ``effective viscosity'', so that \eqref{ssaearly} can be written
\begin{equation}
  \left(2 \,\bar \nu\, H u_x\right)_x - C |u|^{m-1} u = \rho g H h_x.  \label{ssa}
\end{equation}
In form \eqref{ssa} it is understood that the viscosity $\bar\nu$ depends on the velocity solution $u$.

The inequality ``$\,\rho H < - \rho_w b\,$'' is sometimes called the \emph{flotation criterion}.  For grounded ice we know $\rho H > - \rho_w b$ and the driving stress $\tau_d = - \rho g H h_x$ uses $h = H+b$.  On the floating side we know $\rho H < - \rho_w b$ and, by Archimedes principle, we use $h = (1-\rho/\rho_w) H$ in the driving stress.

Equation \eqref{ssa} simplifies if the ice is floating.  The ice surface elevation is proportional to the thickness if the ice is floating.  Also we assume zero resistance ($C=0$) is applied by the ocean.  Thus the SSA becomes
\begin{equation}
   \left(2 \,\bar\nu\, H u_x\right)_x = \rho g (1-\rho/\rho_w) H H_x \label{ssafloat}
\end{equation}
for floating ice.  A useful observation about flow line equation \eqref{ssafloat} is that both left- and right-hand expressions are derivatives; this can be used to build a 1D exact solution.

\subsection*{Steady ice shelf exact solution}  For a steady 1D ice shelf, in which $H_t=0$, the mass continuity equation \eqref{masscont} reduces to $M=(uH)_x$.  Because of the relative simplicity of the SSA equation \eqref{ssafloat} and the steady mass continuity equation for 1D floating ice, the exact velocity and thickness for a steady ice shelf can be computed \cite{vanderVeen83}.  This exact solution depends on the ice thickness $H_g$ and velocity $u_g$ at the grounding line.  For the surface mass balance $M$ we choose a positive constant $M_0$.  These choices determine a unique solution, the derivation of which is left to the exercises.

Supposing $H_g=500$ m, $u_g = 50 \,\text{m}\,\text{a}^{-1}$, and $M_0=30 \,\text{cm}\,\text{a}^{-1}$ we get the results in Figure \ref{fig:steadyshelfprofile}, which are from code \texttt{exactshelf.m} (not shown).  We will use this exact solution to verify a numerical SSA code.  Note that driving stresses are much higher near the grounding line than away from it, and thus the highest longitudinal stresses, strain rates, and thinning rates occur near the grounding line.

\twofig{steadyshelfprofile}{steadyshelfvelocity}{The upper and lower surface elevation (m; left) of the exact ice shelf solution and its velocity (m/a; right); $x=0$ is the grounding line.}

\subsection*{Numerical solution of the SSA}  Suppose the ice thickness is a fixed function $H(x)$.  To find the velocity we must solve the nonlinear PDE \eqref{ssa} or \eqref{ssafloat} for the unknown $u(x)$.  When we do this numerically an iteration is needed because of the nonlinearity.  The simplest iteration idea is to use an initial guess at the velocity, which allows us to compute an effective viscosity and then get a new velocity solution from a linear PDE problem.  Then we recompute the effective viscosity, solve for a new velocity, and repeat until things stop changing.  This is often called a ``Picard'' iteration, in contrast to a ``Newton'' iteration which should converge faster.

Denote the previous velocity iterate as $u^{(k-1)}$ and the current iterate as $u^{(k)}$.  Compute $\bar \nu^{(k-1)} = B |u^{(k-1)}_x|^{1/n-1}$ and define $W^{(k-1)} = 2 \bar \nu^{(k-1)} H$.  Solving this linear elliptic PDE for the unknown $u^{(k)}$ is a Picard iteration for \eqref{ssa}:
\begin{equation}
   \left(W^{(k-1)} u^{(k)}_x\right)_x - C |u^{(k-1)}|^{m-1} u^{(k)} = \rho g H h_x. \label{picardssa}
\end{equation}
If the difference between $u^{(k-1)}$ and $u^{(k)}$ were zero then we would have a solution of \eqref{ssa}, while in practice we stop the iteration when the difference is smaller than some tolerance.

Equation \eqref{picardssa} is a linear boundary value problem.  We can write it abstractly
\begin{equation}
  \left(W(x)\, u_x\right)_x - \alpha(x)\, u = \beta(x)  \label{innerlinear}
\end{equation}
where the functions $W(x)$, $\alpha(x)$, $\beta(x)$ are known.  Equation \eqref{innerlinear} applies on an interval of the $x$-axis.  For one boundary condition we will suppose that $x=x_g$ is a location where the velocity is known, $u(x_g)=u_g$, as in Figure \ref{fig:steadyshelfprofile}.  In the ice shelf case we also have the calving front condition (see Notes and References)
\begin{equation}
  2 B H |u_x|^{1/n - 1} u_x = \frac{1}{2}\rho (1-\rho/\rho_w) g H^2  \label{calvingstress}
\end{equation}
at the end of the ice shelf $x=x_c$.  Boundary condition \eqref{calvingstress} can be solved for $u_x(x_c)=\gamma$ in terms of known quantities including the thickness at the calving front.

Where to get an initial guess $u^{(0)}$?  Generally this may require effort, but we will use these choices for our 1D case.  For floating ice, an initial velocity comes from assuming a uniform strain rate provided by the calving front condition: $u^{(0)}(x) = \gamma (x-x_g) + u_g$.  For grounded ice, we may assume ice is held by basal resistance only: $u^{(0)}(x) = \left(-C^{-1} \rho g H h_x\right)^{1/m}$.

\subsection*{Numerics of the linear boundary value problem}  Suppose equation \eqref{innerlinear} applies on $[x_g,x_c]=[0,L]$.  We choose a grid with equal spacing $\Delta x$ and index $j=1,2,\dots,J+1$, so that $x_1 = 0$ and $x_{J+1} = L$ are endpoints.  The coefficient $W(x)$ is needed on a staggered grid, for stability and accuracy reasons similar to those for the SIA diffusivity.  Our finite difference approximation of \eqref{innerlinear} is, therefore,
\begin{equation}
  \frac{W_{j+1/2} (u_{j+1} - u_j) - W_{j-1/2} (u_{j} - u_{j-1})}{\Delta x^2} - \alpha_j u_j = \beta_j  \label{discreteinnerlinear}
\end{equation}

For the left end boundary condition we have $u_1 = u_g$ given, which is easy to include in the linear system (below).  For the right end boundary condition we have $u_x(L)=\gamma$, which requires more thought.  First introduce a notional point $x_{J+2}$.  Now require $(u_{J+2} - u_J)/(2 \Delta x) = \gamma$, which is a centered approximation to ``$u_x(x_c)=\gamma$.''  Using equation \eqref{discreteinnerlinear} in $j=J+1$ case, eliminate the $u_{J+2}$ variable ``by-hand''.  This determines the form of the last equation in our linear system.

Now observe that each iteration to solve the SSA stress balance has the form
\begin{equation}
   A \mathbf{v} = \mathbf{b}. \label{Aveqb}
\end{equation}
Indeed, at each location $x_1,\dots,x_{J+1}$ we can write an equation, including a row of the matrix $A$ in \eqref{Aveqb}, involving the unknown velocities.  It is this linear system of $J+1$ equations:
\begin{equation}
\begin{bmatrix}
1 &  &  &  &  \\
W_{3/2} & A_{22} & W_{5/2} &  &  \\
 & W_{5/2} & A_{33} &  &  \\
 &  & \ddots & \ddots &  \\
 &  & W_{J-1/2} & A_{JJ} & W_{J+1/2} \\
 &  &  & A_{J+1,J} & A_{J+1,J+1} \\
\end{bmatrix}\,
\begin{bmatrix}
u_1 \\ u_2 \\ u_3 \\ \vdots \\ u_J \\ u_{J+1}
\end{bmatrix}
=
\begin{bmatrix}
u_g \\ \beta_2 \Delta x^2 \\ \beta_3 \Delta x^2 \\ \vdots \\ \beta_J \Delta x^2 \\ b_{J+1}
\end{bmatrix}  \label{discretematrixform}
\end{equation}
The diagonal entries ``$A_{ij}$'' are
  $$A_{22} = -(W_{3/2}+W_{5/2}+\alpha_2 \Delta x^2), \quad \dots, \quad A_{JJ} = -(W_{J-1/2}+W_{J+1/2}+\alpha_J \Delta x^2),$$
except for special cases for the coefficients in the last equation,
  $$A_{J+1,J} = 2 W_{J+1/2}, \quad A_{J+1,J+1} = -(2 W_{J+1/2}+\alpha_{J+1}\Delta x^2).$$
For the right side of the last equation, $b_{J+1} = -2 \gamma \Delta x W_{J+3/2} + \beta_{J+1} \Delta x^2$.

System \eqref{discretematrixform} is a tridiagonal linear system.  But don't bother looking up how to solve such a linear system unless you really need to!  It is fully appropriate to give system \eqref{discretematrixform} to Matlab's linear solver, the ``backslash'' operator $\mathbf{v} = A\, \backslash\, \mathbf{b}$, especially at this initial implementation stage.  In these notes we will not worry further about solving finite linear systems.  We now have a code to solve \eqref{innerlinear} by finite differences and linear algebra, namely \texttt{flowline.m} below.

\minput{flowline}

By ``manufacturing'' exact solutions to \eqref{innerlinear}---see Notes and References---we can test this first piece of our SSA-solving codes before proceeding to solve the actual nonlinear SSA problem.   In fact, results from \texttt{testflowline.m} (not shown) demonstrate that our implemented numerical scheme converges at the optimal rate $O(\Delta x^2)$.

\subsection*{Solving the stress balance for an ice shelf}  The code \texttt{ssaflowline.m} (below) numerically computes the velocity for an ice shelf.  The thickness is assumed to be given, so we are not yet addressing the full, ``coupled'' ice shelf problem, simultaneously solving the applicable mass continuity \eqref{masscont1D} and stress balance \eqref{ssafloat} equations.  We are only solving the latter.

This code implements Picard iteration \eqref{picardssa}, in the floating case, to solve the nonlinear equation \eqref{ssafloat}.  It calls \texttt{ssainit.m} (not shown) to get the initial iterate $u^{(0)}(x)$, as already described, and it calls \texttt{flowline.m} at each iteration.  It also calls small helper functions \texttt{stagav(),regslope(),stagslope()}, at the end of the code, to computed certain gridded values.

\minput{ssaflowline}

Now we can ask precisely: Does \texttt{ssaflowline.m} work correctly?  The exact velocity solution shown in Figure \ref{fig:steadyshelfprofile}, computed by \texttt{exactshelf.m}, allows us to compare the numerical to the exact velocities by finding the maximum difference between them.  For this to work we take the exact thickness shown in Figure \ref{fig:steadyshelfprofile}, also from \texttt{exactshelf.m}.  A convergence comparison, shown in Figure \ref{fig:shelfconv}, is done by codes \texttt{testshelf.m} and \texttt{shelfconv.m} (not shown).  Each circle in the Figure gives the maximum velocity error on a given grid.

\onefig{shelfconv}{The numerical SSA velocity solution from \texttt{ssaflowline.m} converges to the exact solution, at nearly the optimal rate $O(\Delta x^2)$, as the grid is refined from spacing $\Delta x=4$ km to $\Delta x=62$ m.}

Even on the coarsest $\Delta x = 4$ km grid we see in Figure \ref{fig:shelfconv} that the maximum velocity error (i.e.~difference) is less than 1 m/a, while the maximum velocity itself is $\sim 300$ m/a.  We can conclude from this comparison that, at screen resolution, our numerical velocity solutions are essentially identical to that shown in the right part of Figure \ref{fig:steadyshelfprofile}.  There is not even a reason to show the numerical solutions!

\subsection*{Realistic ice shelf modelling}  Real ice shelves have two horizontal variables.  They are frequently confined in bays, and thus they experience ``side drag''.  Their velocities vary spatially and temporally along their grounding lines, which are the curves where the flotation criterion is an equality.  Furthermore real ice shelves have interesting boundary processes, including high basal melt near grounding lines, marine ice basal freeze-on, and fracturing which nears full thickness at the calving front.  It is a bit complicated.

Nonetheless ``diagnostic'' (i.e.~fixed geometry) ice shelf modelling in two horizontal variables, done like the above example where the velocity is unknown but the thickness is known and fixed, is quite successful using only the isothermal SSA model.  For example, Figure \ref{fig:rossquiver} shows a Parallel Ice Sheet Model (PISM) result for the Ross ice shelf, compared to observed velocities.  There is only one tuned parameter, the constant value of the ice hardness $B$.  In this run, observed velocities for grounded ice were applied as boundary conditions.  Many current ice shelf models yield comparable match \cite{MacAyealetal}.

\twofigsizes{rossquiver}{rossscatter}{Results from PISM.  Left: Observed (white) and modeled (black) ice velocities are nearly coincident across the whole Ross ice shelf.  The grounding line is the thin black curve.  Right: In this scatter plot there is one point for each arrow at left.}{3.0in}{3.0in}


\section{A summary of numerical ice sheet modelling}

These notes are brief, and so they give a very incomplete view of numerical models for glaciers and ice sheets.  They do, however, illustrate some general principles about numerical modelling.  One should:
\begin{itemize}
\item Return often to the continuum model.
\item Modularize codes.
\item Test the parts: Is the component robust? Does it show convergence?
\end{itemize}

Regarding the specific ice flow models covered in these notes, here are three high-level points, as a meager conclusion:
\begin{itemize}
\item The mass continuity equation is the part of an ice sheet model which describes how the ice geometry evolves.  It is a kind of transport equation in the map-plane, but with diffusive character at larger spatial scales.  The numerical approach to this equation depends on which is the stress balance which supplies the ice velocity or ice flux.  Mass continuity is a diffusion for frozen bed, large scale flows, and in that case the SIA is a good choice.  Mass continuity is \emph{not} very diffusive for membrane stresses (e.g.~SSA), especially with no basal resistance as in ice shelves.  It has some diffusiveness for ice streams, though how much is hard to quantify.
\item The SIA stress balance is exceptional because it is not horizontally-distributed.  In the SIA, velocity follows immediately by vertical integration of the driving stress.
\item Membrane stress balance equations like the SSA (and the Blatter-Pattyn, hydrostatic, and Stokes models also) determine horizontal velocity from geometry and boundary conditions.  Because of the Glen law these equations are nonlinear, so iteration is necessary.  At each iteration a sparse matrix ``inner'' problem is solved; non-experts should give this matrix problem to a solver package.
\end{itemize}



%\small
\section{Notes} \label{sec:nr}

Recent and recommended books and reviews which extend the continuum modeling content of these notes include \cite{CuffeyPaterson,GreveBlatter2009,SchoofHewitt2013,vanderVeen}.

The SIA model, which was derived by several authors \cite{FowlerLarson1978,Hutter,MorlandJohnson}, follows by scaling the Stokes equations using the aspect ratio $\eps = [H]/[L]$, where $[H]$ is a typical thickness of an ice sheet and $[L]$ is a typical horizontal dimension.  After scaling one drops the terms that are small if $\eps$ is small \cite{Fowler,Hutter}; this is a ``small-parameter argument''.  In one scaling there are no $O(\eps)$ terms in the scaled equations so one only drops $O(\eps^2)$ terms \cite{Fowler}.  The SIA is re-formulated as a well-posed free boundary problem in \cite{JouvetBueler2012}, which provides the correct boundary condition at grounded margins.  The Mahaffy \cite{Mahaffy} scheme for diffusivity used here is not the only one \cite{HindmarshPayne}.

The SSA model \cite{WeisGreveHutter} was derived in \cite{Morland} for ice shelves and in \cite{MacAyeal} for ice streams.  In deriving the SSA, the aspect ratio $\eps$ above is one small parameter but additionally a second parameter describing the magnitude of surface undulations must be assumed to be small  \cite{SchoofStream,SchoofHindmarsh}.  A well-posed model for the emergence of ice streams though till failure, using only the SSA, is in \cite{SchoofStream}.

A key modelling issue omitted in these notes is thermomechanical coupling.  Temperature is important because the ice softness varies by three orders of magnitude in the temperature range relevant to ice sheet modelling.  Ice temperature therefore gives ice sheet dynamics a long memory of past climate, and because the geothermal flux is a significant input in slow-flowing parts of ice sheets.  Equally important, dissipation of gravitational potential energy is a major part of the energy balance, and basal melt in particular.  For example, each year the ice in the Jakobshavn drainage basin in Greenland dissipates enough gravitational potential energy to fully melt more than $1\,\text{km}^3$ of ice \cite{AschwandenBuelerKhroulevBlatter}.  Beautiful evidence that, as a result, outlet glaciers have thick temperate ice is in \cite{Luethietal2009}.  These physical effects motivate modelers to solve the conservation of energy equation simultaneously with the mass conservation (continuity) and momentum conservation (stress balance) equations.  Traditionally the conservation of energy equation uses only temperature as the state variable \cite{BBL}, and this may be suitable for cold ice sheets, but ice sheets are generically polythermal.  Enthalpy methods are a good way to track the energy content of polythermal ice sheets and glaciers \cite{AschwandenBuelerKhroulevBlatter}, though one can also have a separate water-content equation for temperate ice \cite{Greve}.  In any case, the conservation of energy equation is strongly advection-dominated in general \cite{BBL}.

Pressurized basal water is required for most ice sliding.  To model the production of such water in ice sheets one must at least compute the ice base temperature and the basal melt rate through the energy conservation equation \cite{BBssasliding,Clarke05,Raymondenergy,Tulaczyketal2000b}.

One of the most significant issues in modelling ice sheets using shallow models is to describe the ``switch'', in space and time, between shear-dominated and membrane-stress-dominated flow.  It is not a good idea to abruptly switch from the SIA model to the SSA model at the edge of an ice stream, by whatever criterion that switch might be applied, though this has been attempted \cite{HulbeMacAyeal,Ritzetal2001}.  However, ``hybrid'' schemes exist which solve the SIA and SSA everywhere in the ice sheet \cite{BBssasliding,Winkelmannetal2011}, or solve a related vertically-integrated model \cite{Goldberg2011,PollardDeConto}, then combining the stresses or velocities according to different schemes.

``Higher-order'' three-dimensional approximations of the Stokes stress balance, such as the Blatter-Pattyn model \cite{Blatter,Pattyn03}, also use shallow approximations, at minimum including both the most-basic shallow assumption of well-defined thickness (see main text) \emph{and} an assumption of hydrostatic normal stress \cite{GreveBlatter2009}.  Computational limitations generally restrict either the spatial extent, the spatial resolution, or the run duration of these more complete models, primarily because 3D stress balances involve more memory.  Vertically-integrated hybrids can generally be used at higher spatial resolution and longer time scales than higher-order models because the 2D stress balance equations are easier to solve.

As both the SIA and the SSA are derived by small-parameter arguments from the Stokes equations, one might ask whether there is a common shallow antecedent model of both SIA and SSA?  Schoof and Hindmarsh \cite{SchoofHindmarsh} answer that Blatter-Pattyn is one.

Solving the Stokes stress balance itself \cite{JouvetRappaz2011,Lengetal2012,ISMIPHOM} requires explicit accounting for incompressibility through a pressure variable.  Numerical approximations of this stress balance are indefinite, thus harder to solve, essentially because incompressibility is an equality constraint.  In plane flow one can address the incompressibility constraint by using stream functions \cite{BaliseRaymond1985}.  Questions remain about what are the most important deficiencies, relative to the Stokes model, when using either higher-order \cite{ISMIPHOM} or hybrid models.

The finite difference material in these notes should probably be read with reference \cite{MortonMayers} or similar in hand.  The ``main theorem for numerical PDE schemes'' mentioned in the text is the Lax equivalence theorem \cite{MortonMayers}.  Alternative numerical discretization techniques include the finite element \cite{Braess}, finite volume \cite{LeVeque}, and spectral \cite{Trefethen} methods.  Newton iteration for the nonlinear discrete equations is superior to Picard iteration used here, in terms of rapid convergence once iterates are near the solution, but implementation care is needed \cite{Kelley}.

Which are the best numerical models for moving grounding lines?  Even when the minimal SSA stress balance is used, this is still something of an open question \cite{Goldbergetal2009,MISMIP3d2013,MISMIP2012,SchoofMarine1}.  The physics requires at least that the quantities $H$ and $u$ are continuous there, but several stress balance regimes exist near the grounding line, with increasing complexity as one focusses-in on the line \cite{SchoofMarine2}.

Where to find exact solutions for ice flow models?  The textbook Greve and Blatter \cite{GreveBlatter2009} has a few.  Halfar's similarity solution to the SIA \cite{Halfar81,Halfar83} has been generalized to non-zero mass balance \cite{BLKCB}.  There are flow-line \cite{Bodvardsson,vanderVeen83} and cross-flow \cite{SchoofStream} solutions to the SSA model, and one can even construct an exact, steady marine ice sheet in the flow-line case \cite{Bueler2014exactmarine}.  For the Stokes equations themselves there are plane flow solutions for constant viscosity \cite{BaliseRaymond1985}.

As a last resort for numerical verification, one can ``manufacture'' exact solutions by starting with a specified solution and then deriving a source term so that the specified function is actually a solution \cite{Roache}.  There are such manufactured solutions to the thermomechanically-coupled SIA \cite{BBL}, plane flow Blatter-Pattyn model \cite{GlowinskiRappaz}, and Glen-law Stokes equations \cite{JouvetRappaz2011,Lengetal2012,SargentFastook2010}.

\clearpage\newpage
\footnotesize

%\bibliography{ice-bib}
%\bibliographystyle{siam}
\input{notes.bbl}

%\clearpage\newpage
\bigskip
\bigskip
\small
\section*{Exercises}

\newcommand{\exer}[2]{\medskip\noindent \textbf{#1.}\quad #2}

\exer{1}{Assume $f$ has continuous derivatives of all orders.  Show using Taylor's theorem:
  $$f'(x) = \frac{f(x+\Delta) - f(x-\Delta)}{2\Delta} + O(\Delta^2) \quad \text{and} \quad f''(x) = \frac{f(x+\Delta) - 2 f(x) + f(x-\Delta)}{\Delta^2} + O(\Delta^2).$$}

\exer{2}{Sometimes we want finite difference approximations for derivatives in-between grid points.  Continuing exercise \textbf{1}, show $f'(x+(\Delta/2)) = (f(x+\Delta) - f(x))/\Delta + O(\Delta^2)$.}

\exer{3}{Rewrite \texttt{heat.m} using \texttt{for} loops instead of colon notation.  (The only purpose here is to help understand colon notation.)}

\exer{4}{The 1D explicit scheme \eqref{heat1Dfd} for the heat equation, namely $T_j^{n+1} = \mu T_{j+1}^n + (1 - 2 \mu) T_j^n + \mu T_{j-1}^n$, is averaging if stability criterion \eqref{stabcrit} holds.  But of course we must be stepping \emph{forward} in time.  Show that the scheme is not averaging for any values of $\Delta t < 0$.  Try running \texttt{heat.m} backward in time to see what happens.  Moral: \emph{there are no consistent and stable numerical schemes for unstable PDE problems}.}

\exer{5}{Show that when written as a formula for $T_j^{n+1}$, scheme \eqref{implicit1D} has only positive coefficients.  Using \cite{MortonMayers} or other source, as needed, explain why this shows it is unconditionally stable.}

\exer{6}{\emph{This multi-part exercise concerns the numerical treatment of ``$\Div\left(D\,\grad\right)$'' in equation} \eqref{heat}. 
\renewcommand{\labelenumi}{(\alph{enumi})}
\begin{enumerate}
\item Write down the obvious centered $O(\Delta t)+O(\Delta x^2)$ explicit finite difference method for the equation $u_t = D_0 u_{xx} + E_0 u_x$, assuming $D_0>0$ and $E_0$ are constant.  Solve the scheme for the unknown $u_j^{n+1}$. 
\item Stability for the method derived in (a) will occur if $u_j^{n+1}$ is computed from a linear combination of the older values which has all positive (at least, nonnegative) coefficients.  If $|E_0| \gg D_0$, what does this stability assertion say about $\Delta x$?
\item Show that if $D=D(x,y)$ and $u=u(x,y)$ then $\Div \left(D\, \grad u\right) = D \grad^2 u + \grad D \cdot \grad u$.
\item Why do we use the staggered grid for ``$\Div\left(D\,\grad\right)$,'' instead of expanding by the product rule as in part (c)?  Looking at reference \cite{MortonMayers}, or similar, may help you formulate your answer.
\end{enumerate}
}

\exer{7}{Derive the Green's function of the 1D heat equation, namely $T = (4 \pi D t)^{-1/2}\, e^{-x^2/(4Dt)}$, which is a solution to $T_t=D T_{xx}$.  Start by supposing there is a solution of the form $T(t,x) = t^{-1/2} \phi(s)$ where $s=t^{-1/2}x$ is the similarity variable.  Thereby write down an ordinary differential equation in $s$ for $\phi(s)$, and solve it.}

\exer{8}{In the text, and in code \texttt{verifysia.m}, we used Halfar's solution to verify our numerical scheme \texttt{siaflat.m}.  Create the analogous code \texttt{verifyheat.m} to use the Green's function of the 2D heat equation \eqref{heat2D}, namely $T(t,x,y) = (4 \pi D t)^{-1}\, e^{-(x^2+y^2)/(4Dt)}$, to verify \texttt{heatadapt.m}.  You can use the high quality approximation $e^{-A^2}\approx 0$ for $|A|>10$ to choose a rectangular domain in space for which you may use $T=0$ Dirichlet boundary conditions.}

\exer{9}{Is P.~Halfar male or female, and what does the first initial ``P.'' stand for?  (\emph{I don't know the answers to these questions.})}

\exer{10}{Show that formula \eqref{halfar} solves \eqref{sia} in the case of flat bed ($h=H$), zero climatic mass balance ($M=0$), and $n=3$.  You will want to express divergence and gradient in polar coordinates.}

\exer{11}{Show, by appropriate integration of formula \eqref{halfar}, that the volume of ice in the Halfar solution is independent of $t$.}

\exer{12}{In the text it is claimed that any modification of \texttt{siaflat.m} will make the output of \texttt{verifysia.m} show non-convergence, e.g.~the reported average thickness error will not go to zero as the grid is refined.  By randomly altering lines of \texttt{siaflat.m}, or by other methods of your choice, evaluate this claim.}

\exer{13}{Some output from \texttt{verifysia.m} has been suppressed in the text, including a map-plane view of the numerical ice thickness error.  Run the program to see this error map.  Near the grounded margin of an ice sheet this error is much larger than elsewhere.  Why?  Would a higher-order or Stokes model for an ice sheet with an advancing margin have significantly smaller thickness error, on the same grid and supposing we knew a relevant exact solution?}

\exer{14}{Derive the surface kinematical equation \eqref{surfkine} from the mass continuity \eqref{masscont} and base kinematical \eqref{basekine} equations.  You will use the incompressibility of ice, equation \eqref{incompressible}.  Note that the Leibniz rule for differentiating integrals, mentioned in the text, is
  $$\frac{d}{dx}\left(\int_{g(x)}^{f(x)} h(x,y)\,dy\right) = f'(x) h(x,f(x)) - g'(x) h(x,g(x)) + \int_{g(x)}^{f(x)} h_x(x,y)\,dy.$$}

\exer{15}{Let $C_s = A (\rho g (1-r)/4)^n$ where $r=\rho/\rho_w$.  Assume $x_g=0$ is the location of the grounding line.  Derive the two parts of the van der Veen exact ice shelf solution, namely
\begin{align*}
  u &= \left[ u_g^{n+1} + (C_s/M_0) \left((M_0 x + u_g H_g)^{n+1} - (u_g H_g)^{n+1}\right) \right]^{1/(n+1)}, \quad H = (M_0 x + u_g H_g) / u,
\end{align*}
Start from equation \eqref{ssafloat}, and use the fact $(H^2)_x = 2 H H_x$ to generate the first integral of \eqref{ssafloat}.  Also use boundary condition \eqref{calvingstress}.  Get an equation $u_x = C_s H^n$ or equivalent.  On the other hand, note that the mass continuity equation $M_0=(uH)_x$ can be integrated to give the formula $uH = M_0 (x-x_g) + u_g H_g$ for the flux.  Use this expression for $uH$ to find $u(x)$ first, and then write $H(x)$ in terms of $u(x)$ as above.  These exact solutions are from \cite{vanderVeen83}.  They are used in code \texttt{exactshelf.m}.}

\exer{16}{Modify the code \texttt{ssaflowline.m} to solve the ice stream SSA equation \eqref{ssa}.  For boundary conditions it would be reasonable to have fixed velocity at the upstream end, but keep the calving front boundary condition at the downstream end.  I.e.~consider the case where the calving front is at the point where the ice stream reaches flotation.  (See also \cite{Bodvardsson,Bueler2014exactmarine}.)}

\exer{17}{Derive the viscosity form of the flow law \eqref{viscosityflowlaw} from \eqref{flowlaw}, in detail.}

\exer{18}{Derive the hydrostatic \eqref{stresshydrostatic} equation, in detail.}

\end{document}


%\clearpage\newpage
\bigskip
\bigskip
\small
\section*{Exercises}

\newcommand{\exer}[2]{\medskip\noindent \textbf{#1.}\quad #2}

\exer{1}{Assume $f$ has continuous derivatives of all orders.  Show using Taylor's theorem:
  $$f'(x) = \frac{f(x+\Delta) - f(x-\Delta)}{2\Delta} + O(\Delta^2) \quad \text{and} \quad f''(x) = \frac{f(x+\Delta) - 2 f(x) + f(x-\Delta)}{\Delta^2} + O(\Delta^2).$$}

\exer{2}{Sometimes we want finite difference approximations for derivatives in-between grid points.  Continuing exercise \textbf{1}, show $f'(x+(\Delta/2)) = (f(x+\Delta) - f(x))/\Delta + O(\Delta^2)$.}

\exer{3}{Rewrite \texttt{heat.m} using \texttt{for} loops instead of colon notation.  (The only purpose here is to help understand colon notation.)}

\exer{4}{The 1D explicit scheme \eqref{heat1Dfd} for the heat equation, namely $T_j^{n+1} = \mu T_{j+1}^n + (1 - 2 \mu) T_j^n + \mu T_{j-1}^n$, is averaging if stability criterion \eqref{stabcrit} holds.  But of course we must be stepping \emph{forward} in time.  Show that the scheme is not averaging for any values of $\Delta t < 0$.  Try running \texttt{heat.m} backward in time to see what happens.  Moral: \emph{there are no consistent and stable numerical schemes for unstable PDE problems}.}

\exer{5}{Show that when written as a formula for $T_j^{n+1}$, scheme \eqref{implicit1D} has only positive coefficients.  Using \cite{MortonMayers} or other source, as needed, explain why this shows it is unconditionally stable.}

\exer{6}{\emph{This multi-part exercise concerns the numerical treatment of ``$\Div\left(D\,\grad\right)$'' in equation} \eqref{heat}. 
\renewcommand{\labelenumi}{(\alph{enumi})}
\begin{enumerate}
\item Write down the obvious centered $O(\Delta t)+O(\Delta x^2)$ explicit finite difference method for the equation $u_t = D_0 u_{xx} + E_0 u_x$, assuming $D_0>0$ and $E_0$ are constant.  Solve the scheme for the unknown $u_j^{n+1}$. 
\item Stability for the method derived in (a) will occur if $u_j^{n+1}$ is computed from a linear combination of the older values which has all positive (at least, nonnegative) coefficients.  If $|E_0| \gg D_0$, what does this stability assertion say about $\Delta x$?
\item Show that if $D=D(x,y)$ and $u=u(x,y)$ then $\Div \left(D\, \grad u\right) = D \grad^2 u + \grad D \cdot \grad u$.
\item Why do we use the staggered grid for ``$\Div\left(D\,\grad\right)$,'' instead of expanding by the product rule as in part (c)?  Looking at reference \cite{MortonMayers}, or similar, may help you formulate your answer.
\end{enumerate}
}

\exer{7}{Derive the Green's function of the 1D heat equation, namely $T = (4 \pi D t)^{-1/2}\, e^{-x^2/(4Dt)}$, which is a solution to $T_t=D T_{xx}$.  Start by supposing there is a solution of the form $T(t,x) = t^{-1/2} \phi(s)$ where $s=t^{-1/2}x$ is the similarity variable.  Thereby write down an ordinary differential equation in $s$ for $\phi(s)$, and solve it.}

\exer{8}{In the text, and in code \texttt{verifysia.m}, we used Halfar's solution to verify our numerical scheme \texttt{siaflat.m}.  Create the analogous code \texttt{verifyheat.m} to use the Green's function of the 2D heat equation \eqref{heat2D}, namely $T(t,x,y) = (4 \pi D t)^{-1}\, e^{-(x^2+y^2)/(4Dt)}$, to verify \texttt{heatadapt.m}.  You can use the high quality approximation $e^{-A^2}\approx 0$ for $|A|>10$ to choose a rectangular domain in space for which you may use $T=0$ Dirichlet boundary conditions.}

\exer{9}{Is P.~Halfar male or female, and what does the first initial ``P.'' stand for?  (\emph{I don't know the answers to these questions.})}

\exer{10}{Show that formula \eqref{halfar} solves \eqref{sia} in the case of flat bed ($h=H$), zero climatic mass balance ($M=0$), and $n=3$.  You will want to express divergence and gradient in polar coordinates.}

\exer{11}{Show, by appropriate integration of formula \eqref{halfar}, that the volume of ice in the Halfar solution is independent of $t$.}

\exer{12}{In the text it is claimed that any modification of \texttt{siaflat.m} will make the output of \texttt{verifysia.m} show non-convergence, e.g.~the reported average thickness error will not go to zero as the grid is refined.  By randomly altering lines of \texttt{siaflat.m}, or by other methods of your choice, evaluate this claim.}

\exer{13}{Some output from \texttt{verifysia.m} has been suppressed in the text, including a map-plane view of the numerical ice thickness error.  Run the program to see this error map.  Near the grounded margin of an ice sheet this error is much larger than elsewhere.  Why?  Would a higher-order or Stokes model for an ice sheet with an advancing margin have significantly smaller thickness error, on the same grid and supposing we knew a relevant exact solution?}

\exer{14}{Derive the surface kinematical equation \eqref{surfkine} from the mass continuity \eqref{masscont} and base kinematical \eqref{basekine} equations.  You will use the incompressibility of ice, equation \eqref{incompressible}.  Note that the Leibniz rule for differentiating integrals, mentioned in the text, is
  $$\frac{d}{dx}\left(\int_{g(x)}^{f(x)} h(x,y)\,dy\right) = f'(x) h(x,f(x)) - g'(x) h(x,g(x)) + \int_{g(x)}^{f(x)} h_x(x,y)\,dy.$$}

\exer{15}{Let $C_s = A (\rho g (1-r)/4)^n$ where $r=\rho/\rho_w$.  Assume $x_g=0$ is the location of the grounding line.  Derive the two parts of the van der Veen exact ice shelf solution, namely
\begin{align*}
  u &= \left[ u_g^{n+1} + (C_s/M_0) \left((M_0 x + u_g H_g)^{n+1} - (u_g H_g)^{n+1}\right) \right]^{1/(n+1)}, \quad H = (M_0 x + u_g H_g) / u,
\end{align*}
Start from equation \eqref{ssafloat}, and use the fact $(H^2)_x = 2 H H_x$ to generate the first integral of \eqref{ssafloat}.  Also use boundary condition \eqref{calvingstress}.  Get an equation $u_x = C_s H^n$ or equivalent.  On the other hand, note that the mass continuity equation $M_0=(uH)_x$ can be integrated to give the formula $uH = M_0 (x-x_g) + u_g H_g$ for the flux.  Use this expression for $uH$ to find $u(x)$ first, and then write $H(x)$ in terms of $u(x)$ as above.  These exact solutions are from \cite{vanderVeen83}.  They are used in code \texttt{exactshelf.m}.}

\exer{16}{Modify the code \texttt{ssaflowline.m} to solve the ice stream SSA equation \eqref{ssa}.  For boundary conditions it would be reasonable to have fixed velocity at the upstream end, but keep the calving front boundary condition at the downstream end.  I.e.~consider the case where the calving front is at the point where the ice stream reaches flotation.  (See also \cite{Bodvardsson,Bueler2014exactmarine}.)}

\exer{17}{Derive the viscosity form of the flow law \eqref{viscosityflowlaw} from \eqref{flowlaw}, in detail.}

\exer{18}{Derive the hydrostatic \eqref{stresshydrostatic} equation, in detail.}

\end{document}


%\clearpage\newpage
\bigskip
\bigskip
\small
\section*{Exercises}

\newcommand{\exer}[2]{\medskip\noindent \textbf{#1.}\quad #2}

\exer{1}{Assume $f$ has continuous derivatives of all orders.  Show using Taylor's theorem:
  $$f'(x) = \frac{f(x+\Delta) - f(x-\Delta)}{2\Delta} + O(\Delta^2) \quad \text{and} \quad f''(x) = \frac{f(x+\Delta) - 2 f(x) + f(x-\Delta)}{\Delta^2} + O(\Delta^2).$$}

\exer{2}{Sometimes we want finite difference approximations for derivatives in-between grid points.  Continuing exercise \textbf{1}, show $f'(x+(\Delta/2)) = (f(x+\Delta) - f(x))/\Delta + O(\Delta^2)$.}

\exer{3}{Rewrite \texttt{heat.m} using \texttt{for} loops instead of colon notation.  (The only purpose here is to help understand colon notation.)}

\exer{4}{The 1D explicit scheme \eqref{heat1Dfd} for the heat equation, namely $T_j^{n+1} = \mu T_{j+1}^n + (1 - 2 \mu) T_j^n + \mu T_{j-1}^n$, is averaging if stability criterion \eqref{stabcrit} holds.  But of course we must be stepping \emph{forward} in time.  Show that the scheme is not averaging for any values of $\Delta t < 0$.  Try running \texttt{heat.m} backward in time to see what happens.  Moral: \emph{there are no consistent and stable numerical schemes for unstable PDE problems}.}

\exer{5}{Show that when written as a formula for $T_j^{n+1}$, scheme \eqref{implicit1D} has only positive coefficients.  Using \cite{MortonMayers} or other source, as needed, explain why this shows it is unconditionally stable.}

\exer{6}{\emph{This multi-part exercise concerns the numerical treatment of ``$\Div\left(D\,\grad\right)$'' in equation} \eqref{heat}. 
\renewcommand{\labelenumi}{(\alph{enumi})}
\begin{enumerate}
\item Write down the obvious centered $O(\Delta t)+O(\Delta x^2)$ explicit finite difference method for the equation $u_t = D_0 u_{xx} + E_0 u_x$, assuming $D_0>0$ and $E_0$ are constant.  Solve the scheme for the unknown $u_j^{n+1}$. 
\item Stability for the method derived in (a) will occur if $u_j^{n+1}$ is computed from a linear combination of the older values which has all positive (at least, nonnegative) coefficients.  If $|E_0| \gg D_0$, what does this stability assertion say about $\Delta x$?
\item Show that if $D=D(x,y)$ and $u=u(x,y)$ then $\Div \left(D\, \grad u\right) = D \grad^2 u + \grad D \cdot \grad u$.
\item Why do we use the staggered grid for ``$\Div\left(D\,\grad\right)$,'' instead of expanding by the product rule as in part (c)?  Looking at reference \cite{MortonMayers}, or similar, may help you formulate your answer.
\end{enumerate}
}

\exer{7}{Derive the Green's function of the 1D heat equation, namely $T = (4 \pi D t)^{-1/2}\, e^{-x^2/(4Dt)}$, which is a solution to $T_t=D T_{xx}$.  Start by supposing there is a solution of the form $T(t,x) = t^{-1/2} \phi(s)$ where $s=t^{-1/2}x$ is the similarity variable.  Thereby write down an ordinary differential equation in $s$ for $\phi(s)$, and solve it.}

\exer{8}{In the text, and in code \texttt{verifysia.m}, we used Halfar's solution to verify our numerical scheme \texttt{siaflat.m}.  Create the analogous code \texttt{verifyheat.m} to use the Green's function of the 2D heat equation \eqref{heat2D}, namely $T(t,x,y) = (4 \pi D t)^{-1}\, e^{-(x^2+y^2)/(4Dt)}$, to verify \texttt{heatadapt.m}.  You can use the high quality approximation $e^{-A^2}\approx 0$ for $|A|>10$ to choose a rectangular domain in space for which you may use $T=0$ Dirichlet boundary conditions.}

\exer{9}{Is P.~Halfar male or female, and what does the first initial ``P.'' stand for?  (\emph{I don't know the answers to these questions.})}

\exer{10}{Show that formula \eqref{halfar} solves \eqref{sia} in the case of flat bed ($h=H$), zero climatic mass balance ($M=0$), and $n=3$.  You will want to express divergence and gradient in polar coordinates.}

\exer{11}{Show, by appropriate integration of formula \eqref{halfar}, that the volume of ice in the Halfar solution is independent of $t$.}

\exer{12}{In the text it is claimed that any modification of \texttt{siaflat.m} will make the output of \texttt{verifysia.m} show non-convergence, e.g.~the reported average thickness error will not go to zero as the grid is refined.  By randomly altering lines of \texttt{siaflat.m}, or by other methods of your choice, evaluate this claim.}

\exer{13}{Some output from \texttt{verifysia.m} has been suppressed in the text, including a map-plane view of the numerical ice thickness error.  Run the program to see this error map.  Near the grounded margin of an ice sheet this error is much larger than elsewhere.  Why?  Would a higher-order or Stokes model for an ice sheet with an advancing margin have significantly smaller thickness error, on the same grid and supposing we knew a relevant exact solution?}

\exer{14}{Derive the surface kinematical equation \eqref{surfkine} from the mass continuity \eqref{masscont} and base kinematical \eqref{basekine} equations.  You will use the incompressibility of ice, equation \eqref{incompressible}.  Note that the Leibniz rule for differentiating integrals, mentioned in the text, is
  $$\frac{d}{dx}\left(\int_{g(x)}^{f(x)} h(x,y)\,dy\right) = f'(x) h(x,f(x)) - g'(x) h(x,g(x)) + \int_{g(x)}^{f(x)} h_x(x,y)\,dy.$$}

\exer{15}{Let $C_s = A (\rho g (1-r)/4)^n$ where $r=\rho/\rho_w$.  Assume $x_g=0$ is the location of the grounding line.  Derive the two parts of the van der Veen exact ice shelf solution, namely
\begin{align*}
  u &= \left[ u_g^{n+1} + (C_s/M_0) \left((M_0 x + u_g H_g)^{n+1} - (u_g H_g)^{n+1}\right) \right]^{1/(n+1)}, \quad H = (M_0 x + u_g H_g) / u,
\end{align*}
Start from equation \eqref{ssafloat}, and use the fact $(H^2)_x = 2 H H_x$ to generate the first integral of \eqref{ssafloat}.  Also use boundary condition \eqref{calvingstress}.  Get an equation $u_x = C_s H^n$ or equivalent.  On the other hand, note that the mass continuity equation $M_0=(uH)_x$ can be integrated to give the formula $uH = M_0 (x-x_g) + u_g H_g$ for the flux.  Use this expression for $uH$ to find $u(x)$ first, and then write $H(x)$ in terms of $u(x)$ as above.  These exact solutions are from \cite{vanderVeen83}.  They are used in code \texttt{exactshelf.m}.}

\exer{16}{Modify the code \texttt{ssaflowline.m} to solve the ice stream SSA equation \eqref{ssa}.  For boundary conditions it would be reasonable to have fixed velocity at the upstream end, but keep the calving front boundary condition at the downstream end.  I.e.~consider the case where the calving front is at the point where the ice stream reaches flotation.  (See also \cite{Bodvardsson,Bueler2014exactmarine}.)}

\exer{17}{Derive the viscosity form of the flow law \eqref{viscosityflowlaw} from \eqref{flowlaw}, in detail.}

\exer{18}{Derive the hydrostatic \eqref{stresshydrostatic} equation, in detail.}

\end{document}


\begin{comment}
\bigskip
\bigskip
\small
\section*{Exercises}

\newcommand{\exer}[2]{\medskip\noindent \textbf{#1.}\quad #2}

\exer{1}{Assume $f$ has continuous derivatives of all orders.  Show using Taylor's theorem:
  $$f'(x) = \frac{f(x+\Delta) - f(x-\Delta)}{2\Delta} + O(\Delta^2) \quad \text{and} \quad f''(x) = \frac{f(x+\Delta) - 2 f(x) + f(x-\Delta)}{\Delta^2} + O(\Delta^2).$$}

\exer{2}{Sometimes we want finite difference approximations for derivatives in-between grid points.  Continuing exercise \textbf{1}, show $f'(x+(\Delta/2)) = (f(x+\Delta) - f(x))/\Delta + O(\Delta^2)$.}

\exer{3}{To help understand colon notation, rewrite \texttt{heat.m} using \texttt{for} loops instead.}

\exer{4}{The 1D explicit scheme \eqref{heat1Dfd} for the heat equation, namely $T_j^{n+1} = \mu T_{j+1}^n + (1 - 2 \mu) T_j^n + \mu T_{j-1}^n$, is averaging if stability criterion \eqref{stabcrit} holds.  But of course we must be stepping \emph{forward} in time.  Show that the scheme is not averaging for any values of $\Delta t < 0$.  Try running \texttt{heat.m} backward in time to see what happens.  Moral: \emph{there are no consistent and stable numerical schemes for unstable PDE problems}.}

\exer{5}{Show that when written as a formula for $T_j^{n+1}$, implicit scheme \eqref{implicit1D} has only positive coefficients.  Explain why this shows it is unconditionally stable.  Refer to \cite{LeVequeFD,MortonMayers}.}

\exer{6}{\emph{This multi-part exercise concerns the staggered-grid treatment of $\Div\left(D\,\grad\dots\right)$}.
\renewcommand{\labelenumi}{(\alph{enumi})}
\begin{enumerate}
\item Write down the obvious centered $O(\Delta t)+O(\Delta x^2)$ explicit finite difference method for the equation $u_t = D_0 u_{xx} + E_0 u_x$, assuming $D_0>0$ and $E_0$ are constant.  Solve the scheme for the unknown $u_j^{n+1}$. 
\item Stability for the method in (a) occurs if $u_j^{n+1}$ is computed from a linear combination of time $t_n$ values which has all nonnegative coefficients.  If $|E_0| \gg D_0$, what does this stability assertion say about $\Delta x$?
\item Show that if $D=D(x,y)$ and $u=u(x,y)$ then $\Div \left(D\, \grad u\right) = D \grad^2 u + \grad D \cdot \grad u$.
\item Why do we use the staggered grid for ``$\Div\left(D\,\grad\right)$,'' instead of expanding by the product rule as in part (c)?  References \cite{LeVequeFD,MortonMayers} may help you formulate your answer.
\end{enumerate}
}

\exer{7}{Derive the Green's function of the 1D heat equation, namely $T = (4 \pi D t)^{-1/2}\, e^{-x^2/(4Dt)}$, which is a solution to $T_t=D T_{xx}$, as follows:  Suppose there is a solution of the form $T(t,x) = t^{-1/2} \phi(s)$, where $s=t^{-1/2}x$ is the similarity variable, and then apply the chain rule.  Generate an ordinary differential equation for $\phi(s)$, in the single variable $s$, and solve it.}

\exer{8}{In the text, and in the code \texttt{verifysia.m}, we used Halfar's solution to verify our numerical scheme \texttt{siaflat.m}.  Create the analogous code \texttt{verifyheat.m} to use the Green's function (previous exercise) of the 2D heat equation \eqref{heat2D}, to verify \texttt{heatadapt.m}.  You can use the high quality approximation $e^{-A^2}\approx 0$ for $|A|>10$ to choose a rectangular domain in space for which you may use $T=0$ Dirichlet boundary conditions.  (This hack is unnecessary for the SIA!)}

\exer{9}{Show that the Halfar solution formula \eqref{halfar} solves the SIA equation \eqref{sia} in the case of flat bed ($h=H$), zero climatic mass balance ($M=0$), and $n=3$.  You will want to express divergence and gradient in polar coordinates.}

\exer{10}{Show, by integrating formula \eqref{halfar}, and changing variables, that the volume of ice in the Halfar solution is independent of $t$.}

\exer{11}{In the text it is claimed that any modification of \texttt{siaflat.m} will make the output of \texttt{verifysia.m} show non-convergence.  For example, the reported average thickness error will not go to zero at the correct rate, as the grid is refined, if there are coding errors.  By randomly altering lines of \texttt{siaflat.m}, or by other methods of your choice, evaluate this claim.}

\exer{12}{Some output from \texttt{verifysia.m} has been suppressed in the text, including a map-plane view of the numerical ice thickness error.  Run the program to see this error map.  Near the ice sheet margin this error is much larger than elsewhere.  Why?  On the same grids, would a higher-order or Stokes model for an ice sheet with an advancing margin have significantly smaller thickness error, supposing we knew a relevant exact solution?}

\exer{13}{Derive the surface kinematical equation \eqref{surfkine} from the mass continuity \eqref{masscont} and base kinematical \eqref{basekine} equations.  You will use the incompressibility of ice, equation \eqref{incompressible}.  Note that the Leibniz rule for differentiating integrals, mentioned in the text, is
  $$\frac{d}{dx}\left(\int_{g(x)}^{f(x)} h(x,y)\,dy\right) = f'(x) h(x,f(x)) - g'(x) h(x,g(x)) + \int_{g(x)}^{f(x)} h_x(x,y)\,dy.$$}

\exer{14}{Let $C_s = A (\rho g (1-r)/4)^n$ where $r=\rho/\rho_w$.  Assume $x_g=0$ is the location of the grounding line.  Derive the two parts of the van der Veen exact ice shelf solution, namely
\begin{align*}
  u &= \left[ u_g^{n+1} + (C_s/M_0) \left((M_0 x + u_g H_g)^{n+1} - (u_g H_g)^{n+1}\right) \right]^{1/(n+1)}, \quad H = (M_0 x + u_g H_g) / u,
\end{align*}
Start from equation \eqref{ssafloat}, and use the fact $(H^2)_x = 2 H H_x$ to generate the first integral of \eqref{ssafloat}.  Also use boundary condition \eqref{calvingstress}.  Get an equation $u_x = C_s H^n$ or equivalent.  On the other hand, note that the mass continuity equation $M_0=(uH)_x$ can be integrated to give the formula $uH = M_0 (x-x_g) + u_g H_g$ for the flux.  Use this expression for $uH$ to find $u(x)$ first, and then write $H(x)$ in terms of $u(x)$ as above.  These exact solutions are from \cite{vanderVeen83}.  They are used in code \texttt{exactshelf.m}.}

\exer{15}{Modify the code \texttt{ssaflowline.m} to solve the ice stream SSA equation \eqref{ssa}.  For boundary conditions it would be reasonable to have fixed velocity at the upstream end, but keep the calving front boundary condition at the downstream end; the calving front is at the point where the ice stream reaches flotation.  Compare exact solutions in \cite{Bodvardsson,Bueler2014exactmarine}.}

\exer{16}{Derive the hydrostatic \eqref{stresshydrostatic} equation, in detail.}

\exer{17}{Build a better glacier or ice sheet model.}
\end{comment}
\end{document}
