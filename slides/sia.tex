
\section{shallow ice sheets}

\begin{frame}{slow, non-Newtonian, shallow, and sliding}

\begin{itemize}
\item ice sheets have four outstanding properties \emph{as fluids}:
  \begin{enumerate}
  \item slow
  \item non-Newtonian
  \item shallow (usually)
  \item contact slip (sometimes)
  \end{enumerate}
\end{itemize}
\end{frame}


\begin{frame}{regarding ``shallow''}

\begin{itemize}
\item below in \alert{red} is a no-vertical-exaggeration cross section of Greenland at $71^\circ$
\item green and blue: standard vertically-exaggerated cross section
\end{itemize}

\begin{center}
  \includegraphics[width=0.6\textwidth]{green-transect}
\end{center}
\end{frame}


\subsection{shallow ice approx (SIA)}

\begin{frame}{flow model I: non-sliding, isothermal shallow ice approximation = (SIA)}

a model which applies to
\begin{itemize}
\item small depth-to-width ratio (``shallow'') grounded ice sheets
\item on not-too-rough bed topography,
\item whose flow is \emph{not} dominated by sliding and/or liquid water at the base or margin
\end{itemize}

\begin{center}
  \includegraphics[width=0.65\textwidth]{polaris}

\tiny ``Polaris Glacier,'' northwest Greenland, photo 122, Post \& LaChapelle (2000)\nocite{PostLaChapelle}
\end{center}

\end{frame}


\begin{frame}{SIA model equations}

\begin{itemize}
\item \small though the best explanation of the SIA is to use shallowness to simplify the Stokes equations, here we take the simple slogan:\normalsize

\begin{center}
\emph{the SIA uses the formulas from slab-on-a-slope}
\end{center}
\item shear stress approximation:
	$$(\tau_{13},\tau_{23}) = - \rho g (h-z) \nabla h$$
\item let $\mathbf{u} = (u,v)$ be the horizontal velocity
\item use this approximation:
\begin{align*}
\mathbf{u}_z &= 2 A |(\tau_{13},\tau_{23})|^{n-1} (\tau_{13},\tau_{23}) \\
     &= - 2 A (\rho g)^n (h-z)^n |\nabla h|^{n-1} \nabla h
\end{align*}
\item by integrating vertically, in the non-sliding case,
    $$\mathbf{u} = - \frac{2 A (\rho g)^n}{n+1} \left[H^{n+1} - (h-z)^{n+1}\right] |\nabla h|^{n-1} \nabla h$$
\item but mass continuity remains, $H_t = M - \left(\overline{\mathbf{u}} H\right)_x$
\end{itemize}
\end{frame}


\begin{frame}{SIA thickness equation}

\begin{itemize}
\item from last slide, we get the non-sliding, isothermal shallow ice approximation for how thickness changes:
\begin{empheq}[box=\fbox]{equation}
H_t = M + \Div \left(\Gamma H^{n+2} |\grad h|^{n-1} \grad h \right) \label{sia}
\end{empheq}

\vspace{-2mm}
  \begin{itemize}
  \item[$\circ$] where $H$ is ice thickness, $h$ is ice surface elevation, $b$ is bed elevation ($h=H+b$)
  \item[$\circ$] $M$ combines surface and basal mass (im)balance:

     accumulation if $M>0$, ablation if $M<0$
  \item[$\circ$] $n$ is the exponent in the Glen flow law
  \item[$\circ$] $\Gamma = 2 A (\rho g)^n / (n+2)$ is a positive constant
  \end{itemize}
\item numerically solve (1) and you've got a usable model for \dots \emph{the Barnes ice cap} (Mahaffy, 1976)\nocite{Mahaffy}
\end{itemize}
\medskip

\begin{columns}
\begin{column}{0.7\textwidth}
\small
\noindent good questions:
\begin{enumerate}
\item where does equation (1) come from?
\item how to solve it numerically?
\item how to \emph{think} about it?
\end{enumerate}  
\end{column}
\begin{column}{0.3\textwidth}
\includegraphics[width=0.8\textwidth]{mahaffy-profiles}
\end{column}
\end{columns}
\end{frame}


\subsection{analogy w heat equation}

\begin{frame}{heat equation}
\label{slide:heatcompare}

\small
\begin{columns}
\begin{column}{0.6\textwidth}
\begin{itemize}
\item to understand SIA, see heat equation
\item recall Newton's law of cooling
	$$\frac{dT}{dt} = -K (T-T_{\text{ambient}})$$
where $T$ is object temperature and $K$ relates to material and geometry of object (e.g.~cup of coffee)
\item Newton's law for segments of a rod:
\begin{align*}
\frac{dT_j}{dt} &= -K \left(T_j - \frac{1}{2} (T_{j-1} + T_{j+1}) \right) \\
	&= \frac{K}{2} \left(T_{j-1} - 2 T_j + T_{j+1}\right) 
\end{align*}
\item this has limit as segments shrink:
	$$T_t = D T_{xx}$$
\end{itemize}
\end{column}

\begin{column}{0.4\textwidth}
\hfill
\includegraphics[width=0.5\textwidth]{coffee}
\vspace{1.0in}
\includegraphics[width=1.0\textwidth]{heatconduction}
\end{column}
\end{columns}
\end{frame}

\begin{frame}{analogy: SIA versus 2D heat equation}

\begin{itemize}
\item general heat eqn: $T_t = F + \Div (D\, \grad T)$
\item side-by-side comparison:

\medskip
\begin{tabular}{cc}
\scriptsize SIA for ice thickness \, $H(t,x,y)$ & \scriptsize heat eqn for temperature $T(t,x,y)$ \normalsize \medskip \\
	\hspace{-6mm} $H_t = M + \Div \left({\color{red}\Gamma H^{n+2} |\grad h|^{n-1}}\, \grad h \right)$  &  $T_t = F + \Div (D\, \grad T)$
\end{tabular} 

\medskip
\item identify the diffusivity in the SIA:
	$$D = {\color{red}\Gamma H^{n+2} |\grad h|^{n-1}}$$
\item non-sliding shallow ice flow \emph{diffuses} the ice sheet
\item ``issues'' with this analogy:
  \begin{itemize}
  \item[$\circ$]  $D$ depends on solution $H(t,x,y)$
  \item[$\circ$]  $D\to 0$ at margin, where $H\to 0$
  \item[$\circ$]  $D\to 0$ at divides/domes, where $|\grad h|\to 0$
  \end{itemize}
\end{itemize}
\end{frame}


\subsection{finite difference numerics}

\begin{frame}{numerics for heat equation: basic ideas of finite differences}

\begin{itemize}
\item numerical schemes for heat equation are good start for SIA
\item for differentiable $f(x)$ and any $h$, \emph{Taylor's theorem} says
	$$f(x+h) = f(x) + f'(x) h + \frac{1}{2} f''(x) h^2 + \frac{1}{3!} f'''(x) h^3 + \dots$$
\normalsize
\item you can replace ``$h$'' by multiples of $\Delta x$, e.g.:
\small
\begin{align*}
f(x-\Delta x) &= f(x) - f'(x) \Delta x + \frac{1}{2} f''(x) \Delta x^2 - \frac{1}{3!} f'''(x) \Delta x^3 + \dots \\
f(x+2\Delta x) &= f(x) + 2 f'(x) \Delta x + 2 f''(x) \Delta x^2 + \frac{4}{3} f'''(x) \Delta x^3 + \dots
\end{align*}
\normalsize
\item \emph{main idea}:  combine expressions like these to give approximations of derivatives, from values on a grid
\end{itemize}
\end{frame}


\begin{frame}{finite differences for partial derivatives}

\begin{itemize}
\item we want partial derivative expressions, for example with any function $u=u(t,x)$:
\small
\begin{align*}
u_t(t,x) &= \frac{u(t+\Delta t,x) - u(t,x)}{\Delta t} + O(\Delta t), \\
u_t(t,x) &= \frac{u(t+\Delta t,x) - u(t-\Delta t,x)}{2\Delta t} + O(\Delta t^2), \\
u_x(t,x) &= \frac{u(t,x+\Delta x) - u(t,x-\Delta x)}{2\Delta x} + O(\Delta x^2), \\
u_{xx}(t,x) &= \frac{u(t,x+\Delta x) - 2 u(t,x) + u(t,x-\Delta x)}{\Delta x^2} + O(\Delta x^2)
\end{align*}
\normalsize
and so on
\item sometimes we want a derivative in-between grid points:
\small
	$$u_x(t,x+(\Delta x/2)) = \frac{u(t,x+\Delta x) - u(t,x)}{\Delta x} + O(\Delta x^2)$$
\normalsize
\item ``$+O(h^2)$'' is better than ``$+O(h)$'' if $h$ is a small number
\end{itemize}
\end{frame}


\begin{frame}{explicit scheme for heat equation}
\label{slide:explicit}

\begin{itemize}
\item consider 1D heat equation $T_t = D T_{xx}$
\item approximately true:
\small
	$$\frac{T(t+\Delta t,x) - T(t,x)}{\Delta t} \approx D\,\frac{T(t,x+\Delta x) - 2 T(t,x) + T(t,x-\Delta x)}{\Delta x^2}$$
\normalsize
\item the difference between the equation $T_t = D T_{xx}$ and the scheme is $O(\Delta t,\Delta x^2)$ (Morton and Mayers, 2005)\nocite{MortonMayers}
\item notation: $(t_n,x_j)$ is a point in the time-space grid
\item notation: $T_j^n \approx T(t_n,x_j)$  \only<2>{\hfill \alert{$\longleftarrow$ can you distinguish these?}}
\item let $\nu = D \Delta t / (\Delta x)^2$, so ``explicit'' scheme is
\small
	$$T_j^{n+1} = \nu T_{j+1}^n + (1 - 2 \nu) T_j^n + \nu T_{j-1}^n$$
\normalsize
\end{itemize}
\begin{columns}
\begin{column}{0.55\textwidth}
\begin{itemize}
\item ``stencil'' at right \large $\to$ \normalsize
\end{itemize}
\end{column}
\begin{column}{0.45\textwidth}
\includegraphics[width=0.7\textwidth]{expstencil}
\end{column}
\end{columns}
\end{frame}


\begin{frame}{explicit scheme in 2D}

\begin{itemize}
\item recall heat equation two spatial variables (2D):
    $$T_t = D(T_{xx} + T_{yy})$$
\item again: $T_{jk}^n \approx T(t_n,x_j,y_k)$
\item the 2D explicit scheme is
\small
	$$\frac{T_{jk}^{n+1} - T_{jk}^n}{\Delta t} = D\,\left(\frac{T_{j+1,k}^n - 2 T_{jk}^n + T_{j-1,k}^n}{\Delta x^2} + \frac{T_{j,k+1}^n - 2 T_{jk}^n + T_{j,k-1}^n}{\Delta y^2}\right)$$
\end{itemize}

\bigskip
\begin{center}
\includegraphics[width=0.35\textwidth]{exp2dstencil}
\end{center}
\end{frame}


\begin{frame}{implementation}
\label{slide:heatmatlab}

\minput{heat}

\small
\begin{itemize}
\item solves $T_t = D(T_{xx} + T_{yy})$ on square $-1 < x < 1$, $-1 < y < 1$
\item example uses initial condition $T_0(x,y) = e^{-30 r^2}$
\item code uses ``colon notation'' to remove loops (over space)
\item \texttt{>>  heat(1.0,30,30,0.001,20)}

approximates $T$ on $30\times 30$ spatial grid, with $D=1$ and $N=20$ steps of $\Delta t = 0.001$
\end{itemize}
\end{frame}


\begin{frame}{the look of success}

\begin{itemize}
\item solving $T_t = D(T_{xx} + T_{yy})$ on $30\times 30$ grid
\end{itemize}

\bigskip\bigskip
\begin{columns}
\begin{column}{0.5\textwidth}
initial condition $T(0,x,y)$

\bigskip
\begin{center}
\includegraphics[width=1.0\textwidth]{initialheat}
\end{center}
\end{column}
\begin{column}{0.5\textwidth}
approximate solution $T(t,x,y)$ at $t=0.02$ with $\Delta t=0.001$ 

\bigskip
\begin{center}
\includegraphics[width=1.0\textwidth]{finalheat}
\end{center}
\end{column}
\end{columns}
\end{frame}


\begin{frame}{the look of instability}

\begin{itemize}
\item both figures are from solving $T_t = D(T_{xx} + T_{yy})$ on the same space grid, but with slightly different time steps
\end{itemize}

\bigskip\bigskip
\begin{columns}
\begin{column}{0.5\textwidth}
\begin{center}
\includegraphics[width=1.0\textwidth]{stability}

\uncover<2->{$$\frac{D\Delta t}{\Delta x^2}= 0.2$$}
\end{center}
\end{column}
\begin{column}{0.5\textwidth}
\begin{center}
\includegraphics[width=1.0\textwidth]{instability}

\uncover<2->{$$\frac{D\Delta t}{\Delta x^2}= 0.4$$}
\end{center}
\end{column}
\end{columns}
\end{frame}


\begin{frame}{avoid the instability}
\label{slide:maxprinc}

\begin{itemize}
\item recall 1D explicit scheme had the form 
	$$T_j^{n+1} = \nu T_{j+1}^n + (1 - 2 \nu) T_j^n + \nu T_{j-1}^n$$
\item thus the new value $u_j^{n+1}$ is an \emph{average} of the old values, \emph{if the middle coefficient is positive}:
	$$1 - 2 \nu \ge 0 \quad \iff \quad  \frac{D\Delta t}{\Delta x^2} \le \frac{1}{2} \quad \iff \quad \Delta t \le \frac{\Delta x^2}{2 D}$$
\item averaging is always stable because averaged wiggles are always smaller than the original wiggles
\item this condition is a sufficient \emph{stability criterion}
\item so:

\begin{center}
\emph{the result was unstable because the time step was too big}
\end{center}
\end{itemize}
\end{frame}


\begin{frame}{\textsl{adaptive} implementation: guaranteed stability}

\minput{heatadapt}

\begin{itemize}
\item same as \texttt{heat.m} except

\begin{center}
\emph{choose time step from stability criterion}
\end{center}
\end{itemize}\end{frame}


\begin{frame}{alternative instability fix: implicitness}

\begin{itemize}
\item \alert{implicit} methods can be designed to be stable for \emph{any} positive time step $\Delta t$
\end{itemize}

\begin{columns}
\begin{column}{0.7\textwidth}
\small
\begin{itemize}
\item example is \emph{Crank-Nicolson} scheme $\longrightarrow$
\item has smaller error too: $O(\Delta t^2,\Delta x^2)$
\end{itemize}
\normalsize
\end{column}
\begin{column}{0.3\textwidth}
\includegraphics[width=1.2\textwidth]{cnstencil}
\end{column}
\end{columns}

\bigskip
\begin{itemize}
\item \emph{but} you must solve linear (or nonlinear) systems of equations to take each time step
\medskip

\item \scriptsize Donald Knuth has advice for ice sheet modelers: \begin{quote}
\emph{We should forget about small efficiencies \dots: premature optimization is the root of all evil}.
\end{quote}
\end{itemize}
\end{frame}


\begin{frame}{variable diffusivity and time steps}

\begin{itemize}
  \item recall the analogy: \qquad (SIA) $\leftrightarrow$ (heat eqn)
  \item the SIA has a diffusivity which varies in space, so consider a more general heat equation:
  		$$T_t = F + \Div \left(D(x,y) \grad T\right)$$
  \item the explicit method is conditionally stable with the same time step restriction if we evaluate diffusivity $D(x,y)$ at \alert{staggered} grid points:
  \scriptsize
\begin{align*}
\Div \left(D(x,y) \grad u\right) &\approx \frac{D_{j+1/2,k}(T_{j+1,k} - T_{j,k}) - D_{j-1/2,k}(T_{j,k} - T_{j-1,k})}{\Delta x^2} \\
	&\qquad + \frac{D_{j,k+1/2}(T_{j,k+1} - T_{j,k}) - D_{j,k-1/2}(T_{j,k} - T_{j,k-1})}{\Delta y^2}
\end{align*}
\end{itemize}

\vspace{-0.15in}
\small
\begin{columns}
\begin{column}{0.55\textwidth}
in stencil at right:
\begin{itemize}
\item[] diamonds: $T$
\item[] triangles: $D$
\end{itemize}
\end{column}
\begin{column}{0.45\textwidth}
\begin{center}
\includegraphics[width=0.8\textwidth]{diffstencil}
\end{center}
\end{column}
\end{columns}
\end{frame}


\begin{frame}
  \frametitle{general diffusion equation code}

\minputtiny{diffusion}

\small
\begin{itemize}
\item solves abstract diffusion equation $T_t = \Div \left(D(x,y)\, \grad T\right)$
\item user supplies diffusivity on staggered grid
\end{itemize}
\end{frame}


\subsection{solutions}

\begin{frame}{interruption: verification}
\begin{itemize}
\item how do you make sure your \emph{implemented} numerical ice flow code is correct?
  \begin{itemize}
  \item[$\circ$] \emph{technique} 1: don't make any mistakes
  \item[$\circ$] \emph{technique} 2: compare your model with others, and hope that the outliers are the ones with errors
  \item[$\circ$] \emph{technique} 3: build-in a comparison to an exact solution, and actually measure the numerical error $=$ \alert{verification}
  \end{itemize}

\medskip
\item where to get exact solutions for ice flow models?
  \begin{itemize}
  \item[$\circ$] textbooks: Greve and Blatter (2009), van der Veen (2013)
  \item[$\circ$] similarity solns to SIA (Halfar 1983; Bueler et al 2005)
  \item[$\circ$] manufactured solns to thermo-coupled SIA (Bueler et al 2007)
  \item[$\circ$] flowline and cross-flow SSA solns (Bodvardsson, 1955; van der Veen, 1985; Schoof, 2006; Bueler 2014)
  \item[$\circ$] flowline Blatter solns (Glowinski and Rappaz 2003)
  \item[$\circ$] constant viscosity flowline Stokes solns (Ladyzhenskaya 1963, Balise and Raymond 1985)
  \item[$\circ$] manufactured solns to Stokes equations (Sargent and Fastook 2010; Jouvet and Rappaz 2011; Leng et al 2013)
  \end{itemize}
\end{itemize}
\end{frame}


\begin{frame}{an exact solution of heat equation}

\begin{itemize}
\item the simple heat equation in 1D with constant diffusivity $D>0$ is:
	$$T_t = D T_{xx}$$
\item many \emph{exact} solutions to the heat equation are known
\item I'll show the Green's function solution (a.k.a.~``fundamental solution'' or ``heat kernel'')
\item it starts at time $t=0$ with a ``delta function'' of heat at the origin $x=0$ and then it spreads out over time
\item we find it by a method which generalizes to the SIA
\end{itemize}
\end{frame}


\begin{frame}{Green's function of heat equation}

\begin{itemize}
\item the solution is ``self-similar'' over time
\item as time goes it changes shape by
  \begin{itemize}
  \item[$\circ$] shrinking the output (vertical) axis and
  \item[$\circ$] lengthening the input (horizontal) axis
  \end{itemize}
\item \dots but otherwise it is the same shape
\item the integral over $x$ is independent of time
\end{itemize}

\begin{center}
\includegraphics[width=0.5\textwidth]{heatscaling}

\emph{increasing time} \Large $\to$
\end{center}
\end{frame}


\begin{frame}{similarity solutions}

\begin{itemize}
\item Green's function of heat equation in 1D is
	$$T(t,x) = C\, t^{-1/2} e^{-x^2/(4Dt)}$$
\item ``similarity'' variables for 1D heat equation are
	$$s \stackrel{\text{\emph{input scaling}}}{\phantom{\Big|}=\phantom{\Big|}} t^{-1/2} x, \qquad T(t,x) \stackrel{\text{\emph{output scaling}}}{\phantom{\Big|}=\phantom{\Big|}} t^{-1/2} \phi(s)$$
\end{itemize}
\begin{columns}
\begin{column}{0.6\textwidth}
\begin{itemize}
\item \emph{1905}: Einstein discovers that the average distance traveled by particles in thermal motion scales like $\sqrt{t}$, so $s = t^{-1/2}x$ is an invariant
\end{itemize}
\end{column}
\begin{column}{0.4\textwidth}
\begin{center}
\includegraphics[width=0.8\textwidth]{brownian}
\end{center}
\end{column}
\end{columns}

\end{frame}


\subsection{solving the SIA}

\begin{frame}{similarity solution to SIA}

\begin{itemize}
\item \emph{1981}:  P.~Halfar discovers the similarity solution of the SIA in the case of flat bed and no surface mass balance
\item Halfar's 2D solution (1983) for Glen flow law with $n=3$ has scalings
   $$H(t,r)=t^{-1/9} \phi(s), \qquad s = t^{-1/18} r$$
\item \dots so the diffusion of ice really slows down as the shape flattens out!
\end{itemize}
\end{frame}


\begin{frame}{Halfar solution to the SIA: the movie}
\label{slide:plothalfar}

\animategraphics[autoplay,loop,height=7.0cm]{4}{anim/halfar}{0}{26}

\par
\scriptsize 
frames from $t=4$ months to $t = 10^6$ years, equal spaced in \emph{exponential} time
\end{frame}


\begin{frame}{Halfar solution to the SIA: the formula}

\begin{itemize}
\item for $n=3$ the solution formula is:
  $$H(t,r) = H_0 \left(\frac{t_0}{t}\right)^{1/9} \left[1 - \left(\left(\frac{t_0}{t}\right)^{1/18} \frac{r}{R_0}\right)^{4/3}\right]^{3/7}$$
\item the ``characteristic time'' is
  $$t_0 = \frac{1}{18 \Gamma} \left(\frac{7}{4}\right)^3 \frac{R_0^4}{H_0^{7}}$$
if $H_0$, $R_0$ are central height and ice cap radius at $t=t_0$
\item you choose $H_0$ and $R_0$ and then determine $t_0$
\item it is a simple formula to use for verification!
\end{itemize}
\end{frame}


\begin{frame}{is the Halfar solution \emph{good for any modeling}?}

\begin{itemize}
\item John Nye and others (2000)\nocite{NyeIcarus2000} compared different flow laws for the South Polar Cap on Mars
\item they evaluated $\text{CO}_2$ ice and $\text{H}_2\text{O}$ ice softness parameters by comparing the long-time behavior of the corresponding Halfar solutions
\item conclusions:
  \begin{quote}
  \dots none of the three possible [$\text{CO}_2$] flow laws will allow a 3000-m cap, the thickness suggested by stereogrammetry, to survive for $10^7$ years, indicating that the south polar ice cap is probably not composed of pure $\text{CO}_2$ ice \dots the south polar cap probably consists of water ice, with an unknown admixture of dust.
  \end{quote}
\end{itemize}

\end{frame}


\begin{frame}
  \frametitle{computing diffusivity in SIA}

\begin{itemize}
\item back to numerics \dots
\item for numerical stability we compute $D = \Gamma H^{n+2} |\grad h|^{n-1}$ on the staggered grid
\item various schemes proposed \small (Mahaffy, 1976\nocite{Mahaffy}; van der Veen 1999\nocite{vanderVeen}; Hindmarsh and Payne 1996\nocite{HindmarshPayne})
\item all schemes involve
  \begin{itemize}
  \item[$\circ$] averaging $H$
  \item[$\circ$] differencing $h$
  \item[$\circ$] in a ``balanced'' way, for better accuracy,
  \end{itemize}
to get the diffusivity on staggered grid
\end{itemize}

\begin{columns}
\begin{column}{0.65\textwidth}
\begin{itemize}
\item Mahaffy stencil \large $\to$ \normalsize
\end{itemize}
\end{column}

\begin{column}{0.35\textwidth}
  \includegraphics[width=1.0\textwidth]{mahaffystencil}
\end{column}
\end{columns}
\end{frame}


\begin{frame}
  \frametitle{SIA implementation: flat bed case}

\minputtiny{siaflat}

\end{frame}


\begin{frame}[fragile]
\frametitle{verifying SIA code vs Halfar}
\label{slide:verifysia}

\begin{columns}
\begin{column}{0.6\textwidth}
\scriptsize
\begin{verbatim}
octave:40> verifysia(20)
average abs error            = 22.310
maximum abs error            = 227.849
octave:41> verifysia(40)
average abs error            = 9.490
maximum abs error            = 241.470
octave:42> verifysia(80)
average abs error            = 2.800
maximum abs error            = 155.796
octave:43> verifysia(160)
average abs error            = 1.059
maximum abs error            = 109.466
\end{verbatim}
\normalsize

\includegraphics[width=0.8\textwidth]{siaerror}
\end{column}

\begin{column}{0.4\textwidth}
\small
\emph{Trust but verify.}
\medskip

\scriptsize
(Ronald Reagan)

\bigskip\bigskip\bigskip

\includegraphics[width=1.0\textwidth]{eismintone}

\scriptsize \emph{figure 2 in Huybrechts et al.~(1996)}\nocite{EISMINT96}
\end{column}
\end{columns}
\end{frame}


\begin{frame}{demonstrate robustness}

see \texttt{roughice.m}, which calls \texttt{siaflat.m} after setting-up the nasty initial state at left:
\medskip

\begin{columns}
\begin{column}{0.5\textwidth}
\includegraphics[width=1.0\textwidth]{roughinitial}
\end{column}
\begin{column}{0.5\textwidth}
\includegraphics[width=1.0\textwidth]{roughfinal}
\end{column}
\end{columns}

\begin{center}
\includegraphics[width=0.4\textwidth]{roughtimesteps}
\end{center}
\end{frame}


\begin{frame}{model the Antarctic ice sheet}

\normalsize
\begin{itemize}
\item with careful-but-small modifications of \texttt{siaflat.m}, which make a good exercise:
  \begin{itemize}
  \item[$\circ$] observed accumulation as surface mass balance,
  \item[$\circ$] allow non-flat bed (so $H\ne h$),
  \item[$\circ$] compute surface slopes correctly where floating, and
  \item[$\circ$] calve at current calving front location
  \end{itemize}
here are results from this \emph{toy} Antarctic flow model
\item a 2000 model year run on a $\Delta x=50$ km grid; runtime a few seconds
\end{itemize}

\bigskip

\begin{columns}
\begin{column}{0.4\textwidth}
\includegraphics[height=1.75in]{antinitial}
\end{column}
\begin{column}{0.55\textwidth}
\includegraphics[height=1.75in]{antfinal}
\end{column}
\end{columns}
\end{frame}


\begin{comment}
\begin{frame}{final comments on SIA: origin and rigor}

where does the ``shallow ice approximation'' come from?:
\bigskip

\begin{itemize}
\item historically, Fowler and Larson (1978)\nocite{FowlerLarson1978}, Morland and Johnson (1980)\nocite{MorlandJohnson}, and Hutter (1983)\nocite{Hutter} \dots thus recent
\item logically, by a ``small-parameter argument'', based on a small depth-to-width ratio, from the more complete Stokes model for slow ice flow
\item more precisely, by using the small aspect ratio \, $\eps = [H]/[L]$ \, of ice sheets to scale the Stokes model to see which terms make small contributions
\end{itemize}
\end{frame}
\end{comment}
