\documentclass[10pt,dvipsnames]{beamer}

\definecolor{links}{HTML}{2A1B81}
\hypersetup{colorlinks,linkcolor=,urlcolor=links}

\usetheme{metropolis}
\metroset{block=fill}

\usepackage{appendixnumberbeamer}

\usepackage{booktabs,empheq,bbold,bm}
\usepackage[scale=2]{ccicons}

\usepackage{pgfplots}
\usepgfplotslibrary{dateplot}

\usepackage{xspace}
\newcommand{\themename}{\textbf{\textsc{metropolis}}\xspace}

% for python listings; inspired by https://gist.github.com/YidongQIN/a10dd4f72381362aff4257e7a5541d86 
\usepackage{listings}
\usepackage{color}
\definecolor{darkred}{rgb}{0.6,0.0,0.0}
\definecolor{darkgreen}{rgb}{0,0.50,0}
\definecolor{lightblue}{rgb}{0.0,0.42,0.91}
\definecolor{orange}{rgb}{0.99,0.48,0.13}
\definecolor{grass}{rgb}{0.18,0.80,0.18}
\definecolor{pink}{rgb}{0.97,0.15,0.45}
\lstdefinelanguage{PythonPlus}[]{Python}{
  morekeywords=[1]{,as,assert,nonlocal,with,yield,self,True,False,None,} % Python builtin
  morekeywords=[2]{,__init__,__add__,__mul__,__div__,__sub__,__call__,__getitem__,__setitem__,__eq__,__ne__,__nonzero__,__rmul__,__radd__,__repr__,__str__,__get__,__truediv__,__pow__,__name__,__future__,__all__,}, % magic methods
  morekeywords=[3]{,object,type,isinstance,copy,deepcopy,zip,enumerate,reversed,list,set,len,dict,tuple,range,xrange,append,execfile,real,imag,reduce,str,repr,}, % common functions
  morekeywords=[4]{,Exception,NameError,IndexError,SyntaxError,TypeError,ValueError,OverflowError,ZeroDivisionError,}, % errors
  morekeywords=[5]{,ode,fsolve,sqrt,exp,sin,cos,arctan,arctan2,arccos,pi, array,norm,solve,dot,arange,isscalar,max,sum,flatten,shape,reshape,find,any,all,abs,plot,linspace,legend,quad,polyval,polyfit,hstack,concatenate,vstack,column_stack,empty,zeros,ones,rand,vander,grid,pcolor,eig,eigs,eigvals,svd,qr,tan,det,logspace,roll,min,mean,cumsum,cumprod,diff,vectorize,lstsq,cla,eye,xlabel,ylabel,squeeze,}, % numpy / math
}
\lstdefinestyle{colorEX}{
  basicstyle=\ttfamily\small,
  backgroundcolor=\color{white},
  commentstyle=\color{darkgreen}\slshape,
  keywordstyle=\color{blue}\bfseries\itshape,
  keywordstyle=[2]\color{blue}\bfseries,
  keywordstyle=[3]\color{grass},
  keywordstyle=[4]\color{red},
  keywordstyle=[5]\color{orange},
  stringstyle=\color{darkred},
  emphstyle=\color{pink}\underbar,
}
\lstset{style=colorEX}

\newcommand{\bn}{\mathbf{n}}
\newcommand{\bq}{\mathbf{q}}
\newcommand{\bu}{\mathbf{u}}

\newcommand{\hbn}{\hat{\mathbf{n}}}
\newcommand{\hbx}{\hat{\mathbf{x}}}
\newcommand{\hbz}{\hat{\mathbf{z}}}

\newcommand{\bU}{\mathbf{U}}

\newcommand{\bpsi}{\bm{\psi}}
\newcommand{\bzero}{\bm{0}}

\newcommand{\eps}{\epsilon}
\newcommand{\grad}{\nabla}
\newcommand{\Div}{\nabla\cdot}

\newcommand{\rhoi}{\rho_{\text{i}}}
\newcommand{\snew}{s^{\text{new}}}

\newcommand{\comm}[1]{{\footnotesize \hfill \emph{#1}}}

\title{numerical modeling of glaciers}
%\subtitle{A modern beamer theme}
\date{June 2024}
\author{Ed Bueler}
\institute{Porphyry Place, McCarthy, Alaska}
\titlegraphic{\vspace{-1cm}\par\hspace{-1cm}\includegraphics[width=1.5\textwidth]{polaris-overexposed.png}}

\begin{document}
\graphicspath{{figs/}{../figures/}}

\maketitle

\begin{frame}{Outline}
  %\setbeamertemplate{section in toc}[sections numbered]
  %\tableofcontents[hideallsubsections]
  \tableofcontents
\end{frame}

\begin{frame}{online materials}

\begin{itemize}
\item these are my \alert{new slides} \dots\xspace please let me know how it works?
\item in any case, \alert{please ask questions at any time!}
\end{itemize}

\bigskip

\metroset{block=fill}
\begin{block}{online materials}

my slides, notes, codes, and materials are hosted at

\medskip
\centerline{\href{https://github.com/bueler/mccarthy}{\texttt{github.com/bueler/mccarthy}}}
\end{block}

\medskip
{\footnotesize
these slides: {\footnotesize \href{https://github.com/bueler/mccarthy/blob/master/slides/slides-2024.pdf}{\texttt{mccarthy/blob/master/slides/slides-2024.pdf}}}

notes: {\footnotesize \href{https://github.com/bueler/mccarthy/blob/master/notes/notes-2024.pdf}{\texttt{mccarthy/blob/master/notes/notes-2024.pdf}}}

Python codes: {\footnotesize \href{https://github.com/bueler/mccarthy/tree/master/py}{\texttt{mccarthy/tree/master/py/}}}

project 13,14 codes: {\footnotesize \href{https://github.com/bueler/mccarthy/tree/master/stokes}{\texttt{mccarthy/tree/master/stokes/}}}
}
\end{frame}

\section[how does the surface of a glacier move?]{\textbf{how does the surface of a glacier move?} (hour 1)}

\subsection{basic mathematics}

\begin{frame}{surface motion}

\bigskip
\begin{center}
\includegraphics<1>[width=\textwidth]{noboat}
\includegraphics<2>[width=\textwidth]{boat}
\includegraphics<3>[width=\textwidth]{boatplus}
\end{center}

%LIVE really the question is, "what is the most basic thing you can say mathematically at the surface of a glacier?"
%LIVE imagine a magic glacier boat which always floats on the top of the ice.
\end{frame}

\begin{frame}{surface motion: notation}
\begin{center}
\begin{tikzpicture}[scale=1.2]
\node[inner sep=0pt] (domain) at (0,0)
    {\includegraphics[width=0.72\textwidth]{domain}};
\node (a) at (-2.5,0.8) {$a(t,x)$};
\node (b) at (1.5,-1.0) {$b(x)$};
\node (s) at (-1.6,0.0) {$s(t,x)$};
\node (u) at (0.0,-0.5) {$\bu(t,x,z)$};
\end{tikzpicture}
\end{center}

\vspace{-3mm}
\begin{itemize}
\item two spatial dimensions $x,z$ \comm{\dots\, at least for a while}
\item $z = b(x)$: bed elevation (m) \comm{fixed}
\item $z = s(t,x)$: ice surface elevation (m)
\item $\bu(t,x,z)=(u,w)$: ice velocity ($\text{m}\,\text{s}^{-1}$)
\item $a(t,x)$: surface mass balance in ice-equivalent units ($\text{m}\,\text{s}^{-1}$)
%\item {\color{NavyBlue} $z = s(t,x)$}: ice surface elevation (m)
%\item {\color{NavyBlue} $\bu(t,x,z)=(u,w)$}: ice velocity ($\text{m}\,\text{s}^{-1}$)
%\item {\color{Orange} $a(t,x)$}: surface mass balance in ice-equivalent units ($\text{m}\,\text{s}^{-1}$)
\end{itemize}
\end{frame}

\begin{frame}{surface kinematical equation (SKE)}
\begin{center}
\includegraphics[width=0.5\textwidth]{boatplus}
\end{center}

\vspace{-5mm}
\begin{equation*}
\frac{\partial s}{\partial t} = \uncover<2-4>{a} \uncover<3-4>{+w} \uncover<4>{-u \frac{\partial s}{\partial x}}
\end{equation*}

the ice surface moves up and down according to
\begin{itemize}
\item<2-4> surface mass balance
\item<3-4> vertical component of ice velocity
\item<4> horizontal component of ice velocity \emph{and surface slope}
\end{itemize}
\end{frame}

\begin{frame}{surface kinematical equation (SKE) \dots alternate form}
\begin{center}
\includegraphics[width=0.5\textwidth]{boatplus}
\end{center}

\vspace{-5mm}
\begin{equation*}
\frac{\partial s}{\partial t} = a + \bu \cdot \bn_s
\end{equation*}

\begin{itemize}
\item $\bn_s$ is a vector which points upward and is normal (perpendicular) to the ice surface:
	$$\bn_s = \left(-\frac{\partial s}{\partial x}, \,1\right)$$
\item recall $\bu=(u,w)$
\end{itemize}
\end{frame}

\begin{frame}{plan for hour 1: numerical glacier geometry model using SKE}

\begin{itemize}
\item \alert{model:} the surface kinematical equation (SKE)
\begin{equation*}
\frac{\partial s}{\partial t} = a + \bu \cdot \bn_s \qquad \iff \qquad \frac{\partial s}{\partial t} + u \frac{\partial s}{\partial x} = a + w
\end{equation*}
\item \alert{numerical goal:} use a computer program to track the glacier surface $z=s(t,x)$, under some assumptions
    \begin{itemize}
    \item[$\circ$] grid of points $\{x_j\}$
    \item[$\circ$] unknowns are the gridded surface elevations $s_j \approx s(t_n,x_j)$
    \item[$\circ$] assume $a,u,w$ are given functions/values \uncover<2>{\hfill \alert{$\leftarrow$ this needs fixing!}}
    \item[$\circ$] from one short Python program in 1D
    \end{itemize}
\item \alert{problems and issues:}
     \begin{enumerate}
     \item how to compute time-dependent solutions?
     \item accuracy and stability?
     \item how to compute steady state solutions?
     \item practical: debugging, verification, visualization?
     \end{enumerate}
\end{itemize}
\end{frame}


\begin{frame}[standout]
the central object of glacier theory is the

surface kinematical equation

\begin{equation*}
\frac{\partial s}{\partial t} + u \frac{\partial s}{\partial x} = a + w
\end{equation*}
\end{frame}


\subsection{numerics}


\begin{frame}{discretize the SKE 1}

\begin{center}
\includegraphics[width=0.7\textwidth]{surfacenotation}
\end{center}
\end{frame}

\begin{frame}{discretize the SKE 2}
\begin{center}
\includegraphics[width=0.5\textwidth]{surfacenotation}
\end{center}

\begin{itemize}
\item choose time step $\Delta t>0$ and grid spacing $\Delta x>0$
\item in SKE
\begin{equation*}
\frac{\partial s}{\partial t} + u \frac{\partial s}{\partial x} = a + w
\end{equation*}
approximate partial derivatives by finite difference\footnote{helpful textbooks include LeVeque \cite{Leveque2007} and Morton \& Mayers \cite{MortonMayers2005}} quotients:
    $$\frac{\partial s}{\partial t} \approx \frac{\snew_j - s_j}{\Delta t}, \qquad \frac{\partial s}{\partial x} \approx \frac{s_{j+1} - s_{j-1}}{2\Delta x}$$
\phantom{foo}
\end{itemize}
\end{frame}

\begin{frame}{discretize the SKE 3}
\begin{center}
\includegraphics[width=0.5\textwidth]{surfacenotation}
\end{center}

\begin{itemize}
\item SKE
\begin{equation*}
\frac{\partial s}{\partial t} + u \frac{\partial s}{\partial x} = a + w
\end{equation*}
\item becomes a discrete equation:
    $$\frac{\snew_j - s_j}{\Delta t} + u_j \frac{s_{j+1} - s_{j-1}}{2\Delta x} = a_j + w_j$$
\item write as an update which determines $\snew_j$:
	$$\snew_j = s_j - \Delta t\, u_j \frac{s_{j+1}-s_{j-1}}{2\Delta x} + \Delta t (a_j + w_j)$$
\end{itemize}
\end{frame}

\begin{frame}{discretize the SKE 4}

\begin{itemize}
\item \alert{however \dots} \comm{see textbooks for these caveats!}
\item if we implement this ``centered-space'' scheme
	$$\snew_j = s_j - \Delta t\, u_j \frac{s_{j+1}-s_{j-1}}{2\Delta x} + \Delta t (a_j + w_j) \hspace{1.0cm} \leftarrow \text{\emph{\alert{bad}}}$$
then bad things happen!
\item instead, it is known that when the ``advecting velocity'', here $u_j$, is positive then the ``upwind'' version is stable:
	$$\hspace{3mm} \snew_j = s_j - \Delta t\, u_j \frac{s_j-s_{j-1}}{\Delta x} + \Delta t (a_j + w_j) \hfill \hspace{1.2cm} \leftarrow \text{\emph{useful}}$$

    \begin{itemize}
    \item[$\circ$] at least, it is ``conditionally stable'' and the condition is known
    \item<3>[$\circ$] when $u_j<0$ use $\partial s/\partial x \approx (s_{j+1}-s_j)/\Delta x$
    \end{itemize}

\bigskip
\item<2->[] \alert{I will show you these things when we run it!}
\end{itemize}
\end{frame}

\begin{frame}[fragile]\frametitle{discretize the SKE 5}

\begin{itemize}
\item the upwind equation:
    $$\snew_j = s_j - \Delta t\, u_j \frac{s_{j+1}-s_j}{\Delta x} + \Delta t (a_j + w_j)$$
becomes a Python function:
\end{itemize}
\begin{lstlisting}[language=PythonPlus]
def explicitstep(s, x, dt):
    dx = x[1] - x[0]
    snew = s.copy()
    snew[1:] -= dt * u(x[1:]) * (s[1:] - s[:-1]) / dx
    snew[1:] += dt * (a(x[1:]) + w(x[1:]))
    return snew
\end{lstlisting}

\vspace{-2mm}
{\footnotesize
    \begin{itemize}
    \item[$\circ$] \texttt{s} and \texttt{x} are 1D NumPy arrays
    \item[$\circ$] separate functions define $a(x_j)$, $u(x_j)$, $w(x_j)$
    \item[$\circ$] \texttt{x} is equally-spaced
    \item[$\circ$] $j=$\texttt{[1:]} indicates all indices except zero
    \item[$\circ$] note that \, \texttt{snew[0] = s[0]} \, is never changed
    \end{itemize}
}
\end{frame}


\begin{frame}[fragile]
\frametitle{live demo: evolution of glacier surface}
\begin{center}
\includegraphics[width=0.7\textwidth]{frame010}
\end{center}

\bigskip
\begin{block}{live demo}
\begin{itemize}
\item run \texttt{mccarthy/py/surface.py}:
\begin{verbatim}
$ python3 surface.py
$ eog output/frame*.png            # any image viewer
\end{verbatim}
\item the following modifications reveal issues:
    \begin{enumerate}
    \item increase velocity: \quad $u(x)$ \, $\to$ \, $4 u(x)$
    \item or lengthen time-steps: \quad $\Delta t = 1$ a \, $\to$ \, $\Delta t = 2.5$ a
    \end{enumerate}
\end{itemize}
\end{block}
\end{frame}

\begin{frame}{condition for time-stepping stability}
\begin{center}
\includegraphics[width=0.7\textwidth]{steady} FIXME cfl fig
\end{center}

\bigskip
\begin{itemize}
\item FIXME cfl
$$u_j \frac{\Delta t}{\Delta x} \le 1$$

$$\Delta t \le \frac{\Delta x}{\max |u_j|}$$
\end{itemize}
\end{frame}

\begin{frame}[fragile]
\frametitle{live demo cont.: steady state of a glacier surface}
\begin{center}
\includegraphics[width=0.7\textwidth]{steady}
\end{center}

\bigskip
\begin{itemize}
\item SKE in steady-state ($\frac{\partial s}{\partial t}=0$): \quad $\displaystyle u \frac{\partial s}{\partial x} = a + w$
\end{itemize}

\begin{block}{live demo}
\begin{itemize}
\item same run as before, but view steady-state output:
\begin{verbatim}
$ eog output/steady.png
\end{verbatim}
\item the following modification reveals an issue:
    \begin{enumerate}
    \item double SMB: \quad $a(x)$ \, $\to$ \, $2 a(x)$
    \end{enumerate}
\end{itemize}
\end{block}
\end{frame}

\begin{frame}{2D steady state of a glacier surface}

\begin{itemize}
\item this example uses \href{https://www.firedrakeproject.org/}{Firedrake}
{\footnotesize
    \begin{itemize}
    \item[$\circ$] \emph{see me?}
    \item[$\circ$] \emph{or project 6,13,14 people!}
    \end{itemize}
}
\item code:

\texttt{mccarthy/py/surface2d.py}

\item result figure $\rightarrow$

\vspace{-25mm}
\mbox{\hspace{60mm} \includegraphics[width=0.45\textwidth]{surface2d}}

\vspace{-12mm}
\item solves 2D steady-state SKE:
\begin{align*}
&& u \frac{\partial s}{\partial x} + v \frac{\partial s}{\partial y} &= a + w \hspace{40mm}\\
\text{\emph{or}} && (u,v) \cdot \grad s &= a + w \\
\text{\emph{or}} && -\bu \cdot \bn_s &= a
\end{align*}

{\footnotesize
    \begin{itemize}
    \item[$\circ$] where $\bn_s = (-\frac{\partial s}{\partial x},-\frac{\partial s}{\partial y},1) = (-\grad s,1)$
    \item[$\circ$] \alert{but subject to $s \ge b$}
    \end{itemize}
}

\medskip
\item note time-dependent SKE in 2D: \quad $\displaystyle \frac{\partial s}{\partial t} - \bu \cdot \bn_s = a$
\end{itemize}
\end{frame}

\begin{frame}{many issues}
\begin{itemize}
\item FIXME
\end{itemize}
\end{frame}


\section[where does the velocity field come from?]{\textbf{where does the velocity field come from?} (hour 2)}

\subsection{basic mathematics}

\begin{frame}{ice in glaciers is an atypical fluid}

\begin{itemize}
\item if the ice were
  \begin{itemize}
  \item[$\circ$] faster-moving than it actually is, and
  \item[$\circ$] linearly-viscous like liquid water
  \end{itemize}

  then it would be a ``typical'' fluid

\bigskip
\item for typical fluids one uses the Navier-Stokes equations:
\begin{align*}
\nabla \cdot \mathbf{u} &= 0 &&\text{\emph{incompressibility}} \\
\rho \left(\mathbf{u}_t + \mathbf{u}\cdot\nabla \mathbf{u}\right) &= -\nabla p + \nabla \cdot \tau_{ij} + \rho \mathbf{g} &&\text{\emph{stress balance}} \\
2 \nu D\mathbf{u}_{ij} &= \tau_{ij} &&\text{\emph{flow law}}
\end{align*}

\medskip
    \begin{itemize}
    \item[$\circ$] stress balance equation is ``$m a = F$''
    \end{itemize}
\end{itemize}
\end{frame}


\begin{frame}{glaciology as computational fluid dynamics}

\begin{itemize}
\item \alert{yes}, numerical ice sheet flow modelling is ``computational fluid dynamics''
  \begin{itemize}
  \item[$\circ$] it's large-scale like atmosphere and ocean
  \item[$\circ$] \dots\, but it is a weird one
  \end{itemize}
\item consider what makes atmosphere/ocean flow exciting:
  \begin{itemize}
  \item[$\circ$] turbulence
  \item[$\circ$] convection
  \item[$\circ$] coriolis force
  \item[$\circ$] density stratification
  \end{itemize}
\item none of the above list is relevant to ice flow
\item what could be interesting about the flow of slow, cold, stiff, laminar, inert old ice?
  \begin{itemize}
  \item[$\circ$] \emph{ice dynamics!}
  \end{itemize}
\end{itemize}
\end{frame}


\begin{frame}{ice is a slow, shear-thinning fluid}

\begin{itemize}
\item ice fluid is \emph{slow} and \emph{non-Newtonian}
    \begin{itemize}
    \item[$\circ$] ``slow'' is a technical term:
      $$\rho \left(\mathbf{u}_t + \mathbf{u}\cdot\nabla \mathbf{u}\right) \approx 0 \qquad \iff \qquad \begin{pmatrix} \text{forces of inertia} \\ \text{are neglected} \end{pmatrix}$$
    \item[$\circ$] ice is non-Newtonian in a ``shear-thinning'' way
        \begin{itemize}
        \item higher strain rates means lower viscosity
        \item viscosity $\nu$ is not constant
        \end{itemize}
    \end{itemize}

\bigskip
\item thus the standard model is Glen-law Stokes:
\begin{align*}
\nabla \cdot \mathbf{u} &= 0 &&\text{\emph{incompressibility}} \\
0 &= - \nabla p + \nabla \cdot \tau_{ij} + \rho\, \mathbf{g} &&\text{\emph{stress balance}} \\
D\mathbf{u}_{ij} &= A \tau^{n-1} \tau_{ij} &&\text{\emph{flow law}}
\end{align*}

\end{itemize}
\end{frame}


\begin{frame}{``slow'' means no memory of velocity/momentum}

\begin{itemize}
\item note \emph{no time derivatives} in Stokes model:
\small
\begin{align*}
\nabla \cdot \mathbf{u} &= 0 \\
0 &= - \nabla p + \nabla \cdot \tau_{ij} + \rho\, \mathbf{g} \\
D\mathbf{u}_{ij} &= A \tau^{n-1} \tau_{ij}
\end{align*}
\normalsize
\item thus a time-stepping ice sheet code can/must recompute the full velocity field at every time step
  \begin{itemize}
  \item[$\circ$] velocity is \emph{not} required from the previous time step
  \item[$\circ$] velocity is a ``diagnostic'' output not needed for starting or restarting the model
  \end{itemize}
\end{itemize}
\end{frame}


\begin{frame}{plane flow Stokes}

\begin{itemize}
\item again we work in a $x,z$ plane
    \begin{itemize}
    \item[$\circ$] glacier flow line (center line), or a cross-flow plane
    \end{itemize}
\item $n=3$ Stokes equations say:
\begin{empheq}[]{align}
u_x + w_z &= 0 &&\text{\emph{incompressibility}}\notag \\
p_x &= \tau_{11,x} + \tau_{13,z} &&\text{\emph{stress balance} ($x$)} \notag \\
p_z &= \tau_{13,x} - \tau_{11,z} - \rho g &&\text{\emph{stress balance} ($z$)} \notag \\
u_x &= A \tau^2 \tau_{11} &&\text{\emph{flow law (diagonal)}}\notag \\
u_z + w _x &= 2 A \tau^2 \tau_{13} &&\text{\emph{flow law (off-diagonal)}} \notag
\end{empheq}

\vspace{-2mm}
    \begin{itemize}
    \item[$\circ$] \emph{notation}: subscripts $x,z$ denote partial derivatives, $\tau_{13}$ is the ``vertical'' shear stress, $\tau_{11}$ and $\tau_{33}=-\tau_{11}$ are (deviatoric) longitudinal stresses
    \end{itemize}
\item we have five equations in five unknowns ($u,w,p,\tau_{11},\tau_{13}$)
\item this is complicated enough \dots what about in a simplified situation?
\end{itemize}
\end{frame}


\begin{frame}{slab-on-a-slope}

\hfill \includegraphics[width=0.4\textwidth]{slab}

\vspace{-30mm}
\begin{itemize}
\item suppose constant thickness
\item tilt bedrock by angle $\alpha$
\item rotate the coordinates
\item get replacement expressions:
\begin{align*}
\mathbf{g} &= g \sin\alpha\, \hat x - g \cos \alpha \,\hat z \phantom{dslfkj sdkfjlskdjf  sdlfj}\\
p_x &= \tau_{11,x} + \tau_{13,z} + \rho g \sin\alpha \\
p_z &= \tau_{13,x} - \tau_{11,z} - \rho g \cos\alpha
\end{align*}
\item for \alert{slab-on-a-slope} there is \emph{no variation in} $x$:\quad $\partial/\partial x = 0$
\item the equations simplify:
\small
\begin{empheq}[box=\fbox]{align}
w_z &= 0 &   0 &= \tau_{11} \notag \\
\tau_{13,z} &= - \rho g \sin\alpha &   u_z &= 2 A \tau^2 \tau_{13} \notag \\
p_z &= - \rho g \cos\alpha \notag
\end{empheq}
\end{itemize}
\end{frame}


\begin{frame}{slab-on-a-slope 2}

\begin{itemize}
\item add boundary conditions:
	$$w(\text{base})=0, \qquad p(\text{surface})=0, \qquad u(\text{base})=u_0$$
\item by integrating vertically, get:
\begin{align*}
w &= 0 \phantom{asdfklj asldkfjalk asdfkj sdlfkj sldafkj adlfjl sdfakj }\\
p &= \rho g \cos\alpha (H-z) \\
\tau_{13} &= \rho g \sin\alpha (H-z)
\end{align*}

\vspace{-25mm}
\hfill \includegraphics[width=0.4\textwidth]{slabshear}

\vspace{-7mm}
\item $\tau_{13}$ is linear in depth

\medskip
\item from $u_z = 2 A \tau^2 \tau_{13}$ get \alert{velocity formula}
\vspace{-0.05in}
\begin{align*}
u(z) &= u_0 + 2 A (\rho g \sin\alpha)^3 \int_0^z (H-z')^3\,dz' \\
     &= u_0 + \frac{1}{2} A (\rho g \sin\alpha)^3  \left(H^4 - (H-z)^4\right)
\end{align*}
\end{itemize}
\end{frame}


\begin{frame}{slab-on-a-slope 3}

\begin{columns}
\begin{column}{0.6\textwidth}
\begin{itemize}
\item do we believe these equations?
\item velocity formula on last slide gives figure below
\item compare to observations at right
\end{itemize}
\begin{center}
% NOT preserving aspect ratio
\includegraphics[width=0.6\textwidth,height=0.5\textheight]{slabvel}
\end{center}
\end{column}

\begin{column}{0.4\textwidth}
\includegraphics[width=1.0\textwidth]{athabasca-deform}

\medskip
\scriptsize
Velocity profile of the Athabasca Glacier, Canada, derived from inclinometry \cite{SavagePaterson1963}
\end{column}
\end{columns}
\end{frame}


\subsection{numerics}

\begin{frame}{Summary}
FIXME
\end{frame}

\begin{frame}[standout]
  Questions?
\end{frame}

\appendix

\setbeamerfont{bibliography item}{size=\scriptsize}
\setbeamerfont{bibliography entry author}{size=\scriptsize}
\setbeamerfont{bibliography entry title}{size=\scriptsize}
\setbeamerfont{bibliography entry location}{size=\scriptsize}
\setbeamerfont{bibliography entry note}{size=\scriptsize}

\setbeamertemplate{frametitle continuation}{}  % remove the "i"
\begin{frame}[allowframebreaks]
\frametitle{References}
\setbeamertemplate{bibliography item}[text]

  \bibliography{slides}
  \bibliographystyle{alpha}
\end{frame}


\subsection[]{extra slides}


\begin{frame}[standout]
Extra Slides

\begin{itemize}
\item 1. derive SKE from mass conservation
%\item 2. something
\end{itemize}
\end{frame}

\begin{frame}{SKE from mass conservation}
\begin{center}
\includegraphics[width=0.65\textwidth]{skederive.png}
\end{center}

\begin{itemize}
\item these extra slides derive the surface kinematical equation (SKE) from mass conservation applied over a rectangle
\item contrast with standard (dismissive!) derivations \cite{GreveBlatter2009,SchoofHewitt2013}
\end{itemize}
\end{frame}

\begin{frame}{SKE from mass conservation 2}
\begin{center}
\includegraphics[width=0.4\textwidth]{skederive.png}
\end{center}

\begin{itemize}
\item $\Omega = (\text{rectangle centered at} (x_0,z_0)) =[x_0-\delta,x_0+\delta] \times [z_0-\eps,z_0+\eps]$

    \begin{itemize}
    \item[$\circ$] where $s(t_0,x_0)=z_0$
    \end{itemize}
\item let $M(t)$ be the mass of (constant-density) ice within $\Omega$ at time $t$:
   $$M(t) = \int_\Omega \rhoi \mathbb{1}_{\{z<s(t,x)\}}\,dx\,dz = \rhoi \int_{x_0-\delta}^{x_0+\delta} s(t,x) - z_0 + \eps\,dx$$
\item notation: $\bpsi =$ mass flux, $\hbn =$ outward unit normal
\end{itemize}
\end{frame}

\begin{frame}{SKE from mass conservation 3}
\begin{center}
\includegraphics[width=0.4\textwidth]{skederive.png}
\end{center}

\vspace{-3mm}

\begin{itemize}
\item assumptions:
    \begin{itemize}
    \item[$\circ$] there are no mass sources within $\Omega$
    \item[$\circ$] there is an interval of time $t$ so that \,$z_0-\eps < s(t,x) < z_0 + \eps$
    %\item[$\circ$] ice has constant density $\rhoi$
    \item[$\circ$] the ice velocity $\bu(t,x,z)=(u,w)$ is defined on $\{z<s(t,x)\}$
    \item[$\circ$] the SMB $a(t,x)$ is a mass flux only across $\Gamma_t$:
        \begin{itemize}
        \item $\bpsi\big|_{\Gamma_t} = - \rho_i a(t,x) \hbz$  \comm{note $a>0$ is precipitation}
        \end{itemize}
    \item[$\circ$] the other mass fluxes are advective:
        \begin{itemize}
        \item $\bpsi\big|_{\Gamma_b} = \rho_i w(t,x,z_0-\eps) \hbz$
        \item $\bpsi\big|_{\Gamma_{\{\ell,r\}}} = \begin{cases} \rho_i u(t,x_{\{\ell,r\}},z) \hbx, & \text{if } z < s(t,x_{\{\ell,r\}}), \\ \bzero, & \text{otherwise} \end{cases}$
        \end{itemize}
    \end{itemize}
\end{itemize}
\end{frame}

\begin{frame}{SKE from mass conservation 4}

\begin{itemize}
\item mass is conserved:
\begin{align*}
\frac{dM}{dt} &= - \int_{\partial \Omega} \bpsi \cdot \hbn\,ds \\
  &= \underbrace{\rhoi \int_{x_0-\delta}^{x_0+\delta} a(t,x)\,dx}_{\Gamma_t} + \underbrace{\rhoi \int_{x_0-\delta}^{x_0+\delta} w(t,x,z_0-\eps)\,dx}_{\Gamma_b} \\
  &\quad + \underbrace{\rhoi \int_{z_0-\eps}^{s(t,x_0-\delta)} u(t,x_0-\delta,z)\,dz}_{\Gamma_\ell} - \underbrace{\rhoi \int_{z_0-\eps}^{s(t,x_0+\delta)} u(t,x_0+\delta,z)\,dz}_{\Gamma_r}
\end{align*}

\end{itemize}
\end{frame}

\begin{frame}{SKE from mass conservation 5}

\begin{itemize}
\item time derivative:
\begin{align*}
\frac{dM}{dt} &= \lim_{\omega\to 0} \frac{M(t+\omega) - M(t)}{\omega} \\
    &= \lim_{\omega\to 0} \frac{\rhoi}{\omega} \int_\Omega \mathbb{1}_{\{z<s(t+\omega,x)\}} - \mathbb{1}_{\{z<s(t,x)\}}\,dz\,dx \\
    &= \lim_{\omega\to 0} \frac{\rhoi}{\omega} \int_{x_0-\delta}^{x_0+\delta} s(t+\omega,x) - s(t,x)\,dx \\
    &= \rho_i \int_{x_0-\delta}^{x_0+\delta} \frac{\partial s}{\partial t}(t,x)\,dx
\end{align*}
\item assume smoothness sufficient for Taylor expansions on $\Omega$:
\begin{align*}
a(t,x) &= a(t,x_0) + O(\delta) \\
u(t,x,z) &= u(t,x_0,z_0) + O(\delta) + O(\eps) \\
w(t,x,z) &= w(t,x_0,z_0) + O(\delta) + O(\eps) \\
\frac{\partial s}{\partial t}(t,x) &= \frac{\partial s}{\partial t}(t,x_0) + O(\delta)
\end{align*}
\end{itemize}
\end{frame}

\begin{frame}{SKE from mass conservation 6}

\begin{itemize}
\item combine mass conservation, time derivative, and Taylor:
\begin{align*}
2\delta \rhoi \frac{\partial s}{\partial t}(t,x_0) + O(\delta^2) &= 2\delta \rhoi a(t,x_0) + O(\delta^2) \\
  &\quad + 2\delta \rhoi w(t,x_0,z_0) + O(\delta^2) + O(\delta \eps) \\
  &\quad - \rho_i u(t,x_0,z_0) \Big[s(t,x_0+\delta) - s(t,x_0-\delta)\Big] + O(\delta \eps)
\end{align*}
\item divide by $2\rhoi\delta$:
\begin{align*}
\frac{\partial s}{\partial t}(t,x_0) &= a(t,x_0) + w(t,x_0,z_0) - u(t,x_0,z_0) \frac{s(t,x_0+\delta) - s(t,x_0-\delta)}{2\delta} \\
  &\quad  + O(\delta) + O(\eps)
\end{align*}
\item limit $\delta, \eps \to 0$ to get SKE at $(t,x_0,z_0) = (t,x_0,s(t,x_0))$:
  $$\frac{\partial s}{\partial t} = a + w - u \frac{\partial s}{\partial x}$$
\end{itemize}
\end{frame}

\end{document}
